\documentclass[onecolumn,tighten]{aastex6}

\shorttitle{XSPEC Counting Errors}
\slugcomment{Draft, \today}

\usepackage{amsmath}
%\usepackage{hyperref}

% My "standard" TeX aliases
\newcommand*{\mt}{\mathrm}
\newcommand*{\code}{\texttt}

\begin{document}

\title{XSPEC Counting Errors in Plots and Fits}

\author{Aaron Tran}
\affil{Smithsonian Astrophysical Observatory,
60 Garden St. MS-70, Cambridge, MA 02138, USA}
\email{atran@cfa}

\begin{abstract}
XSPEC defaults datum errors to \code{STAT\_ERR} if available, Poisson standard
deviations $\sqrt{N}$ otherwise.
Non-default weights or rebin error propagation override \emph{both} source
and background PHA \code{STAT\_ERR} columns.
Default XSPEC settings work well for fits to Poisson spectra with Poisson
backgrounds (with \code{statistic cstat}) or high-significance binned spectra
with approximately Gaussian errors \citep[but cf.][]{humphrey2009}.
Plotting Poisson spectra with Gaussian error backgrounds may require kludging
to plot correctly propagated errors.
Correct choice of fit statistic generally prevents incorrect error propagation
in most spectrum fitting cases relevant to photon-counting astronomers.
\end{abstract}

\section{Introduction}

XSPEC must calculate sensible errors for plotted error bars, parameter
estimation (fitting), and goodness-of-fit assessment (hypothesis testing),
based on the input spectra keywords and columns, and user settings.
The XSPEC manual descriptions are accurate but terse, and it takes time to
verify that errors are calculated as expected.

Errors must be propagated when:
\begin{itemize}
    \item weighting fits to background-subtracted spectra
    \item plotting errors for background-subtracted spectra
    \item binning data for plotting purposes
\end{itemize}

This writeup considers errors on data, \emph{not} error calculations (confidence
intervals) for model parameters from XSPEC's error command.
I also do not consider:
\begin{itemize}
  \item spectra binned using the GROUPING column
  \item goodness-of-fit assessment
  \item SYS\_ERR usage
  \item inconsistent FITS keywords (e.g. POISSERR=T when STAT\_ERR is given)
\end{itemize}

For general background, please see the XSPEC manual\footnote{\href{http://heasarc.gsfc.nasa.gov/docs/xanadu/xspec/manual/manual.html}{http://heasarc.gsfc.nasa.gov/docs/xanadu/xspec/manual/manual.html}},
\citet{cash1979}, and \citet{arnaud2011}.

\paragraph{Procedure}
I construct source and background spectra with \code{EXPOSURE = AREASCAL =
BACKSCAL = 1} so that plotted and fitted data are integer counts,
then vary XSPEC settings and check that count error bars and fit statistics
change according to the manual and my intuition.
All work is done in XSPEC v12.9.0d.
I have not personally checked all use cases described in this document, but
made some inferences based on XSPEC's behavior in a simple case.


\section{Errors in unbinned spectra}

Suppose you have source spectrum $X_i \pm \xi_{x,i}$ with background
$B_i \pm \xi_{b,i}$ representing counts per channel $i$ with optional
statistical errors $\xi_{x,b}$ from \code{STAT\_ERR}.
For each spectrum, \code{POISSERR=Y} is set if no \code{STAT\_ERR} column is
present.
No \code{GROUPING} column is present in the source spectrum.
We wish to fit both spectra using XSPEC's background handling; i.e., the
background spectrum is linked via the \code{BACKFILE} keyword or XSPEC's
\code{background} command.
From here out, I drop the channel subscript $i$ for clarity.

Each datum has an ``internal'' weight, or error, $\varepsilon$ set by XSPEC's \code{weight} command.
\begin{itemize}
  \item \code{weight standard} sets $\varepsilon_{x,b} = \xi_{x,b}$ if statistical errors are present, else sets $\varepsilon_{x,b} = \{\sqrt{x}, \sqrt{b}\}$
  \item \code{weight gehrels} sets $\varepsilon_{x,b} = 1 + \sqrt{ \{X,B\} + 0.75 }$, \emph{ignoring any statistical errors provided}.
\end{itemize}

The fit statistic, set by XSPEC's \code{statistic} command,
is usually a log-likelihood massaged to asymptote to a $\chi^2$ distribution.
\begin{itemize}
  \item \code{statistic chi} takes $\sqrt{ \varepsilon_x^2 + \varepsilon_b^2 }$ as its weighting (or just $\varepsilon_x$ if no background is present).
  \item \code{statistic cstat} ignores $\varepsilon_{x,b}$ entirely.  The Poisson distribution is defined by just one parameter, model count rate $m_i$, so variance is already folded into the likelihood function.
      Standard versus Gehrels weighting is irrelevant.
In comparison, a Gaussian distribution requires model counts $M_i$ and variance $\sigma_i$.
\footnote{It seems funny that, in our likelihood for Gaussian-distributed data, the model variance $\sigma_i$ is determined by the observed error $\varepsilon$.
  This is something I should understand better.  In a Bayesian context this
  makes more sense; a datum's error translates into spread in allowed model
  parameters, versus the frequentist's sampling from a monolithic model that
  is, seemingly arbitrarily, weighted by our observed error.}
  \item \code{statistic pgstat} takes $\varepsilon_b$ for the background gaussian error and ignores $\varepsilon_x$.
\end{itemize}
The \code{statistic} command does not change datum weights $\varepsilon_{x,b}$,
but it incorporates or ignore weights depending on the chosen statistic.
The \code{statistic} command \emph{does not check whether the current weighting
is appropriate for your data}.
For example, \code{statistic pgstat} used with \code{weight gehrels} will
\emph{incorrectly} use $\varepsilon_b = 1 + \sqrt{B + 0.75}$ in the Gaussian
likelihood calculation.

Finally, plotted error bars are controlled by XSPEC's \code{setplot rebin} command.
I ignore the binning parameters for now and focus on the error calculation for source and background.
\begin{itemize}
  \item \code{setplot rebin , , , quad} plots errors as $\sqrt{ \varepsilon_{x}^2 + \varepsilon_{b}^2 }$, where $\varepsilon_{x,b}$ is set by the weight command.
  \item \code{setplot rebin , , , sqrt} plots errors as $\sqrt{ X + B }$
  \item \code{setplot rebin , , , poiss-1} plots errors as $\sqrt{ \left( 1 + \sqrt{X + 0.75} \right)^2 + \left( 1 + \sqrt{B + 0.75} \right)^2 }$
\end{itemize}
Likewise for \code{poiss-2}, \code{poiss-3}.
Like the weight command, \code{setplot rebin} ignores statistical error unless
\code{setplot rebin , , , quad} is used.
The \code{setplot rebin} command also does not modify datum weights $\varepsilon_{x,b}$.
If the data are rebinned for plotting, then the error calculation is as above,
with the addition that counts are summed, or statistical errors added in
quadrature, following the manual.

So, the precedence of XSPEC commands for error calculation is:
\begin{itemize}
    \item Plotting: \code{setplot rebin} $>$ \code{weight} $>$ \code{PHA file keywords, columns}
    \item Fitting: \code{statistic} $>$ \code{weight} $>$ \code{PHA file keywords, columns}
\end{itemize}
Error calculations and weights are applied identically to both source and
background spectra, for both fits and plots, before source and background
errors are combined in quadrature.


\section{Example}

Suppose you have a Poisson-error spectrum and a Gaussian-error background.
This is relevant to XMM analysis with subtraction of ESAS generated QPB spectra.
From one paper I saw in passing, this can be applied to Fermi-LAT data as well.
%In truth the QPB spectrum is derived from Poisson count data, but it seems
%there are enough filter-wheel closed data so that the errors are roughly
%Gaussian.  And, part of the spectrum is a continuum interpolation that does not
%come with well-defined errors (or, at least, I think ESAS does not provide any
%errors for the interpolated channels).
The best approach, I think, is to fit using:
\begin{verbatim}
weight standard
statistic pgstat
setplot rebin , , , quad
\end{verbatim}
The \code{pgstat} fits will correctly incorporate provided background errors
while folding Poisson probability into the likelihood for low count spectrum
channels, preventing biased parameter estimates.

The plotted error bars are $\sqrt{ X + \varepsilon_{b}^2 }$.
This is OK, but $\sqrt{X}$ underestimates the $1$-$\sigma$ error for small $X$.
To apply the Gehrels approximation for plotted error bars \citep{gehrels1986},
you cannot change error bars with \code{weight} or \code{setplot rebin} because
incorrect background errors will be used in fits and plots respectively.

One way is to write a new \code{STAT\_ERR} column, containing the Gehrels
approximate errors, to your source PHA file.
With \code{pgstat}, the \code{STAT\_ERR} column should be used only for
plotting.
I believe that \code{pgstat} will warn, but should proceed nominally and
ignore the \code{STAT\_ERR} column.
\emph{TODO: I should be testing this soon.}
% TODO TODO test this ni my own code -- fit with / without stat_err column
But, if you use \code{setplot rebin ..., quad} to rebin the data for display,
the Gehrels errors from multiple source channels added in quadrature will be
too large; i.e.:
\[
    1 + \sqrt{ \Sigma_i X_i + 0.75}
    <
    \sqrt{ \sum_i \left( 1 + \sqrt{X_i + 0.75} \right)^2 }
\]
This can be shown by squaring the left side:
\begin{align*}
    \left( 1 + \sqrt{ \Sigma_i X_i + 0.75} \right)^2
        &= 1 + \Sigma_i X_i + 0.75 + 2 \sqrt{ \Sigma_i X_i + 0.75 } \\
%        &< \sum_i \left( 1 + X_i + 0.75 \right) + 2 \sqrt{ \sum_i \left( X_i + 0.75 \right) } \\
        &< \sum_i \left( 1 + X_i + 0.75 \right) + 2 \sum_i \sqrt{ X_i + 0.75 }
        = \sum_i \left( 1 + \sqrt{X_i + 0.75} \right)^2
\end{align*}
using the inequality $\sqrt{ \strut\Sigma_i x_i } < \sum_i \sqrt{x_i}$ for
$x_i \in \mathbb{R}^{+}$.
Using \code{setplot rebin ..., poiss-1} will yield the correct Gehrels error
for the binned source data, but would incorrectly ignore your background
spectrum's Gaussian errors as noted before; the error remains overestimated.

Alternatively, one can rebin to enough counts such that the Gehrels
approximation is unnecessary; i.e., $\sqrt{X}$ is close enough to the
$1$-$\sigma$ error for your needs.
The discrepancy is $26\%$ for $n=10$ and $13\%$ at $n=50$.
The true confidence limits and approximations thereto are given by
\citep[Table 4]{gehrels1986}.

The best would be simply to plot asymmetric, correct error bars in XSPEC.
But that doesn't seem possible at this time.


\section{XSPEC (internal) rebinning}

The \code{weight} command affects rebinning behavior.
XSPEC rebins data to a minimum significance per bin, and I believe that the
significance is simply the plotted $1$-$\sigma$ error.


\section{Errors in externally binned spectra}


How does XSPEC deal with the \code{GROUPING} column?
Often one simply bins to get up to an approximately Gaussian regime, fits using
chi, and moves on with life.
Since life is finite, that is fine in most cases.
But, I would like to understand how XSPEC handles errors for \code{GROUPING} column as
well, to confirm that things work in high and middling count rate cases, and in
cases where the data are not Poisson.

For starters, from an old CIAO website\footnote{\href{http://cxc.harvard.edu/sherpa3.4/faq/xspec_errs.html}{cxc.harvard.edu/sherpa3.4/faq/xspec\_errs.html}}:
\begin{quote}
If the data are grouped then XSPEC calculates the
$\code{GRP\_STAT\_ERR} = \sqrt{\sum \code{STAT\_ERR}^2}$ over the group.
This gives an error which is much larger than the correct one calculated from
the counts in the group.
\end{quote}

\section{Resources}

Most of this work, other than XSPEC finagling, is derived from continuous
re-reading of the XSPEC manual pages and various other web pages.

\begin{itemize}
    \item \href{http://heasarc.gsfc.nasa.gov/docs/xanadu/xspec/manual/XSsetplot.html}{XSPEC setplot manpage}
    \item \href{http://heasarc.gsfc.nasa.gov/docs/xanadu/xspec/manual/XSstatistic.html}{XSPEC statistic manpage}
    \item \href{http://heasarc.gsfc.nasa.gov/docs/xanadu/xspec/manual/XSweight.html}{XSPEC weight manpage}
    \item \href{http://heasarc.gsfc.nasa.gov/docs/xanadu/xspec/manual/XSappendixStatistics.html}{XSPEC fit statistics appendix}
    \item \href{http://heasarc.gsfc.nasa.gov/docs/xanadu/xspec/wstat.ps}{More
        detailed notes on wstat}, found from a \href{https://github.com/sherpa/sherpa/pull/94}{Sherpa thread}
    \item \href{http://hesperia.gsfc.nasa.gov/~schmahl/cash/cash_oddities.html}{Cash statistic simulations}
\end{itemize}

%\acknowledgments

\facility{XMM (EPIC)}
\software{XSPEC}

% ==========
% References
% ==========
\bibliographystyle{aasjournal}
\bibliography{refs-snr}

% ========
% Appendix
% ========
%\clearpage  % Use \clearpage over \newpage
%\appendix
%
%\setcounter{table}{0}
%\renewcommand{\thetable}{A\arabic{table}}
%\setcounter{figure}{0}
%\renewcommand{\thefigure}{A\arabic{figure}}
%
%\section{}


\end{document}
