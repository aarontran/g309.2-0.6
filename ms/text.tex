\documentclass[twocolumn,tighten,trackchanges]{aastex61}
%\documentclass[twocolumn,tighten,linenumbers,trackchanges]{aastex61}
%\documentclass[iop, tighten, apj, numberedappendix]{emulateapj} % Still more compact than aastex

% Customize figures/sizing for 2 column vs. manuscript printout
%%\usepackage{etoolbox}
%%\newtoggle{manuscript}
%%\toggletrue{manuscript}  % Set TRUE if using manuscript / 1-col layout
%%%\togglefalse{manuscript}  % Set FALSE if using 2-col layout

\shorttitle{G309.2$-$0.6 (\today)}  % <~ 44 char
\shortauthors{Alpha, Beta (\today)}  % Max three
%\slugcomment{Draft, \today}

% Packages and commands
\usepackage{amsmath}  % Included in aastex - but want to get eqref
\usepackage{booktabs}
\usepackage{hyperref}
%\usepackage[normalem]{ulem}  % Just to get strikeout text

% My "standard" TeX aliases
\newcommand*{\mt}{\mathrm}
\newcommand*{\unit}[1]{\;\mt{#1}}  % vemod.net/typesetting-units-in-latex
\newcommand*{\abt}{\mathord{\sim}} % tex.stackexchange.com/q/55701
\newcommand*{\ptl}{\partial}
\newcommand*{\del}{\nabla}
\newcommand*\mean[1]{\bar{#1}}
\renewcommand{\vec}[1]{\mathbf{#1}}  % Bold vectors
\newcommand*{\tsup}{\textsuperscript}

% Paper-specific commands
\newcommand*{\nH}{N_{\mathrm{H}}}
\newcommand*{\nHUnits}{\times 10^{22} \unit{cm^{-2}}}
\newcommand*{\TauUnits}{\unit{s\;cm^{-3}}}
\newcommand*{\AV}{A_{\mathrm{V}}}
%\newcommand*{\kB}{k_{\mathrm{B}}}
\newcommand*{\kB}{k}  % vacillating on how to format Boltzmann's constant
\newcommand*{\EM}{\mathrm{EM}}  % Emission measure
\defcitealias{rakowski2001}{RHS01}
\defcitealias{gaensler1998-g309}{GGM98}

\newcommand*{\Gsnr}{G309.2$-$0.6}

% How to format G309.2-0.6?  Three options:
% 1. G309.2-0.6 (regular hyphen/dash)
% 2. G309.2--0.6 (en dash)
% 3. G309.2$-$0.6 (minus in isolation)
% 4. $G309.2-0.6$ (minus as binary operator)
% After inspecting these, I believe the right way is to use the minus sign in
% isolation (which makes sense).  This appears to match the typesetting for
% Gaenser+ 1998 (MNRAS), Rakowski+ 2001 (ApJ), and Safi-Harb+ 2006 (ApJ).
% This makes sense semantically as well.

% DELUXETABLE ADJUSTMENTS
% -----------------------
%\setlength\doublerulesep{0.5pt}  % Default: 1.5pt in aastex61.cls
%\setlength{\tabcolsep}{0pt}

\begin{document}

\title{Spatially Resolved Ejecta in \Gsnr{}}

% Not quite in line with recommended aastex style
\author{
A. Alpha, B. Beta%\altaffilmark{1}
}

\affil{
%\tsup{1}
Smithsonian Astrophysical Observatory, 60 Garden Street MS-70, Cambridge, MA 02138, USA
}

%\received{receipt date}
%\revised{\today}
%\accepted{acceptance date}


\begin{abstract}
We image and fit spectra of ejecta-dominated remnant \Gsnr{} from archival
XMM-Newton data.
\end{abstract}

% (!) no longer used by AAS as of late February, 2016
\keywords{ISM: supernova remnants ---
    ISM: individual objects (SNR \Gsnr{}) ---
    X-rays: ISM
}

% =============================================================================
% Introduction
% =============================================================================
\section{Introduction} \label{sec:intro}

Can we use integrated or spatially resolved X-ray spectra to type
ejecta-dominated or ED-ST transitioning supernova remnants (SNRs)?
In some cases, yes:
\begin{itemize}
    \item coarse abundance ratios of O, Fe, Si/S compared to model SN yields \citep{hughes1995}  % 1995ApJ...444L..81H
    \item high resolution line spectroscopy of O lines in 1E0102.2-7219 \citep{flanagan2004}  % 2004ApJ...605..230F
    \item Ni/Fe and Mn/Fe ratios in 3C397 \citep{yamaguchi2015}
    \item Fe K line centroid energies \citep{yamaguchi2014-iron, patnaude2015}
\end{itemize}
individual remnant studies attempt to guess out the progenitor SN type
from ejecta abundances and other pieces of information.
%Alternatively, \citet{chevalier2005} attempts to type SNe from SNR CSM
%emission.
Optical studies of CSM knots are possible (see references in
\citet{katsuda2015} -- and MANY earlier papers stretching back to 1970s).

Badenes+ 2007: Table 4, ionization timescales of Si/S vs. Fe in young Ia remnants.
Appendix is also sort of helpful.
I'd cite this on stratification... and discussion of Fe absence.

Some studies (i.e. those not emphasizing line centroids, widths, ratios) assume
that abundances inferred from NEI plasma fits are a reasonable probe of ejecta
composition.  Much depends on:
\begin{itemize}
    \item Ejecta structure - which elements are stratified or well mixed, and
        what conditions (ambient density, progenitor type, explosion mechanism)
        control ejecta structure.
        See: Ashall+ 2016 arxiv:1608.05244, stratification in SN1986G.
    \item ambient interstellar or circumstellar medium (ISM, CSM).
        was was the extent and structure of progenitor mass loss, if any?
        (continuous wind, pulses of material, ?)
        nearby molecular clouds, other remnants.
        Interesting recent paper on remnant deformation near GC.
        Yalinewich+ 2016 (arXiv:1608.05904).
    \item Remnant's own dynamics - evolutionary state (ejecta-dominated, Sedov;
        transitioning).
\end{itemize}

One might consider forward folding models and comparing with observed spectra
across the soft X-ray band.
This is basically the approach of \citet{hughes1995}, \citet{badenes2003},
\citet{rakowski2006-g337}.
Some models show observable differences in line ratios, elemental abundances
and the like.  Others are less clear.

We present new spatially resolved X-ray images and CCD-resolution spectra of
\Gsnr{} based on archived XMM-Newton observations totalling $98 \unit{ks}$, with
about half usable after flare removal.

We fit the CCD spectra of morphologically distinct regions of the remnant to
trace variation in ejecta abundances, plasma temperature, and ionization
states.
\textbf{And, what else? TODO: fill in once paper is fleshed out.}

%TODO crude text
Individual object study to build information for more ambitious global studies
of remnant properties and interaction with the ISM -- not undertaken here.
Just one piece of a puzzle.

\subsection{Previous observations of \Gsnr{}}

\begin{figure*}[]
    \plotone{fig/fig_snr_xmm_most_invert.pdf}
    \figcaption{
        Left: $1.4 \unit{GHz}$ arcsinh scaled image of SNR \Gsnr{} from the
        Molonglo Observatory Synthesis Telescope (MOST) Supernova Remnant
        Catalogue \citep{whiteoak1996}.
        Lower-left ellipse shows $43\arcsec$ beam;
        sensitivity is $2 \unit{mJy/beam}$.
        Right: $0.8$--$3.3 \unit{keV}$ XMM-Newton mosaic image of SNR \Gsnr{}
        with logarithmic color scaling and MOST $1.4 \unit{GHz}$ contours
        (0.01, 0.03, 0.1, 0.3 Jy/beam) overlaid.
%        Image is exposure- and background-corrected, binned to 2.5\arcsec
%        pixels, and smoothed by a 2 pixel Gaussian; individual exposures were
%        scaled to MOS2 medium filter response.
        % TODO: colorbars...
    }
    \label{fig:snr}
\end{figure*}

% Radio remnant observations and morphology
\objectname{SNR \Gsnr{}} was
%(J2000 RA 13h46m30s, dec. $-62\arcdeg 54\arcmin 00\arcsec$)  % No need to give
% coordinates -- evident from figures, old publications, and online catalogs.
discovered in Molonglo $408 \unit{MHz}$ and Parkes $5000 \unit{MHz}$ surveys
\citep{day1969, clark1973, green1974, clark1975} and has since been observed
multiple times by southern radio telescopes \citep{caswell1981, kesteven1987,
whiteoak1996}.
% Day, Thomas, Goss (1969 Au. J. Phys. Astrophys. Suppl.)
% flagged \Gsnr{} as a complex source, but did not identify it as a remnant
% or explore further.  Molongo/Parkes surveys may have occurred at the same
% time.  And, yes, it is confused (at the edge) with RCW 80.
\citet{gaensler1998-g309} (hereafter, \citetalias{gaensler1998-g309}) performed a
comprehensive radio study with Australia Telescope Compact Array (ATCA)
$1.344 \unit{GHz}$ continuum and $1.420 \unit{GHz}$ \ion{H}{1} observations at angular
resolution $24\arcsec$.
The radio remnant is a circular shell with two bright lobes to the NE and SW,
hereafter referred to as ``ears'' (Fig \ref{fig:snr}).
% Flux density -- omit, could incorporate later if relevant
% the flux density at $843 \unit{MHz}$ is $6 \unit{Jy}$ \citep{whiteoak1996}.
% Spectral index notes -- omit for now, could incorporate later if relevant
% Comment: spectral indices ~ 0-0.3 are more typical of pulsar powered remnants
% steeper (0.3,0.4-1) more typical for most SNRs.
%The radio spectral index is $0.53 \pm 0.09$ derived by \citet{gaensler1998-g309},
%excluding single-dish Parkes observations at 2700 and 5000 Mhz to avoid
%confusion of source flux with RCW80; a shallower index $0.36 \pm 0.11$ is
%obtained if single dish measurements (with authors' attempted corrections for
%source confusion) are included.
\ion{H}{1} absorption favors remnant distance between $5$--$14 \unit{kpc}$ based on
absorption to the galactic rotation curve's tangent point,
$v_{\mt{LSR}} \sim -50 \unit{km/s}$, and the absence of absorption above
$v_{\mt{LSR}} \sim +40 \unit{km/s}$ \citepalias{gaensler1998-g309}.
% The galactic rotation curve is from \citet{fich1989} with galactic center
% distance $R_0 = 8.5 \unit{kpc}$ and local circular velocity
% $\Theta = 220 \unit{km/s}$.

% ATCA data -- cannot access at http://www.atnf.csiro.au/research/HI/sgps/queryForm.html
% SGPS survey -- taken 1998-2000.
% Emailed a feedback query on 2016 April 13... probably won't have much luck.

% X-ray remnant observation(s) and morphology
X-ray emission was discovered in an Advanced Satellite for Cosmology and
Astrophysics (ASCA) survey of small remnants by \citet{rakowski2001}
(hereafter, \citetalias{rakowski2001}).
The X-ray remnant sits within the radio shell and is brightest towards the
north (Figure \ref{fig:snr}).
A faint X-ray arc on the remnant's northeast limb coincides with the radio
shell's limb and is most apparent in our $0.8$--$1.4 \unit{keV}$ band image.

% Other wavelengths
The field of \Gsnr{} has been surveyed in many wavelengths.
No obvious H$\alpha$, [\ion{S}{2}], or [\ion{O}{3}] emission is seen in three
400 second exposures taken with the $0.6 \unit{m}$ Curtis-Schmidt telescope at
Cerro Tololo Inter-American Observatory in 2001 January
(PI: P. F. Winkler; observers: Gokas, Smith, Winkler).\footnote{\url{http://sites.middlebury.edu/snratlas/g309-2-0-6/}}
% I think it's this proposal (but not sure)
% NOAO proposal 2001A-0331
% http://adsabs.harvard.edu/abs/2001noao.prop..331W
% http://www.noao.edu/dir/q_rep/Q2_Apr_30/FY2001%20Q2%20Jan-Mar%20NOAO%20Quarterly%20Report.pdf
% Some H-alpha emission associated with foreground cluster;
% field is dominated by RCW 80 to northeast.
No $1720.5 \unit{MHz}$ OH maser emission is observed above $40 \unit{mJy}$
\citep{green1997}.
% Mopra CO survey - no public data yet, in works...
No obvious infrared emission appears in Spitzer
$3.6$--$8 \unit{{\mu}m}$ IRAC GLIMPSE or $24 \unit{{\mu}m}$ MIPSGAL mosaics
\citep{churchwell2009, carey2009}.
Dust emission at $100 \unit{{\mu}m}$ (IRAS/ISSA, COBE/DIRBE) shows a
gradient across the remnant, although association with G309 is unknown.  %TODO clarify this
% Interpretation:
% J,H,Ks bands = 1.2, 1.6, 2.2 micron (2MASS) -- cool stars
% 3.6--8 micron (Spitzer) -- cool stars
% 12 micron --
% 22/24 micron (WISE, Spitzer) -- dust
% 70 micron -- more dust (useful for dust SED stuff)
% Melange of potentially useful stuff
The remnant is not detected in GeV by Fermi-LAT \citep{acero2016} or TeV by
HESS \citep{bochow2011}.
The absence of H$\alpha$ and infrared emission is consistent with previous
data reviewed by \citetalias{gaensler1998-g309}.

% HD 119682 - previous interest and subsequent disassociation
The bright X-ray source (1WGA J1346.5$-$6255) within \Gsnr{} is identified with
HD 119682, a member of the X-ray emitting class of $\gamma$ Cas stars.  % TODO WORDING HERE
\citetalias{gaensler1998-g309} and \citetalias{rakowski2001} suggested that
\Gsnr{} could be a jet- or outflow-driven remnant like the SS 443 / W50 system.
% bilobed beats bilobate on Google by 278 to 66 (cf. Google ngrams)
But, HD 119682 is likely a member of the open cluster NGC 5281 at a distance
$\abt 1.4 \unit{kpc}$, based on X-ray and optical astrometry, cluster proper
motions, and X-ray spectrum absorption \citep{rakowski2006-star, safi-harb2007,
torrejon2013}.
% The X-ray fitted absorption of $\nH \sim 0.2 \times 10^{22} \unit{cm^{-2}}$
% is consistent with the inferred distance to HD 119682
% \citep{rakowski2006-star, safi-harb2007, torrejon2013}.

% Gum 48d / RCW 80
North of \Gsnr{}, the \ion{H}{2} region Gum 48d (RCW 80) is well traced in
\ion{H}{1}, polycyclic aromatic hydrocarbon (PAH), and warm dust emission
\citep{karr2009}.
% TODO TODO
\textbf{TODO this H II region is not shown in our image - may not need to
discuss}
The distance to Gum 48d is $\abt 3.5\unit{kpc}$ based on ionized gas emission
velocities measured from the central star system HR 5171 \citep{karr2009}.
Given this distance, and the absence of evidence for interaction with G309,
we assume that Gum 48d also lies in the foreground of \Gsnr{}.
We also ignore the radio wisp above the remnant in Figure~\ref{fig:snr},
which \citet{whiteoak1996} claim to be thermal

% Here's a lovely optical photograph, clearly showing the bright H II region and
% the bluer star cluster NGC 5281 to the south:
% https://it.wikipedia.org/wiki/File:RCW_80.jpg

% Mopra CO J(1-0) survey, resolution 30 arcsec and 0.1 km/s, would be huge!
% Looking at our crude rotation curve -- beyond tangent point, 100 km/s per 10kpc
% is the approximate slope; 0.1 km/s means we have a distance resolution of order
% 10 parsecs, comparable to or smaller than this remnant size.
% Compare, Dame survey has beamwidth ~ 1/8 deg. = 7.5 arcmin = 450 arcsec


% TODO: check out ATCA J134649--625235, mentioned in Gaensler

% Summary / wrap-up
Here, we extend the radio and X-ray studies of \citetalias{gaensler1998-g309} and
\citetalias{rakowski2001} with archived XMM-Newton observations that resolve
the soft X-ray morphology of \Gsnr{}.
%We confirm the previous


%\begin{table*}
%    \centering
%    \caption{XMM Observations of \Gsnr{}\label{tab:obs2}}
%    \begin{tabular}{@{}lrrrlrlr@{}}
    \toprule
    Obs. ID & Dur. & $t_{\mt{\,live},\,\mt{MOS}}$ & $t_{\mt{\,live},\,\mt{PN}}$
        & Date & Rev. & Filter & PI \\
    \midrule
    0087940201 & $40.5$ & $25.3$ & $18.0$ & 2001 August 28 & 315 & Thick & Hughes \\
    0551000201 & $57.3$ & $21.9$ &  $9.0$ & 2009 March 6--7 & 1692 & Medium & Motch \\
    \bottomrule
\end{tabular}

%    \tablecomments{Durations in kiloseconds (ks).
%    Good duration $t_{\mt{live}}$ is central CCD live time after flare
%    filtering; MOS value averages MOS1 and MOS2 times ($\sim 0.6 \unit{ks}$
%    difference in both obsids).
%    Rev. is XMM-Newton orbit (revolution) number.}
%\end{table*}

%\begin{deluxetable*}{@{}lrrrlrlr@{}}
%    \tablecaption{XMM Observations of \Gsnr{}\label{tab:obs}}
%    \tablehead{
%        \colhead{Obs.~ID} & \colhead{Dur.} &
%        \colhead{$t_{\mt{\,live},\,\mt{MOS}}$} &
%        \colhead{$t_{\mt{\,live},\,\mt{PN}}$} &
%        \colhead{Date} & \colhead{Rev.} & \colhead{Filter} & \colhead{PI}
%    }
%    \startdata
%        0087940201 & $40.5$ & $25.3$ & $18.1$ & 2001 August 28 & 315 & Thick & Hughes \\
%        0551000201 & $57.3$ & $22.1$ &  $9.0$ & 2009 March 6--7 & 1692 & Medium & Motch \\
%    \enddata
%    \tablecomments{Durations in kiloseconds (ks).
%    Good duration $t_{\mt{live}}$ is central CCD live time after flare
%    filtering; MOS value averages MOS1 and MOS2 times ($\abt0.6 \unit{ks}$
%    difference in both obsids).
%    Rev.~is XMM-Newton orbit (revolution) number.}
%    % TODO: only when I have nothing better to do, play with new deluxe tables
%    % functionality for column alignment.
%\end{deluxetable*}

% =============================================================================
% Observations + Reduction
% =============================================================================
\section{Observations and Data Reduction} \label{sec:obs}

% Accompanies spectrum extraction AND fitting text
\begin{figure*}[!ht]
    \epsscale{1.15}
    \plotone{fig/fig_snr_fullfov_invert.pdf}
    \figcaption{
        Left: $0.8$--$3.3 \unit{keV}$ image of SNR \Gsnr{} with detected point
        sources (green) and $400\arcsec$ radius source (magenta) and
        $510\arcsec$--$700\arcsec$ radius background (cyan) extraction regions.
        Right: same image, with point sources replaced by interpolated counts
        as described in text.
        Image is exposure- and background-corrected, binned to 2.5\arcsec
        pixels, and smoothed by a 2 pixel Gaussian; individual exposures were
        scaled to a MOS2 medium filter response.  Same procedure was applied
        for Figure~\ref{fig:snr}.
    }
    \label{fig:fov}
\end{figure*}

% TODO I have not thought at all about the effects of optical loading on these
% observations...

% TODO how to refer to each observation in a simple way?
% for now just using ``Hughes'' and ``Motch''
% Possibly, just refer to by year: 2001 and 2009.  Obsid is not that useful
% TBD: kind of inconsistent in text currently...

% Current good time values (2016 April 21. Somewhat old extraction, pipeline
% last re-run in Feb or Mar after SAS v15 release)
%   0087940201 mos1S001 livetime 25.016 ks, ontime 25.290 ks
%   0087940201 mos2S002 livetime 25.621 ks, ontime 25.891 ks
%   0087940201 pnS003   livetime 18.011 ks, ontime 20.979 ks
%   0551000201 mos1S001 livetime 21.621 ks, ontime 21.989 ks
%   0551000201 mos2S002 livetime 22.167 ks, ontime 22.545 ks
%   0551000201 pnS003   livetime  8.981 ks, ontime  9.720 ks
%
% Newer numbers (2017 Jan) from mos1S001-clean.fits... not sure why changed.
%   0087940201 mos1S001 livetime 25.191 ks, ontime 25.470 ks
%   0087940201 mos2S002 livetime 25.500 ks, ontime 25.771 ks
%   0087940201 pnS003   livetime 18.115 ks, ontime 20.099 ks
%   0551000201 mos1S001 livetime 21.792 ks, ontime 22.169 ks
%   0551000201 mos2S002 livetime 22.411 ks, ontime 22.784 ks
%   0551000201 pnS003   livetime  8.981 ks, ontime  9.720 ks


% Event lists, soft proton light curve filtering
% --------------------
% Terse, table-oriented version.
XMM-Newton has observed \Gsnr{} for $97.7 \unit{ks}$ in two pointings.
%(Table~\ref{tab:obs}).
% --------------------
% More verbose version
XMM-Newton has observed \Gsnr{} for $97.73 \unit{ks}$ in two pointings.
Obsid 0087940201 (PI J. P. Hughes) was $40.46 \unit{ks}$ on 2001 August 28
with MOS1/2 in Full Frame mode, PN in Extended Full Frame mode, and XMM's
``thick'' optical filter.
Obsid 0551000201 (PI C. Motch) was $57.27 \unit{ks}$ on 2009 March 6--7
with MOS1/2 in Full Frame mode, PN in Large Window mode, and XMM's ``medium''
optical filter.
%The Motch pointing targeted the foreground Be star HD 119682, but also
%captured \Gsnr{} on the MOS1/2 and PN detectors.
% --------------------
We reduce the data with XMM's Science Analysis System (SAS) v15.0.0 and
Extended SAS (ESAS) v5.9 \citep{snowden2008, kuntz2008}.
\footnote{\url{http://heasarc.gsfc.nasa.gov/docs/xmm/esas/cookbook/xmm-esas.html}}
Strong soft proton flares are filtered by ESAS tasks \texttt{mos-filter} and
\texttt{pn-filter}, which fit a Gaussian to a histogram of time-binned count
rates for each exposure and discard time intervals with count rate $>1.5\sigma$
above the fitted Gaussian mean.  % TODO - sentence needs retooling
We further inspect the $2.5$--$12 \unit{keV}$ field-of-view light curves and
manually cut some brief good-time intervals surrounded by higher count rate
flares.
After filtering, Obs.~ID 0087940201 has $25 \unit{ks}$ and $18 \unit{ks}$ of
usable MOS1/2 and PN exposure respectively.
Obs.~ID 0551000201 has $18 \unit{ks}$ and $9 \unit{ks}$ for MOS1/2 and PN
respectively.
%After flare filtering, our good exposure times are $47 \unit{ks}$
%for MOS1/2 and $27 \unit{ks}$ for PN.
About half to two-thirds of each observation is lost to flares, with more
severe loss for PN exposures.
%Table~\ref{tab:obs} gives the resulting usable times.

% CCD exclusions
In obs.~ID 0551000201, we excluded two MOS CCDs and the entire PN exposure from
our analysis.
MOS1 CCD6 was disabled by a presumed micrometeorite in 2005, and MOS2 CCD5
shows anomalously high soft X-ray background noise \citep[cf.][]{kuntz2008}.
PN in Large Window mode does not collect corner counts, so we cannot
estimate the detector background consistently using ESAS tasks.
Omitting the $9 \unit{ks}$ obs.~ID 0551000201 PN exposure only sacrifices
$\abt 10$--$20\%$ of our available photons.

% Point sources
Point sources with flux $> 10^{-14} \unit{erg\;cm^{-2}\;s^{-1}}$ were
identified and removed with the ESAS task \texttt{cheese}.
The point source detection was run separately on each observation (set of three
exposures), and the two lists were merged to produce our final mask
(Figure~\ref{fig:fov}).
To assist image analysis, we interpolated point source holes in images by
drawing new pixel values from a Poisson distribution with count rate from a
10\arcsec annulus surrounding the source hole.
For spectrum extraction, no such interpolation is applied.
% To re-state question: what is the minimum flux for a point source to be
% detected 99\% of the time in mock data drawn from our model?
% It depends on detector position (effective area) and many other factors.
% Let's do a quick sanity check:
%
% Effective area of PN + MOS1 + MOS2: ~1800 cm^2 over 2-5 keV,
% ~2500 cm^2 over 1-2 keV.
% Suppose a soft source with flux 1e-14 erg/cm^2/s is dominated by ~1 keV
% emission.
% Flux 1e-14 erg/cm^2/s = 6.2e-6 keV/cm^2/s = 6.2e-6 photons/cm^2/s (for 1 keV
% photons only).
% With effective area 2500 cm^2, this nets 0.0156 photons/s
% Finally, with usable exposure 20 ks for combined MOS/PN,
% this yields 312 counts, which should be plenty for a clean detection.
% If I have erred by a factor 10x, 30 counts may still yield a marginal
% detection.  So on-axis point sources of flux 1e-14 can very reasonably be
% detected in our data.  This gets iffy off axis but that's OK.

\subsection{Spectrum Extraction and Modeling} \label{sec:spec-extract-fit}

% Bumping way up to force placement...
\begin{figure*}[!th]
    %\epsscale{1.15}
    \plotone{fig/fig_lineflux_eqwidth.pdf}
    \figcaption{
        Top: line flux images of Mg, Si, and S K$\alpha$ line and continuum.
        All image colormaps are arcsinh scaled with identical scale limits.
        Bottom: equivalent width (EW) of Mg, Si, and S K$\alpha$ lines.
        NaN values (white) away from image centers are caused by noise from
        image subtraction.
        Note that Mg EW range is much smaller than Si and S EW.
        Image spatial scale and centering identical to Figure~\ref{fig:snr}.
%        The lack of spatial correlation between Mg and Si/S EW, the relative
%        uniformity of Mg EW emission, and the limb-brightening of Mg line flux
%        (and associated continuum, per Mg EW image) suggests that Mg K$\alpha$
%        and the soft 1.3--1.4 keV band traces from the shock-heated ambient
%        medium, whereas Si and S K$\alpha$ trace remnant ejecta.
    }
    \label{fig:eqwidth}
\end{figure*}

% Spectrum extraction + grouping, and QPB creation
Observation spectra are created from ``good'' single and double events (i.e.,
\texttt{PATTERN} $\leq 12,4$ for MOS, PN respectively; \texttt{FLAG} $= 0$ for
both MOS and PN) using ESAS tasks \texttt{\{mos,pn\}-spectra}.
We merged MOS1 and MOS2 spectra from each observation and binned merged-MOS and
PN spectra to $\geq 1$ count per energy channel bin.
A detector background spectrum is created with ESAS tasks
\texttt{\{mos,pn\}\_back}, based upon XMM-Newton's filter wheel closed (FWC)
observations and unexposed CCD corner counts from a database of public
observations \citep[Sec.~3.4]{kuntz2008}.
%by (1) computing a quiescent spectrum from unexposed CCD corner counts,
%(2) augmenting the quiescent spectrum with corner counts from public
%observations with similar count rates and spectral hardness, and (3) scaling
%the augmented quiescent spectrum shape to that expected for the source region
%of interest, using the ratio of quiescent spectra across corner / source
%regions from XMM's filter wheel closed (FWC) observations and assuming that the
%ratio of spectra across a given MOS or PN chip is time invariant
%\citep[Sec. 3.4]{kuntz2008}.
The detector background is directly subtracted from observation spectra.

% Spectrum models and fitting
We fit spectra in XSPEC 12.9.0 \citep{arnaud1996} using ISM element abundances
and gas absorption model \texttt{tbnew} v2.3.2
\footnote{\url{http://pulsar.sternwarte.uni-erlangen.de/wilms/research/tbabs/}},
from \citet{wilms2000}.
All spectra were fit with XSPEC's \texttt{pgstat} statistic for
Poisson-distributed data with a Gaussian background (appropriate for the ESAS
detector background model); \texttt{pgstat}, like the Cash statistic
\texttt{cstat}, averts bias associated with $\chi^2$ fitting of binned Poisson
data \citep{humphrey2009}.
Unless otherwise noted, models were fitted to the full soft X-ray spectrum --
$0.3$--$11 \unit{keV}$ for MOS and $0.4$--$11 \unit{keV}$ for PN -- to
constrain both cosmic and instrumental backgrounds.

% Instrumental lines
We model instrumental fluorescence lines as zero-width fixed-energy Gaussians
and assume that relative line strengths (i.e., line ratios) in a given CCD
region are the same between FWC and observation data.
We fit FWC spectra using Gaussian lines on a broken power-law continuum.
For MOS, we model two lines: $1.49$ (Al) and $1.75 \unit{keV}$ (Si).
For PN, we model seven lines: $1.49$ (Al), $4.54$ (Ti), $5.44$ (Cr),
$7.49$ (Ni), $8.05$ (Cu), $8.62$ (Zn), and $8.90 \unit{keV}$ (Cu K$\beta$).
% The PN Ti and Cr lines are not noticeable in observation spectra, but are
% clearly visible in FWC spectra.
Then, to fit observation spectra, we fix all instrumental line ratios and vary
an overall normalization.
% Some spectrum fits could be improved by allowing instrumental line widths and
% centroid energies to vary.
% But, $\Delta \chi$ residuals at instrumental lines are comparable to
% residuals throughout our spectrum, so we do not expect that imperfect line
% model residuals should significantly affect our fit results and
% interpretation.
%
% Wishy washy because we have not quantified or tested this impact...
% But we have no reason to expect major impact.
% Freeing line widths and energies, even within a small range, would add 4--12
% parameters for each exposure.

% Residual soft proton contamination, SWCX
Soft protons on EPIC detectors dominate the X-ray background and are not
removed by flare filtering due to slow time variation.
We model residual soft protons by a power law that bypasses CCD response and
effective area functions following ESAS prescription.
We neglect background solar wind charge exchange (SWCX) emission
\citep{snowden2004, carter2011};
neither observation shows obvious \ion{O}{7} or \ion{O}{8} lines, and any
residual contamination should be folded into our cosmic X-ray background model.
% Omitted -- this text on SWCX checks is not really needed
% Hughes' 2001 observation was not flagged by \citet{carter2011} as containing
% time-variable soft X-ray emission indicative of exospheric SWCX.
% % TODO what is the difference between these types of swcx?
% Modeled magnetospheric SWCX emission decreases by a factor of $10$ over the
% duration of Hughes' 2001 observation, based on XMM-Newton Guest Observer
% Facility (GOF) tool that combines a magnetosphere model \citep{spreiter1966} with
% Advanced Composition Explorer (ACE) solar wind data.
% \footnote{\url{https://heasarc.gsfc.nasa.gov/docs/xmm/scripts/xmm_trend.html}}
% We extracted 0087940201 MOS1 spectra at early and late times in the observation
% and, by eye, saw no obvious differences in soft X-ray emission.
% % TODO I will be re-doing this.
% For the 2009 observation, the XMM GOF magnetosphere model predicts weaker SWCX
% emission compared to the 2001 observation, so we simply neglect SWCX in both
% observations.
% Any unmodeled SWCX should be folded into our X-ray background model, albeit
% imperfectly.
% % NOTE: see my notes from 2016 Jan 13 on spectrum cuts to evaluate SWCX
% % 0551000201 was observed March 2009
% %   Carter/Sembay analysis performed August 2009, so 0551000201 was likely
% %   still in its proprietary period
% % 0087940201 is not listed in Carter/Sembay, but meets basic selection criteria
% % (namely, MOS1/MOS2 operating in full frame mode


% =============================================================================
% Equivalent width images
% =============================================================================

\section{Line flux and equivalent width}

We created line flux and equivalent width (EW) images to characterize bright
Mg, Si, and S K$\alpha$ line emission in the manner of \citet{vink1999} and
\citet{hwang2000}.
For each line-energy band (1.3--1.4 keV, 1.8--1.9 keV, 2.4--2.5 keV), we
selected nearby side-bands from inspection of the integrated spectrum
(Section~\ref{sec:integrated-fit}) to represent continuum emission
(Mg: 1.15--1.25 keV and 1.60--1.65 keV; Si, 1.60--1.65 keV and 1.98--2.05 keV;
S: 1.98--2.05 keV to 2.60--2.70 keV).
We logarithmically averaged the side-band images to estimate each line's
underlying continuum flux, then subtracted continuum to obtain line flux images
and took the ratio of line and continuum to obtain equivalent width images
(Figure~\ref{fig:eqwidth}).

The lack of spatial correlation between Mg and Si/S EW, the relative
uniformity of Mg EW emission, and the limb-brightening of Mg line flux
(and associated continuum, per Mg EW image) suggests that Mg K$\alpha$
and the soft 1.3--1.4 keV band traces emission from the shock-heated ambient
medium, whereas Si and S K$\alpha$ trace remnant ejecta.

We select five regions named ``Core'', ``Lobe SW'', ``Lobe NE'', ``Ridge SE'',
and ``Ridge NW'' for spatially-resolved spectrum fits (Figure~\ref{fig:rgb}).
The emission in Figures~\ref{fig:eqwidth} and \ref{fig:rgb} are heavily binned
and smoothed (excepting the top row of Fig.~\ref{fig:eqwidth}, all images have
$40\arcsec$ pixels smoothed with a 2 pixel Gaussian).
Equivalent width images at $20\arcsec$ resolution better separate the emission
components shown in Figure~\ref{fig:rgb}, and our region boundaries account for
the sharper separation, but the images are extremely noisy and thus less
suitable for display.

\begin{figure}[]
    \epsscale{0.9}
    \plotone{fig/fig_rgb_soft_eqwidth.pdf}
    \figcaption{RGB image distinguishing ejecta and ambient medium emission.
        Red: 1.3--1.4 keV line flux, green: Si K$\alpha$ EW, blue: S K$\alpha$
        EW (Figure~\ref{fig:eqwidth}).
        Image spatial scale and centering identical to Figure~\ref{fig:snr}.
    }
    \label{fig:rgb}
\end{figure}


% =============================================================================
% Spectrum fits
% =============================================================================

\section{Integrated remnant spectrum} \label{sec:integrated-fit}

\begin{figure*}[!ht]
    \plotone{fig/fig_src_bkg_allexps-src-only.pdf}
    %\plotone{fig/fig_src_bkg_0087940201-mos.pdf}
    %\plotone{fig/fig_src_bkg_0087940201-mos-delchi.pdf}
    \figcaption{Integrated source spectra from 0087940201 MOS with best-fit
    model comprising: cosmic X-ray background (CXRB), soft proton background (SP),
    instrumental lines, and NEI plasma for the remnant.
    All plotted spectra are binned towards minimum significance $5\sigma$ with
    maximum 50 channels per bin, but spectra in model fits are only minimally
    binned to $\geq 1 \unit{count}$ (Section~\ref{sec:spec-extract-fit}).
    Note factor of $\abt 2$ difference in soft proton noise between 0087940201
    and 0551000201 MOS exposures.
    }
    \label{fig:src-bkg-fits}
\end{figure*}

% Describe integrated source fit; jump straight to fit and basic numbers
We jointly fit integrated remnant and background spectra
(Figure~\ref{fig:src-bkg-fits}) to obtain a coarse characterization and to
compare against prior ASCA and XMM-Newton spectral analyses; the extraction
regions are 0--400\arcsec and 510--700\arcsec respectively, plotted in
Figure~\ref{fig:fov}.

The remnant is well fitted by an absorbed single-temperature non-equilibrium
ionization (NEI) plasma (XSPEC model \texttt{tbabs\_new * vnei}) with
increased Si and S abundances relative to galactic ISM
(Table~\ref{tab:spectrum-fits}).
%We also consider variation in O, Ne, Mg, Ar, Ca, and Fe abundances.
Increased Mg abundance (constrained by \ion{Mg}{11} He-$\alpha$) formally
improves the fit, but is not clearly required.
%Small bumps at $3.1 \unit{keV}$ and $3.9 \unit{keV}$ suggest \ion{Ar}{17} and
%\ion{Ca}{19} He-$\alpha$; if attributable to line emission, these features
%would require highly super-solar abundances at our best fit temperature and
%ionization timescale.
NEI plasma models accounting for plane shock or Sedov geometries
\citep{borkowski2001} do not yield further insight in our integrated remnant
fit.
An XSPEC \texttt{vpshock} model does not fit significantly better than
the single-ionization-age NEI model ($\mathrm{pgstat}_{\mathrm{vpshock}} = ??$),
and \texttt{vpshock} bounds on ionization age simply bracket the
\texttt{vnei}-inferred mean ionization (Table~\ref{tab:spectrum-fits}).
\textbf{TODO: need to redo vpshock fits}  % TODO
At high electron temperatures $\abt 2$--$3 \unit{keV}$, the
\texttt{vpshock} model should reasonably approximate the more involved
\texttt{vsedov} model of \citet{borkowski2001}, so we do not consider a
\texttt{vsedov} fit with non-uniform electron temperature distribution.
The simpler \texttt{NEI} plasma model suffices for our spectral modeling.

The integrated remnant fit statistic is deceptively good due to the large
number of background counts, both from the sky background spectrum and from the
extended energy range fitted to sample cosmic and instrumental background.
Spatially resolved fits (Section~\ref{sec:subregion-fit}) show more typical fit
tension and better reveal distinct plasma components.

%\begin{table*}[!ht]
%    \centering
%    \caption{Integrated source and background fits, single-temperature NEI
%        model with non-solar abundances \label{tab:src-fits}}
%    \footnotesize
%    % 20161015_src_bkg_grp01_pgstat_mosmerge.json
% 20161015_src_bkg_mg.json
% 20161015_src_bkg_mg-ar-ca_MANUAL.json
% 20161028_src_bkg_mg-si-s-fe_MANUAL.json
% 20161028_src_bkg_o-ne-mg-si-s.json
% 20161026_src_bkg_o-ne-mg-si-s-ar-ca-fe_MANUAL.json
% 20161028_src_bkg_o-ne-mg-si-s-fe_nH-0.7_NOERR.json
% 20161028_src_bkg_mg-si-s_tau-5e13_MANUAL.json
\begin{tabular}{@{}lllllllll@{}}
\toprule
 & A & B & C & D & E & F & G & H \\
\midrule
\multicolumn{9}{c}{Remnant NEI Model} \\
\midrule
$\nH$ & ${2.47}^{+0.15}_{-0.13}$ & ${2.33}^{+0.09}_{-0.17}$ & ${2.48}^{+0.20}_{-0.19}$ & ${2.02}^{+0.13}_{-0.10}$ & ${2.25}^{+0.13}_{-0.12}$ & ${2.10}^{-2.10}_{-2.10}$ & $\textbf{0.70}$ & ${3.12}^{+0.11}_{-0.09}$ \\ [0.5 em]
$\kB T$ & ${1.45}^{+0.21}_{-0.24}$ & ${1.73}^{+0.30}_{-0.14}$ & ${1.25}^{+0.33}_{-0.09}$ & ${1.53}^{+0.22}_{-0.19}$ & ${1.64}^{+0.28}_{-0.13}$ & ${1.08}^{-1.08}_{-1.08}$ & ${2.75}^{-2.75}_{-2.75}$ & ${0.51}^{+0.00}_{-0.02}$ \\ [0.5 em]
$\tau$ & ${2.46}^{+0.47}_{-0.41}$ & ${2.23}^{+0.19}_{-0.34}$ & ${2.77}^{+0.84}_{-0.43}$ & ${2.14}^{+0.25}_{-0.19}$ & ${2.21}^{+0.31}_{-0.24}$ & ${2.66}^{-2.66}_{-2.66}$ & ${1.57}^{-1.57}_{-1.57}$ & $\textbf{5000}$ \\ [0.5 em]
EM & ${6.14}^{+2.03}_{-1.07}$ & ${4.41}^{+0.75}_{-1.37}$ & ${6.29}^{+2.59}_{-1.87}$ & ${5.62}^{+1.20}_{-1.07}$ & ${3.16}^{+0.70}_{-0.56}$ & ${7.50}^{-7.50}_{-7.50}$ & ${2.89}^{-2.89}_{-2.89}$ & ${38.62}^{+5.73}_{-3.23}$ \\ [0.5 em]
O &      &      &      &      & ${12.77}^{+8.33}_{-4.83}$ & ${3.94}^{-3.94}_{-3.94}$ & ${0.05}^{-0.05}_{-0.05}$ &      \\ [0.5 em]
Ne &      &      &      &      & ${1.66}^{+1.18}_{-0.69}$ & ${0.95}^{-0.95}_{-0.95}$ & ${0.00}^{-0.00}_{-0.00}$ &      \\ [0.5 em]
Mg &      & ${1.27}^{+0.11}_{-0.10}$ & ${1.20}^{+0.13}_{-0.14}$ & ${0.74}^{+0.12}_{-0.04}$ & ${1.73}^{+0.57}_{-0.36}$ & ${0.69}^{-0.69}_{-0.69}$ & ${0.33}^{-0.33}_{-0.33}$ & ${1.05}^{+0.10}_{-0.09}$ \\ [0.5 em]
Si & ${4.01}^{+0.23}_{-0.20}$ & ${4.61}^{+0.53}_{-0.29}$ & ${4.61}^{+0.43}_{-0.49}$ & ${3.55}^{+0.23}_{-0.23}$ & ${6.53}^{+2.00}_{-1.15}$ & ${3.88}^{-3.88}_{-3.88}$ & ${3.02}^{-3.02}_{-3.02}$ & ${4.19}^{+0.27}_{-0.24}$ \\ [0.5 em]
S & ${3.63}^{+0.35}_{-0.15}$ & ${4.14}^{+0.59}_{-0.35}$ & ${4.31}^{+0.44}_{-0.47}$ & ${3.83}^{+0.37}_{-0.36}$ & ${6.15}^{+1.91}_{-1.27}$ & ${4.74}^{-4.74}_{-4.74}$ & ${4.35}^{-4.35}_{-4.35}$ & ${4.79}^{+0.30}_{-0.36}$ \\ [0.5 em]
Ar &      &      & ${6.60}^{+2.05}_{-1.67}$ &      &      & ${9.99}^{-9.99}_{-9.99}$ &      &      \\ [0.5 em]
Ca &      &      & ${30.31}^{+16.03}_{-10.57}$ &      &      & ${57.37}^{-57.37}_{-57.37}$ &      &      \\ [0.5 em]
Fe &      &      &      & $< 0.15$ &      & ${0.00}^{-0.00}_{-0.00}$ & ${0.00}^{-0.00}_{-0.00}$ &      \\
\midrule
\multicolumn{9}{c}{X-ray Background} \\
\midrule
$\kB T_{\mt{local}}$ & ${0.26}^{+0.02}_{-0.01}$ & ${0.26}^{+0.02}_{-0.01}$ & ${0.27}^{+0.02}_{-0.01}$ & ${0.26}^{+0.01}_{-0.01}$ & ${0.26}^{+0.01}_{-0.01}$ & ${0.25}^{-0.25}_{-0.25}$ & ${0.25}^{-0.25}_{-0.25}$ & ${0.26}^{+0.01}_{-0.01}$ \\ [0.5 em]
$\mt{EM}_{\mt{local}}$ & ${0.24}^{+0.02}_{-0.01}$ & ${0.24}^{+0.02}_{-0.02}$ & ${0.23}^{+0.02}_{-0.02}$ & ${0.23}^{+0.02}_{-0.02}$ & ${0.22}^{+0.01}_{-0.02}$ & ${0.23}^{-0.23}_{-0.23}$ & ${0.20}^{-0.20}_{-0.20}$ & ${0.18}^{+0.02}_{-0.01}$ \\ [0.5 em]
$N_{\mathrm{H,xgal}}$ & ${1.24}^{+0.73}_{-0.58}$ & ${1.12}^{+0.73}_{-0.45}$ & ${1.09}^{+0.75}_{-0.61}$ & ${1.14}^{+0.79}_{-0.68}$ & ${0.10}^{-0.10}_{-0.10}$ & ${1.17}^{-1.17}_{-1.17}$ & ${1.22}^{-1.22}_{-1.22}$ & ${1.10}^{+1.04}_{-0.54}$ \\ [0.5 em]
$N_{\mathrm{H,ridge}}$ & ${1.38}^{+0.20}_{-0.09}$ & ${1.39}^{+0.19}_{-0.09}$ & ${1.36}^{+0.18}_{-0.09}$ & ${1.41}^{+0.09}_{-0.09}$ & ${1.55}^{+0.10}_{-0.10}$ & ${1.46}^{-1.46}_{-1.46}$ & ${1.40}^{-1.40}_{-1.40}$ & ${1.16}^{+0.06}_{-0.09}$ \\ [0.5 em]
$\kB T_{\mt{ridge}}$ & ${0.75}^{+0.04}_{-0.04}$ & ${0.74}^{+0.12}_{-0.04}$ & ${0.75}^{+0.04}_{-0.03}$ & ${0.70}^{+0.04}_{-0.05}$ & ${0.86}^{+0.03}_{-0.06}$ & ${0.76}^{-0.76}_{-0.76}$ & ${0.68}^{-0.68}_{-0.68}$ & ${0.74}^{+0.03}_{-0.03}$ \\ [0.5 em]
$\mt{EM}_{\mt{ridge}}$ & ${1.92}^{+0.31}_{-0.30}$ & ${1.94}^{+0.32}_{-0.30}$ & ${1.87}^{+0.29}_{-0.35}$ & ${2.12}^{+0.37}_{-0.33}$ & ${1.69}^{+0.27}_{-0.21}$ & ${2.00}^{-2.00}_{-2.00}$ & ${2.08}^{-2.08}_{-2.08}$ & ${1.47}^{+0.26}_{-0.25}$ \\
\midrule
pgstat & 14218.8 & 14201.0 & 14141.1 & 14146.9 & 14125.6 & 14005.1 & 14147.9 & 14486.6 \\
$\chi^2$ & 13119.6 & 13098.4 & 13031.3 & 13043.9 & 13032.5 & 12900.8 & 13055.1 & 13347.3 \\
$\chi^2_{\mt{red}}$ & 1.028 & 1.026 & 1.021 & 1.022 & 1.021 & 1.011 & 1.023 & 1.046 \\
dof & 12764 & 12763 & 12761 & 12762 & 12761 & 12758 & 12761 & 12764 \\
\bottomrule
\end{tabular}

%    \tablecomments{Units:
%        equivalent hydrogen column density $N_H$, $10^{22} \unit{cm^{-2}}$;
%        temperature $\kB T$, keV;
%        ionization timescale $\tau$, $10^{10} \unit{s\;cm^{-3}}$;
%        emission measure (EM), $10^{11} \unit{cm^{-5}}$.
%        Elemental abundances are taken relative to ISM values of
%        \citet[Table 2]{wilms2000}.
%        The emission measure is $10^3 \times $ the usual \texttt{XSPEC} norm
%        $10^{14} (4 \pi D^2)^{-1} \int n_{\mt{H}} n_{\mt{e}} dV$; $D$ is source
%        distance, $n_{\mt{H}}$ is hydrogen number density, and $n_{\mt{e}}$ is
%        electron number density.
%    }
%\end{table*}

% Explain X-ray background model (TODO: use "sky" vs. "cosmic" consistently)
The X-ray sky background model comprises an unabsorbed low-temperature thermal
plasma for both local bubble and stable SWCX emission
\citep{mccammon1990, snowden1990, cravens2000, galeazzi2014, smith2014},
an absorbed thermal plasma for soft diffuse galactic ridge emission
\citep[e.g.][]{kaneda1997}, and an absorbed extragalactic
background power law with photon index $1.4$ \citep{hickox2006}.
The XSPEC model is \texttt{apec + tbnew\_gas*apec + tbnew\_gas*powerlaw}).
We fix the extragalactic power law normalization to
$6.6 \unit{photons\; cm^{-2}\, s^{-1}\, sr^{-1}\, keV^{-1}}$, which
represents $60\%$ of the total extragalactic X-ray background normalization
$10.9 \unit{photons\; cm^{-2}\, s^{-1}\, sr^{-1}\, keV^{-1}}$
\citep{hickox2006} due to removal of bright extragalactic sources.
\footnote{Our point source flux threshold $10^{-14} \unit{erg\;cm^{-2}\;s^{-1}}$
(0.4--7.2 keV band) translates to removal of extragalactic sources brighter than
$1.5 \times 10^{-14} \unit{erg\;cm^{-2}\;s^{-1}}$ (2--10 keV band, assuming
absorbing column $2 \nHUnits$ and power law index $\Gamma = 1.4$).
We integrate the $\log(N)$-$\log(S)$ relation of \citet{moretti2003} to obtain
the excluded brightness.}

Our best fit X-ray sky background has parameters
$\kB T_{\mt{local}} = {0.26}^{+0.02}_{-0.01}$,
%$\mt{EM}_{\mt{local}} = {0.24}^{+0.02}_{-0.01}$,
$N_{\mathrm{H,xgal}} = {1.24}^{+0.73}_{-0.58}$,
$N_{\mathrm{H,ridge}} = {1.38}^{+0.20}_{-0.09}$, and
$\kB T_{\mt{ridge}} = {0.75}^{+0.04}_{-0.04}$.
%$\mt{EM}_{\mt{ridge}} = {1.92}^{+0.31}_{-0.30}$,
where subscript ``lb'' denotes unabsorbed local bubble emission,
``xgal'' denotes galactic absorption of the extragalactic background,
and ``ridge'' denotes absorbed galactic ridge emission.
% MAJOR TODO: need to clean up this text...
% Check model - galactic ridge component has reasonable surface brightness.
The fitted soft galactic ridge component has $2$--$10 \unit{keV}$ surface
brightness $3.1 \times 10^{-12} \unit{erg\;cm^{-2}\;s^{-1}\;deg.^{-2}}$,
somewhat lower than other galactic ridge or bulge fields
\citep{ebisawa2008, revnivtsev2009}.
%$4.8 \times 10^{-11} \unit{erg\;cm^{-2}\;s^{-1}\;deg.^{-2}}$
%at $(l,b) = (28\arcdeg 46\arcmin, -0\arcdeg 20\arcmin)$ \citep{ebisawa2008}
%and
%$7.1 \times 10^{-11} \unit{erg\;cm^{-2}\;s^{-1}\;deg.^{-2}}$
%at $(l,b) = (0.113\arcdeg, -1.424\arcdeg)$ \citep{revnivtsev2009}.
These other fields sample brighter areas of X-ray ridge emission (see the
RXTE/PCA map of \citet{revnivtsev2006}), so a $\abt 10\times$ smaller
brightness seems acceptable given uncertainty in our background modeling from
assumed extragalactic background normalization and high soft proton background.

%\subsection{Annuli spectra}
%
%% Procedure, results
%We extract and fit spectra from $100\arcsec$ wide annuli centered on the
%remnant to discern how thermal plasma emission varies with remnant radius
%(Figure~\ref{fig:annuli-spectra}, Table~\ref{tab:annulus-fit}).
%Each annulus is modeled by an absorbed NEI plasma
%(XSPEC model \texttt{tbabs\_new * vnei}) with Mg, Si, S free and absorbing
%column tied across all annuli.
%X-ray background model parameters are fixed to values from the best integrated
%source fit with Mg, Si, S free (Table~\ref{tab:src-fits}).  % TODO may change...
%We neglect redistribution of counts between annuli due to point spread function
%(PSF) wings as the off-axis half-energy-enclosed width $\abt10$--$20\arcsec$ is
%smaller than our annuli widths.
%
%% Some basic observations on annulus spectra
%The outer annulus fit ($400$--$500\arcsec$) is poorly constrained and merely
%requires some Si and S line emission; the derived ionization timescale,
%temperature, and norm may be unreliable.
%Soft proton parameters from annuli fits qualitatively show expected
%energy-dependent vignetting, with stronger vignetting at soft energies.
%Soft proton contamination at $\abt 0.3$--$1 \unit{keV}$ clearly decreases with
%distance from the aimpoint.
%
%\begin{table*}
%    \centering
%    \caption{Annuli fit \label{tab:annulus-fit}}
%    % 20161019_fiveann_mg.json
\begin{tabular}{@{}rllllll@{}}
\toprule
Annulus & $kT$ & $\tau$ & Mg & Si & S & EM \\
 & (keV) & ($\times 10^{10}$) & (-) & (-) & (-) & $(\times 10^{11})$ \\
\midrule
  $0$--$100\arcsec$ & ${3.9}^{+2.4}_{-1.5}$ & ${1.37}^{+0.21}_{-0.08}$ & ${1.6}^{+0.5}_{-0.5}$ & ${8.3}^{+1.6}_{-1.3}$ & ${8.4}^{+\,(>1.6?)}_{-1.9}$ & ${3.2}^{+1.1}_{-0.7}$ \\
$100$--$200\arcsec$ & ${2.1}^{+0.5}_{-0.4}$ & ${1.94}^{+0.37}_{-0.24}$ & ${1.30}^{+0.23}_{-0.21}$ & ${6.3}^{+0.7}_{-0.7}$ & ${5.7}^{+0.9}_{-0.8}$ & ${4.8}^{+1.3}_{-0.9}$ \\
$200$--$300\arcsec$ & ${2.1}^{+0.4}_{-0.4}$ & ${2.1}^{+0.5}_{-0.2}$ & ${1.6}^{+0.3}_{-0.2}$ & ${4.4}^{+0.5}_{-0.5}$ & ${4.0}^{+0.5}_{-0.6}$ & ${4.7}^{+1.3}_{-0.8}$ \\
$300$--$400\arcsec$ & ${1.7}^{+0.5}_{-0.3}$ & ${2.1}^{+0.8}_{-0.5}$ & ${1.1}^{+0.2}_{-0.2}$ & ${3.2}^{+0.5}_{-0.5}$ & ${2.9}^{+0.9}_{-0.6}$ & ${2.5}^{+0.8}_{-0.5}$ \\
$400$--$500\arcsec$ & ${10}^{+?}_{-3}$ & ${1.3}^{+0.5}_{-0.5}$ & ${1.8}^{+2.1}_{-0.8}$ & ${2.7}^{+2.4}_{-1.0}$ & ${3.5}^{+\,(>6.5?)}_{-2.1}$ & ${0.34}^{+0.16}_{-0.14}$ \\
\bottomrule
\end{tabular}

%    \tablecomments{
%        Outermost ($400$--$500\arcsec$) annulus temperature $\kB T$ is
%        capped at $10 \unit{keV}$ and may not represent the formal best fit.
%        Best fit absorption column is $\nH = {2.51}^{0.09}_{-0.10} \nHUnits$;
%        $\mt{pgstat} = 19335.8$; $\chi^2_{\mt{red}} = 1.210 = 24296.5 / 20073$ ($\chi^2$/dof).
%        Units as in Table~\ref{tab:src-fits}.}
%\end{table*}
%
%\begin{figure*}[]
%    \plotone{fig/fig_fiveann_mg-si-s_0087940201-mos.pdf}
%    \figcaption{Annuli spectra from 0087940201 MOS with best fits; Mg, Si, S free.
%    Table~\ref{tab:annulus-fit} provides NEI fit parameters.}
%    \label{fig:annuli-spectra}
%\end{figure*}
%
%\begin{figure*}[!ht]
%    \plottwo{fig/fig_fiveann_kT.pdf}{fig/fig_fiveann_Tau.pdf}
%    \plottwo{fig/fig_fiveann_Si.pdf}{fig/fig_fiveann_S.pdf}
%    \figcaption{Electron temperature (left) and ionization timescale (right)
%        as a function of radius from annuli spectra fit
%        (Tables~\ref{tab:annulus-fit}).}
%    \label{fig:annulus-pars}
%\end{figure*}
%
%% Radial variation in kT and Tau
%The plasma temperature and ionization timescale vary by $\abt 20$--$50\%$
%across the remnant's projected radius (Figure~\ref{fig:annulus-pars});
%the center plasma is hotter and shows a lower ionization timescale compared to
%the remnant shell.
%The inferred Si, S abundances increase by a factor of $2$-$3$ towards the
%center; Mg shows much less variation.
%These trends are consistent with the standard Sedov picture of early supernova
%remnant evolution, wherein recently shocked ISM and reverse-shocked central
%ejecta should be the ``youngest'' emitting plasma.


% =============================================================================
% Spatially-resolved spectra
% =============================================================================

\begin{table*}[!th]
    \centering
    \caption{Integrated remnant and sub-region fits
        \label{tab:spectrum-fits}}
    \footnotesize
    % 20161015_src_bkg_grp01_pgstat_mosmerge.json
% 20161222_ridge_nw.json
% 20161222_ridge_se.json
% 20161222_core.json
% 20161222_lobe_ne.json
% 20161222_lobe_sw.json
\begin{tabular}{@{}lllllll@{}}
\toprule
 & SNR & Ridge NW & Ridge SE & Core & Lobe NE & Lobe SW \\
\midrule
$\nH$ & ${2.5}^{+0.1}_{-0.1}$ & ${2.8}^{+0.1}_{-0.1}$ & ${2.6}^{+0.3}_{-0.3}$ & ${2.4}^{+0.2}_{-0.2}$ & ${3.8}^{+0.8}_{-1.2}$ & ${1.8}^{+0.2}_{-0.2}$ \\
$\kB T$ & ${1.5}^{+0.2}_{-0.2}$ & ${0.8}^{+0.1}_{-0.1}$ & ${1.3}^{+0.6}_{-0.3}$ & ${2.1}^{+0.6}_{-0.3}$ & ${1.0}^{+1.2}_{-0.3}$ & ${2.7}^{+2.0}_{-0.8}$ \\
$\tau$ & ${2.5}^{+0.5}_{-0.4}$ & ${10.1}^{+5.0}_{-3.5}$ & ${3.0}^{+3.2}_{-1.0}$ & ${1.8}^{+0.3}_{-0.2}$ & ${4.0}^{+3.4}_{-2.0}$ & ${1.1}^{+0.2}_{-0.2}$ \\
EM & ${6.1}^{+2.0}_{-1.1}$ & ${28.0}^{+8.9}_{-7.3}$ & ${5.7}^{+4.0}_{-2.2}$ & ${6.0}^{+1.5}_{-1.3}$ & ${6.4}^{+9.5}_{-4.9}$ & ${2.2}^{+0.8}_{-0.6}$ \\
Si & ${4.0}^{+0.2}_{-0.2}$ & ${2.1}^{+0.2}_{-0.2}$ & ${2.5}^{+0.4}_{-0.4}$ & ${6.6}^{+0.6}_{-0.6}$ & ${7.3}^{+4.4}_{-1.7}$ & ${5.4}^{+0.9}_{-0.7}$ \\
S & ${3.6}^{+0.3}_{-0.2}$ & ${1.8}^{+0.3}_{-0.2}$ & ${1.3}^{+0.5}_{-0.4}$ & ${6.6}^{+1.0}_{-0.8}$ & ${7.8}^{+5.4}_{-2.1}$ & ${8.7}^{+5.2}_{-2.6}$ \\
\midrule
pgstat & 14218.8 & 4898.4 & 4599.1 & 4693.3 & 4138.8 & 4383.6 \\
$\chi^2$ & 13119.6 & 6492.3 & 6026.2 & 6201.4 & 5562.8 & 5933.3 \\
$\chi^2_{\mt{red}}$ & 1.028 & 1.264 & 1.169 & 1.229 & 1.243 & 1.274 \\
dof & 12764 & 5136 & 5157 & 5044 & 4477 & 4658 \\
\bottomrule
\end{tabular}

    \tablecomments{Units:
        equivalent hydrogen column density $N_H$, $10^{22} \unit{cm^{-2}}$;
        temperature $\kB T$, keV;
        ionization timescale $\tau$, $10^{10} \unit{s\;cm^{-3}}$;
        emission measure (EM), $10^{11} \unit{cm^{-5}}$.
        Elemental abundances are relative to ISM values of
        \citet[Table 2]{wilms2000}.
        The emission measure is $10^3 \times $ the \texttt{XSPEC} norm
        $10^{14} (4 \pi D^2)^{-1} \int n_{\mt{H}} n_{\mt{e}} dV$; $D$ is source
        distance, $n_{\mt{H}}$ is hydrogen number density, and $n_{\mt{e}}$ is
        electron number density.
    }
    \tablecomments{SNR (integrated 0--400\arcsec remnant) was fit
        simultaneously with sky background, whereas all sub-region fits
        used fixed cosmic X-ray background model.
    }
\end{table*}

\begin{figure}[]
    \epsscale{1.15}
    \plotone{fig/fig_subregions_0087940201-mos.pdf}
    \figcaption{Subregion spectra}
    \label{fig:subregion-fits}
\end{figure}

\section{Spatially-resolved spectra} \label{sec:subregion-fit}

%Motivated by the spatially distinct central Si/S ejecta emission, we fit
%spectra from small regions within the remnant to better characterize .
%resolved spectroscopy.
%encourages resolved
%spectroscopy.

Sub-region spectrum fits confirm that the spatially distinct emission seen in
Figures~\ref{fig:eqwidth} and \ref{fig:rgb} originate from separated ejecta and
shocked ambient medium.
We fit regional spectra to the same absorbed NEI model
(\texttt{tbabs\_new * vnei}) used for the integrated remnant spectrum fit.
The remnant core and Si/S rich lobes require enhanced Si/S abundances, as
expected.
Ridges require mildly enhanced Si/S abundance, but to a lesser extent than the
ejecta-rich central bar of the remnant.
The emission measure of the NW ridge is $\abt 5\times$ that in the SE,
suggesting asymmetric ambient medium interaction.


% =============================================================================
% Discussion
% =============================================================================
\section{Discussion} \label{sec:disc}

\subsection{Prior descriptions of SNR \Gsnr{}}

% State discrepancy up front
Our fitted absorption column and ionization timescales disagree with prior
X-ray fits, although we find similar temperatures and abundance ratios.
\citet{rakowski2001} obtain $\nH = (0.7 \pm 0.3) \nHUnits$,
$\kB T = 2.0^{+1.0}_{-0.6}$, and
$\tau = 1.5^{+4.7}_{-0.6} \times 10^{11} \unit{s\;cm^{-3}}$
from an absorbed NEI model for ASCA data with enhanced Ne, Mg, Si, S, Ar, Ca,
Fe abundances; the best fit converged towards a pure metal plasma.
\citet{safi-harb2007} report $\nH = 0.65^{+0.45}_{-0.25} \nHUnits$
and $\kB T = 2 \pm 0.6 \unit{keV}$ without providing $\tau$, from
an absorbed NEI model for the 2001 XMM data with
super-solar Si, S and sub-ISM O, Ne, Mg, Ca, Fe.
\citet{rakowski2001} also report emission measure
$(4\pi D^2)^{-1} \int n_e n_H dV = 9.7^{+200}_{-9.7} \times 10^8$ much smaller
than our $\abt 4 \times 10^{11} \unit{cm^{-5}}$; their inferred hydrogen
density is therefore $\abt 10\times$ smaller.
The low value is partially offset by the extremely high ($\abt 30\times$)
abundances inferred by \citet{rakowski2001}.

% Abundances
The relative abundances
  $[\mt{Mg}/\mt{Si}] = 0.23^{+0.09}_{0.08}$ and
  $[\mt{S}/\mt{Si}] = 1.09^{+0.24}_{-0.18}$ from \citet{rakowski2001}
are consistent with our fits, which yield $[\mt{Mg}/\mt{Si}] = 0.28^{+0.04}_{-0.05}$
and $[\mt{S}/\mt{Si}] = 0.92 \pm 0.2$ (this statistical error on
$[\mt{S}/\mt{Si}]$ should be somewhat lower, closer to $\pm 0.1$, because the
parameters covary).
Note that $[\mt{Mg}/\mt{Si}]$ is equal to
\[
    \frac{[\mt{Mg}/\mt{H}] / [\mt{Mg}/\mt{H}]_{\sun}}
         {[\mt{Si}/\mt{H}] / [\mt{Si}/\mt{H}]_{\sun}}
\]
or $[\mt{Mg}/\mt{Mg}_{\sun}] / [\mt{Si}/\mt{Si}_{\sun}]$,
which is the usual ratio of interest.

% Possible causes?
%Could HD 119682 be contaminating other workers' fits?
%\citet{rakowski2001} went to some effort to construct a model for the point
%source contamination, modeling the ASCA X-ray emission as an absorbed power
%law ($\nH = 0.26^{+0.18}_{-0.14} \nHUnits$, $\Gamma = 1.5 \pm 0.3$).
%\citet{safi-harb2007} report $\nH = 0.18^{+0.08}_{-0.07} \nHUnits$ and $\Gamma
%= 1.41^{+0.14}_{-0.22}$ based on finely resolved \textit{Chandra} data, and
%obtained comparable numbers from the same \textit{XMM} data we consider here.
%In short, the point source appears adequately accounted for by previous
%workers, so we disfavor this explanation.

% Explore low abundance fit.
If O, Ne, and Fe abundances are allowed to be sub-solar, we can obtain
reasonable fit with lowered absorption of $\abt 0.7 \nHUnits$.
The extremely sub-solar abundances of O, Ne, and Fe may be explained by a
metal-rich ejecta plasma, wherein metals make up $\gtrsim 10\%$ of the plasma
number density.
%We consider two alternative fits in Table~\ref{tab:src-fits}.
%Fit G fixes absorption $\nH = 0.7 \nHUnits$ to see whether a lower absorption
%is consistent with the data, and to compare to fit results presented by
%\citet{rakowski2001} and \citet{safi-harb2007}.
%Fit H fixes $\tau = 5 \times 10^{13} \TauUnits$ to consider a collisional
%ionization equilibrium plasma \citep{smith2014}.
%Figure~\ref{fig:src-bkg-vary-comp} compares these fits directly to a baseline
%fit with only Mg, Si, S free (same fit as in Figure~\ref{fig:src-bkg-fits}).
Great deal of uncertainty below Mg He-alpha line.
Possible SWCX contamination, degeneracy between Fe-L forest, ejecta or ISM/CSM
O/Ne lines, and interstellar absorption conspire to make this part of the
spectrum more difficult to interpret.

I ascribe the discrepancy to possibly overfitting various elements and the
inability to distinguish discrete plasma components in ASCA data, and so argue
that our more parsimonious analysis makes sense.
In Section~\ref{sec:cloud-absorption} I argue that the increased absorption favored by our
model is consistent with the coarse absorption gradient observed in our five
subregion fits, magnetic field orientation at $d \sim 7 \unit{kpc}$, and the
age estimates from plasma ionization and Sedov-Taylor expansion.
% TODO compare to RCW 80 - any x-ray emission there?...
% TODO clean up this text. Discussion is starting to get interesting!


\subsection{Plasma mass and density}

% Emission measure derived masses
We estimate X-ray emitting plasma mass and density from the integrated remnant
fit emission measure (Table~\ref{tab:spectrum-fits}).
The X-ray emitting plasma density, assuming that the plasma fills a sphere of
radius $400\arcsec$, is:
\begin{equation} \label{eq:density}
    n_{\mt{H}} = 0.1 f^{-1/2} d_{5}^{-1/2} \unit{cm^{-3}}
\end{equation}
with an arbitrary volume filling factor $f \in [0,1]$ and distance relative to
$5 \unit{kpc}$ as $d_{5}$.
Assuming the plasma is hydrogen dominated, with $[\mt{He}/\mt{H}] = 0.1$
\citep{wilms2000}, computing mass $M = 1.4 n_{\mt{H}} m_{H} f V$ yields
\[
    M = 13.7 M_{\sun} f^{1/2} d_{5}^{5/2}
\]
with corresponding ejecta masses $0.018 M_{\sun}$ of silicon, $0.012 M_{\sun}$
of sulfur, and $0.002 M_{\sun}$ of magnesium (attributing a solar abundance to
ISM).

% Mass estimate for a metal-rich plasma
The inferred mass increases if the plasma is metal-rich.
The ion density for a given emission measure is lower than that of an
equivalent hydrogen plasma:
\[
    n_{\mt{ion}} = \sqrt{\frac{4 \pi D^2 \eta}{\langle Z \rangle V}}
\]
where $\eta$ is the emission measure ($10^{14}\times$ the \texttt{XSPEC} norm),
$D$ is source distance, $V$ is source volume, and $\langle Z \rangle$ is mean
atomic number.
But, the mass approximately scales as $\sqrt{2 \langle Z \rangle}$ (assuming no
hydrogen), since
$M_{\mt{tot}} \propto 2 \langle Z \rangle M_{\mt{p}} n_{\mt{ion}}$,
where $M_{\mt{p}}$ is proton mass (approximation for mean nuclide mass) and the
factor $2$ assumes negligible H contribution.
For \Gsnr{}, assuming a spherical volume:
\[
    M_{\mt{tot}} \approx 13.7 M_{\sun} f^{1/2} d_5^{5/2} \langle Z \rangle^{1/2}
\]
where $d_5$ is distance scaled to $5 \unit{kpc}$.
For $\langle Z \rangle \sim 10$ (as an order of magnitude estimate, for a
plasma comprising primarily shocked Si and S), we obtain
$M_{\mt{tot}} \abt 90 M_{\sun} f^{1/2} d_5^{5/2}$.
Although very uncertain, this estimate is rather high for ejecta alone,
although possible if the remnant is close ($d_5 < 1$) and/or the ejecta
distributed in clumps.
Alternatively, shocked ISM/CSM may have mass comparable to that of the ejecta.

% Discuss density inferences
Our inferred density - valid for a homogeneous sphere - should be a lower bound
on the post-forward-shock density.
Therefore the ambient medium density should be at least
$0.025 f^{-1/2} d_{5}^{-1/2} \unit{cm^{-3}}$, assuming a strong forward shock
with compression ratio of four, expected for young ejecta-dominated remnants.
This bound is not very stringent, but is at least a start.
%If shocked plasma is concentrated within a shell of width $r/12$
%% volume fraction (1 - (11/12)^3)
%\textbf{todo:motivate this 1/12 factor}, then the shocked density would be
%$n_{\mt{H}} \sim 0.5 f^{-1/2} d_{5}^{-1/2}$
%and the ambient density bound would be $0.12 f^{-1/2} d_{5}^{-1/2}$,
%which suggests ``typical'' ambient ISM.
%This would also disfavor distance much higher than $5 \unit{kpc}$.

\subsection{Iron non-detection}

Fits with free iron abundance tend towards sub-solar Fe, although the fits
improve only marginally by eye; decreasing Fe abundance does not anchor our
models to salient spectral features.
But, even with solar Fe abundance, we see no Fe K line at $\abt 6.5 \unit{keV}$.
We can set an upper bound on Fe K luminosity and thus constrain the parameter
space for Ia and CC explosions, building on recent work by
\citet{yamaguchi2014-iron} and \citet{patnaude2015}.

We take our best fit (Table~\ref{tab:spectrum-fits} and add a Gaussian line of
normalization $5 \times 10^{-5} \unit{photon\;cm^{-2}\;s^{-1}}$ and
line width $0.05 \unit{keV}$ at $6.55 \unit{keV}$.
This crudely represents a minimum detectable Fe K flux
$\abt 0.01$ to $0.02 \unit{counts\;s^{-1}\;keV^{-1}}$ after folding through
telescope and detector response.
The upper bound on Fe K line luminosity is:
\[
    L_{\mt{Fe-K}} = 1.2 \times 10^{41} d_5^2 \unit{photons\;s^{-1}}
\]
where $d_5$ represents distance scaled to $5 \unit{kpc}$.
\textbf{TODO: clean up this analysis and formalize the non-detection.
We can likely do better with a more careful analysis, say by explicitly
modeling Fe-L and Fe-K emission or calculating errors on a narrow band model of
a power law and gaussian.  Need a rough estimate of Fe line width for this.}

The luminosity $L_{\mt{Fe-K}} \lesssim 10^{41} \unit{photons\;s^{-1}}$
disfavors bright Ia explosions in dense ambient media
$n_0 \sim 1 \unit{cm^{-3}}$; e.g., DDTa in $n_0 \sim 3 \unit{cm^{-3}}$ from
\citet{badenes2003, badenes2005, badenes2006} \citep{yamaguchi2014-iron}
and CC explosion in dense CSM wind with high mass loss rates appropriate for
red supergiants \citep{patnaude2015}.
If $d_5 \sim 2$--$3$, near the \ion{H}{1} absorption-derived distance bound of
\citet{gaensler1998-g309}, the line luminosity
$\lesssim 10^{42} \unit{photons\;s^{-1}}$ is less constraining.
We will argue against such a large distance in subsequent discussion, on
the basis of remnant size and relatively low ionization state.

\begin{figure}[]
    \plotone{../results-interm/20170123_fe-k_derived_core_upper_bound_on_fe_abund.png}
    \figcaption{
        Our upper bound on Fe K luminosity, regardless of model,
        allows us to constrain the abundance of non-detected Fe as a function
        of underlying plasma ionization and temperature.
        We folded ATOMDB \textbf{TODO: version} NEI models through XMM MOS
        response and computed 6-7 keV Fe flux as a function of ionization age
        and plasma temperature.
    }
    \label{fig:fe-bound}
\end{figure}


\subsection{Distance from absorption towards SNR \Gsnr{}} \label{sec:cloud-absorption}

SNR \Gsnr{} lies at the edge of an extensive molecular complex seen in
${}^{12}$CO, ${}^{13}$CO, and C${}^{18}$O J=(1-0) transitions at $v_\mt{LSR} =
-64$ to $-36 \unit{km/s}$ \citep{saito2001}.
Figure 1 of \citet{karr2009} shows the cloud edge striking southeast to
northwest across SNR \Gsnr{}, bisecting its bright radio lobes.
Based on rotation curve of \citet{fich1989}, \citet{saito2001} suggest that the
cloud lies at $4$--$7 \unit{kpc}$ within the Centaurus Arm.

Our soft X-ray absorption change from NE to SW lobes can be attributed to
foreground absorption at the edge of the Centaurus cloud, and our absorption
column measurements of $2$ to $4 \nHUnits$ are fully accounted for by galactic
\ion{H}{1} (dominated by the near Carina arm and Centaurus arm)
and this single, dominant foreground Centaurus cloud.
We infer a cloud H$_{2}$ column density gradient
$4 \times 10^{21} \unit{cm^{-2}}$ to $2.5 \times 10^{22} \unit{cm^{-2}}$
based on CO-to-H$_{2}$ conversion factor
$X_{\mathrm{CO}} = 2 \times 10^{20} \unit{cm^{-2} (K km s^{-1})^{-1}}$
\citep{bolatto2013};
our estimated bounds have uncertainty $\abt 80\%$ assuming 30\% uncertainty in
$X_{\mathrm{CO}}$ and 50\% error in our eyeballed estimate of ${}^{12}$CO
brightness from Fig.~1a of \citet{saito2001}.
% {}^{13}CO: 3.5 to 7 K km/s at SW to 28 K km/s at the NE lobe's outer edge
% {}^{12}CO: just eyeballing, ~20 K km/s at SW to 125 K km/s at NE lobe outer edge
% So, a rise of order 5 to 10x.
We thus attribute the absorption change from NE to SW lobes
$\Delta \nH = 2 \nHUnits$
to foreground absorption in the Centaurus Arm's molecular cloud.
In conjunction with total galactic \ion{H}{1} column $\abt 1.6 \nHUnits$
from the Leiden-Argentine-Bonn Survey and Parkes Galactic All Sky Survey
\citep{kalberla2005, mcclure-griffiths2009},
we predict absorption ranging from $2.4 \nHUnits$ to $6.6 \nHUnits$ across SNR
\Gsnr{}.
These numbers are upper bounds, as we have assumed that all \ion{H}{1} and
${}^{12}$CO emission lies in the foreground.
\textbf{TODO: get Mopra data to make quantitative comparison.  And,
I am using $N_{\mt{tot}} = \nH + 2 N_{\mt{H_{2}}}$.}
% TODO

The fitted absorptions, the lack of evidence for molecular cloud interaction,
and the estimated distance $5--14 \unit{kpc}$ from \ion{H}{1} absorption
suggest that SNR \Gsnr{} lies beyond this cloud.
We thus favor a distance $\gtrsim 5 \unit{kpc}$, and gain confidence that our
soft X-ray spectrum reflects foreground absorption rather than suppressed O,
Ne, and Fe emission.

Can the cloud's position be refined?
\citet{rice2016} have created a thorough catalog of molecular clouds with
distance determinations based on the galactic CO survey of \citet{dame2001}.
Using a contemporary rotation curve, dendrogram-based structure decomposition,
and (1) galactic latitude + (2) size-linewidth relation approach to resolve the
kinematic distance ambiguity,
the largest cloud within $1\deg$ of $l = 309.2\deg$
($l=309.4\deg, b=-0.1\deg$, spatial size $0.36\deg$,
$v_{\mt{LSR}} = -42.7 \unit{km/s}$)
is placed on the ``far'' line-of-sight velocity at $d = 7 \unit{kpc}$ with
corresponding cloud radius $0.09 \unit{kpc}$ \cite[][Table 3]{rice2016}.
% Fig. 17 of Rice+ 2016 show Larson linewidth relation
% Moderate scatter, but it's better than nothing.
This distance determination would place SNR \Gsnr{} around or beyond the
outward-facing edge of the Centaurus arm.
Since \ion{H}{1} spectra towards SNR \Gsnr{} do not show absorption at positive
$v_{\mt{LSR}}$, the upper bound (with substantial uncertainty) on distance
should be $\abt 10 \unit{kpc}$.
An independent contemporary analysis (following the same train of thought:
clustering classification + some resolution of the kinematic distance) also
places the massive Centaurus arm cloud(s) at $\abt 7 \unit{kpc}$
\citet{miville-deschenes2017}.
Unsolicited aside: both papers are very impressive (to this outsider) in scope
and utility.

% LAB HI survey at https://www.astro.uni-bonn.de/hisurvey/profile/index.php
% NW lobe: RA 13 47 12, declination -62 48 00
% SE lobe: RA 13 46 00, declination -62 58 00
% Corresponds to 5 to 10% brighter HI peaks for v_LSR < +50 km/s.
% Slight effect, but not nearly as dramatic as the cloud gradient.
% The integrated HI absorption is 1.6e22 to 1.7e22.

% I would like to get Mopra CO data on this!


% sigma-D estimate of distance
Lastly, from the somewhat questionable $\Sigma$-$D$ relation (calibrated for
shell-type radio remnants), \citet{pavlovic2014} provide a distance estimate of
$5.9 \unit{kpc}$.
This is fraught with peril, but is at least not wildly inconsistent with the
distance estimate we now favor based on foreground absorption.
%See also \citet{huang1985} which presents a $\Sigma$-$D$ relation for
%shell-like remnants also interacting with clouds -- which helps constrain the
%expected brightness.
%
% Shklovskii 1960
% Clark & Caswell 1976
% Milne 1979
% you must understand the argument before rejecting it out of hand.


\subsection{Age dating}

% Age estimates
From our emission-measure-derived plasma density and integrated ionization age,
we infer an age of $7000 f^{1/2} d_{5}^{1/2} \unit{yr}$.

The Sedov-derived age \citep{taylor1950, sedov1959} is calculated using the
numerical constants for a monatomic gas \citet{taylor1950-pt2}.
From the shock radius scaling
$R = (0.34 \unit{pc}) \left( \frac{E_{51}}{n_0} \right)^{1/5} t_{\mt{yr}}^{2/5}$,
we invert for remnant age and assume angular radius $400\arcsec$ to obtain:
\[
    t = (4300 \unit{yr}) d_5^{5/2} E_{51}^{-1/2} n_0^{1/2} .
\]
which is somewhat on the high side for an ejecta dominated remnant.
\textbf{TODO: factor 0.34 is derived from Taylor's numerically derived
$\beta^5 = 2.052$, but better to use closed form solution}

The Sedov age and ionization timescale / norm derived age scale differently
with remnant distance and can help set a lower bound on our distance estimate
(Figure~\ref{fig:age}).
However, the Sedov estimate represents an upper bound on age as the remnant
may still be ejecta-dominated, or transitioning into Sedov expansion.
Based on
\citet{rakowski2001} already considered these estimates, and our current age
estimates are no more stringent than theirs.

When is the Sedov age applicable?
\citet{truelove1999} compute timescales for the transition from free expansion
to Sedov-Taylor dynamics, assuming ambient medium with a power law radial
profile.
For a medium with $\rho_0 \propto r^{-7}$, the transition time
\[
    t_{\mt{ST}} = 0.732 E^{-1/2} M_{\mt{ej}}^{5/6} \rho_0^{-1/3} .
\]
We take ambient density
$\rho_0 < 1.4 m_{\mt{H}} \times \frac{n_{\mt{H}}}{4}
 \approx 0.01 m_{\mt{H}} f^{-1/2} d_{5}^{-1/2} \unit{cm^{-3}}$
assuming a strong shock with compression ratio $4$, and adopting our rough
density estimate from equation~\eqref{eq:density}.
Recalling that the ion density for a shocked metal-rich plasma is lower than
that of shocked ISM, this represents a conservative bound.
We then have:
\[
    % 0.732 * (10^51 erg)^(-1/2) * (1.99e33 grams)^(5/6) * (0.01 cm^-3 * 1.67e-24 grams)^(-1/3)
    t_{\mt{ST}} \lesssim (1600 \unit{yr})
                         E_{\mt{51}}^{-1/2}
                         \left( \frac{M_{\mt{ej}}}{M_{\sun}} \right)^{5/6}
                         f^{1/6} d_{5}^{1/6}
\]
Alternatively, if we simply scale ambient density to $0.1 \unit{cm^{-3}}$, we have:
\begin{equation} \label{eq:sttime-generic}
    % 0.732 * (10^51 erg)^(-1/2) * (1.99e33 grams)^(5/6) * (0.1 cm^-3 * 1.4 * 1.67e-24 grams)^(-1/3)
    t_{\mt{ST}} = (670 \unit{yr})
                  E_{\mt{51}}^{-1/2}
                  \left( \frac{M_{\mt{ej}}}{M_{\sun}} \right)^{5/6}
                  \left( \frac{n_0}{0.1 \unit{cm^{-3}}} \right)^{-1/3}
\end{equation}
Given the uncertainty in density and mass estimates based solely on plasma
emission measure, Figure~\ref{fig:age} shows a range of plausible $t_{\mt{ST}}$
timescales for ambient medium densities $0.1$ to $0.1 \unit{cm^{-3}}$.

\textbf{TODO: possible improvement}, either use the fully stitched together results
of \citet{truelove1999}, or the recent reformulation of \citet{tang2016}.


\begin{figure}[!ht]
    \plotone{fig/fig_age_plot.pdf}
    \figcaption{Red: Sedov-derived age bound with $E_{51} = 1$
        and $n_0 \in [0.1, 1] \unit{cm^{-3}}$.
        Blue: plasma ionization and EM derived age estimate, with filling factor
        $f \in [0.2, 1]$.  Plasma is assumed to be homogenous and spherical
        volume filling.
        Gray: transition time $t_{\mt{ST}}$ range for
        $n_0 \in [0.1, 1] \unit{cm^{-3}}$ from \citet{truelove1999}, assuming
        power-law ambient medium with $n=-7$.
    }
    \label{fig:age}
\end{figure}

\subsection{Galactic environment}

% The absence of H$\alpha$ emission is unilluminating.
% Assuming H$\alpha$ emission to arise by shock propagation through ISM
% containing neutral hydrogen \citep{chevalier1978},  % 1978ApJ...225L..27C
% any of the following factors may contribute: low density of neutral H due to
% progenitor or supernova ionization, low ambient ISM density, or some local
% conditions (extremely high post-shock density?) that rapidly ionize neutral H,
% preventing Balmer line emission.
% Or, the available imagery may just not be deep enough.
% The outcome is that we are unable to constrain shock velocity independently by
% Balmer line width measurement.

% Galactic environment
% The sight line towards \Gsnr{} crosses the tangent of the Centaurus Arm
% (variously Scu-Cen, Crux-Cen, Scu-Crux) and the near and far Carina arm at
% $\abt 1.5 \unit{kpc}$ and $\abt 14 \unit{kpc}$.  Molecular and \ion{H}{1} emission is
% dominated by the Centaurus Arm at line-of-sight velocities $-70$ to
% $-20 \unit{km/s}$ \citep[e.g.,][Figure 4]{dame2011}, and a good number of
% molecular clouds can be identified \citep{rice2016}.  The field is populated
% with unrelated but independently interesting features.

\subsection{Morphology and spatial structure}

Several facets of this remnant draw interest:
\begin{itemize}
    \item Concentration of ejecta interaction in NW ridge; asymmetry with
        respect to SE ridge (we infer much less swept-up mass)
    \item Central bar of shocked ejecta with possible temperature and
        ionization age gradient in NE-SW direction (this is uncertain until I
        specifically explore the parameters / spectral lines that control the
        respective lobe fits, to anchor claims in spectral features)
    \item Outward-jutting, bipolar radio lobes that attain $\abt10\times$
        higher maximum brightness than the circular NW/SE limbs.
        The bipolar lobes represent a $\abt 100\arcsec$ (20--25\%) protrusion
        from the circular shell traced by the NW/SE limbs.
    \item X-ray emission extends into the SW lobe past the circle traced by
        the NW/SE limbs (more easily observed due to decreased absorption
        compared to the NE lobe)
\end{itemize}

\citet{west2016} and \citet{west2017} conducted a thorough modeling of SNRs in
a galactic magnetic field model to observationally determine which of
quasi-parallel or quasi-perpendicular electron acceleration mechanisms better
explains the morphology of bilateral radio shell galactic SNRs.
Much work has been obviously done on this.
In SN 1006, the poster child of particle acceleration, quasi-parallel
acceleration is favored by polarization \citep{reynoso2013} and kinetic MHD
simulation of very high Mach number shock instabilities \citep{caprioli2014-I}.
\textbf{TODO: refine text and references}
But most galactic SNRs are older, and \citet{fulbright1990} argued that
the quasi-parallel remnant morphologies at varying aspect ratio are simply not
observed, compared to the quasi-perpendicular case.
The more sophisticated models of \citet{west2017} favor quasi-perpendicular
acceleration in most remnants on the basis of morphology and literature
distance estimates.

For SNR \Gsnr{}, the models of \citet{west2016} have magnetic field vectors and
polarization striking NE-SW, which would suggest quasi-perpendicular
acceleration in the fainter, circular NW-SE radio limbs.
Motivated by this model, I speculate the following (this discussion is not
test- or data-driven science, but merely a pattern-matching connect-the-dots
exercise):

\begin{itemize}
    \item The faint NW/SE ``ridge'' radio limbs are part of the nominal remnant
        shell, expanding with shock normal perpendicular to the ambient
        magnetic field
    \item The apparent asymmetry in swept-up material could be explained by
        (1) ISM density gradient perpendicular to field lines,
        (2) ISM thermal pressure or magnetic field gradient perpendicular to
        field lines
        (3) Kepler-like bowshock structure? (should we see evidence for
        subsolar abundances in H/He-rich wind?)
        (\textbf{TODO:} I suspect that only density has a chance of being
        dynamically important for a Sedov age remnant, vs. magnetic field, but
        need to check the numbers for different ISM phases
        and different assumed magnetic tensions from swept up B fields... or just find
        relevant references).
        NOTE: such a gradient would produce asymmetry in the NW/SE limbs as
        well; the limbs differ in brightness by a factor 2-3x.
        SE limb is 0.04-0.05 Jy/beam vs. NW limb is 0.015 Jy/beam.
        Is the brightness difference in the right ``direction''?
    \item An independent mechanism is then responsible for creating the bright
        NE/SW lobes (10x brighter than quasi-perpendicular NW/SE ridges,
        at NE lobe peak 0.2 Jy/beam, SW lobe peak 0.1 Jy/beam).
    \item Field alignment parallel to jet or cavity features could ease
        energetic requirements on their creation, IF vectors are mostly in the
        plane of the sky (not sure how to verify this yet short of recomputing
        the \citet{jansson2012} galactic field model)
\end{itemize}

Now how do you create such a thing?
\begin{enumerate}
    \item Jets in the manner of W50/SS433.  \citetalias{gaensler1998-g309} discuss
        this in some detail.  Emitting system is not visible.
    \item Bipolar progenitor wind cavities or pre-existing ISM cavities.
    If cavities, how do you make them emit brightly enough?
    \citet{tsebrenko2013} posit CSM clumps in the ears... can you get them out there?
    blondin, lundqvist, chevalier: http://iopscience.iop.org/article/10.1086/178060/pdf
        seem to focus on really young (order 10-100yr) remnants.
    In each model, are progenitor winds and mass loss rates reasonable and
    consistent with energetics?
    \item asymmetric expansion against an extremely strong magnetic field.
    Work on this case by \citet{insertis1991} and \citet{rozyczka1995}
    assumes extremely strong (mG to G) magnetic fields of relevance near the
    galactic center, but not elsewhere.  So this seems less likely.
\end{enumerate}

Some references on jets (in process of reading up):
H\'{e}nault-Brunet et al. (2012, MNRAS) http://adsabs.harvard.edu/abs/2012MNRAS.420L..13H \\
Seward et al. (2012, ApJ) http://iopscience.iop.org/article/10.1088/0004-637X/759/2/123/pdf \\
de Luca et al. (2006, Science) http://science.sciencemag.org/content/313/5788/814 \\
Heinz et al. (2013, ApJ) http://iopscience.iop.org/article/10.1088/0004-637X/779/2/171/pdf \\
Image of radio blow-out from Circinus X-1 is intriguing.

Approach for estimating lobe/jet energetics:
Heinz (2014, SSRv) https://doi.org/10.1007/s11214-013-0016-4 \\
Tudose et al. (2006, MNRAS) https://doi.org/10.1111/j.1365-2966.2006.10873.x \\
Dubner et al. (1998, ApJ) http://iopscience.iop.org.ezp-prod1.hul.harvard.edu/article/10.1086/300537/pdf

A quick and dirty estimate of a single lobe's luminosity at 5 kpc is
$10^{35} \unit{erg\;s^{-1}}$, which is somewhat comparable to the total jet
power of SS 433 ($10^{39} \unit{erg\;s^{-1}}$) so we're in the right ball park,
where by ball park I mean within a factor of $10^{3}$ or so.

There are at least a few comparably shaped remnants.
G299.2$-$2.9 similarly shows bipolar ear structure, central ejecta bar, and
asymmetry perpendicular to the ear direction \citep{post2014}.

A few sanity checks on this idea:
1. fainter radio limbs' energy budgets should be consistent with Sedov derived
velocity and loss- or escape-limited maximum electron energy
2. optical extinction towards distant stars, or diffuse atomic/molecular gas,
might reveal some kind of emission

Also remember that the absence of H$\alpha$ emission tells us that there cannot
be too many neutrals ahead of the shock.
Either the NW limb is probagating into ionized wind (due to progenitor or
supernova ionization), the ambient ISM density is low on both sides of the
remnant (just lower on SE), or some local conditions (high post-shock density)
rapidly ionize neutral H and preclude detectable Balmer emission.
Or, the available imagery is just not deep enough.
% Note: this duplicates below text on H-alpha...

Polarization is not currently available (RM variation along lines of sight
uncertain)

%\citet{katsuda2015} use, similar to \citet{kosenko2010}, a deprojection method
%to try to back out a radial profile of the ejecta.
%Someone else, \citet{hughes2003} I think, also deprojected DEM L71 emission.
%\textbf{TODO...}

\subsection{Interpretation}

%% Caveats
%\textbf{Caveats:} our NEI model assumes a homogeneous, single-temperature,
%single-ionization age plasma, and does not separate ejecta and shocked ambient
%ISM/CSM emission.
%The best fit parameters probe only the brightest and densest X-ray emitting
%plasma; see \citet{rakowski2006-g337} for more extended discussion.

%We have several distinct observations to interpret:
%\begin{itemize}
%    \item Low ionization timescale
%        $\tau \sim 2 \times 10^{10} \TauUnits \sim 600 \unit{yr\;cm^{-3}}$
%        consistent with other young ejecta-dominated remnants of Ia SNe
%        \citep[][Table 4]{badenes2007}.
%    \item No obvious evidence for separate ISM and ejecta components of
%        distinct temperature and emission age
%    \item Absence of iron K emission, which
%        suggests
%        1. low intrinsic luminosity (low explosion strength and/or low density
%           medium),
%        2. ejecta stratification, and incomplete propagation of the RS through
%           all ejecta (the RS position is of course impacted by explosion strength
%           and ambient density)
%        3. CC abundances with low Fe, or unusual Ia model with little Fe-group
%           burning
%   \item Certainly super-solar Si and S abundances (hallmark of SNRs, and
%       conveniently placed in X-ray mirror passband).
%       Possibly super-solar Ar, Ca lines.
%    \item Curious radio and X-ray shock front morphology.
%\end{itemize}

% ionization age
The ejecta ionization timescale
$\tau \sim 2 \times 10^{10} \TauUnits \sim 600 \unit{yr\;cm^{-3}}$
is consistent with other young ejecta-dominated remnants of Ia SNe
\citep[][Table 4]{badenes2007}.
But, strangely, our Sedov derived age is an order of magnitude larger than
comparable remnants!

% TODO provide stat uncertainty from XSPEC
As ejecta abundances for Si, S are not as strongly variable between Ia and
core-collapse nucleosynthesis yields (versus O or Fe), the relative abundances
of these elements do not clearly constrain the progenitor type.
\textbf{TODO: read more about Ia and CC models and their yields.}

Might be able to get Ar+Ca in center region but that's really it.
I don't believe fits with any other elements.

Since a distance 7 kpc favors older age $\abt 10^{4} \unit{yr}$, Fe
stratification in ejecta seems somewhat less likely.
The Fe/Si ratio could be $\lesssim 0.5$ if all ejecta is shocked and has
comparable plasma state.
Only other way to suppress Fe emission (i.e. hide Fe mass) is to say (1) Fe is
not yet shocked, (2) Fe temperature is low relative to Si/S and not heavily
ionized (say, $\lesssim 0.1 \unit{keV}$, though $kT \sim 1 \unit{keV}$ only
lets you hide Fe/S $\sim$ 1 -- and then we must deal with loads of Fe L-shell
emission).

% NOTE: SW lobe and ridge both show weird shifts in PN spectral lines (Si, S)
% (this dates from Jan/Feb -- to be investigated with now up to date procedures)

% \subsection{Friends of young remnants}
%
% Comparative study?  TBD.
% G350.1-0.3 is young, bright (~100,000 counts)
% G337.2-0.7 appears a bit older (ionization age $10^12$), also Si/S rich but
% allows Fe.
%
% A next step is to study other remnants with little Fe emission.
% Review the Chandra catalog, Green's catalog...

\section{Conclusions}

This is a young to middle-aged ejecta-rich remnant with little iron likely
lying at $d \geq 7 \unit{kpc}$ beyond the thickest Centaurus arm molecular
clouds.
It shows interesting asymmetry in swept-up ejecta and bipolar lobe structure
that we cannot yet explain fully.

\section{Misc}

Questions for self-edification (many things I don't understand):
\begin{itemize}
    \item What is the characteristic cooling timescale (adiabatic expansion?)
        for reverse shock heated ejecta?
        Under what conditions (density, turbulent mixing, ...)
        would conduction or other energy transfer processes play a role?
    \item What is the state of modeling for young remnant evolution?
        Could I fire up CR-Hydro-NEI or some other simple 1-D model,
        obtain densities, ionization times, temperature as a function of
        radius, and construct model remnant images?
        YES: Ferrand+ 2014 show lovely 3D images (don't trace reverse shock all
        the way).
        Lee+ 2014 for CR Hydro NEI
\end{itemize}

\acknowledgments

X acknowledges support by contract ...

This research is based on observations obtained with XMM-Newton, an ESA science
mission with instruments and contributions directly funded by ESA Member States
and NASA.
The MOST is operated by The University of Sydney with support from the
Australian Research Council and the Science Foundation for Physics within The
University of Sydney.

This research has made extensive use of NASA's Astrophysics Data System.
% TODO is software acknowledgement sufficient?
This research used the SNR catalogs of \citet{ferrand2012} and
\citet{green2014}.
This research made use of Astropy, a community-developed core Python package
for Astronomy \citep{astropy2013}.
This research also made use of APLpy, an open-source plotting package for
Python hosted at \url{http://aplpy.github.com}.
This research was expedited in part by Jonathan Sick's \texttt{ads2bibdesk}.

\facility{XMM(EPIC), Molonglo Observatory}
\software{APLpy, Astropy, XSPEC}

%\listofchanges

% ==========
% References
% ==========
\bibliographystyle{aasjournal}
\bibliography{refs-snr}

% ========
% Appendix
% ========
\clearpage  % Use \clearpage over \newpage
\appendix

\setcounter{table}{0}
\renewcommand{\thetable}{A\arabic{table}}
\setcounter{figure}{0}
\renewcommand{\thefigure}{A\arabic{figure}}

\section{Solar wind charge exchange checks}

% TODO explain why we don't care about this
We also neglect low-level solar wind charge exchange (SWCX) contamination,
which may also vary between the two widely-separated XMM pointings.


\section{Sky X-ray background modeling caveats}

The sky X-ray background may not be constant over scales of several arcminutes.
\citet{henley2013} modeled galactic halo emission using 110 XMM-Newton
observations well outside the galactic plane.
Most of their observations were widely separated, but a group of pointings
within $\abt 30\arcmin$ (labeled 103.1--103.27) showed halo temperature
variations $\abt 10$--$20\%$ and emission measure variation within a factor of
$\abt 2$.
% TODO I am totally just eyeballing the numbers right now -- not that useful.
% Possibly remove this.
% Henley 2013: looking at 28 clustered observations well out of the galactic
% plane, sightlines 103.1 to 103.27; 103.8 is two observations.
% SZE SurF project to complement SPT,APEX,ACT study.
Taking background from an annulus around the remnant should average out some
variation, although without knowing the power spectrum, we cannot say anything
quantitative.

The X-ray background parameters from our integrated source and background fit
may be biased by our assumption that a single-temperature NEI plasma adequately
describes integrated remnant emission.
To check this, we fit the background annulus alone, which removes this possible
bias but sacrifices background counts embedded in the source spectrum.
The resulting X-ray background parameters agree within error.
% TODO reference appendix table
% TODO give a sentence to this effect in main body.

\section{Can resolved subregions be fit with solar abundances?}

\begin{table*}[!ht]
    \centering
    \caption{Subregion fits, solar abundances}
    \footnotesize
    % 20161222_ridge_nw_solar.json
% 20161222_ridge_se_solar.json
% 20170109_core_solar.json
% 20170109_lobe_ne_solar.json
% 20170109_lobe_sw_solar.json
\begin{tabular}{@{}llllll@{}}
\toprule
 & Ridge NW & Ridge SE & Core & Lobe NE & Lobe SW \\
\midrule
$\nH$ & ${3.5}^{+0.1}_{-0.1}$ & ${3.8}^{+0.3}_{-0.2}$ & ${5.4}^{+0.1}_{-0.1}$ & ${6.6}^{+0.4}_{-0.4}$ & ${4.6}^{+0.4}_{-0.4}$ \\
$\kB T$ & ${0.7}^{+0.0}_{-0.1}$ & ${0.7}^{+0.1}_{-0.1}$ & ${0.5}^{+0.0}_{-0.0}$ & ${0.5}^{+0.0}_{-0.0}$ & ${0.6}^{+0.1}_{-0.1}$ \\
$\tau$ & ${7.8}^{+4.2}_{-1.3}$ & ${5.6}^{+3.9}_{-1.4}$ & ${9.5}^{+0.8}_{-1.2}$ & ${10.7}^{+3.2}_{-5.3}$ & ${7.3}^{+3.2}_{-2.6}$ \\
EM & ${61.1}^{+19.7}_{-5.4}$ & ${33.0}^{+21.8}_{-8.1}$ & ${342.1}^{+34.0}_{-37.2}$ & ${244.8}^{+130.9}_{-59.6}$ & ${90.4}^{+47.0}_{-36.9}$ \\
\midrule
pgstat & 5090.7 & 4697.9 & 6249.8 & 4492.8 & 4722.3 \\
$\chi^2$ & 6760.0 & 6156.5 & 8413.9 & 6225.6 & 6489.0 \\
$\chi^2_{\mt{red}}$ & 1.316 & 1.193 & 1.667 & 1.390 & 1.392 \\
dof & 5138 & 5159 & 5046 & 4479 & 4660 \\
\bottomrule
\end{tabular}

    \tablecomments{Units as in Table~\ref{tab:spectrum-fits}.}
\end{table*}

\clearpage
\section{Is a nonthermal component compatible with integrated remnant fits?}

% Random question: should nonthermal be hyphenated (non-thermal) or not?
We do not detect nonthermal (power-law-like) X-ray emission.
By eye, we estimate that nonthermal soft X-rays must be less than $\abt 10\%$
of the thermal continuum.

Fits with a single nonthermal component (XSPEC \texttt{powerlaw} or
\texttt{srcut}) limit nonthermal emission to, qualitatively, less than 10\% of
the contribution from thermal emission \citep[cf.][]{reynolds1999}.
Figure~\ref{fig:nonthermal} shows the best fit nonthermal contributions for
each model.
For a power law, we obtain photon index $\Gamma = 4.5 \pm 1$ with
normalization $1.4^{+1.1}_{-0.09} \times 10^{-3} \unit{photons\;s^{-1}\;cm^{-2}\;keV^{-1}}$
at $1 \unit{keV}$.
For \texttt{srcut}, we obtain break frequency $10^{15.7 \pm 0.1}$
corresponding to photon energy cut-off $\abt 0.02 \unit{keV}$.
% Source: 20160630_src_srcutlog_nonsolar_snr_src.txt

Note: the radio spectral index is $0.53$ as derived by \citet{gaensler1998-g309}.
His calculation ignores single dish measurements on the basis that the flux
is confused with RCW 80 emission.  Some authors made efforts to correct their
flux measurements for the confusion; if we include Parkes single dish
measurements the radio index decreases to $0.36 \pm 0.11$.
My own fit obtains $0.52 \pm 0.09$, marginally different than that of
\citet{gaensler1998-g309}, but for simplicity and consistency just use their values.

I also fit the outer annulus emission with a nonthermal component, but the
(rather questionable) fit allowed no nonthermal emission at all.
Given that the supernova remnant's thermal emission is already ill-constrained,
this is not surprising.
I could re-attempt this with a $350$--$450\arcsec$ annulus instead of
$400$--$500\arcsec$.  % TODO - investigate this.

\begin{figure*}[!hb]
    \plotone{fig/fig_src_powerlaw_0087940201-mos1.pdf}
    \plotone{fig/fig_src_srcutlog_0087940201-mos1.pdf}
    \figcaption{Integrated source fit with powerlaw and srcutlog components
        permit only a near-negligible contribution.}
    \label{fig:nonthermal}
\end{figure*}


\clearpage
\section{Does inaccuracy introduced by $\abt 10\%$ error in BACKSCAL ratio for
integrated source fits impact fit results?}

I performed joint fits of integrated source and X-ray background while
varying the BACKSCAL ratio for MOS1S001.
Results in Table~\ref{tab:backscal-hack}.
These fits date from June 2016 and are no longer relevant or correct.

\begin{table*}[!h]
    \centering
    \caption{2009 (Motch) MOS1 BACKSCAL ratio has very little effect on fits
        \label{tab:backscal-hack}}

    \begin{tabular}{@{}lllllll@{}}
    \toprule
    Ratio & $n_\mathrm{H}$ & $kT$ & $\tau$ & Si & S & vnei EM \\
     & ($10^{22} \unit{cm^{-2}}$) & (keV) & ($10^{10} \unit{s\;cm^{-3}}$) & (-) & (-) & (EM units) \\
    \midrule
    % 1 (hack) = results_spec/20160624_src_bkg_hack_eq_one_rerun fit
    \textbf{1}     & ${2.16}^{+0.08}_{-0.11}$ & ${2.29}^{+0.30}_{-0.12}$ & ${1.82}^{+0.14}_{-0.11}$
          & ${3.75}^{+0.18}_{-0.16}$ & ${3.42}^{+0.29}_{-0.25}$ & ${4.5}^{+0.6}_{-0.5} \times 10^{-3}$ \\
    % 0.95 (hack) = results_spec/20160611_src_bkg_rerun, manually entered!
    \textbf{0.95}  & $2.15^{+0.08}_{-0.11}$ & $2.29^{+0.30}_{-0.24}$ & $1.83^{+0.10}_{-0.11}$  % vnei nH, kT, Tau
          & $3.75^{+0.18}_{-0.16}$ & $3.42^{+0.26}_{-0.25}$ & $4.5^{+0.6}_{-0.5} \times 10^{-3}$ \\ % vnei Si,S,norm
    % 0.88 (area ratio) = results_spec/20160624_src_bkg_nohack_rerun fit
    \textbf{0.885} & ${2.15}^{+0.07}_{-0.08}$ & ${2.30}^{+0.29}_{-0.12}$ & ${1.83}^{+0.11}_{-0.11}$
          & ${3.75}^{+0.18}_{-0.16}$ & ${3.42}^{+0.28}_{-0.25}$ & ${4.6}^{+0.5}_{-0.5} \times 10^{-3}$ \\
    \bottomrule
    \end{tabular}

    \quad
    \quad

    \begin{tabular}{@{}llllr@{}}
    \toprule
    Ratio & XRB local $kT$ & XRB $n_\mathrm{H}$ & XRB halo $kT$
           & $\chi^2_{\mathrm{red}} = \chi^2/\mathrm{dof}$ \\
     & (keV) & ($10^{22} \unit{cm^{-2}}$) & (keV) &  \\
    \midrule
    % 1 (hack) = results_spec/20160624_src_bkg_hack_eq_one_rerun fit
    \textbf{1} & ${0.262}^{+0.007}_{-0.007}$ & ${1.33}^{+0.08}_{-0.08}$ & ${0.75}^{+0.04}_{-0.03}$
          & 1.224 = 4611.94/3768 \\
    % 0.95 (hack) = results_spec/20160611_src_bkg_rerun, manually entered!
    \textbf{0.95} & ${0.262}^{+0.007}_{-0.007}$ & $1.32^{+0.08}_{-0.08}$ & $0.75^{+0.04}_{-0.03}$ % XRB
          & 1.213 = 4572.28/3768 \\
    % 0.88 (area ratio) = results_spec/20160624_src_bkg_nohack_rerun fit
    \textbf{0.885} & ${0.262}^{+0.007}_{-0.007}$ & ${1.32}^{+0.08}_{-0.08}$ & ${0.74}^{+0.03}_{-0.03}$
          & 1.206 = 4545.24/3768 \\
    \bottomrule
    \end{tabular}
\end{table*}

% Deluxetable variant

%\floattable
%\begin{deluxetable}{@{}rllllllllll@{}}
%    \rotate
%    \tablecaption{Motch MOS1 BACKSCAL ratio has very little effect on fits
%        \label{table:backscal-hack}}
%    \tablehead{
%          \colhead{Ratio}
%        & \colhead{$n_\mathrm{H}$ }%% $10^{22} \unit{cm^{-2}}$}
%        & \colhead{$kT$ }%% keV}
%        & \colhead{$\tau$ }%% $10^{10} \unit{s\;cm^{-3}}$}
%        & \colhead{Si}
%        & \colhead{S}
%        & \colhead{vnei EM }%% EM units}
%        & \colhead{XRB local $kT$ }%% keV}
%        & \colhead{XRB $n_\mathrm{H}$ }%% $10^{22} \unit{cm^{-2}}$}
%        & \colhead{XRB halo $kT$ }%% keV}
%        & \colhead{$\chi^2_{\mathrm{red}} = \chi^2/\mathrm{dof}$}
%        \\
%          \colhead{}
%        & \colhead{($10^{22} \unit{cm^{-2}}$)}
%        & \colhead{(keV)}
%        & \colhead{($10^{10} \unit{s\;cm^{-3}}$)}
%        & \colhead{(-)}
%        & \colhead{(-)}
%        & \colhead{(EM units)}
%        & \colhead{(keV)}
%        & \colhead{($10^{22} \unit{cm^{-2}}$)}
%        & \colhead{(keV)}
%        & \colhead{}
%      }
%
%    \startdata
%    % 1 (hack) = results_spec/20160624_src_bkg_hack_eq_one_rerun fit
%    \textbf{1}     & ${2.16}^{+0.08}_{-0.11}$ & ${2.29}^{+0.30}_{-0.12}$ & ${1.82}^{+0.14}_{-0.11}$
%          & ${3.75}^{+0.18}_{-0.16}$ & ${3.42}^{+0.29}_{-0.25}$ & ${4.5}^{+0.6}_{-0.5} \times 10^{-3}$
%          & ${0.262}^{+0.007}_{-0.007}$ & ${1.33}^{+0.08}_{-0.08}$ & ${0.75}^{+0.04}_{-0.03}$
%          & 1.224 = 4611.939/3768 \\
%    % 0.95 (hack) = results_spec/20160611_src_bkg_rerun, manually entered!
%    \textbf{0.95}  & $2.15^{+0.08}_{-0.11}$ & $2.29^{+0.30}_{-0.24}$ & $1.83^{+0.10}_{-0.11}$  % vnei nH, kT, Tau
%          & $3.75^{+0.18}_{-0.16}$ & $3.42^{+0.26}_{-0.25}$ & $4.5^{+0.6}_{-0.5} \times 10^{-3}$  % vnei Si,S,norm
%          & ${0.262}^{+0.007}_{-0.007}$ & $1.32^{+0.08}_{-0.08}$ & $0.75^{+0.04}_{-0.03}$ % XRB
%          & 1.213 = 4572.28/3768 \\
%    % 0.88 (area ratio) = results_spec/20160624_src_bkg_nohack_rerun fit
%    \textbf{0.885} & ${2.15}^{+0.07}_{-0.08}$ & ${2.30}^{+0.29}_{-0.12}$ & ${1.83}^{+0.11}_{-0.11}$
%          & ${3.75}^{+0.18}_{-0.16}$ & ${3.42}^{+0.28}_{-0.25}$ & ${4.6}^{+0.5}_{-0.5} \times 10^{-3}$
%          & ${0.262}^{+0.007}_{-0.007}$ & ${1.32}^{+0.08}_{-0.08}$ & ${0.74}^{+0.03}_{-0.03}$
%          & 1.206 = 4545.244/3768 \\
%    \enddata
%\end{deluxetable}

\clearpage
\section{Does fitting four versus five annuli make any difference?}

Answer: no, parameters are identical within error (and the change in parameter
values is much smaller than error).
\textbf{TODO: add a table here showing this result alone}.

I also verified by eye that
(1) all instrumental line normalizations agree to $<1\%$ (often $\lesssim0.1\%$),
(2) remnant model norms agree to $\lesssim 1$\%,
(3) soft proton norms and indices agree to $\lesssim 1\%$.
\textbf{TODO: I need to redo this}
In short, we are good to go.

\begin{figure}[!hb]
  \includegraphics[width={0.5\linewidth}]{fig/fig_fiveann_si-s_0087940201-mos.pdf}
  \includegraphics[width={0.5\linewidth}]{fig/fig_fiveann_mg-si-s_0087940201-mos.pdf} \\
  \figcaption{0087940201 MOS annuli spectra.  Left: Si, S free only.  Right: Mg, Si, S free.}
\end{figure}

\begin{figure}[!hb]
  \includegraphics[width={0.5\linewidth}]{fig/fig_fiveann_si-s_0087940201-pn.pdf}
  \includegraphics[width={0.5\linewidth}]{fig/fig_fiveann_mg-si-s_0087940201-pn.pdf}
  \figcaption{0087940201 PN annuli spectra.  Left: Si, S free only.  Right: Mg, Si, S free.}
\end{figure}

\begin{figure}[!hb]
  \includegraphics[width={0.5\linewidth}]{fig/fig_fiveann_si-s_0551000201-mos.pdf}
  \includegraphics[width={0.5\linewidth}]{fig/fig_fiveann_mg-si-s_0551000201-mos.pdf}
  \figcaption{0551000201 MOS annuli spectra.  Left: Si, S free only.  Right: Mg, Si, S free.}
\end{figure}

\clearpage

\begin{table}
    \centering
    \begin{tabular}{@{}rllllll@{}}
    \toprule
    \multicolumn{7}{c}{Annulus fit (20161019\_fourann.json)} \\
    \midrule
    Annulus & $kT$ & $\tau$ & Mg & Si & S & EM \\
     & (keV) & ($10^{10} \unit{s\;cm^{-3}}$) & (-) & (-) & (-) & $(\times 10^{11})$ \\
    \midrule
      $0$--$100\arcsec$ & ${2.08}^{+0.79}_{-0.44}$ & ${1.45}^{+0.39}_{-0.28}$ &      & ${6.77}^{+0.90}_{-0.78}$ & ${7.13}^{+2.75}_{-1.71}$ & ${5.70}^{+1.18}_{-1.11}$ \\
    $100$--$200\arcsec$ & ${1.53}^{+0.23}_{-0.17}$ & ${2.31}^{+0.43}_{-0.34}$ &      & ${5.25}^{+0.29}_{-0.36}$ & ${4.70}^{+0.64}_{-0.54}$ & ${7.73}^{+1.22}_{-1.22}$ \\
    $200$--$300\arcsec$ & ${1.41}^{+0.20}_{-0.17}$ & ${2.81}^{+0.68}_{-0.45}$ &      & ${3.31}^{+0.24}_{-0.21}$ & ${3.01}^{+0.37}_{-0.30}$ & ${8.63}^{+1.58}_{-1.71}$ \\
    $300$--$400\arcsec$ & ${1.28}^{+0.34}_{-0.11}$ & ${2.57}^{+1.47}_{-0.76}$ &      & ${2.94}^{+0.33}_{-0.30}$ & ${2.66}^{+0.89}_{-0.58}$ & ${3.87}^{+0.57}_{-0.79}$ \\
    \bottomrule
    \end{tabular}
    \tablecomments{
      $\nH = {2.51}^{+0.09}_{-0.10} \nHUnits$;
      $\mt{pgstat} = 19335.8$; $\chi^2_{\mt{red}} = 1.210 = 24296.5 / 20073$ ($\chi^2$/dof).
        Units as in Table~\ref{tab:spectrum-fits}.}
\end{table}

\begin{table}
    \centering
    \begin{tabular}{@{}rllllll@{}}
    \toprule
    \multicolumn{7}{c}{Annulus fit (20161019\_fourann\_mg.json)} \\
    \midrule
    Annulus & $kT$ & $\tau$ & Mg & Si & S & EM \\
     & (keV) & ($10^{10} \unit{s\;cm^{-3}}$) & (-) & (-) & (-) & $(\times 10^{11})$ \\
    \midrule
      $0$--$100\arcsec$ & ${3.82}^{+2.70}_{-1.31}$ & ${1.37}^{+0.21}_{-0.08}$ & ${1.58}^{+0.54}_{-0.41}$ & ${8.20}^{+1.72}_{-1.25}$ & ${8.30}^{-8.30}_{-1.88}$ & ${3.24}^{+0.98}_{-0.71}$ \\
    $100$--$200\arcsec$ & ${2.04}^{+0.54}_{-0.31}$ & ${1.97}^{+0.33}_{-0.28}$ & ${1.28}^{+0.25}_{-0.19}$ & ${6.25}^{+0.77}_{-0.61}$ & ${5.62}^{+0.95}_{-0.73}$ & ${4.93}^{+1.14}_{-1.43}$ \\
    $200$--$300\arcsec$ & ${2.06}^{+0.43}_{-0.35}$ & ${2.15}^{+0.44}_{-0.27}$ & ${1.61}^{+0.26}_{-0.30}$ & ${4.34}^{+0.54}_{-0.41}$ & ${3.93}^{+0.62}_{-0.48}$ & ${4.74}^{+1.18}_{-1.26}$ \\
    $300$--$400\arcsec$ & ${1.72}^{+0.51}_{-0.33}$ & ${2.10}^{+0.73}_{-0.24}$ & ${1.07}^{+0.24}_{-0.20}$ & ${3.18}^{+0.54}_{-0.40}$ & ${2.89}^{+0.89}_{-0.61}$ & ${2.54}^{+0.70}_{-0.56}$ \\
    \bottomrule
    \end{tabular}
    \tablecomments{
      $\nH = {2.26}^{+0.11}_{-0.11} \nHUnits$;
      $\mt{pgstat} = 19298.1$; $\chi^2_{\mt{red}} = 1.207 = 24226.3 / 20069$ ($\chi^2$/dof).
        Units as in Table~\ref{tab:spectrum-fits}.}
\end{table}

\begin{table}
    \centering
    \begin{tabular}{@{}rllllll@{}}
    \toprule
    \multicolumn{7}{c}{Annulus fit (20161019\_fiveann.json)} \\
    \midrule
    Annulus & $kT$ & $\tau$ & Mg & Si & S & EM \\
     & (keV) & ($10^{10} \unit{s\;cm^{-3}}$) & (-) & (-) & (-) & $(\times 10^{11})$ \\
    \midrule
      $0$--$100\arcsec$ & ${2.06}^{+0.76}_{-0.42}$ & ${1.46}^{+0.38}_{-0.29}$ &      & ${6.76}^{+0.91}_{-0.78}$ & ${7.12}^{+2.80}_{-1.73}$ & ${5.77}^{+1.17}_{-1.10}$ \\
    $100$--$200\arcsec$ & ${1.52}^{+0.22}_{-0.09}$ & ${2.32}^{+0.41}_{-0.32}$ &      & ${5.24}^{+0.41}_{-0.36}$ & ${4.69}^{+0.65}_{-0.53}$ & ${7.81}^{+1.21}_{-1.19}$ \\
    $200$--$300\arcsec$ & ${1.40}^{+0.19}_{-0.16}$ & ${2.83}^{+0.44}_{-0.46}$ &      & ${3.31}^{+0.21}_{-0.21}$ & ${3.00}^{+0.36}_{-0.30}$ & ${8.72}^{+1.60}_{-1.34}$ \\
    $300$--$400\arcsec$ & ${1.27}^{+0.32}_{-0.21}$ & ${2.57}^{+1.50}_{-0.76}$ &      & ${2.93}^{+0.34}_{-0.30}$ & ${2.67}^{+0.90}_{-0.56}$ & ${3.91}^{+0.80}_{-0.74}$ \\
    $400$--$500\arcsec$ & ${3.07}^{-3.07}_{-1.08}$ & ${0.68}^{+0.82}_{-0.13}$ &      & ${2.77}^{+0.89}_{-1.12}$ & ${9.88}^{-9.88}_{-8.33}$ & ${0.67}^{+0.21}_{-0.24}$ \\
    \bottomrule
    \end{tabular}
    \tablecomments{
      $\nH = {2.51}^{+0.09}_{-0.10} \nHUnits$;
      $\mt{pgstat} = 25741.5$; $\chi^2_{\mt{red}} = 1.203 = 31637.1 / 26290$ ($\chi^2$/dof).
        Units as in Table~\ref{tab:spectrum-fits}.}
\end{table}

\begin{table}
    \centering
    \begin{tabular}{@{}rllllll@{}}
    \toprule
    \multicolumn{7}{c}{Annulus fit (20161019\_fiveann\_mg.json)} \\
    \midrule
    Annulus & $kT$ & $\tau$ & Mg & Si & S & EM \\
     & (keV) & ($10^{10} \unit{s\;cm^{-3}}$) & (-) & (-) & (-) & $(\times 10^{11})$ \\
    \midrule
      $0$--$100\arcsec$ & ${3.94}^{+2.40}_{-1.47}$ & ${1.37}^{+0.21}_{-0.08}$ & ${1.61}^{+0.51}_{-0.44}$ & ${8.29}^{+1.58}_{-1.34}$ & ${8.38}^{-8.38}_{-1.94}$ & ${3.16}^{+1.11}_{-0.62}$ \\
    $100$--$200\arcsec$ & ${2.08}^{+0.48}_{-0.36}$ & ${1.94}^{+0.37}_{-0.24}$ & ${1.30}^{+0.23}_{-0.21}$ & ${6.30}^{+0.70}_{-0.68}$ & ${5.68}^{+0.87}_{-0.80}$ & ${4.82}^{+1.30}_{-0.85}$ \\
    $200$--$300\arcsec$ & ${2.08}^{+0.38}_{-0.38}$ & ${2.14}^{+0.46}_{-0.26}$ & ${1.62}^{+0.23}_{-0.22}$ & ${4.38}^{+0.47}_{-0.48}$ & ${3.97}^{+0.57}_{-0.53}$ & ${4.65}^{+1.31}_{-0.77}$ \\
    $300$--$400\arcsec$ & ${1.73}^{+0.48}_{-0.34}$ & ${2.10}^{+0.75}_{-0.46}$ & ${1.08}^{+0.23}_{-0.20}$ & ${3.21}^{+0.49}_{-0.43}$ & ${2.92}^{+0.85}_{-0.64}$ & ${2.50}^{+0.77}_{-0.50}$ \\
    $400$--$500\arcsec$ & ${10.00}^{-10.00}_{-2.96}$ & ${1.32}^{+0.52}_{-0.53}$ & ${1.84}^{+2.05}_{-0.83}$ & ${2.67}^{+2.39}_{-0.95}$ & ${3.53}^{-3.53}_{-2.13}$ & ${0.34}^{+0.16}_{-0.14}$ \\
    \bottomrule
    \end{tabular}
    \tablecomments{
      $\nH = {2.25}^{+0.13}_{-0.10} \nHUnits$;
      $\mt{pgstat} = 25699.6$; $\chi^2_{\mt{red}} = 1.201 = 31558.8 / 26285$ ($\chi^2$/dof).
        Units as in Table~\ref{tab:spectrum-fits}.}
\end{table}


\clearpage
\section{Frequently asked questions (or, nitpickables)}

Caveats, disclaimers, and possible points of contention that are not
discussed in the main text for brevity, but should be itemized and reviewed.

\begin{itemize}
    \item \textbf{Why are we using Chandra-derived extragalactic X-ray background
        parameters instead of XMM-derived parameters?}
        Simply, I am more inclined to trust Chandra background modeling and
        background subtraction.
        The \citet{hickox2006} values agree to $\abt 10\%$ of values derived
        from \textit{XMM-Newton}, \textit{Swift}, \textit{ASCA}, and
        \textit{ROSAT} studies
        \citep{chen1997, kushino2002, de-luca2004, moretti2009}.
        \textbf{Do we need to adjust the normalization because XMM resolves
        fewer extragalactic sources than Chandra?}
        If I recall correctly from previous reading (forgot which source(s)) --
        small-number statistics prevail, so a few point sources removed don't
        matter.  It could be more significant for wide-field studies?
        \textbf{Irrelevant now that we have corrected extragalactic background normalization for point source removal -- but not dead yet}
    \item \textbf{Why are halo emission and extragalactic background subject to
        the same absorption?}
        Yes, this is unrealistic.  Halo emission should be absorbed by some
        distance-and-density-averaged column, whereas the extragalactic
        background is attenuated by the complete galactic column.
        But, our X-ray background is decently well fit by our current model,
        and we don't see any need to introduce more parameters.
        The absorptions would only differ by a factor of $\abt 2$ for each
        component, and the discrepancy might be folded into unabsorbed local
        emission or the soft proton components.
        However we have not explored this in depth, and it is possible that our
        assessment of the background would change.
        \textbf{Irrelevant now that we have separated absorption}
\end{itemize}


\end{document}
