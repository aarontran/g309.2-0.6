\documentclass[manuscript]{aastex6}  % one-column, double-spaced GENERATE BIB
%\documentclass[iop, apj, numberedappendix]{emulateapj}

% Customize figures/sizing for 2 column vs. manuscript printout
% TODO - not in HEAD latex install?
%%\usepackage{etoolbox}
%%\newtoggle{manuscript}
%%\toggletrue{manuscript}  % Set TRUE if using manuscript / 1-col layout
%%%\togglefalse{manuscript}  % Set FALSE if using 2-col layout

\shorttitle{Stuff (\today)}  % <~ 44 char
\shortauthors{Draft (\today)}  % Max three
\slugcomment{Draft, \today}

% Packages and commands
\usepackage{amsmath}
\usepackage{booktabs}
\usepackage{hyperref}

\newcommand*{\mt}{\mathrm}
\newcommand*{\unit}[1]{\;\mt{#1}}  % vemod.net/typesetting-units-in-latex
\newcommand*{\abt}{\mathord{\sim}} % tex.stackexchange.com/q/55701
\newcommand*{\ptl}{\partial}
\newcommand*{\del}{\nabla}
\newcommand*\mean[1]{\bar{#1}}
\renewcommand{\vec}[1]{\mathbf{#1}}  % Bold vectors
\newcommand*{\tsup}{\textsuperscript}

% Revision packages and commands
%\usepackage{color}
%\usepackage{soul}
%\newcommand*{\revadd}[1]{\textbf{\color{blue}#1}}
%\newcommand*{\revcut}[1]{\textbf{\color{red}\st{#1}}}

% Paper-specific commands
\defcitealias{rakowski2001}{RHS01}
\defcitealias{gaensler1998}{GGM98}

\begin{document}

\title{Blagh}

\author{
Blah Blah\altaffilmark{1}
Blagh Blagh\altaffilmark{2,3}
}

\affil{
\tsup{1}Blah, BLAH/BLAH MS-Blah, Blah, ME 00000, USA \\
\tsup{2}Blagh, BLAH/BLAH MS-Blagh, Blagh, CA 00000, USA
}

\altaffiltext{3}{Blagh blagh alt affil}


\begin{abstract}
Sketch of a writeup, and my own scribbled notes.
\end{abstract}

% (!) no longer used as of late February, 2016
\keywords{acceleration of particles ---
    ISM: supernova remnants ---
    ISM: individual objects (SNR G309.2-0.6) ---
    X-rays: ISM
%    ISM: magnetic fields ---
%    shock waves
}

\section{Introduction} \label{sec:intro}

\subsection{SNR stuff}

MMSNRs (Rho and Petre, 1998)

Tilley+ (2006) - model anisotropic thermal conduction in remnants in regular
ISM vs. dense medium


Can we confirm that this constitutes a mixed morphology SNR?
(Note, GGM98 came out around the same year as rho/petre)
Rakowski did not label as such but noted ejecta dominated nature

what would be interesting is a comparative study.

G352.7-0.1 (Pannuti+ 2014, XMM/Chandra) exhibits similar features...

G298.6-0.0 (Bamba+ 2015, SuzakuRelevant reference: Kesteven and Caswell (1987)

Plotted with 0.843 GHz radio contours from the Molonglo Observatory Synthesis
Telescope (MOST) supernova remnant catalogue, with resolution $\abt 43\arcsec$
and sensitivity $2$ mJy/beam \\citep{whiteoak1996}.

MOST table 2:
    RA 13h46m40s dec -62 deg. 54 arcmin.
    $S_0.843 = 6.0 Jy$
    $Extent_0.843 = 15\arcmin \times 12\arcmin$
    Surface brightness? (0.843 GHz)? = $5.4e-21 W m^{-2} Hz^{-1} sr^{-1}$
    Relevant reference: Kesteven and Caswell (1987)

%\begin{figure*}[]
%    \plotone{fig_spec_01.pdf} \\
%    \plotone{fig_spec_16.pdf}
%    \caption{Caption goes here}
%    \label{fig:spec}
%\end{figure*}


\section{Observations}

\subsection{Instrumental and Astrophysical Background}

\subsubsection{Instrumental Background}

We follow the ESAS approach (Kuntz and Snowden, 2008).

We augment ESAS by extracting instrumental line ratios from FWC closed data.

We model the following lines...
Mos: 1.49 Al, 1.75 Si
PN lines: 1.49 Al, 7.49 Ni?, 8.05 Cu?, 8.62 Zn?, 8.90 Cu Kbeta?
    (inquire about where 7.11 came from...)

\subsubsection{Astrophysical Background}

Elemental abundances by Wilms et al. (2000) implemented as "tbabs" in XSPEC
(Arnaud 1996).

Will be useful to provide physical units for background emission, and compare
to other measurements.

\paragraph{Soft proton contamination}

Loads of it, everywhere.

\paragraph{Local bubble and galactic halo emission}

Hard to say what variability is like, I think, given sparse data

* Henley and Shelton (2013), XMM-Newton observations of galactic halo emission.
  over 110 sightlines
    $T ~ 2.2 x 10^6 K$
    surface brightness $\abt 0.5-7 \times 10^{-12} \unit{erg cm^{-2} s^{-1} deg^{-2}}$
* Henley and Shelton (2015), XMM and Suzaku observations of shadowing clouds
  to separate foreground (local) and background (halo) emission.

Back of the envelope calculation for our data:
background is $\abt0.1 \unit{ct/s}$ at $1 \unit{keV}$, in source region.
Effective area of MOS is $\abt400 \unit{cm^2}$...
\[
    0.1/400 ct/(s keV cm^2) ~ 2.5e-3 erg / (s erg cm^2)
\]
Angular size of sky region is $\abt 10\arcmin \times 10\arcmin \sim 0.027 \unit{deg^2}$, so we get:
\[
    1.6e-11 erg cm^{-2} s^{-1} deg^{-2}
\]
Given that this background estimate is modeled as a blend of:
local bubble + absorbed hot galactic halo emission + extragalactic power law
(we ignore the "cool" component of galactic halo, likely absorbed since
this remnant is right on galactic plane)
this seems about reasonable.  We should fall right in the range of expected
surface brightness.


Over what lengthscales does the background vary?  Or, how good is our
assumption of a uniform background, in the XMM-Newton FOV?

NOTE: there are 28 clustered observations well out of the galactic plane
that they analyze (sightlines 103.1 to 103.27; 103.8 is two observations)
SZE SurF project to complement SPT,APEX,ACT study.

    Spread in halo T is $\abt10-20\%$.  Column density variation is around 10\%.
    Angular separation between pointings is of order 0.5 degrees $\sim$ 30 arcmin..

    For this study I'm interested in (concerned with?)
    angular variations on slightly shorter scales, of ~10-20 arcmin.  I.e.,
    within a half-degree.

    But I think we are close enough -- "only" factor 2x smaller -- that this
    data might constitute decent probe!



Please also note dispute w.r.t. local bubble: Welsh and Shelton (2009,
Astrophys. and Space Sci.).

Local leo cold cloud -> constraints on local bubble emission
Snowden et al., 2015ApJ...806..119S

Some other interesting reading:
Henley, Shelton, Kwak, Hill, Mac Low (2015) 2015ApJ...800..102H
- modeling gas flow from disk to halo
Peters et al. (2015) 2015ApJ...813L..27P
- more complex modeling works out; plasma for halo emission can be sourced from
  stellar feedback

\paragraph{Solar wind charge exchange}

Solar wind charge exchange -- based on short time cut (give plots in
supplemental material) and analysis by Carter et al. (2008, 2011),
data from 0087940201 are likely unaffected.

\paragraph{Extragalactic background}

ESAS (Snowden/Kuntz) recommends alpha = 1.46 but gives no source.

Chen, Fabien, Gendreau (1997)
  ASCA/ROSAT observations of QSF3 field (RA ~ 55 deg., del ~ -44. deg.; falls
  between LMC and fornax cluster, well out of galactic plane.

  Result: photon index ~ 1.40 to 1.50, depending on how you fit it.
  Single power law fit:
    ROSAT(1-2keV) + ASCA(1-7keV) -> gamma = 1.48 +/- 0.07
    ROSAT(1-2keV) + ASCA(1-3keV) -> gamma = 1.39 +/- 0.11
  Powerlaw + raymond-smith plasma model,
    ROSAT(0.5-2keV) + ASCA(0.4-7keV) -> gamma = 1.46 +/- 0.06

Kuntz, Snowden (200)
    ROSAT - need to skim this

Hickox, Markevitch (2006ApJ...645...95H)
    CDF-N and CDF-S (total 3 Megasec)
    [interestingly, enabled by stowed ACIS calibration observations]
    Required careful background subtraction!
      For such deep Chandra observations, unresolved sky bkg is faint compared
      to quiescent detector background.


\acknowledgments

Blah blah blah.
This research is based on observations obtained with XMM-Newton, an ESA science
mission with instruments and contributions directly funded by ESA Member States
and NASA.
This research has made use of NASA's Astrophysics Data System.
%This research made use of Astropy, a community-developed core Python package
%for Astronomy (Astropy Collaboration, 2013).
%This research also made use of APLpy, an open-source plotting package for
%Python hosted at \href{http://aplpy.github.com}{http://aplpy.github.com}.

{\it Facilities:} \facility{XMM (EPIC)}, \facility{Molonglo Observatory}

%\software{Astropy, ...}

% ==========
% References
% ==========
\bibliographystyle{apj}
\bibliography{refs-snr}

% ========
% Appendix
% ========
%\clearpage
%\newpage
%\appendix
%
%\setcounter{table}{0}
%\renewcommand{\thetable}{A\arabic{table}}
%\setcounter{figure}{0}
%\renewcommand{\thefigure}{A\arabic{figure}}

\end{document}
