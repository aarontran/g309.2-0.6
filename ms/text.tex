\documentclass[twocolumn,tighten,trackchanges]{aastex6}
%\documentclass[iop, apj, numberedappendix]{emulateapj} % Still more compact than aastex

% Customize figures/sizing for 2 column vs. manuscript printout
% TODO - not in HEAD latex install?
%%\usepackage{etoolbox}
%%\newtoggle{manuscript}
%%\toggletrue{manuscript}  % Set TRUE if using manuscript / 1-col layout
%%%\togglefalse{manuscript}  % Set FALSE if using 2-col layout

\shorttitle{Stuff (\today)}  % <~ 44 char
\shortauthors{Draft (\today)}  % Max three
\slugcomment{Draft, \today}

% Packages and commands
%\usepackage{amsmath}  % Included in aastex
\usepackage{booktabs}
\usepackage{hyperref}

% My "standard" TeX aliases
\newcommand*{\mt}{\mathrm}
\newcommand*{\unit}[1]{\;\mt{#1}}  % vemod.net/typesetting-units-in-latex
\newcommand*{\abt}{\mathord{\sim}} % tex.stackexchange.com/q/55701
\newcommand*{\ptl}{\partial}
\newcommand*{\del}{\nabla}
\newcommand*\mean[1]{\bar{#1}}
\renewcommand{\vec}[1]{\mathbf{#1}}  % Bold vectors
\newcommand*{\tsup}{\textsuperscript}

% Paper-specific commands
\newcommand*{\nH}{n_{\mathrm{H}}}
\defcitealias{rakowski2001}{RHS01}
\defcitealias{gaensler1998}{GGM98}

\begin{document}

\title{Fuzzy blobs in G309.2-0.6}

% Not quite in line with recommended aastex style
\author{
Joe Bob\altaffilmark{1},
Joe Smith\altaffilmark{2,3}
}

\affil{
\tsup{1}Smithsonian Astrophysical Observatory, 60 Garden Street MS-70, Cambridge, MA 02138, USA \\
\tsup{2}Department of Basket Weaving, Belafonte Hall, Univ. Calumny, Belabored, CA 00000, USA
}

\altaffiltext{3}{Bananagram crackers}

%\received{receipt date}
%\revised{\today}
%\accepted{acceptance date}


\begin{abstract}
Sketch of a writeup, and a lot of scribbled notes.
\end{abstract}

% (!) no longer used by AAS as of late February, 2016
\keywords{acceleration of particles ---
    ISM: supernova remnants ---
    ISM: individual objects (SNR G309.2-0.6) ---
    X-rays: ISM
%    ISM: magnetic fields ---
%    shock waves
}

\section{Introduction} \label{sec:intro}

Motivations:
(1) typing remnants by ejecta,
(2) resolve ejecta spatial structure (layering, mixing),
(3) resolve interaction with surrounding medium

Individual object study to build information for more ambitious global studies
of remnant properties and interaction with the ISM -- not undertaken here.
Just one piece of a puzzle.

Here we present the first ``modern'' high-resolution x-ray image of this
remnant (as far as the authors are aware) based on two archival XMM-Newton
observations.

Young, ejecta-dominated mixed-morphology remnant?
Seems like a contradiction in terms.

\subsection{SNR stuff}

MMSNRs (Rho and Petre, 1998).  Tilley+ (2006) - model anisotropic thermal
conduction in remnants in regular ISM vs. dense medium.
Is this a mixed morphology SNR?
Answer: not quite.  What other remnants look similar to this one?
(Note, GGM98 came out around the same year as rho/petre)
Rakowski did not label as such but noted ejecta dominated nature.

Possible for MM to have some mixture of ejecta -- how have definitions /
classifications evolved?

MC interaction: what are the most common proxies and
claims?
Kilpatrick, Bieging, Rieke (2016) give a very lucid ALMA study of CO(2-1)
broadening: http://iopscience.iop.org/article/10.3847/0004-637X/816/1/1/pdf

On barrel morphology and magnetic field orientation: Kesteven and Caswell (1987),
Gaensler (1998), West et al. (2016),

General question: how can we safely infer that the ISM is solar in abundance,
outside of our area?

A comparative study may be interesting.
G352.7-0.1 (Pannuti+ 2014, XMM/Chandra) exhibits similar features...
G298.6-0.0 (Bamba+ 2015, Suzaku)


\subsection{G309.2-0.6 morphology and environment}

% What are the most salient features of this remnant?
% Using SIMBAD coordinates, in agreement (1 arcsec off, doesn't matter) with Green's catalog...
G309.2-0.6 (RA 13 46 30, dec -62 54 00) was discovered in XXX radio
survey \citep{someone} as a circular shell with two bright lobes to the NE and
SW (hereafter referred to as ``ears'');
the angular extent is about $15\arcmin$ by $12\arcmin$, and the radio flux at
$843 \unit{MHz}$ is $\abt 6 \unit{Jy}$ for a surface brightness of
$5.4\times10^{-21} \unit{W \; m^{-2} \; Hz^{-1} \; sr^{-1}}$
(CITATION... Kesteven and Caswell 1987? what survey to use.)
% TODO (I think this data is from MOST survey)
The X-ray remnant sits within the radio shell and is brightest towards the
north (Figure \ref{fig:snr}).
A faint X-ray arc on the remnant's northeast limb coincides with the radio
shell's limb, interior to the bright SE ear; the arc is most visible in our
$0.8$--$1.4 \unit{keV}$ band image.
The X-ray emission appears anti-correlated with radio emission.
Both radio ears are X-ray dark.
In the circular shell, the SE limb outshines the NE limb; the opposite is true
in X-rays.  %% TODO show this with spectra

% More about this remnant
The remnant distance is estimated to be within $5$--$14 \unit{kpc}$ by
\citet{gaensler98} (hereafter, \citet{GGM98}) based on HI absorption features
out to $v_{\mt{LSR}} \sim -50 \unit{km/s}$ at the galactic rotation curve's
tangent point, and a lack of absorption at $v_{\mt{LSR}} \sim +40 \unit{km/s}$.
\citet{GGM98} state the lower distance bound as $5.4 \pm 1.6 \unit{kpc}$, which
is more strongly established than the upper bound due to relatively clear
absorption at negative velocities.


and \citet{rakowski01}
(hereafter, \citet{RHS01}) present comprehensive radio (ATCA) and X-ray (ASCA)
studies respectively.
the most HI absorption.
Distance 5-10kpc or whatever, compare to x-ray brightness.

% marco09 == 2009AdSpR..44..348M
% torrejon13 == 2013ApJ...765...13T

% Introduce bright source
A bright ROSAT source (1WGA J1346.5--6255) within G309.2-0.6 is the foreground
Be star HD 119682.  Previous studies, including \citet{GGM98} and
\citet{RHS01}, sought a potential connection between the source and the
remnant as the X-ray morphology is similar to the jet-driven remnant W50.
However, Chandra and optical astrometry, hard thermal X-ray spectrum fits, and
proper motion established that HD 119682 is most likely a member of the open
cluster NGC 5281 at a distance $\abt 1.4 \unit{kpc}$
\citep{rakowski06, safiharb07}.
Three distinct studies using XMM, Chandra ACIS-S imaging, and Chandra HETGS
fit the X-ray spectrum of HD 119682 and agreed on an absorption column
$\nH \sim 0.2 \times 10^{22} \unit{cm^{-2}}$, consistent with the inferred
distance to HD 119682 \citep{rakowski06, safiharb07, torrejon13}.

% I don't see that stellar classification work (e.g., Marco 2009)
% necessarily says anything about distance to the star.
% But it provides circumstantial evidence if such blue stragglers are
% preferentially associated with globular clusters -- I have no idea if that's true.

% Absorption -> distance provides a strong constraint.
% absorping column through the bright -60km/s feature
% (Cen/Scutum/something-or-other milky way arm) is probably about 1e22.
% given that nH calculators are spitting out absorptions of ~ 1-2e22.

% I don't understand how stellar distances are calculated
% (my best guess right now is that, e.g., Kharchenko+ 2005 is placing
% cluster stars on an isochrone after making an extinction correction,
% then converting apparent to absolute magnitudes -> distance modulus
% which is basically distance.
% distance to cluster: kharchenko et al. 2005A&A...438.1163K
% they get distance = 1.108 kpc, i don't know how?!
% Kharchenko+ use E(B-V) = 0.27 mag. vs. Lotkin+'s value 0.225 mag.
% partial answer -- interstellar extinction and isochrones...

% BUT, that's OK because absorption and the converging stellar classification
% as a gamma cas analog / blue straggler, strongly favor that this is a
% foreground star.

% NON-condensed version
%\citet{rakowski06} state O9e stellar classification from
%    Magellan H-alpha (<$0.75\arcsec$), Chandra ACIS-S; Chandra/XMM CCD spectrum fits
%    Sanner 2001:
%        cluster distance 1.58 +/- 0.15 kpc,
%        HD 119682 membership probability ~ 0.78 from proper motion
%    Levenhagen and Leister 2006:
%        spectroscopic parameters of HD 119682
%
%    Rakowski+ state O9.7 classification based on T 31910 +/- 550 K from
%        Levenhagen/Leister and "updated determinations of O star effective
%        temperatures that account for line blanketing (Martins+ 2002)"
%
%    Basically - positional overlap -- ruling out neighboring stars to 0.006 pc
%    (1240 AU)... assuming bright enough i guess (magnitude limited?... i don't
%    know anything about stars yet).  Better than 1 arcsec agreement.
%
%    How did people determine luminosity??????
%    Doesn't that assume a distance?...
%    - what is an isochrone
%    - what "are" blue stragglers
%    - what is a bolometric luminosity
%    - how do you distinguish late O vs. early B star, does it even matter
%    - how did levenhagen/leister deduce a mass?... stellar evolutionary tracks...
%
%\citet{safi-harb07} state Be classification...


% TODO: check out ATCA J134649--625235, mentioned in Gaensler

% Nearby Gum region, possible interaction...

Plotted with 0.843 GHz radio contours from the Molonglo Observatory Synthesis
Telescope (MOST) supernova remnant catalogue, with resolution $\abt 43\arcsec$
and sensitivity $2$ mJy/beam \citep{whiteoak1996}.

Let's do the usual review of observations.

Green (1974, Fig. 12) -- really fuzzy blob like thing, possibly corresponds to
the two bright lobes.  In the paper, labeled as number 22. This paper also
lists three accompanying references concerning this remnant (numbered 8,9,16).
    8 = Clark, Caswell, Green (1973; Nature 246, 28)
    9 = Clark, Caswell, Green (1974; Austr. J. Phys. in prep.)
    16 = Green (1972; PhD thesis, Sydney Univ., unpublished)
The table lists alternate name RCW80 but may be due to confusion with nearby HII
region -- clarify this w/ more recent papers.

Green's catalog: http://www.mrao.cam.ac.uk/surveys/snrs/snrs.G309.2-0.6.html


There is ALSO a recent study of the foreground OB association kinematics:
2009MNRAS.400..518M

ASCA data (1999 April) were presented by Rakowski+
favored high kT, Ar/Ne/Mg about solar relative to S/Si, far-subsolar Fe/Ca
(I would argue that ASCA, like XMM, doesn't have the SNR to probe this
well -- but would need to show with simulation / counting statistics)

Environs: Karr, Manoj, Ohashi (2009) -- study of Gum 48d, evolved HII region to
north.  H-alpha, GLIMPSE 8 micron (PAH), MIPSGAL 24 micron (dust grains)
http://iopscience.iop.org/article/10.1088/0004-637X/697/1/133/pdf

Note multiple radio data sets:
MOST 843 MHz
ATCA 1344 MHz, 1420 MHz (HI); 20 arcsec resolution (best available that I know of)
Southern Gal. Plane Survey 1420 MHz beam 1 arcmin. (1.5x larger than MOST)
Parkes 2417 MHz survey data?..
Spitzer data: http://sha.ipac.caltech.edu/applications/Spitzer/SHA/

Just by eye we are shooting along Carina-Sagittarius arm,
based on Fig. 3 of Momany et al. (2008 A\&A), which in turn is based on COBE
dust data from Fig. 19 of Drimmel and Spergel (2001, ApJ 556 181)

A useful tactic gleaned from Dubner+ 2013 (http://arxiv.org/pdf/1305.1275v1.pdf): see if
we can get an nH map for this part of the sky.
It will be pretty low resolution, but that's OK.

%
%Safi-Harb report:
%* vnei fit with supersolar Si/S, subsolar O/Ne/Mg/Ca/Fe.
%  kT = 2 +/- 0.6 keV,
%  nH = 0.65 (+0.45/-0.25) x 10^22 cm^{-2}
%
%  This is extremely odd to me -- does not match my results at ALL.
%
%  Suggest mismatch with Rakowski et al. due to confusion w/ bright star
%  main effect is to add a fair amount of broad soft emission
%
%* multiple arguments for distance to HD119682
%  spectrum fit -> nH ~ 0.2 x 10^22 cm^{-2}
%  based on both Chandra and XMM-Newton fits
%
%Rakowski 2001
%    kT ~ 2 keV,
%    nH = 2 x 10^22
%    Tau ~ 4.6e3 cm^{-3} year = 1.45e11 cm^{-3} s
%
%Safi-Harb don't discuss how background modeling is done, nor instrumental lines
%etc -- which I imagine would affect results...
%Rakowski+ does thoroughly explain their model for modeling out point source
%contamination.
%
%Safi-Harb region is basically the "north clump"

%\begin{figure*}[]
%    \plotone{fig_spec_01.pdf} \\
%    \plotone{fig_spec_16.pdf}
%    \caption{Caption goes here}
%    \label{fig:spec}
%\end{figure*}


\section{Observations and Data Reduction} \label{sec:obs}

%% High-level summary of our usable data

XMM-Newton's European Photon Imaging Camera (EPIC) observed G309.2-0.6 for a
total of $97.73 \unit{ks}$ in two pointings on 2001 August 28 and 2009 March
6--7.  The first pointing (obsid 0087940201; PI John P. Hughes) was $40.46
\unit{ks}$ with MOS1/2 in Full Frame mode, PN in Extended Full Frame mode, and
XMM's ``thick'' optical filter, as part of a study of ejecta-dominated galactic
remnants.
The second pointing (obsid 0551000201; PI Christian Motch) was $57.27
\unit{ks}$  with MOS1/2 in Full Frame mode, PN in Large Window mode, and XMM's
``medium'' optical filter.  The proposal targeted the X-ray bright foreground
Be star HD 119682, but as a byproduct captured the remnant G309.2-0.6.  The NE
edge of the remnant is cut-off by the disabled MOS1 CCD3 and the PN Large
Window mode boundary, but a majority of the remnant's area and its brightest
emission are captured on the central CCDs of all three detectors.

\begin{table*}
    \centering
    \caption{XMM Observations of G309.2-0.6}
    %% TODO use AASTEX table environment -- allow column hiding etc...
    \begin{tabular}{@{}lrrrrrrr@{}}
        \toprule
        Obs. ID & Dur. & Good dur. & Obs. date & XMM rev. & Pos. ang. & Filter & PI \\
                & (ks)     & (ks)                &      &          & (deg.)    &        &    \\
        \midrule
        0087940201 & $40.46$ & $\abt25$ & 2001 August 28 & 315 & 311.... & Thick & J. P. Hughes \\
        0551000201 & $57.27$ & $\abt15$ & 2009 March 6--7 & 1692 & 138.5... & Medium & C. Motch \\
        \bottomrule
    \end{tabular}
    %% TODO this is kinda a bad comment -- clean this table, if it is used in
    %% final product
    \tablecomments{Good duration is the sum of good time intervals after
    standard flare filtering; see text}
\end{table*}

%% TODO I have not thought at all about the effects of optical loading
%% on these observations...

We reprocess the data with XMM Science Analysis System (SAS) v15.0.0, released
on 2016 February 4, using standard tasks: cifbuild, emchain, epchain,
including an execution of epchain for out-of-time events.

Point source removal -- in particular, the excision region for the bright
foreground star HD 119682 is \textbf{XXX}
star

After SAS/ESAS processing as detailed below, we obtain X and Y good times
(Table 1) with a total of XXX counts and XXX counts

The data are ??? affected by soft proton flares.

Two CCDs in the Motch dataset were unusable -- MOS1 CCD6 after the ~2005

%% Omission of 0551000201 PN
Because no corner counts were recorded by the PN in 0551000201,
and the SP contamination is even more severe than for MOS1/MOS2 in 0551000201
(our basic filtering criterion yielded X amount of good time, interspersed with
x)
we excluded that exposure from our analysis.  This is a $\abt 10\%$ loss of
exposure time.

\subsection{Data filtering}

%% ESAS -- flare filtering, SAS flagging, QPB generation
We use the Extended SAS software (Kuntz and Snowden, 2008)
to prepare the data, filter out soft proton flare intervals (as identified by X
criterion by SAS task?),
apply SAS event criterion flags (PATTERN, FLAG)

\subsection{Instrumental Background}

(briefly explain FWC data and XMM instrumental lines in a short paragraph)

%% ESAS
We use ESAS tasks ? and ? to model the MOS and PN quiescent particle background
(QPB).
The QPB represents continuum detector emission from XMM FWC closed data,
specifically selected where the FWC corner spectra have spectral hardness
similar to that of the current observation (Kuntz and Snowden, 2008, Sec. ??).
In the regions where fluorescent instrumental lines are strong, ESAS
interpolates the continuum.  We subtract the QPB directly from observation
spectra.

%% FWC instrumental line fits
To model instrumental lines, we fit FWC spectra with zero-width, fixed-energy
gaussians and assume that the relative line strengths (i.e., line ratios) are
the same between FWC and observation spectra.
For MOS exposures, we fit FWC spectra to a broken power-law continuum with two
Gaussian lines at $1.49$ (Al) and $1.75 \unit{keV}$ (Si).
For PN, we fit seven Gaussian lines at $1.49$ (Al), $4.54$ (Ti), $5.44$ (Cr),
$7.49$ (Ni), $8.05$ (Cu), $8.62$ (Zn), and $8.90 \unit{keV}$ (Cu K-beta).
The PN Ti and Cr lines are not obvious in observation spectra, but
appear clearly in FWC spectra.
To fit instrumental lines in observation spectra, we import and fix line
normalizations from FWC fits, then vary a constant prefactor for all
instrumental lines in a given exposure.

\subsection{Astrophysical Background}

Elemental abundances by Wilms et al. (2000)
and the "tbabs" model in XSPEC (Arnaud 1996).

Will be useful to provide physical units for background emission, and compare
to other measurements.

\paragraph{Soft proton contamination}

Loads of it, everywhere.
ESAS approach: qpb includes only highly energetic CR penetration.
http://heasarc.gsfc.nasa.gov/docs/xmm/esas/cookbook/xmm-esas.html#foot986
the "contamination" essentially captures slowly varying part of SP background.

\paragraph{Local bubble and galactic halo emission}

Hard to say what variability is like, I think, given sparse data

* Henley and Shelton (2013), XMM-Newton observations of galactic halo emission.
  over 110 sightlines
    $T ~ 2.2 x 10^6 K$
    surface brightness $\abt 0.5-7 \times 10^{-12} \unit{erg cm^{-2} s^{-1} deg^{-2}}$
* Henley and Shelton (2015), XMM and Suzaku observations of shadowing clouds
  to separate foreground (local) and background (halo) emission.

Back of the envelope calculation for our data:
background is $\abt0.1 \unit{ct/s}$ at $1 \unit{keV}$, in source region.
Effective area of MOS is $\abt400 \unit{cm^2}$...
\[
    0.1/400 ct/(s keV cm^2) ~ 2.5e-3 erg / (s erg cm^2)
\]
Angular size of sky region is $\abt 10\arcmin \times 10\arcmin \sim 0.027 \unit{deg^2}$, so we get:
\[
    1.6e-11 erg cm^{-2} s^{-1} deg^{-2}
\]
Given that this background estimate is modeled as a blend of:
local bubble + absorbed hot galactic halo emission + extragalactic power law
(we ignore the "cool" component of galactic halo, likely absorbed since
this remnant is right on galactic plane)
this seems about reasonable.  We should fall right in the range of expected
surface brightness.


Over what lengthscales does the background vary?  Or, how good is our
assumption of a uniform background, in the XMM-Newton FOV?

NOTE: there are 28 clustered observations well out of the galactic plane
that they analyze (sightlines 103.1 to 103.27; 103.8 is two observations)
SZE SurF project to complement SPT,APEX,ACT study.

    Spread in halo T is $\abt10-20\%$.  Column density variation is around 10\%.
    Angular separation between pointings is of order 0.5 degrees $\sim$ 30 arcmin..

    For this study I'm interested in (concerned with?)
    angular variations on slightly shorter scales, of ~10-20 arcmin.  I.e.,
    within a half-degree.

    But I think we are close enough -- "only" factor 2x smaller -- that this
    data might constitute decent probe!



Please also note dispute w.r.t. local bubble: Welsh and Shelton (2009,
Astrophys. and Space Sci.).

Local leo cold cloud -> constraints on local bubble emission
Snowden et al., 2015ApJ...806..119S

Some other interesting reading:
Henley, Shelton, Kwak, Hill, Mac Low (2015) 2015ApJ...800..102H
- modeling gas flow from disk to halo
Peters et al. (2015) 2015ApJ...813L..27P
- more complex modeling works out; plasma for halo emission can be sourced from
  stellar feedback

\paragraph{Solar wind charge exchange}

Solar wind charge exchange -- based on short time cut (give plots in
supplemental material) and analysis by Carter et al. (2008, 2011),
data from 0087940201 are likely unaffected.

\paragraph{Extragalactic background}

ESAS (Snowden/Kuntz) recommends alpha = 1.46 but gives no source.

Chen, Fabien, Gendreau (1997)
  ASCA/ROSAT observations of QSF3 field (RA ~ 55 deg., del ~ -44. deg.; falls
  between LMC and fornax cluster, well out of galactic plane.

  Result: photon index ~ 1.40 to 1.50, depending on how you fit it.
  Single power law fit:
    ROSAT(1-2keV) + ASCA(1-7keV) -> gamma = 1.48 +/- 0.07
    ROSAT(1-2keV) + ASCA(1-3keV) -> gamma = 1.39 +/- 0.11
  Powerlaw + raymond-smith plasma model,
    ROSAT(0.5-2keV) + ASCA(0.4-7keV) -> gamma = 1.46 +/- 0.06

Similar analysis:
Kushino et al. 2002 (PASJ), another ASCA study.

Kuntz, Snowden (200)
    ROSAT - need to skim this

Hickox, Markevitch (2006ApJ...645...95H)
    CDF-N and CDF-S (total 3 Megasec)
    [interestingly, enabled by stowed ACIS calibration observations]
    Required careful background subtraction!
      For such deep Chandra observations, unresolved sky bkg is faint compared
      to quiescent detector background.

\section{Integrated and spatially resolved spectral fits} \label{sec:spec}

Quick discussion of vpshock, vsedov fit results.
Discuss assumptions and arguments laid out by Borkowski, Lyerly, Reynolds
(2001, ApJ 548:820).
For a young-ish shock, ejecta-dominated, hot, we do expect vnei to be a good
approximation (immediate post-shock electron temperature ~ 1/2000th proton
temperature, Tau initial -> 0).
Therefore forget overly detailed modeling.

Question:
what happens if I assumed fixed (well-mixed) plasma state and attempt to infer
absorption variation?
What happens if I fix absorption and attempt to probe plasma state?

Be sure to give \textbf{qualitative} explanation of why we do not obtain high
ionization timescales -- sure, XSPEC may favor low Tau, but must what portion
of our spectrum disfavors high ionization age. (provides a stable footing from
which to make further predictions, or give a clear way to falsify our claims)

\section{Morphology}

This remnant is relatively understudied, so no previous work has actually
probed the remnant's X-ray morphology with the available archival XMM data.

Rakowski et al. found an unresolved blob blended with the bright O star.

We resolve the bulk of SNR emission to a bright blob towards the north of the
remnant (assuming that the remnant may be delineated by the MOST contours),
part of a slightly dimmer ridge arcing towards the northwest.
The south is relatively dark.

\section{Spatially resolved spectral fits}

Describe individual fits, annulus fits, etc.

We find X regions to be consistent with background (no obvious emission)

Note: for now some numbers are entered in by hand...

Main observation: nH values are generally quite high (2-3e22).

If nH is low, we can still get reasonable kT and Si/S abundances
by driving down O,Ne,Fe emission. (based on north clump fits)
IF we force solar O/Ne/Fe, lower nH (below 2) quickly starts to look
unphysical.

north clump fit is ok not amazing.
SE dark gives OK fit with both Si/S solar...

Fits with: PN power law frozen, generally Si or both Si/S thawed


Figure 6 of http://arxiv.org/pdf/1412.1169v1.pdf (Hughes white paper)
where does this come from? fit of lines with approaching/receding redshifts...
strongest signal expect in center of remnant.

----> maybe could explain our spectra

Lack of H-alpha emission from Gaensler (should verify yourself).
Balmer shocks taken as evidence for progenitor NOT photo-ionizing too much
stuff.  So Balmer shocks at small radii can be taken as evidence that the
progenitor cannot have ionized "too much" stuff.
But, the absence of Balmer shocks doesn't tell us as much.
The shock could be weak, the density could be low, the material could be fully
ionized.



% TODO TODO full remnant fit could use an update, and systematic script to dump fits
\begin{table*}
    \centering
    \caption{G309.2-0.6 -- sub-source region fits}
    \begin{tabular}{@{}lrrrrrr@{}}
        \toprule
        Region & $n_\mathrm{H}$             & $kT$  & $\tau$                & Si  & S   & $\chi^2_{\mathrm{red}} (\mathrm{dof}$) \\
               & ($10^{22} \unit{cm^{-2}}$) & (keV) & ($\unit{s\;cm^{-3}}$) & (-) & (-) &  \\
        \midrule
        full remnant  & 2.19 & 1.56 & 2.55e+10 &  4.19 & 3.73 & 1.201 (3768) \\
        \midrule
        north clump & 2.92 & 0.81 & 7.28e+10 & 4.78 & 4.62 & 1.298 (692) \\  % File src_north_clump_row.tex
        E lobe & 3.67 & 0.80 & 6.01e+10 & 3.37 & 2.43 & 0.958 (207) \\  % File src_E_lobe_row.tex
        SW lobe & 1.91 & 2.41 & 1.35e+10 & 3.91 & 4.80 & 1.352 (438) \\  % File src_SW_lobe_row.tex
        SE dark & 2.49 & 0.67 & 9.24e+10 & 2.82 & 2.33 & 0.988 (222) \\  % File src_SE_dark_row.tex
        \midrule
        ridge & 2.89 & 0.62 & 1.30e+11 & 2.28 & 2.06 & 1.416 (225) \\  % File src_ridge_row.tex
        SE dark ridge & 4.56 & 0.43 & 7.64e+10 & 1(fr) & 1(fr) & 1.159 (104) \\  % File src_SE_ridge_dark_row.tex
        \bottomrule
    \end{tabular}
\end{table*}

\begin{table*}
    \centering
    \caption{G309.2-0.6 -- sub-source region fits, $n_H=2.5$ fixed}
    \begin{tabular}{@{}lrrrrrr@{}}
        \toprule
        Region & $n_\mathrm{H}$             & $kT$  & $\tau$                & Si  & S   & $\chi^2_{\mathrm{red}} (\mathrm{dof}$) \\
               & ($10^{22} \unit{cm^{-2}}$) & (keV) & ($\unit{s\;cm^{-3}}$) & (-) & (-) &  \\
        \midrule
        north clump & 2.50 & 1.07 & 4.79e+10 & 5.73 & 5.22 & 1.339 (693) \\  % File src_north_clump_nH-2.5_row.tex
        E lobe & 2.50 & 1.74 & 2.71e+10 & 5.47 & 3.69 & 1.016 (208) \\  % File src_E_lobe_nH-2.5_row.tex
        SW lobe & 2.50 & 1.27 & 1.37e+10 & 3.46 & 5.78 & 1.429 (439) \\  % File src_SW_lobe_nH-2.5_row.tex
        SE dark & 2.50 & 0.68 & 8.80e+10 & 2.79 & 2.30 & 0.983 (223) \\  % File src_SE_dark_nH-2.5_row.tex
        \midrule
        ridge & 2.50 & 0.81 & 8.69e+10 & 2.63 & 2.10 & 1.467 (226) \\  % File src_ridge_nH-2.5_row.tex
        SE dark ridge & 2.50 & 0.78 & 2.51e+11 & 1(fr) & 1(fr) & 1.232 (105) \\  % File src_SE_ridge_dark_nH-2.5_row.tex
        \bottomrule
    \end{tabular}
\end{table*}

\begin{table*}
    \centering
    \caption{G309.2-0.6 -- annulus fits}
    \begin{tabular}{@{}lrrrrrr@{}}
        \toprule
        Region & $n_\mathrm{H}$             & $kT$  & $\tau$                & Si  & S   & $\chi^2_{\mathrm{red}} (\mathrm{dof}$) \\
               & ($10^{22} \unit{cm^{-2}}$) & (keV) & ($\unit{s\;cm^{-3}}$) & (-) & (-) &  \\
        \midrule
        $0$--$100\arcsec$ & 1.59 & 9.52 & 1.68e+10 & 2.34 & 2.01 & 1.893 (356) \\  % File ann_000_100_row.tex
        $100$--$200\arcsec$ & 2.17 & 1.68 & 2.24e+10 & 5.96 & 5.56 & 1.189 (559) \\  % File ann_100_200_row.tex
        $200$--$300\arcsec$ & 2.86 & 0.82 & 7.33e+10 & 3.31 & 3.21 & 1.268 (792) \\  % File ann_200_300_row.tex
        $300$--$400\arcsec$ & 2.59 & 0.73 & 1.03e+11 & 4.01 & 3.76 & 1.276 (799) \\  % File ann_300_400_row.tex
        $400$--$500\arcsec$ & 4.77 & 0.40 & 8.90e+10 & 4.34 & 10.99 & 1.186 (822) \\  % File ann_400_500_row.tex
        \midrule
        $0$--$100\arcsec$ (ONeMg) & 1.59 & 6.22 & 1.72e+10 & 1.97 & 1.42 & 1.543 (353) \\  % File ann_000_100_ONeMg_row.tex
        \bottomrule
    \end{tabular}
\end{table*}

\begin{table*}
    \centering
    \caption{G309.2-0.6 -- annulus fits, $n_H=2.5$ fixed}
    \begin{tabular}{@{}lrrrrrr@{}}
        \toprule
        Region & $n_\mathrm{H}$             & $kT$  & $\tau$                & Si  & S   & $\chi^2_{\mathrm{red}} (\mathrm{dof}$) \\
               & ($10^{22} \unit{cm^{-2}}$) & (keV) & ($\unit{s\;cm^{-3}}$) & (-) & (-) &  \\
        \midrule
        $0$--$100\arcsec$ & 2.50 & 2.33 & 1.58e+10 & 1.74 & 1.27 & 2.756 (357) \\  % File ann_000_100_nH-2.5_row.tex
        $100$--$200\arcsec$ & 2.50 & 1.22 & 2.72e+10 & 5.29 & 5.04 & 1.210 (560) \\  % File ann_100_200_nH-2.5_row.tex
        $200$--$300\arcsec$ & 2.50 & 1.07 & 4.51e+10 & 3.83 & 3.56 & 1.291 (793) \\  % File ann_200_300_nH-2.5_row.tex
        $300$--$400\arcsec$ & 2.50 & 0.78 & 9.05e+10 & 4.11 & 3.74 & 1.275 (800) \\  % File ann_300_400_nH-2.5_row.tex
        $400$--$500\arcsec$ & 2.50 & 0.99 & 3.99e+10 & 8.57 & 10.19 & 1.188 (823) \\  % File ann_400_500_nH-2.5_row.tex
        \bottomrule
    \end{tabular}
\end{table*}

Fits to regions, systematically varying nH.

\begin{table*}
    \centering
    \caption{sub-region fits, varying $n_H$}
    \begin{tabular}{@{}lrrrrrr@{}}
        \toprule
        Region & $n_\mathrm{H}$             & $kT$  & $\tau$                & Si  & S   & $\chi^2_{\mathrm{red}} (\mathrm{dof}$) \\
               & ($10^{22} \unit{cm^{-2}}$) & (keV) & ($\unit{s\;cm^{-3}}$) & (-) & (-) &  \\
        \midrule
        north clump & 1.50 & 42.92 & 2.10e+10 & 5.20 & 9.15 & 2.085 (693) \\  % File src_north_clump_nH-1.5_row.tex
        north clump & 2.00 & 1.73 & 2.79e+10 & 7.51 & 6.71 & 1.543 (693) \\  % File src_north_clump_nH-2.0_row.tex
        north clump & 2.50 & 1.07 & 4.79e+10 & 5.73 & 5.22 & 1.339 (693) \\  % File src_north_clump_nH-2.5_row.tex
        north clump & 3.00 & 0.77 & 7.84e+10 & 4.64 & 4.56 & 1.297 (693) \\  % File src_north_clump_nH-3.0_row.tex
        \midrule
        E lobe & 1.50 & 54.24 & 2.20e+10 & 3.02 & 6.61 & 1.193 (208) \\  % File src_E_lobe_nH-1.5_row.tex
        E lobe & 2.00 & 3.06 & 2.08e+10 & 7.37 & 5.24 & 1.114 (208) \\  % File src_E_lobe_nH-2.0_row.tex
        E lobe & 2.50 & 1.74 & 2.71e+10 & 5.47 & 3.69 & 1.016 (208) \\  % File src_E_lobe_nH-2.5_row.tex
        E lobe & 3.00 & 1.19 & 3.49e+10 & 4.35 & 2.96 & 0.970 (208) \\  % File src_E_lobe_nH-3.0_row.tex
        \midrule
        SW lobe & 1.50 & 5.62 & 1.38e+10 & 4.58 & 5.43 & 1.410 (439) \\  % File src_SW_lobe_nH-1.5_row.tex
        SW lobe & 2.00 & 2.18 & 1.34e+10 & 3.78 & 4.76 & 1.352 (439) \\  % File src_SW_lobe_nH-2.0_row.tex
        SW lobe & 2.50 & 1.27 & 1.37e+10 & 3.46 & 5.78 & 1.429 (439) \\  % File src_SW_lobe_nH-2.5_row.tex
        SW lobe & 3.00 & 0.91 & 1.51e+10 & 3.21 & 7.07 & 1.553 (439) \\  % File src_SW_lobe_nH-3.0_row.tex
        \midrule
        SE dark & 1.50 & 1.73 & 2.92e+10 & 4.97 & 3.22 & 1.146 (223) \\  % File src_SE_dark_nH-1.5_row.tex
        SE dark & 2.00 & 0.97 & 5.73e+10 & 3.51 & 2.45 & 1.019 (223) \\  % File src_SE_dark_nH-2.0_row.tex
        SE dark & 2.50 & 0.68 & 8.80e+10 & 2.79 & 2.30 & 0.983 (223) \\  % File src_SE_dark_nH-2.5_row.tex
        SE dark & 3.00 & 0.44 & 1.70e+11 & 2.83 & 3.54 & 1.002 (223) \\  % File src_SE_dark_nH-3.0_row.tex
        \midrule
        ridge & 1.50 & 2.39 & 2.55e+10 & 5.34 & 3.64 & 2.500 (226) \\  % File src_ridge_nH-1.5_row.tex
        ridge & 2.00 & 1.35 & 3.61e+10 & 3.41 & 2.44 & 1.789 (226) \\  % File src_ridge_nH-2.0_row.tex
        ridge & 2.50 & 0.81 & 8.69e+10 & 2.63 & 2.10 & 1.467 (226) \\  % File src_ridge_nH-2.5_row.tex
        ridge & 3.00 & 0.58 & 1.51e+11 & 2.21 & 2.08 & 1.413 (226) \\  % File src_ridge_nH-3.0_row.tex
        \midrule
        SE dark ridge & 1.50 & 1.73 & 1.21e+11 & 1.00 & 1.00 & 1.344 (105) \\  % File src_SE_ridge_dark_nH-1.5_row.tex
        SE dark ridge & 2.00 & 1.36 & 6.70e+10 & 1.00 & 1.00 & 1.284 (105) \\  % File src_SE_ridge_dark_nH-2.0_row.tex
        SE dark ridge & 2.50 & 0.78 & 2.51e+11 & 1.00 & 1.00 & 1.232 (105) \\  % File src_SE_ridge_dark_nH-2.5_row.tex
        SE dark ridge & 3.00 & 0.79 & 5.14e+10 & 1.00 & 1.00 & 1.196 (105) \\  % File src_SE_ridge_dark_nH-3.0_row.tex
        \bottomrule
    \end{tabular}
\end{table*}

\begin{table*}
    \centering
    \caption{annulus fits, varying $n_H$}
    \begin{tabular}{@{}lrrrrrr@{}}
        \toprule
        Region & $n_\mathrm{H}$             & $kT$  & $\tau$                & Si  & S   & $\chi^2_{\mathrm{red}} (\mathrm{dof}$) \\
               & ($10^{22} \unit{cm^{-2}}$) & (keV) & ($\unit{s\;cm^{-3}}$) & (-) & (-) &  \\
        \midrule
        000--100$\arcsec$ & 1.50 & 9.79 & 1.71e+10 & 2.50 & 2.19 & 1.917 (357) \\  % File ann_000_100_nH-1.5_row.tex
        000--100$\arcsec$ & 2.00 & 4.08 & 1.57e+10 & 2.04 & 1.56 & 2.144 (357) \\  % File ann_000_100_nH-2.0_row.tex
        000--100$\arcsec$ & 2.50 & 2.33 & 1.58e+10 & 1.74 & 1.27 & 2.756 (357) \\  % File ann_000_100_nH-2.5_row.tex
        000--100$\arcsec$ & 3.00 & 1.54 & 1.79e+10 & 1.54 & 1.12 & 3.407 (357) \\  % File ann_000_100_nH-3.0_row.tex
        \midrule
        100--200$\arcsec$ & 1.50 & 4.84 & 1.68e+10 & 8.49 & 8.54 & 1.373 (560) \\  % File ann_100_200_nH-1.5_row.tex
        100--200$\arcsec$ & 2.00 & 1.99 & 2.04e+10 & 6.51 & 6.10 & 1.199 (560) \\  % File ann_100_200_nH-2.0_row.tex
        100--200$\arcsec$ & 2.50 & 1.22 & 2.72e+10 & 5.29 & 5.04 & 1.210 (560) \\  % File ann_100_200_nH-2.5_row.tex
        100--200$\arcsec$ & 3.00 & 0.85 & 4.58e+10 & 4.48 & 4.40 & 1.307 (560) \\  % File ann_100_200_nH-3.0_row.tex
        \midrule
        200--300$\arcsec$ & 1.50 & 4.10 & 1.89e+10 & 7.00 & 6.63 & 1.777 (793) \\  % File ann_200_300_nH-1.5_row.tex
        200--300$\arcsec$ & 2.00 & 1.72 & 2.76e+10 & 4.97 & 4.49 & 1.432 (793) \\  % File ann_200_300_nH-2.0_row.tex
        200--300$\arcsec$ & 2.50 & 1.07 & 4.51e+10 & 3.83 & 3.56 & 1.291 (793) \\  % File ann_200_300_nH-2.5_row.tex
        200--300$\arcsec$ & 3.00 & 0.75 & 8.71e+10 & 3.17 & 3.16 & 1.268 (793) \\  % File ann_200_300_nH-3.0_row.tex
        \midrule
        300--400$\arcsec$ & 1.50 & 43.16 & 2.06e+10 & 3.24 & 5.12 & 1.396 (800) \\  % File ann_300_400_nH-1.5_row.tex
        300--400$\arcsec$ & 2.00 & 1.24 & 3.68e+10 & 5.05 & 4.31 & 1.295 (800) \\  % File ann_300_400_nH-2.0_row.tex
        300--400$\arcsec$ & 2.50 & 0.78 & 9.05e+10 & 4.11 & 3.74 & 1.275 (800) \\  % File ann_300_400_nH-2.5_row.tex
        300--400$\arcsec$ & 3.00 & 0.58 & 1.51e+11 & 3.59 & 3.73 & 1.282 (800) \\  % File ann_300_400_nH-3.0_row.tex
        \midrule
        400--500$\arcsec$ & 1.50 & 42.91 & 2.00e+10 & 5.80 & 13.09 & 1.187 (823) \\  % File ann_400_500_nH-1.5_row.tex
        400--500$\arcsec$ & 2.00 & 43.10 & 1.75e+10 & 4.70 & 12.63 & 1.188 (823) \\  % File ann_400_500_nH-2.0_row.tex
        400--500$\arcsec$ & 2.50 & 0.99 & 3.99e+10 & 8.57 & 10.19 & 1.188 (823) \\  % File ann_400_500_nH-2.5_row.tex
        400--500$\arcsec$ & 3.00 & 0.78 & 4.22e+10 & 6.78 & 9.34 & 1.187 (823) \\  % File ann_400_500_nH-3.0_row.tex
        \bottomrule
    \end{tabular}
\end{table*}




NOTE: SW lobe and ridge both show weird shifts in PN spectral lines (Si, S)


\acknowledgments

Blah blah blah.
X acknowledges support by contract ...
This research is based on observations obtained with XMM-Newton, an ESA science
mission with instruments and contributions directly funded by ESA Member States
and NASA.
This research has made use of NASA's Astrophysics Data System.
%This research made use of Astropy, a community-developed core Python package
%for Astronomy (Astropy Collaboration, 2013).
%This research also made use of APLpy, an open-source plotting package for
%Python hosted at \href{http://aplpy.github.com}{http://aplpy.github.com}.

{\it Facilities:} \facility{XMM (EPIC)}, \facility{Molonglo Observatory}

%\software{Astropy, ...}

% ==========
% References
% ==========
\bibliographystyle{apj}
\bibliography{refs-snr}

% ========
% Appendix
% ========
%\clearpage
%\newpage
%\appendix
%
%\setcounter{table}{0}
%\renewcommand{\thetable}{A\arabic{table}}
%\setcounter{figure}{0}
%\renewcommand{\thefigure}{A\arabic{figure}}

\listofchanges

\end{document}
