\documentclass[preprint2,tighten,trackchanges]{aastex6}
%\documentclass[iop, tighten, apj, numberedappendix]{emulateapj} % Still more compact than aastex

% Customize figures/sizing for 2 column vs. manuscript printout
%%\usepackage{etoolbox}
%%\newtoggle{manuscript}
%%\toggletrue{manuscript}  % Set TRUE if using manuscript / 1-col layout
%%%\togglefalse{manuscript}  % Set FALSE if using 2-col layout

\shorttitle{G309 (\today)}  % <~ 44 char
\shortauthors{Alpha, Beta (\today)}  % Max three
\slugcomment{Draft, \today}

% Packages and commands
\usepackage{amsmath}  % Included in aastex - but want to get eqref
\usepackage{booktabs}
\usepackage{hyperref}
\usepackage[normalem]{ulem}  % Just to get strikeout text

% My "standard" TeX aliases
\newcommand*{\mt}{\mathrm}
\newcommand*{\unit}[1]{\;\mt{#1}}  % vemod.net/typesetting-units-in-latex
\newcommand*{\abt}{\mathord{\sim}} % tex.stackexchange.com/q/55701
\newcommand*{\ptl}{\partial}
\newcommand*{\del}{\nabla}
\newcommand*\mean[1]{\bar{#1}}
\renewcommand{\vec}[1]{\mathbf{#1}}  % Bold vectors
\newcommand*{\tsup}{\textsuperscript}

% Paper-specific commands
\newcommand*{\nH}{N_{\mathrm{H}}}
\newcommand*{\nHUnits}{\times 10^{22} \unit{cm^{-2}}}
\newcommand*{\TauUnits}{\unit{s\;cm^{-3}}}
\newcommand*{\AV}{A_{\mathrm{V}}}
%\newcommand*{\kB}{k_{\mathrm{B}}}
\newcommand*{\kB}{k}  % vacillating on how to format Boltzmann's constant
\newcommand*{\EM}{\mathrm{EM}}  % Emission measure
\defcitealias{rakowski2001}{RHS01}
\defcitealias{gaensler1998-g309}{GGM98}

\begin{document}

\title{Spatially Resolved Ejecta in G309.2-0.6}

% Not quite in line with recommended aastex style
\author{
A. Alpha, B. Beta%\altaffilmark{1}
}

\affil{
%\tsup{1}
Smithsonian Astrophysical Observatory, 60 Garden Street MS-70, Cambridge, MA 02138, USA
}

%\received{receipt date}
%\revised{\today}
%\accepted{acceptance date}


\begin{abstract}
We image and fit spectra of ejecta-dominated remnant G309.2-0.6 from archival
XMM-Newton data.
\end{abstract}

% (!) no longer used by AAS as of late February, 2016
\keywords{ISM: supernova remnants ---
    ISM: individual objects (SNR G309.2-0.6) ---
    X-rays: ISM
}

% =============================================================================
% Introduction
% =============================================================================
\section{Introduction} \label{sec:intro}

Can we use integrated or spatially resolved X-ray spectra to type
ejecta-dominated or ED-ST transitioning supernova remnants (SNRs)?
In some cases, yes:
\begin{itemize}
    \item coarse abundance ratios of O, Fe, Si/S compared to model SN yields \citep{hughes1995}  % 1995ApJ...444L..81H
    \item high resolution line spectroscopy of O lines in 1E0102.2-7219 \citep{flanagan2004}  % 2004ApJ...605..230F
    \item Ni/Fe and Mn/Fe ratios in 3C397 \citep{yamaguchi2015}
    \item Fe K line centroid energies \citep{yamaguchi2014-iron, patnaude2015}
\end{itemize}
Most individual remnant studies attempt to guess out the progenitor SN type
from ejecta abundances and other pieces of information.
Alternatively, \citet{chevalier2005} attempts to type SNe from SNR CSM
emission.
Optical studies of CSM knots are possible (see references in
\citet{katsuda2015}).

Badenes+ 2007: Table 4, ionization timescales of Si/S vs. Fe in young Ia remnants.
Appendix is also sort of helpful.
I'd cite this on stratification... and discussion of Fe absence.

Some studies (i.e. those not emphasizing line centroids, widths, ratios) assume
that abundances inferred from NEI plasma fits are a reasonable probe of ejecta
composition.  Much depends on:
\begin{itemize}
    \item Ejecta structure - which elements are stratified or well mixed, and
        what conditions (ambient density, progenitor type, explosion mechanism)
        control ejecta structure.
        See: Ashall+ 2016 arxiv:1608.05244, stratification in SN1986G.
    \item ambient interstellar or circumstellar medium (ISM, CSM).
        was was the extent and structure of progenitor mass loss, if any?
        (continuous wind, pulses of material, ?)
        nearby molecular clouds, other remnants.
        Interesting recent paper on remnant deformation near GC.
        Yalinewich+ 2016 (arXiv:1608.05904).
    \item Remnant's own dynamics - evolutionary state (ejecta-dominated, Sedov;
        transitioning).
\end{itemize}

One might consider forward folding models and comparing with observed spectra
across the soft X-ray band.
This is basically the approach of \citet{hughes1995}, \citet{badenes2003},
\citet{rakowski2006-g337}.
Some models show observable differences in line ratios, elemental abundances
and the like.  Others are less clear.

We present a spatially resolved X-ray image and CCD-resolution spectra of
G309.2-0.6 based on two archived XMM-Newton observations totalling
$98 \unit{ks}$, with about half usable after flare removal.

We fit the CCD spectra of morphologically distinct regions of the remnant to
trace variation in ejecta abundances, plasma temperature, and ionization
states.

Individual object study to build information for more ambitious global studies
of remnant properties and interaction with the ISM -- not undertaken here.
Just one piece of a puzzle.

\subsection{Previous observations of G309.2-0.6}

\begin{figure*}[]
    \plotone{fig/fig_snr_xmm_most_invert.pdf}
    %\caption{Broadband 0.8--3.3 keV image of SNR G309.2-0.6.}
    % figcaption is an aastex alias -- cannot use with emulateapj
    \figcaption{Left: XMM broadband 0.8--3.3 keV image of SNR G309.2-0.6,
        smoothed and log scaled with $1.4 \unit{GHz}$ radio contours
        (0.01, 0.03, 0.1, 0.3 Jy).
        Right: $1.4 \unit{GHz}$ image, arcsinh scaled, from the Molonglo
        Observatory Synthesis Telescope (MOST) Supernova Remnant Catalogue
        \citep{whiteoak1996}.  Resolution: $\abt 43\arcsec$, sensitivity $2
        \unit{mJy/beam}$.}
    \label{fig:snr}
\end{figure*}

% Radio remnant observations and morphology
\objectname{SNR G309.2-0.6} (J2000 RA 13h46m30s, dec. $-62\arcdeg 54\arcmin 00\arcsec$) was
discovered in Molonglo $408 \unit{MHz}$ and Parkes $5000 \unit{MHz}$ surveys
\citep{day1969, clark1973, green1974, clark1975} and has since been observed
multiple times by southern radio telescopes \citep{caswell1981, kesteven1987,
whiteoak1996}.
% Day, Thomas, Goss (1969 Au. J. Phys. Astrophys. Suppl.)
% flagged G309.2-0.6 as a complex source, but did not identify it as a remnant
% or explore further.  Molongo/Parkes surveys may have occurred at the same
% time.  And, yes, it is confused (at the edge) with RCW 80.
\citet{gaensler1998-g309} (hereafter, \citetalias{gaensler1998-g309}) performed the most
comprehensive radio study to date with Australia Telescope Compact Array (ATCA)
$1.344 \unit{GHz}$ continuum and $1.420 \unit{GHz}$ HI observations at angular
resolution $24\arcsec$.
The radio remnant is a circular shell with two bright lobes to the NE and SW,
hereafter referred to as ``ears'' (Fig \ref{fig:snr}).
% Flux density -- omit, could incorporate later if relevant
% the flux density at $843 \unit{MHz}$ is $6 \unit{Jy}$ \citep{whiteoak1996}.
% Spectral index notes -- omit for now, could incorporate later if relevant
% Comment: spectral indices ~ 0-0.3 are more typical of pulsar powered remnants
% steeper (0.3,0.4-1) more typical for most SNRs.
%The radio spectral index is $0.53 \pm 0.09$ derived by \citet{gaensler1998-g309},
%excluding single-dish Parkes observations at 2700 and 5000 Mhz to avoid
%confusion of source flux with RCW80; a shallower index $0.36 \pm 0.11$ is
%obtained if single dish measurements (with authors' attempted corrections for
%source confusion) are included.
HI absorption favors remnant distance between $5$--$14 \unit{kpc}$ based on
absorption to the galactic rotation curve's tangent point,
$v_{\mt{LSR}} \sim -50 \unit{km/s}$, and the absence of absorption above
$v_{\mt{LSR}} \sim +40 \unit{km/s}$ \citepalias{gaensler1998-g309}.
% The galactic rotation curve is from \citet{fich1989} with galactic center
% distance $R_0 = 8.5 \unit{kpc}$ and local circular velocity
% $\Theta = 220 \unit{km/s}$.

% ATCA data -- cannot access at http://www.atnf.csiro.au/research/HI/sgps/queryForm.html
% SGPS survey -- taken 1998-2000.
% Emailed a feedback query on 2016 April 13... probably won't have much luck.

% X-ray remnant observation(s) and morphology
X-ray emission was discovered in an Advanced Satellite for Cosmology and
Astrophysics (ASCA) survey of small remnants by \citet{rakowski2001}
(hereafter, \citetalias{rakowski2001}).
The X-ray remnant sits within the radio shell and is brightest towards the
north (Figure \ref{fig:snr}).
A faint X-ray arc on the remnant's northeast limb coincides with the radio
shell's limb and is most apparent in our $0.8$--$1.4 \unit{keV}$ band image.

% Other wavelengths
The field of G309.2-0.6 has been surveyed in many wavelengths.
No obvious H$\alpha$, [S II], or [O III] filaments appear in 3x 400 s exposures
taken with the 0.6 m Curtis-Schmidt telescope at Cerro Tololo Inter-American
Observatory in 2001 January (PI: P. F. Winkler; observers: Gokas, Smith,
Winkler).
% I think it's this proposal (but not sure)
% NOAO proposal 2001A-0331
% http://adsabs.harvard.edu/abs/2001noao.prop..331W
% http://www.noao.edu/dir/q_rep/Q2_Apr_30/FY2001%20Q2%20Jan-Mar%20NOAO%20Quarterly%20Report.pdf
% Some H-alpha emission associated with foreground cluster;
% field is dominated by RCW 80 to northeast.
\footnote{\url{http://sites.middlebury.edu/snratlas/g309-2-0-6/}}
No $1720.5 \unit{MHz}$ OH maser emission is observed above $40 \unit{mJy}$
\citep{green1997}.
% Mopra CO survey - no public data yet, in works...
No obvious infrared emission appears in Spitzer
$3.6$--$8 \unit{{\mu}m}$ IRAC GLIMPSE or $24 \unit{{\mu}m}$ MIPSGAL mosaics
\citep{churchwell2009, carey2009}.
Dust emission at $100 \unit{{\mu}m}$ (IRAS/ISSA, COBE/DIRBE) shows a
gradient across the remnant, although association with G309 is unknown.
% Interpretation:
% J,H,Ks bands = 1.2, 1.6, 2.2 micron (2MASS) -- cool stars
% 3.6--8 micron (Spitzer) -- cool stars
% 12 micron --
% 22/24 micron (WISE, Spitzer) -- dust
% 70 micron -- more dust (useful for dust SED stuff)
% Melange of potentially useful stuff
The remnant is not detected in GeV by Fermi-LAT \citep{acero2016} or TeV by
HESS \citep{bochow2011}.
The absence of H$\alpha$ and infrared emission is consistent with previous
data reviewed by \citetalias{gaensler1998-g309}.


\subsection{Environment}

% HD 119682 - previous interest and subsequent disassociation
A bright ROSAT source (1WGA J1346.5--6255) within G309.2-0.6 is the foreground
Be star HD 119682.
The source's position within a bilobed radio remnant led
\citetalias{gaensler1998-g309} and \citetalias{rakowski2001} to suggest that
G309.2-0.6 might be a jet- or outflow-driven remnant, similar to the SS 443 /
W50 system.
% bilobed beats bilobate on Google by 278 to 66 (cf. Google ngrams)
But, HD 119682 is likely a member of the open cluster NGC 5281 at a distance
$\abt 1.4 \unit{kpc}$, based on X-ray and optical astrometry, cluster proper
motions, and X-ray spectrum absorption \citep{rakowski2006-star, safi-harb2007,
torrejon2013}.
% The X-ray fitted absorption of $\nH \sim 0.2 \times 10^{22} \unit{cm^{-2}}$
% is consistent with the inferred distance to HD 119682
% \citep{rakowski2006-star, safi-harb2007, torrejon2013}.

% Gum 48d / RCW 80
The H II region Gum 48d (RCW 80), north of G309.2-0.6, is well traced in HI,
polycyclic aromatic hydrocarbon (PAH), and warm dust emission
\citep{karr2009}.
The distance to Gum 48d is $\abt 3.5\unit{kpc}$ based on ionized gas emission
velocities measured from the central star system HR 5171 \citep{karr2009}.
Given this distance, and the absence of evidence for interaction with G309,
we assume that Gum 48d also lies in the foreground of G309.2-0.6.
We also ignore the north-south wisp of radio emission that visually joins the
radio remnant and H II region in Figure~\ref{fig:snr}.

% Centaurus CO cloud -- push this to discussion...
\citet{saito2001} imaged
an extensive molecular CO complex in ${}^{12}$CO, ${}^{13}$CO, and C${}^{18}$O
J=(1-0) transitions at $v_\mt{LSR} = -64$ to $-36 \unit{km/s}$.
\textbf{TODO} this would need some CO to ISM density conversion estimate so
that we could estimate a density gradient across the remnant, if it were
associated with this cloud (which we think it is not).
For this see Bolatto, Wolfire and Leroy (2013).
It is clear that remnants expanding into a density gradient need not look that
asymmetric \citep{hnatyk1999, williams2013}, so \textbf{this is speculative and may be
outright wrong}.

% Here's a lovely optical photograph, clearly showing the bright H II region and
% the bluer star cluster NGC 5281 to the south:
% https://it.wikipedia.org/wiki/File:RCW_80.jpg

% Mopra CO J(1-0) survey, resolution 30 arcsec and 0.1 km/s, would be huge!
% Looking at our crude rotation curve -- beyond tangent point, 100 km/s per 10kpc
% is the approximate slope; 0.1 km/s means we have a distance resolution of order
% 10 parsecs, comparable to or smaller than this remnant size.
% Compare, Dame survey has beamwidth ~ 1/8 deg. = 7.5 arcmin = 450 arcsec


% TODO: check out ATCA J134649--625235, mentioned in Gaensler

% Summary / wrap-up
Here, we extend the radio and X-ray studies of \citetalias{gaensler1998-g309} and
\citetalias{rakowski2001} with archived XMM-Newton observations that spatially
resolve the X-ray morphology of G309.2-0.6.
We confirm the previous


\begin{table*}
    \centering
    \caption{XMM Observations of G309.2-0.6\label{tab:obs}}
    \begin{tabular}{@{}lrrrlrlr@{}}
    \toprule
    Obs. ID & Dur. & $t_{\mt{\,live},\,\mt{MOS}}$ & $t_{\mt{\,live},\,\mt{PN}}$
        & Date & Rev. & Filter & PI \\
    \midrule
    0087940201 & $40.5$ & $25.3$ & $18.0$ & 2001 August 28 & 315 & Thick & Hughes \\
    0551000201 & $57.3$ & $21.9$ &  $9.0$ & 2009 March 6--7 & 1692 & Medium & Motch \\
    \bottomrule
\end{tabular}

    \tablecomments{Durations in kiloseconds (ks).
    Good duration $t_{\mt{live}}$ is central CCD live time after flare
    filtering; MOS value averages MOS1 and MOS2 times ($\sim 0.6 \unit{ks}$
    difference in both obsids).
    Rev. is XMM-Newton orbit (revolution) number.}
\end{table*}

% =============================================================================
% Observations + Reduction
% =============================================================================
\section{Observations and Data Reduction} \label{sec:obs}

XMM-Newton has observed G309.2-0.6 for $97.7 \unit{ks}$ in two pointings
(Table~\ref{tab:obs}).
%The 2009 observation targeted the Be star HD 119682, but also captured
%G309.2-0.6 on the MOS1/2 and PN detectors.
After background flare filtering, our good exposure times are $47 \unit{ks}$
for MOS1/2 and $27 \unit{ks}$ for PN.
% CCD exclusions
In obs. ID 0551000201, we further excluded two MOS CCDs and the entire PN
exposure from our spectral analysis.
MOS1 CCD6 was disabled by a presumed micrometeorite in 2005, and MOS2 CCD5
shows anomalously high soft X-ray background noise \citep[cf.][]{kuntz2008}.
The PN in Large Window mode does not collect corner counts, so we cannot
estimate the detector background consistently using ESAS tasks.
Omitting the $9 \unit{ks}$ PN exposure only sacrifices $\abt 10--20\%$ of our
available photons, as the exposure was heavily flare contaminated.

% More verbose version
%XMM-Newton has observed G309.2-0.6 for $97.73 \unit{ks}$ in two pointings.
%Obsid 0087940201 (PI J. P. Hughes) was $40.46 \unit{ks}$ on 2001 August 28
%with MOS1/2 in Full Frame mode, PN in Extended Full Frame mode, and XMM's
%``thick'' optical filter.
%Obsid 0551000201 (PI C. Motch) was $57.27 \unit{ks}$ on 2009 March 6--7
%with MOS1/2 in Full Frame mode, PN in Large Window mode, and XMM's ``medium''
%optical filter.
%The Motch pointing targeted the foreground Be star HD 119682, but also
%captured G309.2-0.6 on the MOS1/2 and PN detectors.

% TODO I have not thought at all about the effects of optical loading on these
% observations...

% TODO how to refer to each observation in a simple way?
% for now just using ``Hughes'' and ``Motch''
% Possibly, just refer to by year: 2001 and 2009.  Obsid is not that useful
% TBD: kind of inconsistent in text currently...

% Current good time values (2016 April 21. Somewhat old extraction, pipeline
% last re-run in Feb or Mar after SAS v15 release)
%   0087940201 mos1S001 livetime 25.016 ks, ontime 25.290 ks
%   0087940201 mos2S002 livetime 25.621 ks, ontime 25.891 ks
%   0087940201 pnS003   livetime 18.011 ks, ontime 20.979 ks
%   0551000201 mos1S001 livetime 21.621 ks, ontime 21.989 ks
%   0551000201 mos2S002 livetime 22.167 ks, ontime 22.545 ks
%   0551000201 pnS003   livetime  8.981 ks, ontime  9.720 ks

% Event lists, soft proton light curve filtering
We reduce the data with XMM's Science Analysis System (SAS) v15.0.0 and
Extended SAS (ESAS) v5.9 \citep{snowden2008, kuntz2008}.
\footnote{\url{http://heasarc.gsfc.nasa.gov/docs/xmm/esas/cookbook/xmm-esas.html}}
Strong soft proton flares are filtered by ESAS tasks \texttt{mos-filter} and
\texttt{pn-filter}, which fit a Gaussian to a histogram of time-binned count
rates for each exposure and discard time intervals where the count rate is
$1.5\sigma$ above the fitted Gaussian mean.  % TODO - sentence needs retooling
We further inspect the $2.5$--$12 \unit{keV}$ field-of-view light curves and
manually cut some brief good-time intervals surrounded by higher count rate
flares.
About half to two-thirds of each observation is lost to flares, with more
severe loss for PN exposures.
The resulting live times (good time intervals less CCD readout time) are
given in Table~\ref{tab:obs}.

% Point sources
We remove point sources to a limiting flux of
$10^{-14} \unit{erg\;cm^{-2}\;s^{-1}}$ using the ESAS task \texttt{cheese},
configured to use the SAS task \texttt{region} in
\texttt{radiusstyle='contour'} model.
We apply the point source detection to each exposure separately, then merge the
resulting lists.
\textbf{TODO: is this truly the limiting flux, given the high background?
What is the minimum flux for a point source to be detected 99\% of the time in
data drawn from our model?
False positives are OK, but want to check rate of false negatives at given
flux.}
\textbf{TODO: create images with and without pt source exclusions...}

% Accompanies spectrum extraction AND fitting text
\begin{figure*}[!ht]
    \epsscale{0.8}
    \plotone{fig/fig_snr_regs-fiveann_invert.pdf}
    \figcaption{Unsmoothed 0.8--3.3 keV image of SNR G309.2-0.6 with
    $100\arcsec$ wide annuli and background region;
    the magenta $400\arcsec$ circle demarcates the extraction region for
    integrated remnant fits.}
    \label{fig:regions}
\end{figure*}

\subsection{Spectrum Extraction and Modeling} \label{sec:spec-extract-fit}

% Spectrum and QPB creation
Spectra are created from ``good'' single and double events (i.e.,
\texttt{PATTERN} $\leq 12,4$ for MOS, PN respectively; \texttt{FLAG} $= 0$ for
both MOS and PN) using ESAS tasks \texttt{\{mos,pn\}-spectra}.
We fit all spectra in XSPEC 12.9.0d \citep{arnaud1996}
using T{\"u}bingen-Boulder ISM elemental abundances \citep{wilms2000}
and galactic absorption model \texttt{tbnew} v2.3.2
\footnote{\url{http://pulsar.sternwarte.uni-erlangen.de/wilms/research/tbabs/}},
which is the latest version of the gas absorption model of \citet{wilms2000}.
We hereafter refer to baseline ISM abundances as ``solar'', although they
deviate slightly from solar abundances, to simplify description.
\textbf{TODO: ``super solar-neighborhood-ISM abundances'' or
``sub MW-ISM abundances'' are unambiguous but hard to read.  Any better way?}

% QPB
The detector background is modeled by ESAS tasks \texttt{\{mos,pn\}\_back}
by (1) computing a quiescent spectrum from unexposed CCD corner counts,
(2) augmenting the quiescent spectrum with corner counts from public
observations with similar count rates and spectral hardness, and (3) scaling
the augmented quiescent spectrum shape to that expected for the source region
of interest, using the ratio of quiescent spectra across corner / source
regions from XMM's filter wheel closed (FWC) observations and assuming that the
ratio of spectra across a given MOS or PN chip is time invariant
\citep[Sec. 3.4]{kuntz2008}.
The detector background is directly subtracted from observation spectra.
\textbf{TODO: could this be shortened or omitted?}  % TODO

% Spectrum grouping, fitting, and modeling
We merged MOS1 and MOS2 spectra from each observation and binned MOS and PN
spectra to at least 1 count per bin.
All spectra were fit with XSPEC's \texttt{pgstat} statistic for
Poisson-distributed data with a Gaussian background (appropriate for the ESAS
detector background model); \texttt{pgstat}, like the Cash statistic
\texttt{cstat}, averts bias associated with $\chi^2$ fitting of binned Poisson
data \citep{humphrey2009}.

% Instrumental lines
We model instrumental lines as zero-width, fixed-energy Gaussians and assume
that relative line strengths (i.e., line ratios) in a given CCD region are the
same between FWC and observation data.
The FWC spectra are well fit by Gaussian lines on a broken power-law continuum.
For MOS exposures, we fit $1.49$ (Al) and $1.75 \unit{keV}$ (Si) lines.
For PN, we fit seven lines: $1.49$ (Al), $4.54$ (Ti), $5.44$ (Cr), $7.49$ (Ni),
$8.05$ (Cu), $8.62$ (Zn), and $8.90 \unit{keV}$ (Cu K$\beta$).
% The PN Ti and Cr lines are not noticeable in observation spectra, but are
% clearly visible in FWC spectra.
To fit observation spectra, we adopt line normalizations from FWC spectrum fits
and vary a single constant pre-factor for all instrumental lines in a given
instrument exposure.
% Some spectrum fits could be improved by allowing instrumental line widths and
% centroid energies to vary.
% But, $\Delta \chi$ residuals at instrumental lines are comparable to
% residuals throughout our spectrum ethe rest of our spectral fits, so we do not expect that
% imperfect line model residuals should significantly affect our fit results and
% interpretation.
%
% Wishy washy because we have not quantified or tested this impact...
% But we have no reason to expect major impact.
% Freeing line widths and energies, even within a small range, would add 4--12
% parameters for each exposure.
% if time allows, try this for one baseline spectrum, and show no substantial
% effect.

% Residual soft proton contamination, SWCX
Soft protons incident on EPIC detectors dominate the X-ray background
and are not removed by flare filtering due to their slow time variation.
We model residual soft protons by a power law that bypasses CCD response and
effective area functions, following ESAS procedure.
We neglect background solar wind charge exchange (SWCX) emission
\citep{snowden2004, carter2011};
neither observation shows obvious \ion{O}{7} or \ion{O}{8} lines, and any
residual contamination should be folded into our X-ray background model.
% Omitted -- this text on SWCX checks is not really needed
% Hughes' 2001 observation was not flagged by \citet{carter2011} as containing
% time-variable soft X-ray emission indicative of exospheric SWCX.
% % TODO what is the difference between these types of swcx?
% Modeled magnetospheric SWCX emission decreases by a factor of $10$ over the
% duration of Hughes' 2001 observation, based on XMM-Newton Guest Observer
% Facility (GOF) tool that combines a magnetosphere model \citep{spreiter1966} with
% Advanced Composition Explorer (ACE) solar wind data.
% \footnote{\url{https://heasarc.gsfc.nasa.gov/docs/xmm/scripts/xmm_trend.html}}
% We extracted 0087940201 MOS1 spectra at early and late times in the observation
% and, by eye, saw no obvious differences in soft X-ray emission.
% % TODO I will be re-doing this.
% For the 2009 observation, the XMM GOF magnetosphere model predicts weaker SWCX
% emission compared to the 2001 observation, so we simply neglect SWCX in both
% observations.
% Any unmodeled SWCX should be folded into our X-ray background model, albeit
% imperfectly.
% % NOTE: see my notes from 2016 Jan 13 on spectrum cuts to evaluate SWCX
% % 0551000201 was observed March 2009
% %   Carter/Sembay analysis performed August 2009, so 0551000201 was likely
% %   still in its proprietary period
% % 0087940201 is not listed in Carter/Sembay, but meets basic selection criteria
% % (namely, MOS1/MOS2 operating in full frame mode

% =============================================================================
% Spectrum fits
% =============================================================================

\section{Spatially resolved spectral fits} \label{sec:spec}

\subsection{Integrated spectrum} \label{sec:src-bkg}

% Describe integrated source fit; jump straight to fit and basic numbers
We first jointly fit integrated remnant (0--400\arcsec) and background
(510--700\arcsec) spectra (Figure~\ref{fig:regions}).
The remnant is well fit by an absorbed single-temperature non-equilibrium
ionization (NEI) plasma (XSPEC model \texttt{tbabs\_new * vnei}) with
increased Si and S abundances relative to local galactic ISM
(Figure~\ref{fig:src-bkg-fits}, Table~\ref{tab:src-fits}), as fits with solar
(ISM) abundances shows clear residual \ion{Si}{13} and \ion{S}{15} He-$\alpha$
emission.
We also consider variation in O, Ne, Mg, Ar, Ca, and Fe abundances.
The \ion{Mg}{11} He-$\alpha$ line is evident but does not clearly require
super-solar Mg abundance.
Small bumps at $3.1 \unit{keV}$ and $3.9 \unit{keV}$ suggest \ion{Ar}{17} and
\ion{Ca}{19} He-$\alpha$; if attributable to line emission, these features
would require highly super-solar abundances at our best fit temperature and
ionization timescale.

\begin{figure*}[!ht]
    \plotone{fig/fig_src_bkg_allexps-src-only.pdf}
    %\plotone{fig/fig_src_bkg_0087940201-mos.pdf}
    %\plotone{fig/fig_src_bkg_0087940201-mos-delchi.pdf}
    \figcaption{Integrated source spectra from 0087940201 MOS with best-fit
    model comprising: cosmic X-ray background (CXRB), soft proton background (SP),
    instrumental lines, and NEI plasma for the remnant.
    All plotted spectra are binned towards minimum significance $5\sigma$ with
    maximum 50 channels per bin, though fits only bin to $\geq 1 \unit{count}$
    per bin as discussed in Section~\ref{sec:spec-extract-fit}.
    }
    \label{fig:src-bkg-fits}
\end{figure*}

\begin{table*}[!ht]
    \centering
    \caption{Integrated source and background fits, single-temperature NEI
        model with non-solar abundances \label{tab:src-fits}}
    \footnotesize
    % 20161015_src_bkg_grp01_pgstat_mosmerge.json
% 20161015_src_bkg_mg.json
% 20161015_src_bkg_mg-ar-ca_MANUAL.json
% 20161028_src_bkg_mg-si-s-fe_MANUAL.json
% 20161028_src_bkg_o-ne-mg-si-s.json
% 20161026_src_bkg_o-ne-mg-si-s-ar-ca-fe_MANUAL.json
% 20161028_src_bkg_o-ne-mg-si-s-fe_nH-0.7_NOERR.json
% 20161028_src_bkg_mg-si-s_tau-5e13_MANUAL.json
\begin{tabular}{@{}lllllllll@{}}
\toprule
 & A & B & C & D & E & F & G & H \\
\midrule
\multicolumn{9}{c}{Remnant NEI Model} \\
\midrule
$\nH$ & ${2.47}^{+0.15}_{-0.13}$ & ${2.33}^{+0.09}_{-0.17}$ & ${2.48}^{+0.20}_{-0.19}$ & ${2.02}^{+0.13}_{-0.10}$ & ${2.25}^{+0.13}_{-0.12}$ & ${2.10}^{-2.10}_{-2.10}$ & $\textbf{0.70}$ & ${3.12}^{+0.11}_{-0.09}$ \\ [0.5 em]
$\kB T$ & ${1.45}^{+0.21}_{-0.24}$ & ${1.73}^{+0.30}_{-0.14}$ & ${1.25}^{+0.33}_{-0.09}$ & ${1.53}^{+0.22}_{-0.19}$ & ${1.64}^{+0.28}_{-0.13}$ & ${1.08}^{-1.08}_{-1.08}$ & ${2.75}^{-2.75}_{-2.75}$ & ${0.51}^{+0.00}_{-0.02}$ \\ [0.5 em]
$\tau$ & ${2.46}^{+0.47}_{-0.41}$ & ${2.23}^{+0.19}_{-0.34}$ & ${2.77}^{+0.84}_{-0.43}$ & ${2.14}^{+0.25}_{-0.19}$ & ${2.21}^{+0.31}_{-0.24}$ & ${2.66}^{-2.66}_{-2.66}$ & ${1.57}^{-1.57}_{-1.57}$ & $\textbf{5000}$ \\ [0.5 em]
EM & ${6.14}^{+2.03}_{-1.07}$ & ${4.41}^{+0.75}_{-1.37}$ & ${6.29}^{+2.59}_{-1.87}$ & ${5.62}^{+1.20}_{-1.07}$ & ${3.16}^{+0.70}_{-0.56}$ & ${7.50}^{-7.50}_{-7.50}$ & ${2.89}^{-2.89}_{-2.89}$ & ${38.62}^{+5.73}_{-3.23}$ \\ [0.5 em]
O &      &      &      &      & ${12.77}^{+8.33}_{-4.83}$ & ${3.94}^{-3.94}_{-3.94}$ & ${0.05}^{-0.05}_{-0.05}$ &      \\ [0.5 em]
Ne &      &      &      &      & ${1.66}^{+1.18}_{-0.69}$ & ${0.95}^{-0.95}_{-0.95}$ & ${0.00}^{-0.00}_{-0.00}$ &      \\ [0.5 em]
Mg &      & ${1.27}^{+0.11}_{-0.10}$ & ${1.20}^{+0.13}_{-0.14}$ & ${0.74}^{+0.12}_{-0.04}$ & ${1.73}^{+0.57}_{-0.36}$ & ${0.69}^{-0.69}_{-0.69}$ & ${0.33}^{-0.33}_{-0.33}$ & ${1.05}^{+0.10}_{-0.09}$ \\ [0.5 em]
Si & ${4.01}^{+0.23}_{-0.20}$ & ${4.61}^{+0.53}_{-0.29}$ & ${4.61}^{+0.43}_{-0.49}$ & ${3.55}^{+0.23}_{-0.23}$ & ${6.53}^{+2.00}_{-1.15}$ & ${3.88}^{-3.88}_{-3.88}$ & ${3.02}^{-3.02}_{-3.02}$ & ${4.19}^{+0.27}_{-0.24}$ \\ [0.5 em]
S & ${3.63}^{+0.35}_{-0.15}$ & ${4.14}^{+0.59}_{-0.35}$ & ${4.31}^{+0.44}_{-0.47}$ & ${3.83}^{+0.37}_{-0.36}$ & ${6.15}^{+1.91}_{-1.27}$ & ${4.74}^{-4.74}_{-4.74}$ & ${4.35}^{-4.35}_{-4.35}$ & ${4.79}^{+0.30}_{-0.36}$ \\ [0.5 em]
Ar &      &      & ${6.60}^{+2.05}_{-1.67}$ &      &      & ${9.99}^{-9.99}_{-9.99}$ &      &      \\ [0.5 em]
Ca &      &      & ${30.31}^{+16.03}_{-10.57}$ &      &      & ${57.37}^{-57.37}_{-57.37}$ &      &      \\ [0.5 em]
Fe &      &      &      & $< 0.15$ &      & ${0.00}^{-0.00}_{-0.00}$ & ${0.00}^{-0.00}_{-0.00}$ &      \\
\midrule
\multicolumn{9}{c}{X-ray Background} \\
\midrule
$\kB T_{\mt{local}}$ & ${0.26}^{+0.02}_{-0.01}$ & ${0.26}^{+0.02}_{-0.01}$ & ${0.27}^{+0.02}_{-0.01}$ & ${0.26}^{+0.01}_{-0.01}$ & ${0.26}^{+0.01}_{-0.01}$ & ${0.25}^{-0.25}_{-0.25}$ & ${0.25}^{-0.25}_{-0.25}$ & ${0.26}^{+0.01}_{-0.01}$ \\ [0.5 em]
$\mt{EM}_{\mt{local}}$ & ${0.24}^{+0.02}_{-0.01}$ & ${0.24}^{+0.02}_{-0.02}$ & ${0.23}^{+0.02}_{-0.02}$ & ${0.23}^{+0.02}_{-0.02}$ & ${0.22}^{+0.01}_{-0.02}$ & ${0.23}^{-0.23}_{-0.23}$ & ${0.20}^{-0.20}_{-0.20}$ & ${0.18}^{+0.02}_{-0.01}$ \\ [0.5 em]
$N_{\mathrm{H,xgal}}$ & ${1.24}^{+0.73}_{-0.58}$ & ${1.12}^{+0.73}_{-0.45}$ & ${1.09}^{+0.75}_{-0.61}$ & ${1.14}^{+0.79}_{-0.68}$ & ${0.10}^{-0.10}_{-0.10}$ & ${1.17}^{-1.17}_{-1.17}$ & ${1.22}^{-1.22}_{-1.22}$ & ${1.10}^{+1.04}_{-0.54}$ \\ [0.5 em]
$N_{\mathrm{H,ridge}}$ & ${1.38}^{+0.20}_{-0.09}$ & ${1.39}^{+0.19}_{-0.09}$ & ${1.36}^{+0.18}_{-0.09}$ & ${1.41}^{+0.09}_{-0.09}$ & ${1.55}^{+0.10}_{-0.10}$ & ${1.46}^{-1.46}_{-1.46}$ & ${1.40}^{-1.40}_{-1.40}$ & ${1.16}^{+0.06}_{-0.09}$ \\ [0.5 em]
$\kB T_{\mt{ridge}}$ & ${0.75}^{+0.04}_{-0.04}$ & ${0.74}^{+0.12}_{-0.04}$ & ${0.75}^{+0.04}_{-0.03}$ & ${0.70}^{+0.04}_{-0.05}$ & ${0.86}^{+0.03}_{-0.06}$ & ${0.76}^{-0.76}_{-0.76}$ & ${0.68}^{-0.68}_{-0.68}$ & ${0.74}^{+0.03}_{-0.03}$ \\ [0.5 em]
$\mt{EM}_{\mt{ridge}}$ & ${1.92}^{+0.31}_{-0.30}$ & ${1.94}^{+0.32}_{-0.30}$ & ${1.87}^{+0.29}_{-0.35}$ & ${2.12}^{+0.37}_{-0.33}$ & ${1.69}^{+0.27}_{-0.21}$ & ${2.00}^{-2.00}_{-2.00}$ & ${2.08}^{-2.08}_{-2.08}$ & ${1.47}^{+0.26}_{-0.25}$ \\
\midrule
pgstat & 14218.8 & 14201.0 & 14141.1 & 14146.9 & 14125.6 & 14005.1 & 14147.9 & 14486.6 \\
$\chi^2$ & 13119.6 & 13098.4 & 13031.3 & 13043.9 & 13032.5 & 12900.8 & 13055.1 & 13347.3 \\
$\chi^2_{\mt{red}}$ & 1.028 & 1.026 & 1.021 & 1.022 & 1.021 & 1.011 & 1.023 & 1.046 \\
dof & 12764 & 12763 & 12761 & 12762 & 12761 & 12758 & 12761 & 12764 \\
\bottomrule
\end{tabular}

    \tablecomments{Units:
        equivalent hydrogen column density $N_H$, $10^{22} \unit{cm^{-2}}$;
        temperature $\kB T$, keV;
        ionization timescale $\tau$, $10^{10} \unit{s\;cm^{-3}}$;
        emission measure (EM), $10^{11} \unit{cm^{-5}}$.
        Elemental abundances are taken relative to ISM values of
        \citet[Table 2]{wilms2000}.
        The emission measure is $10^3 \times $ the usual \texttt{XSPEC} norm
        $10^{14} (4 \pi D^2)^{-1} \int n_{\mt{H}} n_{\mt{e}} dV$; $D$ is source
        distance, $n_{\mt{H}}$ is hydrogen number density, and $n_{\mt{e}}$ is
        electron number density.
    }
\end{table*}

% Explain X-ray background model
The X-ray sky background model comprises an unabsorbed low-temperature thermal
plasma for both local bubble and stable SWCX emission
\citep{mccammon1990, snowden1990, cravens2000, galeazzi2014, smith2014},
an absorbed thermal plasma for soft diffuse galactic ridge emission
\citep[e.g.][]{kaneda1997}, and an absorbed extragalactic
background power law with photon index $1.4$ \citep{hickox2006}.
The XSPEC model is \texttt{apec + tbnew\_gas*apec + tbnew\_gas*powerlaw}).
We fix the extragalactic power law normalization to
$6.6 \unit{photons\; cm^{-2}\, s^{-1}\, sr^{-1}\, keV^{-1}}$, which
represents $60\%$ of the total extragalactic X-ray background normalization
$10.9 \unit{photons\; cm^{-2}\, s^{-1}\, sr^{-1}\, keV^{-1}}$
\citep{hickox2006} due to removal of bright extragalactic sources.
\footnote{Our point source flux cut-off $10^{-14} \unit{erg\;cm^{-2}\;s^{-1}}$
(0.4-7.2 keV band) translates to removal of extragalactic sources brighter than
$1.5 \times 10^{-14} \unit{erg\;cm^{-2}\;s^{-1}}$ (2-10 keV band, assuming
absorbing column $2 \nHUnits$ and power law index $\Gamma = 1.4$).
We integrate the $\log(N)$-$\log(S)$ relation of \citet{moretti2003} to obtain
the excluded brightness.}
\textbf{TODO: shorten and clarify text?}  % TODO

% Check model - galactic ridge component has reasonable surface brightness.
The fitted soft galactic ridge component has $2$--$10 \unit{keV}$ surface
brightness $3.1 \times 10^{-12} \unit{erg\;cm^{-2}\;s^{-1}\;deg.^{-2}}$ (from
normalization $(1.94 \pm 0.3) \times 10^{-3} \times 10^{14} \unit{cm^{-5}}$ for
a 400\arcsec radius circular FOV).
Our inferred ridge surface brightness is somewhat lower than
other galactic ridge or bulge fields --
$4.8 \times 10^{-11} \unit{erg\;cm^{-2}\;s^{-1}\;deg.^{-2}}$
at $(l,b) = (28\arcdeg 46\arcmin, -0\arcdeg 20\arcmin)$ \citep{ebisawa2008}
and
$7.1 \times 10^{-11} \unit{erg\;cm^{-2}\;s^{-1}\;deg.^{-2}}$
at $(l,b) = (0.113\arcdeg, -1.424\arcdeg)$ \citep{revnivtsev2009}.
These other fields sample brighter areas of X-ray ridge emission (see the
RXTE/PCA map presented by \citet{revnivtsev2006}), so a $\abt 10\times$ smaller
brightness seems plausible, given significant uncertainty in our background
modeling due to assumed extragalactic background normalization and high soft
proton background.

% vpshock discussion
We consider an alternative model that accounts for an ionization age gradient
behind a plane shock \citep{borkowski2001}.
The \texttt{vpshock} model fits the spectrum just as well
($\mathrm{pgstat} = ???$) as the single-ionization-age NEI model, and the
\texttt{vpshock} bounds on ionization age bracket the inferred mean ionization
from \texttt{NEI} (Table~\ref{tab:src-fits}.
\textbf{TODO: need to redo vpshock fits}
At high electron temperatures $\abt 2$--$3 \unit{keV}$, the
\texttt{vpshock} model should reasonably approximate the more involved
\texttt{vsedov} model of \citet{borkowski2001}, so we do not consider a
\texttt{vsedov} fit with non-uniform electron temperature distribution.
The simpler \texttt{NEI} plasma model suffices for our spectral modeling.

% ISM component TODO to be redone
We see no particular evidence for a distinct NEI (ISM) plasma of different
temperature or ionization state.
\textbf{TODO fits with ISM component to be rerun}.
%If we add an ISM component, represented by a NEI plasma with ISM abundances,
%the fit improves slightly.
%But fitted parameter values jump around quite a lot.
%The ISM NEI components can be wildly different (ionization age
%$10^{9}$--$10^{10}$, temperature $0.55$ to $8.63 \unit{keV}$,
%normalization (scaled emission measure) $2 \times 10^{-3}$ to $3 \times
%10^{-2}$), depending on which elemental abundances are free in the ejecta
%plasma component.  In short, very ill-constrained.
%We don't see any spectral features that can unequivocally be attributed to a
%distinct mass of plasma with different temperature, abundance, and ionization
%characteristics.

\textbf{Although fits have not all been redone, it is pretty clear that adding
additional components will not improve the fit substantially
We can argue for one model over another on the basis of f-test / similar...}

We consider two alternative fits in Table~\ref{tab:src-fits}.
Fit G fixes absorption $\nH = 0.7 \nHUnits$ to see whether a lower absorption
is consistent with the data, and to compare to fit results presented by
\citet{rakowski2001} and \citet{safi-harb2007}.
Fit H fixes $\tau = 5 \times 10^{13} \TauUnits$ to consider a collisional
ionization equilibrium plasma \citep{smith2014}.
Figure~\ref{fig:src-bkg-vary-comp} compares these fits directly to a baseline
fit with only Mg, Si, S free (same fit as in Figure~\ref{fig:src-bkg-fits}).

%TODO clean up text
Not shown: fit H with a CIE plasma does alter Si and S He $\alpha$ / He $\beta$
line ratios, introducing some tension.

\begin{figure*}[!ht]
    \plotone{fig/fig_src_bkg_tau-5e13_nH-0d7_allexp.pdf}
    \figcaption{Fits with varied absorption and ionization timescale
        also reasonably describe the data, provided that elemental abundances
        comprising the $\lesssim 1 \unit{keV}$ continuum are allowed to float.
        Solid curves plot total spectrum model (remnant with x-ray background,
        soft proton, and instrumental line contribution as in
        Figure~\ref{fig:src-bkg-fits}); dashed curves plot remnant
        absorbed NEI plasma contribution alone.
        Fit parameters (B, G, H) are given in Table~\ref{tab:src-fits}.
    }
    \label{fig:src-bkg-vary-comp}
\end{figure*}

\subsection{Annuli spectra}

% Procedure, results
We extract and fit spectra from $100\arcsec$ wide annuli centered on the
remnant to discern how thermal plasma emission varies with remnant radius
(Figure~\ref{fig:annuli-spectra}, Table~\ref{tab:annulus-fit}).
Each annulus is modeled by an absorbed NEI plasma
(XSPEC model \texttt{tbabs\_new * vnei}) with Mg, Si, S free and absorbing
column tied across all annuli.
X-ray background model parameters are fixed to values from the best integrated
source fit with Mg, Si, S free (Table~\ref{tab:src-fits}).  % TODO may change...
We neglect redistribution of counts between annuli due to point spread function
(PSF) wings as the off-axis half-energy-enclosed width $\abt10$--$20\arcsec$ is
smaller than our annuli widths.

% Some basic observations on annulus spectra
The outer annulus fit ($400$--$500\arcsec$) is poorly constrained and merely
requires some Si and S line emission; the derived ionization timescale,
temperature, and norm may be unreliable.
Soft proton parameters from annuli fits qualitatively show expected
energy-dependent vignetting, with stronger vignetting at soft energies.
Soft proton contamination at $\abt 0.3$--$1 \unit{keV}$ clearly decreases with
distance from the aimpoint.

\begin{table*}
    \centering
    \caption{Annuli fit \label{tab:annulus-fit}}
    % 20161019_fiveann_mg.json
\begin{tabular}{@{}rllllll@{}}
\toprule
Annulus & $kT$ & $\tau$ & Mg & Si & S & EM \\
 & (keV) & ($\times 10^{10}$) & (-) & (-) & (-) & $(\times 10^{11})$ \\
\midrule
  $0$--$100\arcsec$ & ${3.9}^{+2.4}_{-1.5}$ & ${1.37}^{+0.21}_{-0.08}$ & ${1.6}^{+0.5}_{-0.5}$ & ${8.3}^{+1.6}_{-1.3}$ & ${8.4}^{+\,(>1.6?)}_{-1.9}$ & ${3.2}^{+1.1}_{-0.7}$ \\
$100$--$200\arcsec$ & ${2.1}^{+0.5}_{-0.4}$ & ${1.94}^{+0.37}_{-0.24}$ & ${1.30}^{+0.23}_{-0.21}$ & ${6.3}^{+0.7}_{-0.7}$ & ${5.7}^{+0.9}_{-0.8}$ & ${4.8}^{+1.3}_{-0.9}$ \\
$200$--$300\arcsec$ & ${2.1}^{+0.4}_{-0.4}$ & ${2.1}^{+0.5}_{-0.2}$ & ${1.6}^{+0.3}_{-0.2}$ & ${4.4}^{+0.5}_{-0.5}$ & ${4.0}^{+0.5}_{-0.6}$ & ${4.7}^{+1.3}_{-0.8}$ \\
$300$--$400\arcsec$ & ${1.7}^{+0.5}_{-0.3}$ & ${2.1}^{+0.8}_{-0.5}$ & ${1.1}^{+0.2}_{-0.2}$ & ${3.2}^{+0.5}_{-0.5}$ & ${2.9}^{+0.9}_{-0.6}$ & ${2.5}^{+0.8}_{-0.5}$ \\
$400$--$500\arcsec$ & ${10}^{+?}_{-3}$ & ${1.3}^{+0.5}_{-0.5}$ & ${1.8}^{+2.1}_{-0.8}$ & ${2.7}^{+2.4}_{-1.0}$ & ${3.5}^{+\,(>6.5?)}_{-2.1}$ & ${0.34}^{+0.16}_{-0.14}$ \\
\bottomrule
\end{tabular}

    \tablecomments{
        Outermost ($400$--$500\arcsec$) annulus temperature $\kB T$ is
        capped at $10 \unit{keV}$ and may not represent the formal best fit.
        Best fit absorption column is $\nH = {2.51}^{0.09}_{-0.10} \nHUnits$;
        $\mt{pgstat} = 19335.8$; $\chi^2_{\mt{red}} = 1.210 = 24296.5 / 20073$ ($\chi^2$/dof).
        Units as in Table~\ref{tab:src-fits}.}
\end{table*}

\begin{figure*}[]
    \plotone{fig/fig_fiveann_mg-si-s_0087940201-mos.pdf}
    \figcaption{Annuli spectra from 0087940201 MOS with best fits; Mg, Si, S free.
    Table~\ref{tab:annulus-fit} provides NEI fit parameters.}
    \label{fig:annuli-spectra}
\end{figure*}

\begin{figure*}[!ht]
    \plottwo{fig/fig_fiveann_kT.pdf}{fig/fig_fiveann_Tau.pdf}
    \plottwo{fig/fig_fiveann_Si.pdf}{fig/fig_fiveann_S.pdf}
    \figcaption{Electron temperature (left) and ionization timescale (right)
        as a function of radius from annuli spectra fit
        (Tables~\ref{tab:annulus-fit}).}
    \label{fig:annulus-pars}
\end{figure*}

% Radial variation in kT and Tau
The plasma temperature and ionization timescale vary by $\abt 20$--$50\%$
across the remnant's projected radius (Figure~\ref{fig:annulus-pars});
the center plasma is hotter and shows a lower ionization timescale compared to
the remnant shell.
The inferred Si, S abundances increase by a factor of $2$-$3$ towards the
center; Mg shows much less variation.
These trends are consistent with the standard Sedov picture of early supernova
remnant evolution, wherein recently shocked ISM and reverse-shocked central
ejecta should be the ``youngest'' emitting plasma.


\subsection{Equivalent-width guided local spectra}


Equivalent width images better demarcate regions of ejecta emission enhanced
relative to the continuum, by definition.

\begin{figure*}[]
    \plotone{fig/fig_bar-ridge-lobe_0087940201-mos.pdf}
    \figcaption{Sub-region fits}
    \label{fig:bar-lobe-ridge-fits}
\end{figure*}

\begin{table}[!ht]
    \centering
    \caption{Fits sampling interior ejecta and putative shocked ambient material
        \label{tab:bar-lobe-ridge-fits}}
    \footnotesize
    % 20161220_bar_si-s.json
% 20161220_lobe_si-s.json
% 20161220_ridge_si-s.json
% 20161220_ridge_solar.json
\begin{tabular}{@{}lllll@{}}
\toprule
 & Bar & Lobe & Ridge & Ridge (solar) \\
\midrule
$\nH$ & ${2.52}^{+0.22}_{-0.30}$ & ${1.89}^{+0.23}_{-0.18}$ & ${2.88}^{+0.15}_{-0.14}$ & ${3.62}^{+0.11}_{-0.09}$ \\ [0.5 em]
$\kB T$ & ${2.05}^{+1.15}_{-0.41}$ & ${2.58}^{+1.31}_{-0.83}$ & ${0.79}^{+0.11}_{-0.08}$ & ${0.68}^{+0.02}_{-0.03}$ \\ [0.5 em]
$\tau$ & ${1.82}^{+0.35}_{-0.27}$ & ${1.17}^{+0.23}_{-0.23}$ & ${9.2}^{+4.5}_{-2.9}$ & ${8.3}^{+1.9}_{-1.4}$ \\ [0.5 em]
EM & ${4.0}^{+1.3}_{-1.3}$ & ${2.5}^{+1.1}_{-0.6}$ & ${30.9}^{+9.3}_{-7.4}$ & ${79.}^{+19.}_{-7.}$ \\ [0.5 em]
Si & ${8.60}^{+1.19}_{-0.90}$ & ${5.34}^{+0.84}_{-0.69}$ & ${2.28}^{+0.20}_{-0.19}$ &      \\ [0.5 em]
S & ${9.01}^{+1.72}_{-1.27}$ & ${8.04}^{+5.20}_{-2.39}$ & ${2.00}^{+0.27}_{-0.24}$ &      \\
\midrule
pgstat & 5106.3 & 4319.5 & 4643.2 & 4893.3 \\
$\chi^2$ & 6792.9 & 5760.4 & 6005.3 & 6333.0 \\
$\chi^2_{\mt{red}}$ & 1.279 & 1.257 & 1.221 & 1.287 \\
dof & 5313 & 4582 & 4918 & 4920 \\
\bottomrule
\end{tabular}

    \tablecomments{Units as in Table~\ref{tab:src-fits}.}
\end{table}


% =============================================================================
% Discussion
% =============================================================================
\section{Discussion} \label{sec:disc}

\subsection{Prior work}

% State discrepancy up front
Our fitted absorption column and ionization timescales disagree with prior
X-ray fits, although we find similar temperatures and abundance ratios.
\citet{rakowski2001} obtain $\nH = (0.7 \pm 0.3) \nHUnits$,
$\kB T = 2.0^{+1.0}_{-0.6}$, and
$\tau = 1.5^{+4.7}_{-0.6} \times 10^{11} \unit{s\;cm^{-3}}$
from an absorbed NEI model for ASCA data with enhanced Ne, Mg, Si, S, Ar, Ca,
Fe abundances; the best fit converged towards a pure metal plasma.
\citet{safi-harb2007} report $\nH = 0.65^{+0.45}_{-0.25} \nHUnits$
and $\kB T = 2 \pm 0.6 \unit{keV}$ without providing $\tau$, from
an absorbed NEI model for the 2001 XMM data with
super-solar Si, S and sub-ISM O, Ne, Mg, Ca, Fe.
\citet{rakowski2001} also report emission measure
$(4\pi D^2)^{-1} \int n_e n_H dV = 9.7^{+200}_{-9.7} \times 10^8$ much smaller
than our $\abt 4 \times 10^{11} \unit{cm^{-5}}$; their inferred hydrogen
density is therefore $\abt 10\times$ smaller.
The low value is partially offset by the extremely high ($\abt 30\times$)
abundances inferred by \citet{rakowski2001}.

% Abundances
The relative abundances
  $[\mt{Mg}/\mt{Si}] = 0.23^{+0.09}_{0.08}$ and
  $[\mt{S}/\mt{Si}] = 1.09^{+0.24}_{-0.18}$ from \citet{rakowski2001}
are consistent with our fits, which yield $[\mt{Mg}/\mt{Si}] = 0.28^{+0.04}_{-0.05}$
and $[\mt{S}/\mt{Si}] = 0.92 \pm 0.2$ (this statistical error on
$[\mt{S}/\mt{Si}]$ should be somewhat lower, closer to $\pm 0.1$, because the
parameters covary).
Note that $[\mt{Mg}/\mt{Si}]$ is equal to
\[
    \frac{[\mt{Mg}/\mt{H}] / [\mt{Mg}/\mt{H}]_{\sun}}
         {[\mt{Si}/\mt{H}] / [\mt{Si}/\mt{H}]_{\sun}}
\]
or $[\mt{Mg}/\mt{Mg}_{\sun}] / [\mt{Si}/\mt{Si}_{\sun}]$,
which is the usual ratio of interest.

% Possible causes?
%Could HD 119682 be contaminating other workers' fits?
%\citet{rakowski2001} went to some effort to construct a model for the point
%source contamination, modeling the ASCA X-ray emission as an absorbed power
%law ($\nH = 0.26^{+0.18}_{-0.14} \nHUnits$, $\Gamma = 1.5 \pm 0.3$).
%\citet{safi-harb2007} report $\nH = 0.18^{+0.08}_{-0.07} \nHUnits$ and $\Gamma
%= 1.41^{+0.14}_{-0.22}$ based on finely resolved \textit{Chandra} data, and
%obtained comparable numbers from the same \textit{XMM} data we consider here.
%In short, the point source appears adequately accounted for by previous
%workers, so we disfavor this explanation.

% Explore low abundance fit.
If O, Ne, and Fe abundances are allowed to be sub-solar, there exists a
reasonable fit with lowered absorption of $\abt 0.7 \nHUnits$
(Table~\ref{tab:src-fits}, Fit~G).
The extremely sub-solar abundances of O, Ne, and Fe may be explained by a
metal-rich ejecta plasma, wherein metals make up $\gtrsim 10\%$ of the plasma
number density.
We similarly find reasonable, if disfavored, fit with increased $\tau = 5
\times 10^{13} \TauUnits$.


\subsection{Plasma mass and density}

% Emission measure derived masses
We estimate X-ray emitting plasma mass and density from the integrated remnant
fit emission measure (Table~\ref{tab:src-fits}).
The X-ray emitting plasma density, assuming that the plasma fills a sphere of
radius $400\arcsec$, is:
\begin{equation} \label{eq:density}
    n_{\mt{H}} = 0.1 f^{-1/2} d_{5}^{-1/2} \unit{cm^{-3}}
\end{equation}
with an arbitrary volume filling factor $f \in [0,1]$ and distance relative to
$5 \unit{kpc}$ as $d_{5}$.
Assuming the plasma is hydrogen dominated, with $[\mt{He}/\mt{H}] = 0.1$
\citep{wilms2000}, computing mass $M = 1.4 n_{\mt{H}} m_{H} f V$ yields
\[
    M = 13.7 M_{\sun} f^{1/2} d_{5}^{5/2}
\]
with corresponding ejecta masses $0.018 M_{\sun}$ of silicon, $0.012 M_{\sun}$
of sulfur, and $0.002 M_{\sun}$ of magnesium (attributing a solar abundance to
ISM).

% Mass estimate for a metal-rich plasma
The inferred mass increases if the plasma is metal-rich.
The ion density for a given emission measure is lower than that of an
equivalent hydrogen plasma:
\[
    n_{\mt{ion}} = \sqrt{\frac{4 \pi D^2 \eta}{\langle Z \rangle V}}
\]
where $\eta$ is the emission measure ($10^{14}\times$ the \texttt{XSPEC} norm),
$D$ is source distance, $V$ is source volume, and $\langle Z \rangle$ is mean
atomic number.
But, the mass approximately scales as $\sqrt{2 \langle Z \rangle}$ (assuming no
hydrogen), since
$M_{\mt{tot}} \propto 2 \langle Z \rangle M_{\mt{p}} n_{\mt{ion}}$,
where $M_{\mt{p}}$ is proton mass (approximation for mean nuclide mass) and the
factor $2$ assumes negligible H contribution.
For G309.2-0.6, assuming a spherical volume:
\[
    M_{\mt{tot}} \approx 13.7 M_{\sun} f^{1/2} d_5^{5/2} \langle Z \rangle^{1/2}
\]
where $d_5$ is distance scaled to $5 \unit{kpc}$.
For $\langle Z \rangle \sim 10$ (as an order of magnitude estimate, for a
plasma comprising primarily shocked Si and S), we obtain
$M_{\mt{tot}} \abt 90 M_{\sun} f^{1/2} d_5^{5/2}$.
Although very uncertain, this estimate is rather high for ejecta alone,
although possible if the remnant is close ($d_5 < 1$) and/or the ejecta
distributed in clumps.
Alternatively, shocked ISM/CSM may have mass comparable to that of the ejecta.

% Discuss density inferences
Our inferred density - valid for a homogeneous sphere - should be a lower bound
on the post-forward-shock density.
Therefore the ambient medium density should be at least
$0.025 f^{-1/2} d_{5}^{-1/2} \unit{cm^{-3}}$, assuming a strong forward shock
with compression ratio of four, expected for young ejecta-dominated remnants.
This bound is not very stringent, but is at least a start.
%If shocked plasma is concentrated within a shell of width $r/12$
%% volume fraction (1 - (11/12)^3)
%\textbf{todo:motivate this 1/12 factor}, then the shocked density would be
%$n_{\mt{H}} \sim 0.5 f^{-1/2} d_{5}^{-1/2}$
%and the ambient density bound would be $0.12 f^{-1/2} d_{5}^{-1/2}$,
%which suggests ``typical'' ambient ISM.
%This would also disfavor distance much higher than $5 \unit{kpc}$.

\subsection{Iron non-detection}

Fits allowing non-solar iron abundance favor zero iron contribution, although
by eye a solar Fe abundance fit appears plausible.
But, even with solar Fe abundance, we see no Fe K line at $\abt 6.5 \unit{keV}$.
We can set an upper bound on Fe K luminosity and thus constrain the parameter
space for Ia and CC explosions, building on recent work by
\citet{yamaguchi2014-iron} and \citet{patnaude2015}.

We take our best fit (Table~\ref{tab:src-fits} and add a Gaussian line of
normalization $5 \times 10^{-5} \unit{photon\;cm^{-2}\;s^{-1}}$ and
line width $0.05 \unit{keV}$ at $6.55 \unit{keV}$.
This crudely represents a minimum detectable Fe K flux
$\abt 0.01$ to $0.02 \unit{counts\;s^{-1}\;keV^{-1}}$ after folding through
telescope and detector response.
The upper bound on Fe K line luminosity is:
\[
    L_{\mt{Fe-K}} = 1.2 \times 10^{41} d_5^2 \unit{photons\;s^{-1}}
\]
where $d_5$ represents distance scaled to $5 \unit{kpc}$.
\textbf{TODO: clean up this analysis and formalize the non-detection.
We can likely do better with a more careful analysis, say by explicitly
modeling Fe-L and Fe-K emission or calculating errors on a narrow band model of
a power law and gaussian.  Need a rough estimate of Fe line width for this.}

The luminosity $L_{\mt{Fe-K}} \lesssim 10^{41} \unit{photons\;s^{-1}}$
disfavors bright Ia explosions in dense ambient media
$n_0 \sim 1 \unit{cm^{-3}}$; e.g., DDTa in $n_0 \sim 3 \unit{cm^{-3}}$ from
\citet{badenes2003, badenes2005, badenes2006} \citep{yamaguchi2014-iron}
and CC explosion in dense CSM wind with high mass loss rates appropriate for
red supergiants \citep{patnaude2015}.
If $d_5 \sim 2$--$3$, near the HI-absorption-derived distance bound of
\citet{gaensler1998-g309}, the line luminosity
$\lesssim 10^{42} \unit{photons\;s^{-1}}$ is less constraining.
We will argue against such a large distance in subsequent discussion, on
the basis of remnant size and relatively low ionization state.

\subsection{Morphology and spatial structure}

% Discuss morphology in broad terms
The broadband $0.8$--$3.3 \unit{keV}$ remnant mostly fills the radio remnant,
excepting the remnant's southeast limb and edges of the radio ears
(Figures~\ref{fig:snr}).
X-ray emission is concentrated at and behind the northwest limb, which is
somewhat dimmer in radio compared to the southeast limb and ears.
Figure~\ref{fig:rgb} traces the spatial distribution of Mg, Si, and S, the
three elements with resolved He$\alpha$-like line emission throughout the
remnant.

% Spatial distribution of line emission
Mg emission concentrates along the northwest limb, and possibly along the
southeast limb as well.
Both Si and S concentrate towards the remnant center.
Towards the southwest and southeast limbs, small clumps of Mg and Si/S appear
similarly separated.

These images of Mg, Si, S may not necessarily trace ejecta distribution.
Our annulus fits show that temperature and ionization age vary with radius by a
factor $\abt 2 \times$, so line emission may vary independently of ejecta
abundance.
Nevertheless, variation in plasma conditions is also worth investigating.

some show that radial variation in temperature and ionization state

should be small, and not significantly impact emission.
\textbf{TODO: We have not shown this for azimuthal variation.}
\textbf{The simplest solution is to explore with resolved spectroscopy.}

\begin{figure*}[!ht]
    \plotone{fig/fig_narrowband.png}
    \figcaption{Heavily smoothed images of Mg, Si, S He$\alpha$ line emission
%        Smoothed RGB image (top left) of G309 sampling Mg, Si, S
%        He$\alpha$ lines.
%        Red, top right: $1.3$--$1.4 \unit{keV}$ (Mg).
%        Green, bottom left: $1.7$--$1.8 \unit{keV}$ (Si).
%        Blue, bottom right: $2.3$--$2.4 \unit{keV}$ (S).
        MOST contours are 0.01, 0.02, 0.05, 0.1, 0.2 Jy in all images.
        The images are not adjusted for background and should be considered
        qualitative only.}
    \label{fig:rgb}
\end{figure*}

If the radio traces the forward shock front, then there are relatively
large gaps between FS and CD along the sides of the remnant, with the
notable exception of the northwest limb (which also corresponds with
the brightest X-ray and faintest radio emission).
This may offer clues about the dynamical state of the remnant, although
\textit{Chandra} would be far better for this sort of study.

% Outer annulus study
The outer ($300$--$400\arcsec$) annulus temperature $\abt 2 \unit{keV}$
suggests strong shock heating (shock velocity $\gtrsim 1000 \unit{km/s}$).
It is unclear whether emission arises from ejecta or shocked ISM, but in either
case requires the shock to be dynamically young.

\citet{katsuda2015} use, similar to \citet{kosenko2010}, a deprojection method
to try to back out a radial profile of the ejecta.
Someone else, \citet{hughes2003} I think, also deprojected DEM L71 emission.
\textbf{TODO...}

\subsection{Age dating}

% Age estimates
From our emission-measure-derived plasma density and integrated ionization age,
we infer an age of $7000 f^{1/2} d_{5}^{1/2} \unit{yr}$.

The Sedov-derived age \citep{taylor1950, sedov1959} is calculated using the
numerical constants for a monatomic gas \citet{taylor1950-pt2}.
From the shock radius scaling
$R = (0.34 \unit{pc}) \left( \frac{E_{51}}{n_0} \right)^{1/5} t_{\mt{yr}}^{2/5}$,
we invert for remnant age and assume angular radius $400\arcsec$ to obtain:
\[
    t = (4300 \unit{yr}) d_5^{5/2} E_{51}^{-1/2} n_0^{1/2} .
\]
which is somewhat on the high side for an ejecta dominated remnant.
\textbf{TODO: factor 0.34 is derived from Taylor's numerically derived
$\beta^5 = 2.052$, but better to use closed form solution}

The Sedov age and ionization timescale / norm derived age scale differently
with remnant distance and can help set a lower bound on our distance estimate
(Figure~\ref{fig:age}).
However, the Sedov estimate represents an upper bound on age as the remnant
may still be ejecta-dominated, or transitioning into Sedov expansion.
Based on
\citet{rakowski2001} already considered these estimates, and our current age
estimates are no more stringent than theirs.

When is the Sedov age applicable?
\citet{truelove1999} compute timescales for the transition from free expansion
to Sedov-Taylor dynamics, assuming ambient medium with a power law radial
profile.
For a medium with $\rho_0 \propto r^{-7}$, the transition time
\[
    t_{\mt{ST}} = 0.732 E^{-1/2} M_{\mt{ej}}^{5/6} \rho_0^{-1/3} .
\]
We take ambient density
$\rho_0 < 1.4 m_{\mt{H}} \times \frac{n_{\mt{H}}}{4}
 \approx 0.01 m_{\mt{H}} f^{-1/2} d_{5}^{-1/2} \unit{cm^{-3}}$
assuming a strong shock with compression ratio $4$, and adopting our rough
density estimate from equation~\eqref{eq:density}.
Recalling that the ion density for a shocked metal-rich plasma is lower than
that of shocked ISM, this represents a conservative bound.
We then have:
\[
    % 0.732 * (10^51 erg)^(-1/2) * (1.99e33 grams)^(5/6) * (0.01 cm^-3 * 1.67e-24 grams)^(-1/3)
    t_{\mt{ST}} \lesssim (1600 \unit{yr})
                         E_{\mt{51}}^{-1/2}
                         \left( \frac{M_{\mt{ej}}}{M_{\sun}} \right)^{5/6}
                         f^{1/6} d_{5}^{1/6}
\]
Alternatively, if we simply scale ambient density to $0.1 \unit{cm^{-3}}$, we have:
\begin{equation} \label{eq:sttime-generic}
    % 0.732 * (10^51 erg)^(-1/2) * (1.99e33 grams)^(5/6) * (0.1 cm^-3 * 1.4 * 1.67e-24 grams)^(-1/3)
    t_{\mt{ST}} = (670 \unit{yr})
                  E_{\mt{51}}^{-1/2}
                  \left( \frac{M_{\mt{ej}}}{M_{\sun}} \right)^{5/6}
                  \left( \frac{n_0}{0.1 \unit{cm^{-3}}} \right)^{-1/3}
\end{equation}
Given the uncertainty in density and mass estimates based solely on plasma
emission measure, Figure~\ref{fig:age} shows a range of plausible $t_{\mt{ST}}$
timescales for ambient medium densities $0.1$ to $0.1 \unit{cm^{-3}}$.

\textbf{TODO: possible improvement}, either use the fully stitched together results
of \citet{truelove1999}, or the recent reformulation of \citet{tang2016}.


\begin{figure}[!ht]
    \plotone{fig/fig_age_plot.pdf}
    \figcaption{Red: Sedov-derived age bound with $E_{51} = 1$
        and $n_0 \in [0.1, 1] \unit{cm^{-3}}$.
        Blue: plasma ionization and EM derived age estimate, with filling factor
        $f \in [0.2, 1]$.  Plasma is assumed to be homogenous and spherical
        volume filling.
        Gray: transition time $t_{\mt{ST}}$ range for
        $n_0 \in [0.1, 1] \unit{cm^{-3}}$ from \citet{truelove1999}, assuming
        power-law ambient medium with $n=-7$.
    }
    \label{fig:age}
\end{figure}

\subsection{Galactic environment}

The absence of H$\alpha$ emission is unilluminating.
Assuming H$\alpha$ emission to arise by shock propagation through ISM
containing neutral hydrogen \citep{chevalier1978},  % 1978ApJ...225L..27C
any of the following factors may contribute: low density of neutral H due to
progenitor or supernova ionization, low ambient ISM density, or some local
conditions (extremely high post-shock density?) that rapidly ionize neutral H,
preventing Balmer line emission.
Or, the available imagery may just not be deep enough.
The outcome is that we are unable to constrain shock velocity independently by
Balmer line width measurement.

% Galactic environment
The sight line towards G309.2-0.6 crosses the tangent of the Centaurus Arm
(variously Scu-Cen, Crux-Cen, Scu-Crux) and the near and far Carina arm at
$\abt 1.5 \unit{kpc}$ and $\abt 14 \unit{kpc}$.  Molecular and HI emission is
dominated by the Centaurus Arm at line-of-sight velocities $-70$ to
$-20 \unit{km/s}$ \citep[e.g.,][Figure 4]{dame2011}, and a good number of
molecular clouds can be identified \citep{rice2016}.  The field is populated
with unrelated but independently interesting features.

% sigma-D estimate of distance
From the somewhat questionable $\Sigma$-$D$ relation, \citet{pavlovic2014}
provide a distance estimate of $5.9 \unit{kpc}$.
This is fraught with peril.
At least the relation of \citet{pavlovic2014} is calibrated for shell-type
radio remnants.
%See also \citet{huang1985} which presents a $\Sigma$-$D$ relation for
%shell-like remnants also interacting with clouds -- which helps constrain the
%expected brightness.
I wonder if there's some way to adjust the results based on local density
estimates, or just blindly regress against multiplicative combinations of other
salient physical variables.
Nevertheless, present the result, since it is at least partially physically
motivated.
% Shklovskii 1960
% Clark & Caswell 1976
% Milne 1979
% you must understand the argument before rejecting it out of hand.

\subsection{Interpretation}

% Caveats
\textbf{Caveats:} our NEI model assumes a homogeneous, single-temperature,
single-ionization age plasma, and does not separate ejecta and shocked ambient
ISM/CSM emission.
The best fit parameters probe only the brightest and densest X-ray emitting
plasma; see \citet{rakowski2006-g337} for more extended discussion.

Nevertheless, we suggest that our observations are consistent with two
scenarios:
1. a dynamically young remnant comparable to historical galactic SNRs at
relatively close distance, probably past the near Carina arm
2. a young to middle-aged remnant at large distance.

We have several distinct observations to interpret:
\begin{itemize}
    \item Low ionization timescale
        $\tau \sim 2 \times 10^{10} \TauUnits \sim 600 \unit{yr\;cm^{-3}}$
        consistent with other young ejecta-dominated remnants of Ia SNe
        \citep[][Table 4]{badenes2007}.
    \item No obvious evidence for separate ISM and ejecta components of
        distinct temperature and emission age
    \item Absence of iron K emission, which
        suggests
        1. low intrinsic luminosity (low explosion strength and/or low density
           medium),
        2. ejecta stratification, and incomplete propagation of the RS through
           all ejecta (the RS position is of course impacted by explosion strength
           and ambient density)
        3. CC abundances with low Fe, or unusual Ia model with little Fe-group
           burning
   \item Certainly super-solar Si and S abundances (hallmark of SNRs, and
       conveniently placed in X-ray mirror passband).
       Possibly super-solar Ar, Ca lines.
    \item Curious radio and X-ray shock front morphology.
\end{itemize}

% ionization age
The ionization timescale
$\tau \sim 2 \times 10^{10} \TauUnits \sim 600 \unit{yr\;cm^{-3}}$
is consistent with other young ejecta-dominated remnants of Ia SNe
\citep[][Table 4]{badenes2007}.

% TODO provide stat uncertainty from XSPEC
As ejecta abundances for Si, S are not as strongly variable between Ia and
core-collapse nucleosynthesis yields (versus O or Fe), the relative abundances
of these elements do not clearly constrain the progenitor type.
\textbf{TODO: read more about Ia and CC models and their yields.}
We cannot really constrain other elemental emission, and so do not attempt to
type G309.2-0.6 on the basis of abundance measurements.

% Comparison to old fits

If the absorption is truly that high, it is just marginally consistent with
the galactic column computed from HI and dust maps.
For G309.2-0.6, the \textit{Swift} galactic $\nH$ calculator implementation of
\citet{willingale2013} provides an expected column density $1.81 \nHUnits$.
%Dust scattering can bias X-ray spectrum fits for a purely absorbing column
%$\abt25\%$ its true value \citep{corrales2016}, although
%some scattered light is recaptured in our extended source region spectra.
%The \textit{spectrum fit value} of $\nH$ may also be estimated from empirical
%linear regression against optical extinction $\AV$ \citep[e.g.,][]{foight2016}
%-- which benefits from a strong sampling of sources within the galactic plane,
%and empirical validation -- but does not offer physical insight into the
%discrepancy. We find approximate agreement ($2$--$3 \nHUnits$) with the various
%relations.
%\textbf{TODO: for now detailed accounting of absorption is beyond the scope of
%this work}

% Discuss HI brightness profiles
We could use HI brightness profiles along the line-of-sight to set some upper
bounds on ambient density.
Consider two cases for degeneracy: assume all emission with $v_{\mt{LSR}} < 0$
on either (1) nearby-arm of rotation curve, or (2) far-arm of rotation curve.
\textbf{also revisit willingale assumptions on ISM density.  Estimate for H2
contribution is much different than \citet{yamaguchi2012}; see Sec. 3.1.
not sure if the willingale approach is valid in the galactic plane.}




% NOTE: SW lobe and ridge both show weird shifts in PN spectral lines (Si, S)
% (this dates from Jan/Feb -- to be investigated with now up to date procedures)

\subsection{Friends of young remnants}

Comparative study?  TBD.
G350.1-0.3 is young, bright (~100,000 counts)
G337.2-0.7 appears a bit older (ionization age $10^12$), also Si/S rich but
allows Fe.

A next step is to study other remnants with little Fe emission.
Review the Chandra catalog, Green's catalog...


\section{Conclusions}

Young, ejecta-dominated remnant with little iron.

\section{Misc}

Questions for self-edification (many things I don't understand):
\begin{itemize}
    \item What is the characteristic cooling timescale (adiabatic expansion?)
        for reverse shock heated ejecta?
        Under what conditions (density, turbulent mixing, ...)
        would conduction or other energy transfer processes play a role?
    \item What is the state of modeling for young remnant evolution?
        Could I fire up CR-Hydro-NEI or some other simple 1-D model,
        obtain densities, ionization times, temperature as a function of
        radius, and construct model remnant images?
        YES: Ferrand+ 2014 show lovely 3D images (don't trace reverse shock all
        the way).
        Lee+ 2014 for CR Hydro NEI
\end{itemize}

\acknowledgments

X acknowledges support by contract ...

This research is based on observations obtained with XMM-Newton, an ESA science
mission with instruments and contributions directly funded by ESA Member States
and NASA.
The MOST is operated by The University of Sydney with support from the
Australian Research Council and the Science Foundation for Physics within The
University of Sydney.

This research has made extensive use of NASA's Astrophysics Data System.
% TODO is software acknowledgement sufficient?
This research used the SNR catalogs of \citet{ferrand2012} and
\citet{green2014}.
This research made use of Astropy, a community-developed core Python package
for Astronomy \citep{astropy2013}.
This research also made use of APLpy, an open-source plotting package for
Python hosted at \url{http://aplpy.github.com}.
This research was expedited in part by Jonathan Sick's \texttt{ads2bibdesk}.

\facility{XMM(EPIC), Molonglo Observatory}
% Chandra(ACIS) -- pending
\software{APLpy, Astropy, XSPEC}

%\listofchanges

% ==========
% References
% ==========
\bibliographystyle{aasjournal}
\bibliography{refs-snr}

% ========
% Appendix
% ========
\clearpage  % Use \clearpage over \newpage
\appendix

\setcounter{table}{0}
\renewcommand{\thetable}{A\arabic{table}}
\setcounter{figure}{0}
\renewcommand{\thefigure}{A\arabic{figure}}

\section{Solar wind charge exchange checks}

% TODO explain why we don't care about this
We also neglect low-level solar wind charge exchange (SWCX) contamination,
which may also vary between the two widely-separated XMM pointings.


\section{Sky X-ray background modeling caveats}

The sky X-ray background may not be constant over scales of several arcminutes.
\citet{henley2013} modeled galactic halo emission using 110 XMM-Newton
observations well outside the galactic plane.
Most of their observations were widely separated, but a group of pointings
within $\abt 30\arcmin$ (labeled 103.1--103.27) showed halo temperature
variations $\abt 10$--$20\%$ and emission measure variation within a factor of
$\abt 2$.
% TODO I am totally just eyeballing the numbers right now -- not that useful.
% Possibly remove this.
% Henley 2013: looking at 28 clustered observations well out of the galactic
% plane, sightlines 103.1 to 103.27; 103.8 is two observations.
% SZE SurF project to complement SPT,APEX,ACT study.
Taking background from an annulus around the remnant should average out some
variation, although without knowing the power spectrum, we cannot say anything
quantitative.

The X-ray background parameters from our integrated source and background fit
may be biased by our assumption that a single-temperature NEI plasma adequately
describes integrated remnant emission.
To check this, we fit the background annulus alone, which removes this possible
bias but sacrifices background counts embedded in the source spectrum.
The resulting X-ray background parameters agree within error.
% TODO reference appendix table
% TODO give a sentence to this effect in main body.

\clearpage
\section{Is a nonthermal component compatible with integrated remnant fits?}

% Random question: should nonthermal be hyphenated (non-thermal) or not?
We do not detect nonthermal (power-law-like) X-ray emission.
By eye, we estimate that nonthermal soft X-rays must be less than $\abt 10\%$
of the thermal continuum.

Fits with a single nonthermal component (XSPEC \texttt{powerlaw} or
\texttt{srcut}) limit nonthermal emission to, qualitatively, less than 10\% of
the contribution from thermal emission \citep[cf.][]{reynolds1999}.
Figure~\ref{fig:nonthermal} shows the best fit nonthermal contributions for
each model.
For a power law, we obtain photon index $\Gamma = 4.5 \pm 1$ with
normalization $1.4^{+1.1}_{-0.09} \times 10^{-3} \unit{photons\;s^{-1}\;cm^{-2}\;keV^{-1}}$
at $1 \unit{keV}$.
For \texttt{srcut}, we obtain break frequency $10^{15.7 \pm 0.1}$
corresponding to photon energy cut-off $\abt 0.02 \unit{keV}$.
% Source: 20160630_src_srcutlog_nonsolar_snr_src.txt

Note: the radio spectral index is $0.53$ as derived by \citet{gaensler1998-g309}.
His calculation ignores single dish measurements on the basis that the flux
is confused with RCW 80 emission.  Some authors made efforts to correct their
flux measurements for the confusion; if we include Parkes single dish
measurements the radio index decreases to $0.36 \pm 0.11$.
My own fit obtains $0.52 \pm 0.09$, marginally different than that of
\citet{gaensler1998-g309}, but for simplicity and consistency just use their values.

I also fit the outer annulus emission with a nonthermal component, but the
(rather questionable) fit allowed no nonthermal emission at all.
Given that the supernova remnant's thermal emission is already ill-constrained,
this is not surprising.
I could re-attempt this with a $350$--$450\arcsec$ annulus instead of
$400$--$500\arcsec$.  % TODO - investigate this.

\begin{figure*}[!hb]
    \plotone{fig/fig_src_powerlaw_0087940201-mos1.pdf}
    \plotone{fig/fig_src_srcutlog_0087940201-mos1.pdf}
    \figcaption{Integrated source fit with powerlaw and srcutlog components
        permit only a near-negligible contribution.}
    \label{fig:nonthermal}
\end{figure*}


\clearpage
\section{Does inaccuracy introduced by $\abt 10\%$ error in BACKSCAL ratio for
integrated source fits impact fit results?}

I performed joint fits of integrated source and X-ray background while
varying the BACKSCAL ratio for MOS1S001.
Results in Table~\ref{tab:backscal-hack}.
These fits date from June 2016 and are no longer relevant or correct.

\begin{table*}[!h]
    \centering
    \caption{2009 (Motch) MOS1 BACKSCAL ratio has very little effect on fits
        \label{tab:backscal-hack}}

    \begin{tabular}{@{}lllllll@{}}
    \toprule
    Ratio & $n_\mathrm{H}$ & $kT$ & $\tau$ & Si & S & vnei EM \\
     & ($10^{22} \unit{cm^{-2}}$) & (keV) & ($10^{10} \unit{s\;cm^{-3}}$) & (-) & (-) & (EM units) \\
    \midrule
    % 1 (hack) = results_spec/20160624_src_bkg_hack_eq_one_rerun fit
    \textbf{1}     & ${2.16}^{+0.08}_{-0.11}$ & ${2.29}^{+0.30}_{-0.12}$ & ${1.82}^{+0.14}_{-0.11}$
          & ${3.75}^{+0.18}_{-0.16}$ & ${3.42}^{+0.29}_{-0.25}$ & ${4.5}^{+0.6}_{-0.5} \times 10^{-3}$ \\
    % 0.95 (hack) = results_spec/20160611_src_bkg_rerun, manually entered!
    \textbf{0.95}  & $2.15^{+0.08}_{-0.11}$ & $2.29^{+0.30}_{-0.24}$ & $1.83^{+0.10}_{-0.11}$  % vnei nH, kT, Tau
          & $3.75^{+0.18}_{-0.16}$ & $3.42^{+0.26}_{-0.25}$ & $4.5^{+0.6}_{-0.5} \times 10^{-3}$ \\ % vnei Si,S,norm
    % 0.88 (area ratio) = results_spec/20160624_src_bkg_nohack_rerun fit
    \textbf{0.885} & ${2.15}^{+0.07}_{-0.08}$ & ${2.30}^{+0.29}_{-0.12}$ & ${1.83}^{+0.11}_{-0.11}$
          & ${3.75}^{+0.18}_{-0.16}$ & ${3.42}^{+0.28}_{-0.25}$ & ${4.6}^{+0.5}_{-0.5} \times 10^{-3}$ \\
    \bottomrule
    \end{tabular}

    \quad
    \quad

    \begin{tabular}{@{}llllr@{}}
    \toprule
    Ratio & XRB local $kT$ & XRB $n_\mathrm{H}$ & XRB halo $kT$
           & $\chi^2_{\mathrm{red}} = \chi^2/\mathrm{dof}$ \\
     & (keV) & ($10^{22} \unit{cm^{-2}}$) & (keV) &  \\
    \midrule
    % 1 (hack) = results_spec/20160624_src_bkg_hack_eq_one_rerun fit
    \textbf{1} & ${0.262}^{+0.007}_{-0.007}$ & ${1.33}^{+0.08}_{-0.08}$ & ${0.75}^{+0.04}_{-0.03}$
          & 1.224 = 4611.94/3768 \\
    % 0.95 (hack) = results_spec/20160611_src_bkg_rerun, manually entered!
    \textbf{0.95} & ${0.262}^{+0.007}_{-0.007}$ & $1.32^{+0.08}_{-0.08}$ & $0.75^{+0.04}_{-0.03}$ % XRB
          & 1.213 = 4572.28/3768 \\
    % 0.88 (area ratio) = results_spec/20160624_src_bkg_nohack_rerun fit
    \textbf{0.885} & ${0.262}^{+0.007}_{-0.007}$ & ${1.32}^{+0.08}_{-0.08}$ & ${0.74}^{+0.03}_{-0.03}$
          & 1.206 = 4545.24/3768 \\
    \bottomrule
    \end{tabular}
\end{table*}

% Deluxetable variant

%\floattable
%\begin{deluxetable}{@{}rllllllllll@{}}
%    \rotate
%    \tablecaption{Motch MOS1 BACKSCAL ratio has very little effect on fits
%        \label{table:backscal-hack}}
%    \tablehead{
%          \colhead{Ratio}
%        & \colhead{$n_\mathrm{H}$ }%% $10^{22} \unit{cm^{-2}}$}
%        & \colhead{$kT$ }%% keV}
%        & \colhead{$\tau$ }%% $10^{10} \unit{s\;cm^{-3}}$}
%        & \colhead{Si}
%        & \colhead{S}
%        & \colhead{vnei EM }%% EM units}
%        & \colhead{XRB local $kT$ }%% keV}
%        & \colhead{XRB $n_\mathrm{H}$ }%% $10^{22} \unit{cm^{-2}}$}
%        & \colhead{XRB halo $kT$ }%% keV}
%        & \colhead{$\chi^2_{\mathrm{red}} = \chi^2/\mathrm{dof}$}
%        \\
%          \colhead{}
%        & \colhead{($10^{22} \unit{cm^{-2}}$)}
%        & \colhead{(keV)}
%        & \colhead{($10^{10} \unit{s\;cm^{-3}}$)}
%        & \colhead{(-)}
%        & \colhead{(-)}
%        & \colhead{(EM units)}
%        & \colhead{(keV)}
%        & \colhead{($10^{22} \unit{cm^{-2}}$)}
%        & \colhead{(keV)}
%        & \colhead{}
%      }
%
%    \startdata
%    % 1 (hack) = results_spec/20160624_src_bkg_hack_eq_one_rerun fit
%    \textbf{1}     & ${2.16}^{+0.08}_{-0.11}$ & ${2.29}^{+0.30}_{-0.12}$ & ${1.82}^{+0.14}_{-0.11}$
%          & ${3.75}^{+0.18}_{-0.16}$ & ${3.42}^{+0.29}_{-0.25}$ & ${4.5}^{+0.6}_{-0.5} \times 10^{-3}$
%          & ${0.262}^{+0.007}_{-0.007}$ & ${1.33}^{+0.08}_{-0.08}$ & ${0.75}^{+0.04}_{-0.03}$
%          & 1.224 = 4611.939/3768 \\
%    % 0.95 (hack) = results_spec/20160611_src_bkg_rerun, manually entered!
%    \textbf{0.95}  & $2.15^{+0.08}_{-0.11}$ & $2.29^{+0.30}_{-0.24}$ & $1.83^{+0.10}_{-0.11}$  % vnei nH, kT, Tau
%          & $3.75^{+0.18}_{-0.16}$ & $3.42^{+0.26}_{-0.25}$ & $4.5^{+0.6}_{-0.5} \times 10^{-3}$  % vnei Si,S,norm
%          & ${0.262}^{+0.007}_{-0.007}$ & $1.32^{+0.08}_{-0.08}$ & $0.75^{+0.04}_{-0.03}$ % XRB
%          & 1.213 = 4572.28/3768 \\
%    % 0.88 (area ratio) = results_spec/20160624_src_bkg_nohack_rerun fit
%    \textbf{0.885} & ${2.15}^{+0.07}_{-0.08}$ & ${2.30}^{+0.29}_{-0.12}$ & ${1.83}^{+0.11}_{-0.11}$
%          & ${3.75}^{+0.18}_{-0.16}$ & ${3.42}^{+0.28}_{-0.25}$ & ${4.6}^{+0.5}_{-0.5} \times 10^{-3}$
%          & ${0.262}^{+0.007}_{-0.007}$ & ${1.32}^{+0.08}_{-0.08}$ & ${0.74}^{+0.03}_{-0.03}$
%          & 1.206 = 4545.244/3768 \\
%    \enddata
%\end{deluxetable}

\clearpage
\section{Does fitting four versus five annuli make any difference?}

Answer: no, parameters are identical within error (and the change in parameter
values is much smaller than error).
\textbf{TODO: add a table here showing this result alone}.

I also verified by eye that
(1) all instrumental line normalizations agree to $<1\%$ (often $\lesssim0.1\%$),
(2) remnant model norms agree to $\lesssim 1$\%,
(3) soft proton norms and indices agree to $\lesssim 1\%$.
\textbf{TODO: I need to redo this}
In short, we are good to go.

\begin{figure}[!hb]
  \includegraphics[width={0.5\linewidth}]{fig/fig_fiveann_si-s_0087940201-mos.pdf}
  \includegraphics[width={0.5\linewidth}]{fig/fig_fiveann_mg-si-s_0087940201-mos.pdf} \\
  \figcaption{0087940201 MOS annuli spectra.  Left: Si, S free only.  Right: Mg, Si, S free.}
\end{figure}

\begin{figure}[!hb]
  \includegraphics[width={0.5\linewidth}]{fig/fig_fiveann_si-s_0087940201-pn.pdf}
  \includegraphics[width={0.5\linewidth}]{fig/fig_fiveann_mg-si-s_0087940201-pn.pdf}
  \figcaption{0087940201 PN annuli spectra.  Left: Si, S free only.  Right: Mg, Si, S free.}
\end{figure}

\begin{figure}[!hb]
  \includegraphics[width={0.5\linewidth}]{fig/fig_fiveann_si-s_0551000201-mos.pdf}
  \includegraphics[width={0.5\linewidth}]{fig/fig_fiveann_mg-si-s_0551000201-mos.pdf}
  \figcaption{0551000201 MOS annuli spectra.  Left: Si, S free only.  Right: Mg, Si, S free.}
\end{figure}

\clearpage

\begin{table}
    \centering
    \begin{tabular}{@{}rllllll@{}}
    \toprule
    \multicolumn{7}{c}{Annulus fit (20161019\_fourann.json)} \\
    \midrule
    Annulus & $kT$ & $\tau$ & Mg & Si & S & EM \\
     & (keV) & ($10^{10} \unit{s\;cm^{-3}}$) & (-) & (-) & (-) & $(\times 10^{11})$ \\
    \midrule
      $0$--$100\arcsec$ & ${2.08}^{+0.79}_{-0.44}$ & ${1.45}^{+0.39}_{-0.28}$ &      & ${6.77}^{+0.90}_{-0.78}$ & ${7.13}^{+2.75}_{-1.71}$ & ${5.70}^{+1.18}_{-1.11}$ \\
    $100$--$200\arcsec$ & ${1.53}^{+0.23}_{-0.17}$ & ${2.31}^{+0.43}_{-0.34}$ &      & ${5.25}^{+0.29}_{-0.36}$ & ${4.70}^{+0.64}_{-0.54}$ & ${7.73}^{+1.22}_{-1.22}$ \\
    $200$--$300\arcsec$ & ${1.41}^{+0.20}_{-0.17}$ & ${2.81}^{+0.68}_{-0.45}$ &      & ${3.31}^{+0.24}_{-0.21}$ & ${3.01}^{+0.37}_{-0.30}$ & ${8.63}^{+1.58}_{-1.71}$ \\
    $300$--$400\arcsec$ & ${1.28}^{+0.34}_{-0.11}$ & ${2.57}^{+1.47}_{-0.76}$ &      & ${2.94}^{+0.33}_{-0.30}$ & ${2.66}^{+0.89}_{-0.58}$ & ${3.87}^{+0.57}_{-0.79}$ \\
    \bottomrule
    \end{tabular}
    \tablecomments{
      $\nH = {2.51}^{+0.09}_{-0.10} \nHUnits$;
      $\mt{pgstat} = 19335.8$; $\chi^2_{\mt{red}} = 1.210 = 24296.5 / 20073$ ($\chi^2$/dof).
        Units as in Table~\ref{tab:src-fits}.}
\end{table}

\begin{table}
    \centering
    \begin{tabular}{@{}rllllll@{}}
    \toprule
    \multicolumn{7}{c}{Annulus fit (20161019\_fourann\_mg.json)} \\
    \midrule
    Annulus & $kT$ & $\tau$ & Mg & Si & S & EM \\
     & (keV) & ($10^{10} \unit{s\;cm^{-3}}$) & (-) & (-) & (-) & $(\times 10^{11})$ \\
    \midrule
      $0$--$100\arcsec$ & ${3.82}^{+2.70}_{-1.31}$ & ${1.37}^{+0.21}_{-0.08}$ & ${1.58}^{+0.54}_{-0.41}$ & ${8.20}^{+1.72}_{-1.25}$ & ${8.30}^{-8.30}_{-1.88}$ & ${3.24}^{+0.98}_{-0.71}$ \\
    $100$--$200\arcsec$ & ${2.04}^{+0.54}_{-0.31}$ & ${1.97}^{+0.33}_{-0.28}$ & ${1.28}^{+0.25}_{-0.19}$ & ${6.25}^{+0.77}_{-0.61}$ & ${5.62}^{+0.95}_{-0.73}$ & ${4.93}^{+1.14}_{-1.43}$ \\
    $200$--$300\arcsec$ & ${2.06}^{+0.43}_{-0.35}$ & ${2.15}^{+0.44}_{-0.27}$ & ${1.61}^{+0.26}_{-0.30}$ & ${4.34}^{+0.54}_{-0.41}$ & ${3.93}^{+0.62}_{-0.48}$ & ${4.74}^{+1.18}_{-1.26}$ \\
    $300$--$400\arcsec$ & ${1.72}^{+0.51}_{-0.33}$ & ${2.10}^{+0.73}_{-0.24}$ & ${1.07}^{+0.24}_{-0.20}$ & ${3.18}^{+0.54}_{-0.40}$ & ${2.89}^{+0.89}_{-0.61}$ & ${2.54}^{+0.70}_{-0.56}$ \\
    \bottomrule
    \end{tabular}
    \tablecomments{
      $\nH = {2.26}^{+0.11}_{-0.11} \nHUnits$;
      $\mt{pgstat} = 19298.1$; $\chi^2_{\mt{red}} = 1.207 = 24226.3 / 20069$ ($\chi^2$/dof).
        Units as in Table~\ref{tab:src-fits}.}
\end{table}

\begin{table}
    \centering
    \begin{tabular}{@{}rllllll@{}}
    \toprule
    \multicolumn{7}{c}{Annulus fit (20161019\_fiveann.json)} \\
    \midrule
    Annulus & $kT$ & $\tau$ & Mg & Si & S & EM \\
     & (keV) & ($10^{10} \unit{s\;cm^{-3}}$) & (-) & (-) & (-) & $(\times 10^{11})$ \\
    \midrule
      $0$--$100\arcsec$ & ${2.06}^{+0.76}_{-0.42}$ & ${1.46}^{+0.38}_{-0.29}$ &      & ${6.76}^{+0.91}_{-0.78}$ & ${7.12}^{+2.80}_{-1.73}$ & ${5.77}^{+1.17}_{-1.10}$ \\
    $100$--$200\arcsec$ & ${1.52}^{+0.22}_{-0.09}$ & ${2.32}^{+0.41}_{-0.32}$ &      & ${5.24}^{+0.41}_{-0.36}$ & ${4.69}^{+0.65}_{-0.53}$ & ${7.81}^{+1.21}_{-1.19}$ \\
    $200$--$300\arcsec$ & ${1.40}^{+0.19}_{-0.16}$ & ${2.83}^{+0.44}_{-0.46}$ &      & ${3.31}^{+0.21}_{-0.21}$ & ${3.00}^{+0.36}_{-0.30}$ & ${8.72}^{+1.60}_{-1.34}$ \\
    $300$--$400\arcsec$ & ${1.27}^{+0.32}_{-0.21}$ & ${2.57}^{+1.50}_{-0.76}$ &      & ${2.93}^{+0.34}_{-0.30}$ & ${2.67}^{+0.90}_{-0.56}$ & ${3.91}^{+0.80}_{-0.74}$ \\
    $400$--$500\arcsec$ & ${3.07}^{-3.07}_{-1.08}$ & ${0.68}^{+0.82}_{-0.13}$ &      & ${2.77}^{+0.89}_{-1.12}$ & ${9.88}^{-9.88}_{-8.33}$ & ${0.67}^{+0.21}_{-0.24}$ \\
    \bottomrule
    \end{tabular}
    \tablecomments{
      $\nH = {2.51}^{+0.09}_{-0.10} \nHUnits$;
      $\mt{pgstat} = 25741.5$; $\chi^2_{\mt{red}} = 1.203 = 31637.1 / 26290$ ($\chi^2$/dof).
        Units as in Table~\ref{tab:src-fits}.}
\end{table}

\begin{table}
    \centering
    \begin{tabular}{@{}rllllll@{}}
    \toprule
    \multicolumn{7}{c}{Annulus fit (20161019\_fiveann\_mg.json)} \\
    \midrule
    Annulus & $kT$ & $\tau$ & Mg & Si & S & EM \\
     & (keV) & ($10^{10} \unit{s\;cm^{-3}}$) & (-) & (-) & (-) & $(\times 10^{11})$ \\
    \midrule
      $0$--$100\arcsec$ & ${3.94}^{+2.40}_{-1.47}$ & ${1.37}^{+0.21}_{-0.08}$ & ${1.61}^{+0.51}_{-0.44}$ & ${8.29}^{+1.58}_{-1.34}$ & ${8.38}^{-8.38}_{-1.94}$ & ${3.16}^{+1.11}_{-0.62}$ \\
    $100$--$200\arcsec$ & ${2.08}^{+0.48}_{-0.36}$ & ${1.94}^{+0.37}_{-0.24}$ & ${1.30}^{+0.23}_{-0.21}$ & ${6.30}^{+0.70}_{-0.68}$ & ${5.68}^{+0.87}_{-0.80}$ & ${4.82}^{+1.30}_{-0.85}$ \\
    $200$--$300\arcsec$ & ${2.08}^{+0.38}_{-0.38}$ & ${2.14}^{+0.46}_{-0.26}$ & ${1.62}^{+0.23}_{-0.22}$ & ${4.38}^{+0.47}_{-0.48}$ & ${3.97}^{+0.57}_{-0.53}$ & ${4.65}^{+1.31}_{-0.77}$ \\
    $300$--$400\arcsec$ & ${1.73}^{+0.48}_{-0.34}$ & ${2.10}^{+0.75}_{-0.46}$ & ${1.08}^{+0.23}_{-0.20}$ & ${3.21}^{+0.49}_{-0.43}$ & ${2.92}^{+0.85}_{-0.64}$ & ${2.50}^{+0.77}_{-0.50}$ \\
    $400$--$500\arcsec$ & ${10.00}^{-10.00}_{-2.96}$ & ${1.32}^{+0.52}_{-0.53}$ & ${1.84}^{+2.05}_{-0.83}$ & ${2.67}^{+2.39}_{-0.95}$ & ${3.53}^{-3.53}_{-2.13}$ & ${0.34}^{+0.16}_{-0.14}$ \\
    \bottomrule
    \end{tabular}
    \tablecomments{
      $\nH = {2.25}^{+0.13}_{-0.10} \nHUnits$;
      $\mt{pgstat} = 25699.6$; $\chi^2_{\mt{red}} = 1.201 = 31558.8 / 26285$ ($\chi^2$/dof).
        Units as in Table~\ref{tab:src-fits}.}
\end{table}


\clearpage
\section{Frequently asked questions (or, nitpickables)}

Caveats, disclaimers, and possible points of contention that are not
discussed in the main text for brevity, but should be itemized and reviewed.

\begin{itemize}
    \item \textbf{Why are we using Chandra-derived extragalactic X-ray background
        parameters instead of XMM-derived parameters?}
        Simply, I am more inclined to trust Chandra background modeling and
        background subtraction.
        The \citet{hickox2006} values agree to $\abt 10\%$ of values derived
        from \textit{XMM-Newton}, \textit{Swift}, \textit{ASCA}, and
        \textit{ROSAT} studies
        \citep{chen1997, kushino2002, de-luca2004, moretti2009}.
        \textbf{Do we need to adjust the normalization because XMM resolves
        fewer extragalactic sources than Chandra?}
        If I recall correctly from previous reading (forgot which source(s)) --
        small-number statistics prevail, so a few point sources removed don't
        matter.  It could be more significant for wide-field studies?
        \textbf{Irrelevant now that we have corrected extragalactic background normalization for point source removal -- but not dead yet}
    \item \textbf{Why are halo emission and extragalactic background subject to
        the same absorption?}
        Yes, this is unrealistic.  Halo emission should be absorbed by some
        distance-and-density-averaged column, whereas the extragalactic
        background is attenuated by the complete galactic column.
        But, our X-ray background is decently well fit by our current model,
        and we don't see any need to introduce more parameters.
        The absorptions would only differ by a factor of $\abt 2$ for each
        component, and the discrepancy might be folded into unabsorbed local
        emission or the soft proton components.
        However we have not explored this in depth, and it is possible that our
        assessment of the background would change.
        \textbf{Irrelevant now that we have separated absorption}
\end{itemize}


\end{document}
