\documentclass[preprint2,tighten,trackchanges]{aastex6}
%\documentclass[iop, tighten, apj, numberedappendix]{emulateapj} % Still more compact than aastex

% Customize figures/sizing for 2 column vs. manuscript printout
%%\usepackage{etoolbox}
%%\newtoggle{manuscript}
%%\toggletrue{manuscript}  % Set TRUE if using manuscript / 1-col layout
%%%\togglefalse{manuscript}  % Set FALSE if using 2-col layout

\shorttitle{G309 (\today)}  % <~ 44 char
\shortauthors{Alpha, Beta (\today)}  % Max three
\slugcomment{Draft, \today}

% Packages and commands
%\usepackage{amsmath}  % Included in aastex
\usepackage{booktabs}
\usepackage{hyperref}
\usepackage[normalem]{ulem}  % Just to get strikeout text

% My "standard" TeX aliases
\newcommand*{\mt}{\mathrm}
\newcommand*{\unit}[1]{\;\mt{#1}}  % vemod.net/typesetting-units-in-latex
\newcommand*{\abt}{\mathord{\sim}} % tex.stackexchange.com/q/55701
\newcommand*{\ptl}{\partial}
\newcommand*{\del}{\nabla}
\newcommand*\mean[1]{\bar{#1}}
\renewcommand{\vec}[1]{\mathbf{#1}}  % Bold vectors
\newcommand*{\tsup}{\textsuperscript}

% Paper-specific commands
\newcommand*{\nH}{N_{\mathrm{H}}}
\newcommand*{\nHUnits}{\times 10^{22} \unit{cm^{-2}}}
\newcommand*{\TauUnits}{\unit{s\;cm^{-3}}}
\newcommand*{\AV}{A_{\mathrm{V}}}
%\newcommand*{\kB}{k_{\mathrm{B}}}
\newcommand*{\kB}{k}  % vacillating on how to format Boltzmann's constant
\newcommand*{\EM}{\mathrm{EM}}  % Emission measure
\defcitealias{rakowski2001}{RHS01}
\defcitealias{gaensler1998-g309}{GGM98}

\begin{document}

\title{Spatially Resolved Ejecta in G309.2-0.6}

% Not quite in line with recommended aastex style
\author{
A. Alpha, B. Beta%\altaffilmark{1}
}

\affil{
%\tsup{1}
Smithsonian Astrophysical Observatory, 60 Garden Street MS-70, Cambridge, MA 02138, USA
}

%\received{receipt date}
%\revised{\today}
%\accepted{acceptance date}


\begin{abstract}
We image and fit spectra of G309.2-0.6 from archival XMM-Newton data.
The remnant is ejecta-dominated and hotter than expected for its HI-inferred
distance and hence physical size.

Sketch of a writeup, and a lot of scribbled notes.
\end{abstract}

% (!) no longer used by AAS as of late February, 2016
\keywords{ISM: supernova remnants ---
    ISM: individual objects (SNR G309.2-0.6) ---
    X-rays: ISM
}

% =============================================================================
% Introduction
% =============================================================================
\section{Introduction} \label{sec:intro}

Can we use integrated or spatially resolved X-ray spectra to type
ejecta-dominated or ED-ST transitioning supernova remnants (SNRs)?
In some cases, yes:
\begin{itemize}
    \item coarse abundance ratios of O, Fe, Si/S compared to model SN yields \citep{hughes1995}  % 1995ApJ...444L..81H
    \item high resolution line spectroscopy of O lines in 1E0102.2-7219 \citep{flanagan2004}  % 2004ApJ...605..230F
    \item Ni/Fe and Mn/Fe ratios in 3C397 \citep{yamaguchi2015}
    \item Fe K line centroid energies \citep{yamaguchi2014-iron, patnaude2015}
\end{itemize}
Most individual remnant studies attempt to guess out the progenitor SN type
from ejecta abundances and other pieces of information.
Alternatively, \citet{chevalier2005} attempts to type SNe from SNR CSM
emission.
Optical studies of CSM knots are possible (see references in
\citet{katsuda2015}).

Badenes+ 2007: Table 4, ionization timescales of Si/S vs. Fe in young Ia remnants.
Appendix is also sort of helpful.
I'd cite this on stratification... and discussion of Fe absence.

Some studies (i.e. those not emphasizing line centroids, widths, ratios) assume
that abundances inferred from NEI plasma fits are a reasonable probe of ejecta
composition.  Much depends on:
\begin{itemize}
    \item Ejecta structure - which elements are stratified or well mixed, and
        what conditions (ambient density, progenitor type, explosion mechanism)
        control ejecta structure.
        See: Ashall+ 2016 arxiv:1608.05244, stratification in SN1986G.
    \item ambient interstellar or circumstellar medium (ISM, CSM).
        was was the extent and structure of progenitor mass loss, if any?
        (continuous wind, pulses of material, ?)
        nearby molecular clouds, other remnants.
        Interesting recent paper on remnant deformation near GC.
        Yalinewich+ 2016 (arXiv:1608.05904).
    \item Remnant's own dynamics - evolutionary state (ejecta-dominated, Sedov;
        transitioning).
\end{itemize}

One might consider forward folding models and comparing with observed spectra.
This is basically the approach of \citet{hughes1995}, \citet{badenes2003},
\citet{rakowski2006-g337}, \citet{patnaude2015}.
Some models show observable differences in line ratios, elemental abundances
and the like.  Others are less clear.

Question: why do we not model Ia ejecta with a metal plasma?  No expectation of
H emission unless from shocked ISM.
Similar q for CC.  Would not expect that much H...

We present a spatially resolved X-ray image and CCD-resolution spectra of
G309.2-0.6 based on two archived XMM-Newton observations totalling
$98 \unit{ks}$, with about half usable after flare removal.

We fit the CCD spectra of morphologically distinct regions of the remnant to
trace variation in ejecta abundances, plasma temperature, and ionization
states.

Individual object study to build information for more ambitious global studies
of remnant properties and interaction with the ISM -- not undertaken here.
Just one piece of a puzzle.

\subsection{Previous observations of G309.2-0.6}

\begin{figure*}[]
    \plotone{fig/fig_snr_xmm_most_invert.pdf}
    %\caption{Broadband 0.8--3.3 keV image of SNR G309.2-0.6.}
    % figcaption is an aastex alias -- cannot use with emulateapj
    \figcaption{Left: XMM broadband 0.8--3.3 keV image of SNR G309.2-0.6,
        smoothed and log scaled with $1.4 \unit{GHz}$ radio contours
        (0.01, 0.03, 0.1, 0.3 Jy).
        Right: $1.4 \unit{GHz}$ image, arcsinh scaled, from the Molonglo
        Observatory Synthesis Telescope (MOST) Supernova Remnant Catalogue
        \citep{whiteoak1996}.  Resolution: $\abt 43\arcsec$, sensitivity $2
        \unit{mJy/beam}$.}
    \label{fig:snr}
\end{figure*}

% Radio remnant observations and morphology
\objectname{SNR G309.2-0.6} (J2000 RA 13h46m30s, dec. $-62\arcdeg 54\arcmin 00\arcsec$) was
discovered in Molonglo $408 \unit{MHz}$ and Parkes $5000 \unit{MHz}$ surveys
\citep{day1969, clark1973, green1974, clark1975} and has since been observed
multiple times by southern radio telescopes \citep{caswell1981, kesteven1987,
whiteoak1996}.
% Day, Thomas, Goss (1969 Au. J. Phys. Astrophys. Suppl.)
% flagged G309.2-0.6 as a complex source, but did not identify it as a remnant
% or explore further.  Molongo/Parkes surveys may have occurred at the same
% time.  And, yes, it is confused (at the edge) with RCW 80.
\citet{gaensler1998-g309} (hereafter, \citetalias{gaensler1998-g309}) performed the most
comprehensive radio study to date with Australia Telescope Compact Array (ATCA)
$1.344 \unit{GHz}$ continuum and $1.420 \unit{GHz}$ HI observations at angular
resolution $24\arcsec$.
The radio remnant is a circular shell with two bright lobes to the NE and SW,
hereafter referred to as ``ears'' (Fig \ref{fig:snr}).
% Flux density -- omit, could incorporate later if relevant
% the flux density at $843 \unit{MHz}$ is $6 \unit{Jy}$ \citep{whiteoak1996}.
% Spectral index notes -- omit for now, could incorporate later if relevant
% Comment: spectral indices ~ 0-0.3 are more typical of pulsar powered remnants
% steeper (0.3,0.4-1) more typical for most SNRs.
%The radio spectral index is $0.53 \pm 0.09$ derived by \citet{gaensler1998-g309},
%excluding single-dish Parkes observations at 2700 and 5000 Mhz to avoid
%confusion of source flux with RCW80; a shallower index $0.36 \pm 0.11$ is
%obtained if single dish measurements (with authors' attempted corrections for
%source confusion) are included.
HI absorption favors remnant distance between $5$--$14 \unit{kpc}$ based on
absorption to the galactic rotation curve's tangent point,
$v_{\mt{LSR}} \sim -50 \unit{km/s}$, and the absence of absorption above
$v_{\mt{LSR}} \sim +40 \unit{km/s}$ \citepalias{gaensler1998-g309}.
% The galactic rotation curve is from \citet{fich1989} with galactic center
% distance $R_0 = 8.5 \unit{kpc}$ and local circular velocity
% $\Theta = 220 \unit{km/s}$.

% ATCA data -- cannot access at http://www.atnf.csiro.au/research/HI/sgps/queryForm.html
% SGPS survey -- taken 1998-2000.
% Emailed a feedback query on 2016 April 13... probably won't have much luck.

% X-ray remnant observation(s) and morphology
X-ray emission was discovered in an Advanced Satellite for Cosmology and
Astrophysics (ASCA) survey of small remnants by \citet{rakowski2001}
(hereafter, \citetalias{rakowski2001}).
The X-ray remnant sits within the radio shell and is brightest towards the
north (Figure \ref{fig:snr}).
A faint X-ray arc on the remnant's northeast limb coincides with the radio
shell's limb and is most apparent in our $0.8$--$1.4 \unit{keV}$ band image.

% Other wavelengths
The field of G309.2-0.6 has been surveyed in many wavelengths.
No obvious H$\alpha$, [S II], or [O III] filaments appear in 3x 400 s exposures
taken with the 0.6 m Curtis-Schmidt telescope at Cerro Tololo Inter-American
Observatory in 2001 January (PI: P. F. Winkler; observers: Gokas, Smith,
Winkler).
% I think it's this proposal (but not sure)
% NOAO proposal 2001A-0331
% http://adsabs.harvard.edu/abs/2001noao.prop..331W
% http://www.noao.edu/dir/q_rep/Q2_Apr_30/FY2001%20Q2%20Jan-Mar%20NOAO%20Quarterly%20Report.pdf
% Some H-alpha emission associated with foreground cluster;
% field is dominated by RCW 80 to northeast.
\footnote{\url{http://sites.middlebury.edu/snratlas/g309-2-0-6/}}
No $1720.5 \unit{MHz}$ OH maser emission is observed above $40 \unit{mJy}$
\citep{green1997}.
% Mopra CO survey - no public data yet, in works...
No obvious infrared emission appears in Spitzer
$3.6$--$8 \unit{{\mu}m}$ IRAC GLIMPSE or $24 \unit{{\mu}m}$ MIPSGAL mosaics
\citep{churchwell2009, carey2009}.
Dust emission at $100 \unit{{\mu}m}$ (IRAS/ISSA, COBE/DIRBE) shows a
gradient across the remnant, although association with G309 is unknown.
% Interpretation:
% J,H,Ks bands = 1.2, 1.6, 2.2 micron (2MASS) -- cool stars
% 3.6--8 micron (Spitzer) -- cool stars
% 12 micron --
% 22/24 micron (WISE, Spitzer) -- dust
% 70 micron -- more dust (useful for dust SED stuff)
% Melange of potentially useful stuff
The remnant is not detected in GeV by Fermi-LAT \citep{acero2016} or TeV by
HESS \citep{bochow2011}.
The absence of H$\alpha$ and infrared emission is consistent with previous
data reviewed by \citetalias{gaensler1998-g309}.


\subsection{Environment}

% HD 119682 - previous interest and subsequent disassociation
A bright ROSAT source (1WGA J1346.5--6255) within G309.2-0.6 is the foreground
Be star HD 119682.
The source's position within a bilobed radio remnant led
\citetalias{gaensler1998-g309} and \citetalias{rakowski2001} to suggest that
G309.2-0.6 might be a jet- or outflow-driven remnant, similar to the SS 443 /
W50 system.
% bilobed beats bilobate on Google by 278 to 66 (cf. Google ngrams)
But, HD 119682 is likely a member of the open cluster NGC 5281 at a distance
$\abt 1.4 \unit{kpc}$, based on X-ray and optical astrometry, cluster proper
motions, and X-ray spectrum absorption \citep{rakowski2006-star, safi-harb2007,
torrejon2013}.
% The X-ray fitted absorption of $\nH \sim 0.2 \times 10^{22} \unit{cm^{-2}}$
% is consistent with the inferred distance to HD 119682
% \citep{rakowski2006-star, safi-harb2007, torrejon2013}.

% Gum 48d / RCW 80
The H II region Gum 48d (RCW 80), north of G309.2-0.6, is well traced in HI,
polycyclic aromatic hydrocarbon (PAH), and warm dust emission
\citep{karr2009}.
The distance to Gum 48d is $\abt 3.5\unit{kpc}$ based on ionized gas emission
velocities measured from the central star system HR 5171 \citep{karr2009}.
Given this distance, and the absence of evidence for interaction with G309,
we assume that Gum 48d also lies in the foreground of G309.2-0.6.
We also ignore the north-south wisp of radio emission that visually joins the
radio remnant and H II region in Figure~\ref{fig:snr}.

% Centaurus CO cloud -- push this to discussion...
\citet{saito2001} imaged
an extensive molecular CO complex in ${}^{12}$CO, ${}^{13}$CO, and C${}^{18}$O
J=(1-0) transitions at $v_\mt{LSR} = -64$ to $-36 \unit{km/s}$.
\textbf{TODO} this would need some CO to ISM density conversion estimate so
that we could estimate a density gradient across the remnant, if it were
associated with this cloud (which we think it is not).
For this see Bolatto, Wolfire and Leroy (2013).
It is clear that remnants expanding into a density gradient need not look that
asymmetric \citep{hnatyk1999, williams2013}, so \textbf{this is speculative and may be
outright wrong}.

% Here's a lovely optical photograph, clearly showing the bright H II region and
% the bluer star cluster NGC 5281 to the south:
% https://it.wikipedia.org/wiki/File:RCW_80.jpg

% Mopra CO J(1-0) survey, resolution 30 arcsec and 0.1 km/s, would be huge!
% Looking at our crude rotation curve -- beyond tangent point, 100 km/s per 10kpc
% is the approximate slope; 0.1 km/s means we have a distance resolution of order
% 10 parsecs, comparable to or smaller than this remnant size.
% Compare, Dame survey has beamwidth ~ 1/8 deg. = 7.5 arcmin = 450 arcsec


% TODO: check out ATCA J134649--625235, mentioned in Gaensler

% Summary / wrap-up
Here, we extend the radio and X-ray studies of \citetalias{gaensler1998-g309} and
\citetalias{rakowski2001} with archived XMM-Newton observations that spatially
resolve the X-ray morphology of G309.2-0.6.
We confirm the previous


% =============================================================================
% Observations + Reduction
% =============================================================================
\section{Observations and Data Reduction} \label{sec:obs}

XMM-Newton observed G309.2-0.6 for $97.7 \unit{ks}$ in two pointings
(Table~\ref{tab:obs}), which we hereafter reference by year.
The 2009 observation targeted the Be star HD 119682, but also captured
G309.2-0.6 on the MOS1/2 and PN detectors.
After background flare filtering, our good exposure times are $47 \unit{ks}$
for MOS1/2 and $27 \unit{ks}$ for PN.

% More verbose version
%XMM-Newton has observed G309.2-0.6 for $97.73 \unit{ks}$ in two pointings.
%Obsid 0087940201 (PI J. P. Hughes) was $40.46 \unit{ks}$ on 2001 August 28
%with MOS1/2 in Full Frame mode, PN in Extended Full Frame mode, and XMM's
%``thick'' optical filter.
%Obsid 0551000201 (PI C. Motch) was $57.27 \unit{ks}$ on 2009 March 6--7
%with MOS1/2 in Full Frame mode, PN in Large Window mode, and XMM's ``medium''
%optical filter.
%The Motch pointing targeted the foreground Be star HD 119682, but also
%captured G309.2-0.6 on the MOS1/2 and PN detectors.

% TODO I have not thought at all about the effects of optical loading on these
% observations...

\begin{table*}
    \centering
    \caption{XMM Observations of G309.2-0.6\label{tab:obs}}
    \begin{tabular}{@{}lrrrlrlr@{}}
        \toprule
        Obs. ID & Dur. & $t_{\mt{\,live},\,\mt{MOS}}$ & $t_{\mt{\,live},\,\mt{PN}}$
            & Date & Rev. & Filter & PI \\
        \midrule
        0087940201 & $40.5$ & $25.3$ & $18.0$ & 2001 August 28 & 315 & Thick & Hughes \\
        0551000201 & $57.3$ & $21.9$ &  $9.0$ & 2009 March 6--7 & 1692 & Medium & Motch \\
        \bottomrule
    \end{tabular}
    \tablecomments{Durations in kiloseconds (ks).
    Good duration $t_{\mt{live}}$ is central CCD live time after flare
    filtering; MOS value averages MOS1 and MOS2 times ($\sim 0.6 \unit{ks}$
    difference in both obsids).
    Rev. is XMM-Newton orbit (revolution) number.}
\end{table*}

% TODO how to refer to each observation in a simple way?
% for now just using ``Hughes'' and ``Motch''
% Possibly, just refer to by year: 2001 and 2009.  Obsid is not that useful
% TBD: kind of inconsistent in text currently...

% Current good time values (2016 April 21. Somewhat old extraction, pipeline
% last re-run in Feb or Mar after SAS v15 release)
%   0087940201 mos1S001 livetime 25.016 ks, ontime 25.290 ks
%   0087940201 mos2S002 livetime 25.621 ks, ontime 25.891 ks
%   0087940201 pnS003   livetime 18.011 ks, ontime 20.979 ks
%   0551000201 mos1S001 livetime 21.621 ks, ontime 21.989 ks
%   0551000201 mos2S002 livetime 22.167 ks, ontime 22.545 ks
%   0551000201 pnS003   livetime  8.981 ks, ontime  9.720 ks

% Event lists, soft proton light curve filtering
We reduce the data with XMM's Science Analysis System (SAS) v15.0.0 and
Extended SAS (ESAS) v5.9 \citep{snowden2008, kuntz2008}.
\footnote{\url{http://heasarc.gsfc.nasa.gov/docs/xmm/esas/cookbook/xmm-esas.html}}
Strong soft proton flares are filtered by ESAS tasks \texttt{mos-filter} and
\texttt{pn-filter}, which fit a Gaussian to a histogram of time-binned count
rates for each exposure and discard time intervals where the count rate is
$1.5\sigma$ above the fitted Gaussian mean.  % TODO - sentence needs retooling
We further inspect the $2.5$--$12 \unit{keV}$ field-of-view light curves and
manually cut some brief good-time intervals surrounded by higher count rate
flares.
About half to two-thirds of each observation is lost to flares, with more
severe loss for PN exposures.
The resulting live times (good time intervals less CCD readout time) are
given in Table~\ref{tab:obs}.

% Point sources
We remove point sources with the ESAS task \texttt{cheese}
to a limiting flux of $10^{-14} \unit{erg\;cm^{-2}\;s^{-1}}$.
\textbf{TODO: merge pt source removal lists from both obsids}  %TODO
Of note, HD 119682 requires an exclusion mask $\sim 1\arcmin$ in radius.
\textbf{TODO: create images with and without pt source exclusions...}

% CCD exclusions
In obs. ID 0551000201, we excluded two MOS CCDs and the entire PN exposure from
our spectral analysis.
MOS2 CCD5 showed anomalously high soft X-ray background noise in its corner
counts \citep[cf.][]{kuntz2008}; MOS1 CCD6 was disabled by a presumed
micrometeorite in 2005.
The PN in Large Window mode does not collect corner counts, so we are unable to
estimate the detector background using ESAS; moreover, the good time is
$\lesssim 10\%$ of our available exposures due to severe flare contamination.


\subsection{Spectrum Extraction and Modeling}

% Spectrum and QPB creation
Spectra are created from ``good'' single and double events (i.e.,
\texttt{PATTERN} $\leq 12,4$ for MOS, PN respectively; \texttt{FLAG} $= 0$ for
both MOS and PN), using ESAS tasks \texttt{\{mos,pn\}-spectra}.
We fit all spectra in XSPEC 12.9.0d \citep{arnaud1996}
using T{\"u}bingen-Boulder ISM elemental abundances \citep{wilms2000}
and galactic absorption model \texttt{tbnew} v2.3.2
\footnote{http://pulsar.sternwarte.uni-erlangen.de/wilms/research/tbabs/},
which is the latest version of the gas absorption model of \citet{wilms2000}.

% QPB
The detector background is modeled by ESAS tasks \texttt{\{mos,pn\}\_back}
by (1) computing a quiescent spectrum from unexposed CCD corner counts,
(2) augmenting the quiescent spectrum with corner counts from public
observations with similar count rates and spectral hardness, and (3) scaling
the augmented quiescent spectrum shape to that expected for the source region
of interest, using the ratio of quiescent spectra across corner / source
regions from XMM's filter wheel closed (FWC) observations and assuming that the
ratio of spectra across a given MOS or PN chip is time invariant
\citep[Sec. 3.4]{kuntz2008}.
% TODO this can be shortened or maybe omitted...

% Instrumental lines
We model instrumental lines as zero-width, fixed-energy Gaussians and assume
that relative line strengths (i.e., line ratios) in a given CCD region are the
same between FWC and observation data.
The FWC spectra are well fit by Gaussian lines on a broken power-law continuum.
For MOS exposures, we fit $1.49$ (Al) and $1.75 \unit{keV}$ (Si) lines.
For PN, we fit seven lines: $1.49$ (Al), $4.54$ (Ti), $5.44$ (Cr), $7.49$ (Ni),
$8.05$ (Cu), $8.62$ (Zn), and $8.90 \unit{keV}$ (Cu K$\beta$).
% The PN Ti and Cr lines are not noticeable in observation spectra, but are
% clearly visible in FWC spectra.
Our spectrum fits adopt line normalizations from FWC spectrum fits and then
vary a single constant pre-factor for all instrumental lines in a given
exposure.
% TODO a caveat.  \ion{Si}{1} [Si I] K alpha is ~1.74 keV vs. \ion{Si}{13} [Si XIII] at 1.85 keV
Some spectrum fits could be improved by slightly varying instrumental line
energies or broadening the line widths.
However, adding further free parameters will slow error calculation.
Systematic error from our instrumental line fits does not weight fits
significantly, based on weighted and scaled $\Delta \chi$ residuals.

% SWCX
We neglect solar wind charge exchange (SWCX) emission, which can affect
XMM-Newton observations \citep{snowden2004, carter2011}.
Neither observation shows obvious \ion{O}{7} or \ion{O}{8} line emission associated with
SWCX.
Hughes' 2001 observation was not flagged by \citet{carter2011} as containing
time-variable soft X-ray emission indicative of exospheric SWCX.
% TODO what is the difference between these types of swcx?
Modeled magnetospheric SWCX emission decreases by a factor of $10$ over the
duration of Hughes' 2001 observation, based on XMM-Newton Guest Observer
Facility (GOF) tool that combines a magnetosphere model \citep{spreiter1966} with
Advanced Composition Explorer (ACE) solar wind data.
\footnote{\url{https://heasarc.gsfc.nasa.gov/docs/xmm/scripts/xmm_trend.html}}
We extracted 0087940201 MOS1 spectra at early and late times in the observation
and, by eye, saw no obvious differences in soft X-ray emission.
% TODO I will be re-doing this.
For the 2009 observation, the XMM GOF magnetosphere model predicts weaker SWCX
emission compared to the 2001 observation, so we simply neglect SWCX in both
observations.
Any unmodeled SWCX should be folded into our X-ray background model, albeit
imperfectly.
% NOTE: see my notes from 2016 Jan 13 on spectrum cuts to evaluate SWCX
% 0551000201 was observed March 2009
%   Carter/Sembay analysis performed August 2009, so 0551000201 was likely
%   still in its proprietary period
% 0087940201 is not listed in Carter/Sembay, but meets basic selection criteria
% (namely, MOS1/MOS2 operating in full frame mode

% Residual soft proton contamination
Residual soft protons incident on EPIC detectors dominate the X-ray background
and are not removed by flare filtering due to their slow time variation.
We model residual soft protons by a power law that bypasses CCD response and
effective area functions, following ESAS procedure.


% =============================================================================
% Spectrum fits
% =============================================================================
\section{Integrated and spatially resolved spectral fits} \label{sec:spec}

\subsection{Integrated spectrum} \label{sec:src-bkg}

\begin{figure*}[]
    \epsscale{0.8}
    \plotone{fig/fig_snr_regs-fiveann_invert.pdf}
    \figcaption{Unsmoothed 0.8--3.3 keV image of SNR G309.2-0.6 with
    $100\arcsec$ wide annuli and background region;
    the magenta $400\arcsec$ circle demarcates the extraction region for
    integrated remnant fits.}
    \label{fig:regions}
\end{figure*}

% Introduce -- joint source + background fit, source model, regions & spectra
We jointly fit integrated remnant (0--400\arcsec) and background
(510--700\arcsec) spectra (Figure~\ref{fig:regions}).
We model the integrated remnant as a single temperature, non-equilibrium
ionization (NEI) plasma (XSPEC model \texttt{tbabs\_new * vnei}, after
\citet{hamilton1983}).  % Better reference, phrasing?
Our X-ray sky background model comprises an unabsorbed local bubble thermal
plasma \textbf{CITATION}, an absorbed thermal plasma for diffuse galactic ridge
emission \textbf{CITATION: worrell, ebisawa, revnivtsev}, and an absorbed extragalactic background power law
(XSPEC model \texttt{apec + tbnew\_gas*(apec + powerlaw)}).
We allow the background plasma temperatures and absorbing column to vary, but
fix the extragalactic power law to photon index $1.4$ and normalization
$10.9 \unit{cm^{-2}\, s^{-1}\, sr^{-1}\, keV^{-1}}$ \citep{hickox2006}.
% TODO text on extragalactic power law flux fix could be shortened
The extragalactic power law normalization is decreased by 39\% to account for
our bright point source flux cut-off of $10^{-14} \unit{erg\;cm^{-2}\;s^{-1}}$
in the 0.4-7.2 keV band, as done by \citet{revnivtsev2009}.
The flux cut-off for extragalactic sources is
$1.46 \times 10^{-14} \unit{erg\;cm^{-2}\;s^{-1}}$ assuming
absorbing column $2 \nHUnits$ and power law spectrum with index $1.4$.
We then integrate the log(N)-log(S) relation of \citet{moretti2003} to obtain
the excluded flux.

The fitted model components, with representative source and background spectra,
are shown in Figure~\ref{fig:src-bkg-fits}.
Our best fit parameters are given in Table~\ref{tab:src-fit}.

\begin{figure*}[!ht]
    \plotone{fig/fig_src_bkg_0087940201-mos1_top.pdf}
    \plotone{fig/fig_src_bkg_0087940201-mos1_bottom.pdf}
    \figcaption{Integrated source and background spectra with fits for
        0087940201 MOS1 (one of five exposures fitted)}
    \label{fig:src-bkg-fits}
\end{figure*}

\begin{table*}[!ht]
    \centering
    \caption{Integrated source and background fits\label{tab:src-fit}}
    \scriptsize
    \begin{tabular}{@{}lllllllllllr@{}}
  \toprule
  \multicolumn{7}{c}{Remnant} & \multicolumn{3}{c}{Background} & \\
  \cmidrule(r){1-7} \cmidrule(lr){8-10}
  $N_\mt{H}$ & $kT$ & $\tau$ & Mg & Si & S & EM
    & $\kB T_{\mt{local}}$ & $N_\mt{H}$ & $\kB T_{\mt{halo}}$
    & $\chi^2_{\mt{red}}$ & $\chi^2 / \mt{dof}$ \\
  ($\times 10^{22}$) & (keV) & ($\times 10^{10}$) & (-) & (-) & (-) & ($\times 10^{14}$)
    & (keV) & ($\times 10^{22}$) & (keV) &  &  \\
  \midrule
  ${2.15}^{+0.07}_{-0.08}$ & ${2.3}^{+0.3}_{-0.1}$ & ${1.83}^{+0.12}_{-0.11}$ & \textbf{1 (fixed)}       & ${3.75}^{+0.18}_{-0.16}$ & ${3.42}^{+0.28}_{-0.25}$ & $\left(4.6^{+0.5}_{-0.5}\right)\times10^{-3}$
    & ${0.26}^{+0.007}_{-0.007}$ & ${1.32}^{+0.08}_{-0.08}$ & ${0.74}^{+0.03}_{-0.03}$ & 1.206 & 4545.24 / 3768 \\
  ${2.03}^{+0.10}_{-0.09}$ & ${2.9}^{+0.5}_{-0.4}$ & ${1.75}^{+0.11}_{-0.10}$ & ${1.24}^{+0.11}_{-0.10}$ & ${4.2}^{+0.3}_{-0.3}$ & ${3.9}^{+0.4}_{-0.4}$ & $\left(3.5^{+0.6}_{-0.4}\right)\times10^{-3}$
    & ${0.26}^{+0.01}_{-0.01}$ & ${1.33}^{+0.08}_{-0.08}$ & ${0.74}^{+0.03}_{-0.03}$ & 1.202 & 4528.51 / 3767 \\
  \bottomrule
\end{tabular}

    \tablecomments{Units:
        equivalent hydrogen column density $N_H$, $10^{22} \unit{cm^{-2}}$;
        ionization timescale $\tau$, $10^{10} \unit{s\;cm^{-3}}$;
        emission measure (EM), $10^{14} \unit{cm^{-5}}$.
        The (scaled) emission measure is the usual XSPEC norm, defined as
        $(4 \pi D^2)^{-1} \int n_{\mt{H}} n_{\mt{e}} dV$
        where $D$ is source distance and all quantities are in CGS units.
        Elemental abundances are relative to ISM values of \citet[Table 2]{wilms2000}.
    }
\end{table*}

% Parameter discussion (TODO this will need rewriting after I re-run fits with
% new x-ray background model.)
Our galactic ridge component has $2$--$10 \unit{keV}$ surface brightness
$3.6 \times 10^{-12} \unit{erg\;cm^{-2}\;s^{-1}\;deg.^{-2}}$, somewhat lower
than measured brightnesses $4.8 \times 10^{-11}$ and $7.1 \times 10^{-11}$ in
other galactic ridge or bulge fields \citep{ebisawa2008, revnivtsev2009}.
But, these other fields sample brighter areas of X-ray emission (see the
RXTE/PCA map presented by \citet{revnivtsev2006}); therefore a decrease in
brightness of order $10$ in our field -- in conjunction with some uncertainty
in assumed extragalactic background normalization -- seems plausible.
% The integrated $2$--$10 \unit{keV}$ diffuse flux, based on a Chandra
% galactic plane observation at $(l,b) = (+28.45\arcdeg, -0.2\arcdeg)$,
% is $1.0 \times 10^{-10} \unit{erg\;s^{-1}\;cm^{-2}\;deg^{-2}}$
% \citep{ebisawa2001}.
%
% A subsequent Suzaku study of the same field revised this downwards to
% $4.8 \times 10^{-11} \unit{erg\;s^{-1}\;cm^{-2}\;deg^{-2}}$ (Suzaku; Chandra
% agrees within 20\%) after excluding (1) extragalactic background,
% and (2) point sources brighter than
% $2 \times 10^{-13} \unit{erg\;s^{-1}\;cm^{02}}$
% \citep{ebisawa2008}.
%
% \citet{revnivtsev2009} find total 2-10 keV brightness 8.6e-11 erg/s/cm^2/deg^2
% in the ``HRES'' field at l = 0.08, b = -1.42.
% CXB contribution 1.5e-11 erg/s/cm^2/deg^2 (down from 2.2e-11, because of
% absence of brightest sources which account for ~31\% of extragal CXB.
% We should make a similar correction in our work based on the pt source
% brightness we are able to resolve.)
% Then ridge emission is 7e-11.

% Varying abundances
We vary other elemental abundances that are less obviously constrained by our
data, and our results are shown in Table~\ref{tab:src-varyabund} and
Figure~\ref{fig:src-varyabund}.
A variety of elemental abundances yield reasonable fits.
A significantly lower absorption is allowed with O, Ne, and Fe free.

\begin{table*}[!ht]
    \centering
    \caption{Source fit with varying abundances \label{tab:src-varyabund}}

\begin{tabular}{@{}lllllllll@{}}
\toprule
 & Stock & Mg & ONeMgFe & All... & OMg & NeMg & MgFe & ONeMg \\
\midrule
\multicolumn{9}{c}{Remnant plasma} \\
\midrule
$\nH$   & ${2.15}^{+0.07}_{-0.08}$ & ${2.03}^{+0.10}_{-0.09}$ & ${0.83}^{+0.16}_{-0.05}$ & ${0.84}^{+0.08}_{-0.07}$
            & ${2.03}^{+0.06}_{-0.08}$ & ${1.80}^{+0.08}_{-0.07}$ & ${1.84}^{+0.10}_{-0.09}$ & ${2.06}^{+0.09}_{-0.11}$ \\ [0.5em]
$\kB T$ & ${2.3}^{+0.3}_{-0.1}$    & ${2.9}^{+0.5}_{-0.4}$ & ${2.9}^{+0.4}_{-0.3}$ & ${2.5}^{+0.2}_{-0.3}$
            & ${2.4}^{+0.5}_{-0.3}$ & ${2.7}^{+0.4}_{-0.3}$ & ${2.3}^{+0.4}_{-0.4}$ & ${2.4}^{+0.3}_{-0.1}$ \\ [0.5em]
$\tau$  & ${1.83}^{+0.12}_{-0.11}$ & ${1.75}^{+0.11}_{-0.10}$ & ${1.49}^{+0.09}_{-0.09}$ & ${1.60}^{+0.11}_{-0.10}$
            & ${1.78}^{+0.11}_{-0.11}$ & ${1.65}^{+0.10}_{-0.09}$ & ${1.72}^{+0.16}_{-0.10}$ & ${1.79}^{+0.12}_{-0.10}$ \\ [0.5em]
EM      & ${4.6}^{+0.5}_{-0.5}$ & ${3.6}^{+0.6}_{-0.5}$ & ${3.8}^{+0.4}_{-0.4}$ & ${4.0}^{+0.4}_{-0.4}$
        & ${3.6}^{+0.4}_{-0.5}$ & ${3.8}^{+0.5}_{-0.4}$ & ${4.5}^{+1.0}_{-0.6}$ & ${3.5}^{+0.5}_{-0.5}$ \\ [0.5em]
O  &                          &                          & $< 0.11$                 & $< 0.03$                 & ${7.0}^{+1.3}_{-1.2}$    &                          &                          & ${8.1}^{+3.5}_{-2.8}$ \\ [0.5em]
Ne &                          &                          & $< 0.05$                 & $< 0.10$                 &                          & ${0.14}^{+0.17}_{-0.14}$ &                          & ${1.2}^{+0.6}_{-0.5}$ \\ [0.5em]
Mg &                          & ${1.24}^{+0.11}_{-0.10}$ & ${0.30}^{+0.05}_{-0.03}$ & ${0.31}^{+0.04}_{-0.03}$ & ${1.28}^{+0.11}_{-0.10}$ & ${0.92}^{+0.10}_{-0.09}$ & ${0.81}^{+0.11}_{-0.11}$ & ${1.4}^{+0.2}_{-0.3}$ \\ [0.5em]
Si & ${3.75}^{+0.18}_{-0.16}$ & ${4.2}^{+0.3}_{-0.3}$    & ${2.50}^{+0.14}_{-0.11}$ & ${2.51}^{+0.13}_{-0.07}$ & ${4.5}^{+0.3}_{-0.3}$    & ${3.6}^{+0.3}_{-0.2}$    & ${3.4}^{+0.2}_{-0.3}$    & ${4.7}^{+0.7}_{-0.5}$ \\ [0.5em]
S  & ${3.42}^{+0.28}_{-0.25}$ & ${3.9}^{+0.4}_{-0.3}$    & ${3.6}^{+0.4}_{-0.2}$    & ${3.5}^{+0.3}_{-0.2}$    & ${4.2}^{+0.4}_{-0.3}$    & ${3.8}^{+0.3}_{-0.4}$    & ${3.5}^{+0.3}_{-0.3}$    & ${4.4}^{+0.6}_{-0.5}$ \\ [0.5em]
Ar &                          &                          &                          & ${3.9}^{+1.0}_{-0.9}$    &                          &                          &                          &      \\ [0.5em]
Ca &                          &                          &                          & ${14.2}^{+5.3}_{-4.3}$   &                          &                          &                          &      \\ [0.5em]
Fe &                          &                          & ${0.06}^{+0.03}_{-0.02}$ & ${0.05}^{+0.02}_{-0.02}$ &                          &                          & ${0.19}^{+0.15}_{-0.13}$ &      \\ [0.5em]
Ni &                          &                          &                          & ${0.30}^{+0.37}_{-0.30}$ &                          &                          &                          &      \\
\midrule
\multicolumn{9}{c}{X-ray Background} \\
\midrule
$\kB T_{\mt{local}}$   & ${0.26}^{+0.01}_{-0.01}$ & ${0.26}^{+0.01}_{-0.01}$ & ${0.25}^{+0.01}_{-0.01}$ & ${0.25}^{+0.01}_{-0.01}$ & ${0.25}^{+0.01}_{-0.01}$ & ${0.26}^{+0.01}_{-0.01}$ & ${0.26}^{+0.01}_{-0.01}$ & ${0.25}^{+0.01}_{-0.01}$ \\ [0.5em]
$\mt{EM}_{\mt{local}}$ & ${0.30}^{+0.02}_{-0.02}$ & ${0.30}$ & ${0.27}$ & ${0.25}$ & ${0.30}^{+0.02}_{-0.02}$ & ${0.30}^{+0.02}_{-0.02}$ & ${0.29}^{+0.02}_{-0.02}$ & ${0.30}^{+0.02}_{-0.02}$ \\ [0.5em]
$\nH$                  & ${1.32}^{+0.08}_{-0.08}$ & ${1.33}^{+0.08}_{-0.08}$ & ${1.37}^{+0.08}_{-0.08}$ & ${1.33}^{+0.08}_{-0.08}$ & ${1.37}^{+0.07}_{-0.07}$ & ${1.39}^{+0.07}_{-0.07}$ & ${1.34}^{+0.08}_{-0.08}$ & ${1.37}^{+0.07}_{-0.07}$ \\ [0.5em]
$\kB T_{\mt{halo}}$    & ${0.74}^{+0.03}_{-0.03}$ & ${0.74}^{+0.03}_{-0.03}$ & ${0.64}^{+0.03}_{-0.03}$ & ${0.62}^{+0.03}_{-0.03}$ & ${0.81}^{+0.04}_{-0.04}$ & ${0.79}^{+0.05}_{-0.04}$ & ${0.71}^{+0.03}_{-0.06}$ & ${0.81}^{+0.04}_{-0.04}$ \\ [0.5em]
$\mt{EM}_{\mt{halo}}$  & ${2.19}^{+0.29}_{-0.26}$ & ${2.25}$ & ${2.65}$ & ${2.62}$ & ${2.17}^{+0.27}_{-0.25}$ & ${2.25}^{+0.28}_{-0.26}$ & ${2.35}^{+0.30}_{-0.27}$ & ${2.17}^{+0.27}_{-0.25}$ \\ [0.5em]
$\chi^2$ & 4545.2 & 4528.5 & 4421.5 & 4347.5 & 4447.1 & 4492.5 & 4486.8 & 4446.7 \\ [0.5em]
$\chi^2_{\mt{red}}$ & 1.206 & 1.202 & 1.175 & 1.156 & 1.181 & 1.193 & 1.191 & 1.181 \\
\bottomrule
\end{tabular}

    \tablecomments{Units as in Table~\ref{tab:src-fit}, except EM is in units of $10^{11}$; i.e., $10^3 \times$ the usual XSPEC norm.}
\end{table*}


\begin{figure*}[!ht]
    \plotone{fig/fig_src_varabund_0087940201-mos1_pt1.pdf}
    \plotone{fig/fig_src_varabund_0087940201-mos1_pt2.pdf}
    \plotone{fig/fig_src_varabund_0087940201-mos1_pt3.pdf}
    \figcaption{Integrated source fits with varying abundances (top plot label
        should read Si,S \textbf{TODO}), as labeled, for
        0087940201 MOS1 (one of five exposures fitted)}
    \label{fig:src-varyabund}
\end{figure*}

\begin{figure*}[!ht]
    \plotone{fig/fig_src_varabund_0087940201-pn_pt1.pdf}
    \plotone{fig/fig_src_varabund_0087940201-pn_pt2.pdf}
    \plotone{fig/fig_src_varabund_0087940201-pn_pt3.pdf}
    \figcaption{Same as Figure~\ref{fig:src-varyabund}, but 0087940201 PN exposure.}
    \label{fig:src-varyabund-pn}
\end{figure*}




\subsection{Annuli spectra}

% Procedure, results
We extract and fit spectra from $100\arcsec$ wide annuli centered on the
remnant to discern how thermal plasma emission varies with remnant radius
(Figure~\ref{fig:regions}).
Each annulus is modeled by an absorbed NEI plasma
(XSPEC model \texttt{tbabs\_new * vnei}) with absorption column tied across
all annuli.
The X-ray background is fixed to the integrated remnant fit model
(Table~\ref{tab:src-fit}).
We neglect redistribution of counts between annuli due to point spread function
(PSF) wings as the off-axis half-energy-enclosed width $\abt10$--$20\arcsec$ is
smaller than our annuli widths.
Annulus spectra from $0$--$500\arcsec$ are shown in
Figure~\ref{fig:annuli-spectra}, and fit parameters for the interior four
annuli are in Table~\ref{tab:annulus-fit}.
The best fit absorption (not tabulated) is $2.11^{+0.04}_{-0.07} \nHUnits$.

% outer annulus handling
The outer annulus fit ($400$--$500\arcsec$) was very poorly constrained;
two fits with slightly different X-ray background parameters, which agreed
within error, yielded disparate fits ($kT = 0.6$ vs. $10 \unit{keV}$,
Si and S abundances $\abt10$ vs. $\abt2$,
ionization timescale $\tau = 1 \times 10^{10}$ vs. $5 \times 10^{13} \TauUnits$).
Some Si and S line emission in the outer annulus is required, without too much
underlying bremsstrahlung continuum, hence the inferred super-solar abundances.
Discarding the outer annulus has no impact on fits to interior annuli, so our
subsequent fits and analysis consider only four annuli.

% center "annulus" handling
The remnant center ($0$--$100\arcsec$) shows excess soft X-rays
($<1 \unit{keV}$) compared to our NEI model.
In Table~\ref{tab:annulus-varycenter}, we attempt to explain the central excess
by allowing sub- or super-ISM O, Ne, and Fe abundances
Although additional elements improve the fit, it is not clear that a particular
combination of abundances can fully explain the model discrepancy.
Significant O line emission can be produced by SWCX, but the absence of soft
X-ray excess outside the central region disfavors this explanation (as we would
expect SWCX to fill the field of view).

\begin{table}
    \centering
    \caption{Annuli fit \label{tab:annulus-fit}}
    \begin{tabular}{@{}rlllll@{}}
        \toprule
        Annulus & $kT$ & $\tau$ & Si & S & EM \\
         & (keV) & ($\times 10^{10}$) & (-) & (-) & ($\times 10^{14}$)\\
        \midrule
        % 20160706_fourann_stock
        $0$--$100\arcsec$ & ${3.8}^{+0.6}_{-0.4}$ & ${1.57}^{+0.14}_{-0.13}$ & ${1.98}^{+0.14}_{-0.13}$ & ${1.5}^{+0.3}_{-0.3}$ & $1.21 \times 10^{-3}$ \\
        $100$--$200\arcsec$ & ${2.4}^{+0.2}_{-0.3}$ & ${1.74}^{+0.16}_{-0.14}$ & ${5.7}^{+0.4}_{-0.3}$ & ${5.5}^{+0.7}_{-0.6}$ & $1.00 \times 10^{-3}$ \\
        $200$--$300\arcsec$ & ${2.1}^{+0.2}_{-0.2}$ & ${2.16}^{+0.30}_{-0.21}$ & ${4.1}^{+0.3}_{-0.2}$ & ${3.8}^{+0.4}_{-0.3}$ & $1.74 \times 10^{-3}$ \\
        $300$--$400\arcsec$ & ${2.5}^{+0.5}_{-0.4}$ & ${1.74}^{+0.37}_{-0.28}$ & ${3.4}^{+0.4}_{-0.3}$ & ${3.1}^{+0.9}_{-0.6}$ & $0.92 \times 10^{-3}$ \\
        \bottomrule
    \end{tabular}
    \tablecomments{The best fit absorption is $2.11^{+0.04}_{-0.07} \nHUnits$,
        and $\chi^2_{\mt{red}} = 1.432 = 3581.34/2501$.
        Units as in Table~\ref{tab:src-fit}.}
\end{table}

\begin{figure*}[]
    \plotone{fig/fig_fiveann_0087940201-mos1.pdf}
    \figcaption{Annuli spectra with fits from 0087940201 MOS1 (one of five exposures fitted).
        Only Si and S abundances are free; all others are fixed to ISM
    Table~\ref{tab:annulus-fit} provides NEI fit parameters for four of five annuli.}
    \label{fig:annuli-spectra}
\end{figure*}

\begin{figure*}[]
    \plotone{fig/fig_fiveann_0087940201-mos1-delchi.pdf}
    \figcaption{Annuli spectra fit scaled residuals ($\Delta\chi$)
        from 0087940201 MOS1 (one of five exposures fitted), to accompany
        Figure~\ref{fig:annuli-spectra}.}
    \label{fig:annuli-delchi}
\end{figure*}


% SANITY check: do soft proton power law parameters show expected behavior?
% EXPECT: vignetted (lower norm), harder spectrum farther off-axis
% SP vignetting differs from the standard photon vignetting.
We confirm that soft proton parameters from annuli fits qualitatively show
expected energy-dependent vignetting, with stronger vignetting at soft energies
\ref{fig:sp}.
% TODO citation
% TODO build this...

\begin{figure}[!h]
    \plotone{fig/fig_sp.png}
    \figcaption{Fitted soft proton power laws from annuli fits (TODO: ``stock''
        fits, temporary).  Power law for $200$--$300\arcsec$ overlaps
        $400$--$500\arcsec$ plot.}
    \label{fig:sp}
\end{figure}

\begin{table*}[!ht]
    \centering
    \caption{Varying center abundances\label{tab:annulus-varycenter}}
    \scriptsize
\begin{tabular}{@{}lllllllllllr@{}}
\toprule
Fit & $n_\mathrm{H}$ & $kT$ & $\tau$
        & O & Ne & Mg & Si & S & Fe
        & $\chi^2_{\mt{red}}$ & $\chi^2 / (\mt{dof})$ \\
    & $\times 10^{22}$ & keV & $\times 10^{10}$
        & - & - & - & - & - & -
        & & \\
\midrule
5std    & ${2.10}^{+0.05}_{-0.05}$ & ${3.82}^{+0.64}_{-0.64}$ & ${1.57}^{+0.16}_{-0.15}$
    &                 &                 &                          & ${1.98}^{+0.13}_{-0.16}$ & ${1.45}^{+0.29}_{-0.29}$ &
    & 1.385 & 4603.25/3323 \\
    % File 20160701_fiveann_row.tex
4std    & ${2.11}^{+0.04}_{-0.07}$ & ${3.79}^{+0.55}_{-0.42}$ & ${1.56}^{+0.14}_{-0.13}$
    &                 &                 &                          & ${1.98}^{+0.14}_{-0.13}$ & ${1.45}^{+0.30}_{-0.26}$ &
    & 1.432 & 3581.34/2501 \\
    % File 20160706_fourann_stock_row.tex

4Mg     & ${2.15}^{+0.04}_{-0.09}$ & ${2.95}^{+0.36}_{-0.41}$ & ${1.50}^{+0.18}_{-0.17}$
    &                 &                 & ${0.50}^{+0.08}_{-0.08}$ & ${1.63}^{+0.13}_{-0.11}$ & ${1.25}^{+0.33}_{-0.26}$ &
    & 1.403 & 3508.26/2500 \\
    % File 20160708_fourann_center-mg-free_ONeMgFe_row.tex

\midrule

4Mg,O   & ${2.10}^{+0.04}_{-0.06}$ & ${2.64}^{+0.55}_{-0.43}$ & ${1.57}^{+0.19}_{-0.18}$
    & ${9.5}^{+1.6}_{-1.5}$ &                 & ${0.57}^{+0.09}_{-0.08}$ & ${1.86}^{+0.15}_{-0.14}$ & ${1.35}^{+0.34}_{-0.28}$ &
    & 1.365 & 3411.25/2499 \\
    % File 20160708_fourann_center-mg-o-free_ONeMgFe_row.tex

4Mg,Ne  & ${2.14}^{+0.09}_{-0.07}$ & ${3.02}^{+0.56}_{-0.45}$ & ${1.50}^{+0.17}_{-0.17}$
    &                 & ${1.05}^{+0.39}_{-0.26}$ & ${0.51}^{+0.10}_{-0.08}$ & ${1.66}^{+0.17}_{-0.15}$ & ${1.27}^{+0.33}_{-0.26}$ &
    & 1.404 & 3507.97/2499 \\
    % File 20160708_fourann_center-mg-ne-free_ONeMgFe_row.tex

4Mg,Fe  & ${2.22}^{+0.06}_{-0.11}$ & ${3.99}^{+1.09}_{-0.47}$ & ${1.63}^{+0.16}_{-0.15}$
    &                 &                 & ${0.67}^{+0.14}_{-0.11}$ & ${2.01}^{+0.28}_{-0.20}$ & ${1.32}^{+0.29}_{-0.24}$ & ${1.88}^{+0.48}_{-0.42}$
    & 1.399 & 3497.19/2499 \\
    % File 20160708_fourann_center-mg-fe-free_ONeMgFe_row.tex

\midrule

4Mg,O,Ne& ${2.28}^{+0.04}_{-0.03}$ & ${3.24}^{+0.56}_{-0.26}$ & ${1.50}^{+0.14}_{-0.13}$
    & ${36}^{+18}_{-9}$ & ${4.7}^{+2.6}_{-1.0}$ & ${1.41}^{+0.64}_{-0.27}$ & ${3.68}^{+1.09}_{-0.58}$ & ${2.68}^{+0.71}_{-0.61}$ &
    & 1.348 & 3367.95/2498 \\
    % File 20160708_fourann_center-mg-o-ne-free_ONeMgFe_row.tex

4Mg,O,Fe  & ${2.21}^{+0.06}_{-0.03}$ & ${4.15}^{+0.75}_{-0.57}$ & ${1.81}^{+0.13}_{-0.14}$
    & ${32}^{+22}_{-12}$ &                 & ${1.28}^{+0.4}_{-0.2}$ & ${3.88}^{+2.2}_{-0.73}$ & ${2.19}^{+0.38}_{-0.47}$ & ${4.20}^{+2.6}_{-0.58}$
    & 1.351 & 3374.89/2498 \\
    % From 20160712_fourann_center-mg-o-fe-free fits, manually

\bottomrule
\end{tabular}
\tablecomments{Units as in Table~\ref{tab:src-fit}.}
\end{table*}

\begin{figure*}[!ht]
    \epsscale{0.8}
    \plotone{fig/fig_fourann-varycenter_0087940201-mos1.pdf}
    \figcaption{Center ($0$--$100\arcsec$) spectrum fits with different
        elemental abundances freed (O, Ne, Mg, Fe).
        Spectra from 0087940201 MOS1 only (one of five exposures fitted).
        Dashed curve indicates ``fiducial'' fit with Mg, Si, S freed.}
    \label{fig:annuli-varycenter-spectra}
\end{figure*}



\section{Discussion}

\subsection{Assumptions and prior work}

% probably can omit or place one sentence elsewhere
\textbf{standard vnei disclaimer}
The single temperature and ionization age \texttt{vnei} model ignores spatial
structure in temperature, ionization, and ejecta abundances.
Nevertheless, it fits the remnant reasonably well, consistent with the
subsequent annulus fits yielding ionization timescales and temperatures varying
by $\abt50\%$ at most (Table~\ref{tab:annulus-fit}).

% vpshock discussion
A \texttt{vpshock} model, which accounts for varying ionization age integrated
behind a plane shock \citep{borkowski2001}, yields a nearly identical fit with
$\chi^2_{\mt{red}} = 1.203$ versus $\chi^2_{\mt{red}} = 1.206$ for the
\texttt{vnei} model.
The \texttt{vpshock} ionization age range spans
$\abt 0$ to $3.89 \times 10^{10}$, whereas the \texttt{vnei} sits nearly in the
middle at $1.83 \times 10^{10}$.
In effect, the \texttt{vnei} average over ionization age provides a reasonable
fit; or, variation in ionization age has little effect on the model spectrum.
This makes sense for the relatively low ionization timescales seen in G309
(compared to more evolved supernova remnants).
Given the high electron temperatures $\abt 2$--$3 \unit{keV}$, the
\texttt{vpshock} model should reasonably approximate the more involved
\texttt{vsedov} model of \citet{borkowski2001}, so we do not consider a
\texttt{vsedov} fit with non-uniform electron temperature distribution.

% ionization age
Our ionization age $n_e t \sim 1.5$--$2 \times 10^{10} \unit{s\;cm^{-3}}$
is, qualitatively, anchored by the Si and S He$\alpha$ and Ly$\alpha$ line ratios.
\textbf{TODO calculate line centroids, widths, and fluxes; look at atomic
physics (see Hiroya's plots or \citet{hwang2000}) to back out ionization time}
Fits with $n_e t = 10^{11}$ fixed require lowered
electron temperature $\sim 1 \unit{keV}$ to reproduce the Si and S emission.
The ionization state implied by Si and S emission, along with our fitted ISM or
sub-ISM iron abundance, matches measurements of other young ejecta-dominated
remnants of Ia SNe \citep[][Table 4]{badenes2007}.

% Abundances
I note that fit with Ar,Ca,Ni free yields a significantly better fit (per
f-test), compared to baseline of Si,S,Mg,O,Ne,Fe free only.
The f-test yields F statistic value 21.3 with probability $10^{-13}$.
\textbf{TODO: I would dispute that Ca, Ni have any effect, but I think Ar can be constrained.}
Nothing saves you from garbage in, garbage out, though.

% Interpretation
As a baseline, we kind of expect that most thermal continuum arises from
shocked ISM and/or CSM.
The alternative is a pure metal plasma, which could describe ejecta.
Rakowski+ get very different emission measure with a pure metal plasma.

% ISM component
If we add an ISM component, represented by a NEI plasma with ISM abundances,
the fit improves slightly.
But fitted parameter values jump around quite a lot.
The ISM NEI components can be wildly different (ionization age
$10^{9}$--$10^{10}$, temperature $0.55$ to $8.63 \unit{keV}$,
normalization (scaled emission measure) $2 \times 10^{-3}$ to $3 \times
10^{-2}$), depending on which elemental abundances are free in the ejecta
plasma component.  In short, very ill-constrained.
We don't see any spectral features that can unequivocally be attributed to a
distinct mass of plasma with different temperature, abundance, and ionization
characteristics.
\textbf{the fix is to build one of those confidence contour maps,
after vetting errors and incorporating more systematics.
or use some kinda bayesian analysis}

% Comparison to old fits
Our fitted absorption column and ionization timescales disagree with prior
X-ray fits.
\citet{rakowski2001} obtain $\nH = (0.7 \pm 0.3) \nHUnits$,
$\kB T = 2.0^{+1.0}_{-0.6}$, and
$\tau = 1.5^{+4.7}_{-0.6} \times 10^{11} \unit{s\;cm^{-3}}$
from an absorbed NEI model for ASCA data with enhanced Ne, Mg, Si, S, Ar, Ca,
Fe abundances; the best fit converged towards a pure metal plasma
(H, He $\to 0$), and taking abundances relative to Si they favored
under-abundant Ne, Mg, Ar relative to S.
\citet{safi-harb2007} report $N_H = 0.65^{+0.45}_{-0.25} \nHUnits$.
and $\kB T = 2 \pm 0.6 \unit{keV}$ without providing $\tau$, from
an absorbed NEI model for the 2001 XMM data with
super-ISM Si, S and sub-ISM O, Ne, Mg, Ca, Fe.

The relative abundances $[Mg/Si] = 0.23^{+0.09}_{0.08}$ and $[S/Si] =
1.09^{+0.24}_{-0.18}$ from \citet{rakowski2001} are consistent with our fits
($0.29 \pm 0.03$ and $0.93 \pm 0.12$ respectively).  Similarly our Si, S, Mg
abundances are consistent with the qualitative statement of supersolar
abundances by \citet{safi-harb2007}.

\citet{rakowski2001} report emission measure
$(4\pi D^2)^{-1} \int n_e n_H dV = 9.7^{+200}_{-9.7} \times 10^8$, much smaller
than our $\abt 4 \times 10^{11} \unit{cm^{-5}}$; their inferred hydrogen
density is therefore $10\times$ smaller.
The low value is partially offset by the extreme ($\abt 30\times$) abundances
inferred by \citet{rakowski2001}, and the lack of constraint on bremsstrahlung
continuum as their fit favored a pure metal plasma.

The discrepant absorption and ionization timescale could arise from degeneracy
with respect to absorption, ionization time, and total metallicity (H
abundance), particularly as \citet{rakowski2001} had to explicitly model point
source contamination from HD 119682, which dominates
below $\abt 1 \unit{keV}$ and above $\abt 3$--$4 \unit{keV}$.
The XMM data fit by \citet{safi-harb2007} should be free of HD 119682
contamination, but we cannot speak to the absorption discrepancy without
knowledge of their background and instrumental line modeling and the fitted
ionization timescale.
Although our fits are limited by soft proton contamination at high and
low energy ranges as well, XMM's spatial and spectral resolution allow us to
excise point source emission and better constrain the bremsstrahlung continuum.

Alternatively, our absorption can be as low as $0.8 \times 10^{22}$ if
O, Ne, and Fe are highly subsolar (Table~\ref{tab:src-varyabund}).
\textbf{This is being investigated further.}
Ratios with respect to Si make sense, but the allowance of a thermal continuum
disagrees with Rakowski+.

If the absorption is truly that high, it is just marginally consistent with
the galactic column computed from HI and dust maps.
For G309.2-0.6, the \textit{Swift} galactic $\nH$ calculator implementation of
\citet{willingale2013} provides an expected column density $1.81 \nHUnits$.
%Dust scattering can bias X-ray spectrum fits for a purely absorbing column
%$\abt25\%$ its true value \citep{corrales2016}, although
%some scattered light is recaptured in our extended source region spectra.
%The \textit{spectrum fit value} of $\nH$ may also be estimated from empirical
%linear regression against optical extinction $\AV$ \citep[e.g.,][]{foight2016}
%-- which benefits from a strong sampling of sources within the galactic plane,
%and empirical validation -- but does not offer physical insight into the
%discrepancy. We find approximate agreement ($2$--$3 \nHUnits$) with the various
%relations.
%\textbf{TODO: for now detailed accounting of absorption is beyond the scope of
%this work}

The absence of H$\alpha$ emission is unilluminating.
Assuming H$\alpha$ emission to arise by shock propagation through ISM
containing neutral hydrogen \citep{chevalier1978},  % 1978ApJ...225L..27C
any of the following factors may contribute: low density of neutral H due to
progenitor or supernova ionization, low ambient ISM density, or some local
conditions (extremely high post-shock density?) that rapidly ionize neutral H,
preventing Balmer line emission.
Or, the available imagery may just not be deep enough.
The outcome is that we are unable to constrain shock velocity independently by
Balmer line width measurement.


\subsection{Morphology}

% Discuss morphology in broad terms
The broadband $0.8$--$3.3 \unit{keV}$ remnant mostly fills the radio remnant,
excepting the remnant's southeast limb and edges of the radio ears
(Figures~\ref{fig:snr}).
X-ray emission is concentrated at and behind the northwest limb, which is
somewhat dimmer in radio compared to the southeast limb and ears.

% Spatial distribution of line emission
The line emission broadly follows the broadband emission (particularly the
dominant Si line), but Mg and S appear to be localized to the north and east
(Figure~\ref{fig:rgb}).
Both Si and S are concentrated in the northern clump (\textbf{TODO:
green-band Si image is overexposed, obscuring detail}), although Si more
visibly fills the remnant.
Mg appears localized to a large northwestern clump and possibly some towards
the southeast X-ray limb, although the significance of this possible spatial
enhancement is questionable.

% log scale, 6e-7 to 10e-6 cts/sec/cm^2 whatever
% contrast = 3.890, bias = 0.605
% smooth = 12
\begin{figure*}[!ht]
    \plottwo{fig/fig_rgb.png}{fig/fig_rgb_mg.png}
    \plottwo{fig/fig_rgb_si.png}{fig/fig_rgb_s.png}
    \figcaption{Smoothed RGB image (top left) of G309 sampling Mg, Si, S
        He$\alpha$ lines.
        Red, top right: $1.3$--$1.4 \unit{keV}$ (Mg).
        Green, bottom left: $1.7$--$1.8 \unit{keV}$ (Si).
        Blue, bottom right: $2.3$--$2.4 \unit{keV}$ (S).
        MOST contours are 0.01, 0.02, 0.05, 0.1, 0.2 Jy in all images.
        The images are not adjusted for background and should be considered
        qualitative only.}
    \label{fig:rgb}
\end{figure*}

\textbf{TODO: calculate mirror asymmetry and compare to \citet{lopez2011}.}


\subsection{Dynamical character}

% Radial variation in kT and Tau
The plasma temperature and ionization timescale vary by $\abt 20$--$50\%$
across annuli (Figure~\ref{fig:kt-tau-radius}).
The center and limb plasma is hotter and more recently shocked than the bright
$200$--$300\arcsec$ annulus.
Although these variations may be spurious if systematic errors dominate, the
observed radial variation is at least consistent with the standard picture of
supernova remnant evolution, wherein recently shocked ISM and central
reverse-shocked central ejecta should be the ``youngest'' emitting plasma.

\begin{figure*}[!ht]
    \plottwo{fig/fig_kt_radius.pdf}{fig/fig_tau_radius.pdf}
    \figcaption{Electron temperature (left) and ionization timescale (right)
        as a function of radius from annuli spectra fit
        (Tables~\ref{tab:annulus-fit}, \ref{tab:annulus-varycenter}).}
    \label{fig:kt-tau-radius}
\end{figure*}

% Emission measure derived masses
We estimate X-ray emitting plasma mass and density from the integrated remnant
fit emission measure (Table~\ref{tab:src-fit}).
Assuming that the plasma fills a spherical volume of radius $400\arcsec$,
\[
    n_{\mt{H}} = 0.11 f^{-1/2} d_{5}^{-1/2} \unit{cm^{-3}}
\]
with volume filling factor $f \in [0,1]$ and distance relative to $5
\unit{kpc}$ as $d_{5}$.
We then estimate mass $M = 1.4 n_{\mt{H}} m_{H} f V$:
\[
    M = 10.3 M_{\sun} f^{1/2} d_{5}^{5/2}
\]
with corresponding ejecta masses $0.016 M_{\sun}$ of silicon, $0.011 M_{\sun}$
of sulfur, and $0.005 M_{\sun}$ of magnesium.
% TODO explicate factor 1.4 for H,He...
% TODO provide stat uncertainty from XSPEC
As ejecta abundances for Mg, Si, S are not as strongly variable between Ia and
core-collapse nucleosynthesis yields (versus O or Fe), the relative abundances
of these elements do not clearly constrain the progenitor type.
\textbf{TODO: read more about Ia and CC models and their yields.}

% Discuss density inferences
Our inferred density - valid for a homogeneous sphere - should be a lower bound
on the post-forward-shock density.
Therefore the ambient medium density should be at least
$0.025 f^{-1/2} d_{5}^{-1/2} \unit{cm^{-3}}$, assuming a strong forward shock
with compression ratio of four, expected for young ejecta-dominated remnants.
This bound is not very stringent, but is at least a start.
If shocked plasma is concentrated within a shell of width $r/12$
% volume fraction (1 - (11/12)^3)
\textbf{todo:motivate this 1/12 factor}, then the shocked density would be
$n_{\mt{H}} \sim 0.5 f^{-1/2} d_{5}^{-1/2}$
and the ambient density bound would be $0.12 f^{-1/2} d_{5}^{-1/2}$,
which suggests ``typical'' ambient ISM.
This would also disfavor distance much higher than $5 \unit{kpc}$.
% TODO ...
\citet{katsuda2015} use, similar to \citet{kosenko2010}, a deprojection method
to try to back out a radial profile of the ejecta.

% Discuss HI brightness profiles
We could use HI brightness profiles along the line-of-sight to set some upper
bounds on ambient density.
Consider two cases for degeneracy: assume all emission with $v_{\mt{LSR}} < 0$
on either (1) nearby-arm of rotation curve, or (2) far-arm of rotation curve.
\textbf{also revisit willingale assumptions on ISM density.  Estimate for H2
contribution is much different than \citet{yamaguchi2012}; see Sec. 3.1.
not sure if the willingale approach is valid in the galactic plane.}

\subsection{A young remnant}

The high electron temperature, low ionization age, enhanced ejecta abundances,
low emitting plasma mass all point towards a dynamically young remnant.
G309 is somewhat center-filled in X-rays and clearly shell-like in radio,
consistent with the definition of a mixed-morphology (MM) remnant
\citep{rho1998}.
But, the plasma is relatively hot and recently shocked for a MM remnant, and
the surface brightness peaks off-center.
The remnant's age estimates, fitted X-ray plasma parameters and brightness
distribution are consistent with an ejecta-dominated young remnant, which
does not require complex cloud or thermal conduction models as proposed for MM
remnants.

The outer ($300$--$400\arcsec$) annulus temperature of $2 \unit{keV}$
suggests strong-shock heating (shock velocity $\gtrsim 1000 \unit{km/s}$).
It is unclear whether emission arises from ejecta or shocked ISM, but in either
case requires the shock to be dynamically young.

The age inferred from the emitting plasma norm and integrated ionization age
$1.75 \TauUnits$ is $5000 f^{1/2} d_{5}^{1/2} \unit{yr}$.
Enhanced central Si and S emission suggests that the reverse shock has
traversed most of the remnant.

The Sedov-derived age \citep{taylor1950, sedov1959} is calculated using the
numerical constants for a monatomic gas \citet{taylor1950-pt2}.
From the shock radius scaling
$R = (0.34 \unit{pc}) \left( \frac{E_{51}}{n_0} \right)^{1/5} t_{\mt{yr}}^{2/5}$,
we invert for remnant age and assume angular radius $400\arcsec$ to obtain:
\[
    t = (4300 \unit{yr}) d_5^{5/2} E_{51}^{-1/2} n_0^{1/2} .
\]
which is somewhat on the high side for an ejecta dominated remnant.
\textbf{TODO: factor 0.34 is derived from $\beta^5 = 2.052$, but would be preferable to draw value
from closed form solution.  See my paper notes.}

The Sedov age and ionization timescale / norm derived age scale differently
with remnant distance and can help set a lower bound on our distance estimate
(Figure~\ref{fig:age}).
However, the Sedov estimate represents an upper bound on age as the remnant
may still be ejecta-dominated, or transitioning into Sedov expansion.
\citet{rakowski2001} already considered these estimates, and our current age
estimates are no more stringent than theirs.

Todo: compare to characteristic Sedov-Taylor time of \citet{truelove1999}.

\begin{figure}[!ht]
    \plotone{fig/fig_age_plot.pdf}
    \figcaption{Red: Sedov-derived age bound with $E_{51} = 1$
        and $n_0 \in [0.1, 1] \unit{cm^{-3}}$.
        Blue: plasma ionization and EM derived age estimate, with filling factor
        $f \in [0.2, 1]$.  Plasma is assumed to be homogenous and spherical
        volume filling.}
    \label{fig:age}
\end{figure}


\textbf{TODO: review ejecta-dominated free expansion and transition into Sedov}
% Vink: http://arxiv.org/pdf/1112.0576v2.pdf
% Chevalier 1972?: ...
% Truelove McKee: http://adsabs.harvard.edu/abs/1999ApJS..120..299T
% Tang Chevalier: http://arxiv.org/pdf/1607.06391v1.pdf


% NOTE: SW lobe and ridge both show weird shifts in PN spectral lines (Si, S)
% (this dates from Jan/Feb -- to be investigated with now up to date procedures)

\subsection{Friends of young remnants}

Comparative study?  TBD.
G350.1-0.3 is young, bright (~100,000 counts)
G337.2-0.7 appears a bit older (ionization age $10^12$), also Si/S rich but
allows Fe.

\section{Distance Inference}

Yada yada yada.

% Galactic environment
The sight line towards G309.2-0.6 crosses the tangent of the Centaurus Arm
(variously Scu-Cen, Crux-Cen, Scu-Crux) and the near and far Carina arm at
$\abt 1.5 \unit{kpc}$ and $\abt 14 \unit{kpc}$.  Molecular and HI emission is
dominated by the Centaurus Arm at line-of-sight velocities $-70$ to
$-20 \unit{km/s}$ \citep[e.g.,][Figure 4]{dame2011}, and a good number of
molecular clouds can be identified \citep{rice2016}.  The field is populated
with unrelated but independently interesting features.

% sigma-D estimate of distance
From the somewhat questionable $\Sigma$-$D$ relation, \citet{pavlovic2014}
provide a distance estimate of $5.9 \unit{kpc}$.
This is fraught with peril.
At least the relation of \citet{pavlovic2014} is calibrated for shell-type
radio remnants.
%See also \citet{huang1985} which presents a $\Sigma$-$D$ relation for
%shell-like remnants also interacting with clouds -- which helps constrain the
%expected brightness.
I wonder if there's some way to adjust the results based on local density
estimates, or just blindly regress against multiplicative combinations of other
salient physical variables.
Nevertheless, present the result, since it is at least partially physically
motivated.
% Shklovskii 1960
% Clark & Caswell 1976
% Milne 1979
% you must understand the argument before rejecting it out of hand.

\section{Conclusions}

TBD

\section{Misc}

Questions for self-edification (many things I don't understand):
\begin{itemize}
    \item What is the characteristic cooling timescale (adiabatic expansion?)
        for reverse shock heated ejecta?
        Under what conditions (density, turbulent mixing, ...)
        would conduction or other energy transfer processes play a role?
    \item What is the state of modeling for young remnant evolution?
        Could I fire up CR-Hydro-NEI or some other simple 1-D model,
        obtain densities, ionization times, temperature as a function of
        radius, and construct model remnant images?
        YES: Ferrand+ 2014 show lovely 3D images (don't trace reverse shock all
        the way).
        Lee+ 2014 for CR Hydro NEI
\end{itemize}

\acknowledgments

X acknowledges support by contract ...

This research is based on observations obtained with XMM-Newton, an ESA science
mission with instruments and contributions directly funded by ESA Member States
and NASA.
The MOST is operated by The University of Sydney with support from the
Australian Research Council and the Science Foundation for Physics within The
University of Sydney.

This research has made extensive use of NASA's Astrophysics Data System.
% TODO is software acknowledgement sufficient?
This research used the SNR catalog of \citet{ferrand2012}.
This research made use of Astropy, a community-developed core Python package
for Astronomy \citep{astropy2013}.
This research also made use of APLpy, an open-source plotting package for
Python hosted at \url{http://aplpy.github.com}.
This research was expedited in part by Jonathan Sick's \texttt{ads2bibdesk}.

\facility{XMM(EPIC), Molonglo Observatory}
% Chandra(ACIS) -- pending
\software{APLpy, Astropy, XSPEC}

%\listofchanges

% ==========
% References
% ==========
\bibliographystyle{aasjournal}
\bibliography{refs-snr}

% ========
% Appendix
% ========
\clearpage  % Use \clearpage over \newpage
\appendix

\setcounter{table}{0}
\renewcommand{\thetable}{A\arabic{table}}
\setcounter{figure}{0}
\renewcommand{\thefigure}{A\arabic{figure}}

\section{Solar wind charge exchange checks}

% TODO explain why we don't care about this
We also neglect low-level solar wind charge exchange (SWCX) contamination,
which may also vary between the two widely-separated XMM pointings.


\section{Sky X-ray background modeling caveats}

The sky X-ray background may not be constant over scales of several arcminutes.
\citet{henley2013} modeled galactic halo emission using 110 XMM-Newton
observations well outside the galactic plane.
Most of their observations were widely separated, but a group of pointings
within $\abt 30\arcmin$ (labeled 103.1--103.27) showed halo temperature
variations $\abt 10$--$20\%$ and emission measure variation within a factor of
$\abt 2$.
% TODO I am totally just eyeballing the numbers right now -- not that useful.
% Possibly remove this.
% Henley 2013: looking at 28 clustered observations well out of the galactic
% plane, sightlines 103.1 to 103.27; 103.8 is two observations.
% SZE SurF project to complement SPT,APEX,ACT study.
Taking background from an annulus around the remnant should average out some
variation, although without knowing the power spectrum, we cannot say anything
quantitative.

The X-ray background parameters from our integrated source and background fit
may be biased by our assumption that a single-temperature NEI plasma adequately
describes integrated remnant emission.
To check this, we fit the background annulus alone, which removes this possible
bias but sacrifices background counts embedded in the source spectrum.
The resulting X-ray background parameters agree within error.
% TODO reference appendix table
% TODO give a sentence to this effect in main body.

\section{Is a nonthermal component compatible with integrated remnant fits?}

% Random question: should nonthermal be hyphenated (non-thermal) or not?
We do not detect nonthermal (power-law-like) X-ray emission.
By eye, we estimate that nonthermal soft X-rays must be less than $\abt 10\%$
of the thermal continuum.

Fits with a single nonthermal component (XSPEC \texttt{powerlaw} or
\texttt{srcut}) limit nonthermal emission to, qualitatively, less than 10\% of
the contribution from thermal emission \citep[cf.][]{reynolds1999}.
Figure~\ref{fig:nonthermal} shows the best fit nonthermal contributions for
each model.
For a power law, we obtain photon index $\Gamma = 4.5 \pm 1$ with
normalization $1.4^{+1.1}_{-0.09} \times 10^{-3} \unit{photons\;s^{-1}\;cm^{-2}\;keV^{-1}}$
at $1 \unit{keV}$.
For \texttt{srcut}, we obtain break frequency $10^{15.7 \pm 0.1}$
corresponding to photon energy cut-off $\abt 0.02 \unit{keV}$.
% Source: 20160630_src_srcutlog_nonsolar_snr_src.txt

Note: the radio spectral index is $0.53$ as derived by \citet{gaensler1998-g309}.
His calculation ignores single dish measurements on the basis that the flux
is confused with RCW 80 emission.  Some authors made efforts to correct their
flux measurements for the confusion; if we include Parkes single dish
measurements the radio index decreases to $0.36 \pm 0.11$.
My own fit obtains $0.52 \pm 0.09$, marginally different than that of
\citet{gaensler1998-g309}, but for simplicity and consistency just use their values.

I also fit the outer annulus emission with a nonthermal component, but the
(rather questionable) fit allowed no nonthermal emission at all.
Given that the supernova remnant's thermal emission is already ill-constrained,
this is not surprising.
I could re-attempt this with a $350$--$450\arcsec$ annulus instead of
$400$--$500\arcsec$.  % TODO - investigate this.

\begin{figure*}[]
    \plotone{fig/fig_src_powerlaw_0087940201-mos1.pdf}
    \plotone{fig/fig_src_srcutlog_0087940201-mos1.pdf}
    \figcaption{Integrated source fit with powerlaw and srcutlog components
        permit only a near-negligible contribution.}
    \label{fig:nonthermal}
\end{figure*}


\section{Does inaccuracy introduced by $\abt 10\%$ error in BACKSCAL ratio for
integrated source fits impact fit results?}

I performed joint fits of integrated source and X-ray background while
varying the BACKSCAL ratio for MOS1S001.
Results in Table~\ref{tab:backscal-hack}.

\begin{table*}[!h]
    \centering
    \caption{2009 (Motch) MOS1 BACKSCAL ratio has very little effect on fits
        \label{tab:backscal-hack}}

    \begin{tabular}{@{}lllllll@{}}
    \toprule
    Ratio & $n_\mathrm{H}$ & $kT$ & $\tau$ & Si & S & vnei EM \\
     & ($10^{22} \unit{cm^{-2}}$) & (keV) & ($10^{10} \unit{s\;cm^{-3}}$) & (-) & (-) & (EM units) \\
    \midrule
    % 1 (hack) = results_spec/20160624_src_bkg_hack_eq_one_rerun fit
    \textbf{1}     & ${2.16}^{+0.08}_{-0.11}$ & ${2.29}^{+0.30}_{-0.12}$ & ${1.82}^{+0.14}_{-0.11}$
          & ${3.75}^{+0.18}_{-0.16}$ & ${3.42}^{+0.29}_{-0.25}$ & ${4.5}^{+0.6}_{-0.5} \times 10^{-3}$ \\
    % 0.95 (hack) = results_spec/20160611_src_bkg_rerun, manually entered!
    \textbf{0.95}  & $2.15^{+0.08}_{-0.11}$ & $2.29^{+0.30}_{-0.24}$ & $1.83^{+0.10}_{-0.11}$  % vnei nH, kT, Tau
          & $3.75^{+0.18}_{-0.16}$ & $3.42^{+0.26}_{-0.25}$ & $4.5^{+0.6}_{-0.5} \times 10^{-3}$ \\ % vnei Si,S,norm
    % 0.88 (area ratio) = results_spec/20160624_src_bkg_nohack_rerun fit
    \textbf{0.885} & ${2.15}^{+0.07}_{-0.08}$ & ${2.30}^{+0.29}_{-0.12}$ & ${1.83}^{+0.11}_{-0.11}$
          & ${3.75}^{+0.18}_{-0.16}$ & ${3.42}^{+0.28}_{-0.25}$ & ${4.6}^{+0.5}_{-0.5} \times 10^{-3}$ \\
    \bottomrule
    \end{tabular}

    \quad
    \quad

    \begin{tabular}{@{}llllr@{}}
    \toprule
    Ratio & XRB local $kT$ & XRB $n_\mathrm{H}$ & XRB halo $kT$
           & $\chi^2_{\mathrm{red}} = \chi^2/\mathrm{dof}$ \\
     & (keV) & ($10^{22} \unit{cm^{-2}}$) & (keV) &  \\
    \midrule
    % 1 (hack) = results_spec/20160624_src_bkg_hack_eq_one_rerun fit
    \textbf{1} & ${0.262}^{+0.007}_{-0.007}$ & ${1.33}^{+0.08}_{-0.08}$ & ${0.75}^{+0.04}_{-0.03}$
          & 1.224 = 4611.94/3768 \\
    % 0.95 (hack) = results_spec/20160611_src_bkg_rerun, manually entered!
    \textbf{0.95} & ${0.262}^{+0.007}_{-0.007}$ & $1.32^{+0.08}_{-0.08}$ & $0.75^{+0.04}_{-0.03}$ % XRB
          & 1.213 = 4572.28/3768 \\
    % 0.88 (area ratio) = results_spec/20160624_src_bkg_nohack_rerun fit
    \textbf{0.885} & ${0.262}^{+0.007}_{-0.007}$ & ${1.32}^{+0.08}_{-0.08}$ & ${0.74}^{+0.03}_{-0.03}$
          & 1.206 = 4545.24/3768 \\
    \bottomrule
    \end{tabular}
\end{table*}

% Deluxetable variant

%\floattable
%\begin{deluxetable}{@{}rllllllllll@{}}
%    \rotate
%    \tablecaption{Motch MOS1 BACKSCAL ratio has very little effect on fits
%        \label{table:backscal-hack}}
%    \tablehead{
%          \colhead{Ratio}
%        & \colhead{$n_\mathrm{H}$ }%% $10^{22} \unit{cm^{-2}}$}
%        & \colhead{$kT$ }%% keV}
%        & \colhead{$\tau$ }%% $10^{10} \unit{s\;cm^{-3}}$}
%        & \colhead{Si}
%        & \colhead{S}
%        & \colhead{vnei EM }%% EM units}
%        & \colhead{XRB local $kT$ }%% keV}
%        & \colhead{XRB $n_\mathrm{H}$ }%% $10^{22} \unit{cm^{-2}}$}
%        & \colhead{XRB halo $kT$ }%% keV}
%        & \colhead{$\chi^2_{\mathrm{red}} = \chi^2/\mathrm{dof}$}
%        \\
%          \colhead{}
%        & \colhead{($10^{22} \unit{cm^{-2}}$)}
%        & \colhead{(keV)}
%        & \colhead{($10^{10} \unit{s\;cm^{-3}}$)}
%        & \colhead{(-)}
%        & \colhead{(-)}
%        & \colhead{(EM units)}
%        & \colhead{(keV)}
%        & \colhead{($10^{22} \unit{cm^{-2}}$)}
%        & \colhead{(keV)}
%        & \colhead{}
%      }
%
%    \startdata
%    % 1 (hack) = results_spec/20160624_src_bkg_hack_eq_one_rerun fit
%    \textbf{1}     & ${2.16}^{+0.08}_{-0.11}$ & ${2.29}^{+0.30}_{-0.12}$ & ${1.82}^{+0.14}_{-0.11}$
%          & ${3.75}^{+0.18}_{-0.16}$ & ${3.42}^{+0.29}_{-0.25}$ & ${4.5}^{+0.6}_{-0.5} \times 10^{-3}$
%          & ${0.262}^{+0.007}_{-0.007}$ & ${1.33}^{+0.08}_{-0.08}$ & ${0.75}^{+0.04}_{-0.03}$
%          & 1.224 = 4611.939/3768 \\
%    % 0.95 (hack) = results_spec/20160611_src_bkg_rerun, manually entered!
%    \textbf{0.95}  & $2.15^{+0.08}_{-0.11}$ & $2.29^{+0.30}_{-0.24}$ & $1.83^{+0.10}_{-0.11}$  % vnei nH, kT, Tau
%          & $3.75^{+0.18}_{-0.16}$ & $3.42^{+0.26}_{-0.25}$ & $4.5^{+0.6}_{-0.5} \times 10^{-3}$  % vnei Si,S,norm
%          & ${0.262}^{+0.007}_{-0.007}$ & $1.32^{+0.08}_{-0.08}$ & $0.75^{+0.04}_{-0.03}$ % XRB
%          & 1.213 = 4572.28/3768 \\
%    % 0.88 (area ratio) = results_spec/20160624_src_bkg_nohack_rerun fit
%    \textbf{0.885} & ${2.15}^{+0.07}_{-0.08}$ & ${2.30}^{+0.29}_{-0.12}$ & ${1.83}^{+0.11}_{-0.11}$
%          & ${3.75}^{+0.18}_{-0.16}$ & ${3.42}^{+0.28}_{-0.25}$ & ${4.6}^{+0.5}_{-0.5} \times 10^{-3}$
%          & ${0.262}^{+0.007}_{-0.007}$ & ${1.32}^{+0.08}_{-0.08}$ & ${0.74}^{+0.03}_{-0.03}$
%          & 1.206 = 4545.244/3768 \\
%    \enddata
%\end{deluxetable}

\section{Does fitting four versus five annuli make any difference?}

Answer: no, parameters are identical within error (and the change in parameter
values is much smaller than error).
This is already shown in Table~\ref{tab:annulus-varycenter}.
For thoroughness, I give fit parameters from ``stock'' fits (only Si and S
free for each annulus) in Tables~\ref{tab:fiveann-stock} and
\ref{tab:fourann-stock}.

I also verified by eye that
(1) all instrumental line normalizations agree to $<1\%$ (often $\lesssim0.1\%$),
(2) remnant model norms agree to $\lesssim 1$\%,
(3) soft proton norms and indices agree to $\lesssim 1\%$.

In short, we are good to go.

\begin{table*}
    \centering
    \caption{Five annulus fit \label{tab:fiveann-stock}}
    \begin{tabular}{@{}rlllll@{}}
        \toprule
        Annulus & $n_\mathrm{H}$ & $kT$ & $\tau$ & Si & S \\
         & ($10^{22} \unit{cm^{-2}}$) & (keV) & ($10^{10} \unit{s\;cm^{-3}}$) & (-) & (-) \\
        \midrule
        % 20160701_fiveann.json
        $0$--$100\arcsec$   & ${2.10}^{+0.05}_{-0.05}$ & ${3.82}^{+0.51}_{-0.47}$ & ${1.57}^{+0.15}_{-0.13}$ & ${1.98}^{+0.14}_{-0.13}$ & ${1.45}^{+0.30}_{-0.26}$ \\
        $100$--$200\arcsec$ &                          & ${2.43}^{+0.36}_{-0.34}$ & ${1.72}^{+0.16}_{-0.14}$ & ${5.70}^{+0.39}_{-0.35}$ & ${5.49}^{+0.69}_{-0.59}$ \\
        $200$--$300\arcsec$ &                          & ${2.09}^{+0.25}_{-0.26}$ & ${2.16}^{+0.23}_{-0.22}$ & ${4.10}^{+0.26}_{-0.22}$ & ${3.78}^{+0.41}_{-0.35}$ \\
        $300$--$400\arcsec$ &                          & ${2.52}^{+0.61}_{-0.52}$ & ${1.73}^{+0.37}_{-0.28}$ & ${3.38}^{+0.42}_{-0.30}$ & ${3.08}^{+0.87}_{-0.59}$ \\
        $400$--$500\arcsec$ &                          & ${0.62}^{+0.20}_{-0.09}$ & ${\left(5 \times 10^3\right)}^{?}$                  & ${8.30}^{>1.7?}_{-2.49}$ & $>7.19$ \\
            % S abundance was ${10.00}^{?}_{-2.81}$
        \bottomrule
    \end{tabular}
\end{table*}

\begin{table*}
    \centering
    \caption{Four annulus fit \label{tab:fourann-stock}}
    \begin{tabular}{@{}rlllll@{}}
        \toprule
        Annulus & $n_\mathrm{H}$ & $kT$ & $\tau$ & Si & S \\
         & ($10^{22} \unit{cm^{-2}}$) & (keV) & ($10^{10} \unit{s\;cm^{-3}}$) & (-) & (-) \\
        \midrule
        % 20160706_fourann_stock
        $000$--$100\arcsec$ & ${2.11}^{+0.04}_{-0.07}$ & ${3.79}^{+0.55}_{-0.42}$ & ${1.57}^{+0.14}_{-0.13}$ & ${1.98}^{+0.14}_{-0.13}$ & ${1.45}^{+0.30}_{-0.26}$ \\
        $100$--$200\arcsec$ &                       & ${2.38}^{+0.24}_{-0.29}$ & ${1.74}^{+0.16}_{-0.14}$ & ${5.7}^{+0.4}_{-0.3}$ & ${5.5}^{+0.7}_{-0.6}$ \\
        $200$--$300\arcsec$ &                       & ${2.08}^{+0.24}_{-0.12}$ & ${2.16}^{+0.30}_{-0.21}$ & ${4.1}^{+0.3}_{-0.2}$ & ${3.8}^{+0.4}_{-0.3}$ \\
        $300$--$400\arcsec$ &                       & ${2.45}^{+0.53}_{-0.44}$ & ${1.74}^{+0.37}_{-0.28}$ & ${3.4}^{+0.4}_{-0.3}$ & ${3.1}^{+0.9}_{-0.6}$ \\
        \bottomrule
    \end{tabular}
\end{table*}


\section{Frequently asked questions (or, nitpickables)}

Here are caveats, disclaimers, and possible points of contention that are not
discussed in the main text for brevity, but should be itemized and reviewed.

\begin{itemize}
    \item \textbf{Why are we using Chandra-derived extragalactic X-ray background
        parameters instead of XMM-derived parameters?}
        Simply, I am more inclined to trust Chandra background modeling and
        background subtraction.
        The \citet{hickox2006} values agree to $\abt 10\%$ of values derived
        from \textit{XMM-Newton}, \textit{Swift}, \textit{ASCA}, and
        \textit{ROSAT} studies
        \citep{chen1997, kushino2002, de-luca2004, moretti2009}.
        \textbf{Do we need to adjust the normalization because XMM resolves
        fewer extragalactic sources than Chandra?}
        If I recall correctly from previous reading (forgot which source(s)) --
        small-number statistics prevail, so a few point sources removed don't
        matter.  It could be more significant for wide-field studies?
    \item \textbf{Why are halo emission and extragalactic background subject to
        the same absorption?}
        Yes, this is unrealistic.  Halo emission should be absorbed by some
        distance-and-density-averaged column, whereas the extragalactic
        background is attenuated by the complete galactic column.
        But, our X-ray background is decently well fit by our current model,
        and we don't see any need to introduce more parameters.
        The absorptions would only differ by a factor of $\abt 2$ for each
        component, and the discrepancy might be folded into unabsorbed local
        emission or the soft proton components.
        However we have not explored this in depth, and it is possible that our
        assessment of the background would change.
\end{itemize}


%\begin{table*}
%    \centering
%    \caption{G309.2-0.6 -- sub-source region fits, $n_H=2.5$ fixed}
%    \begin{tabular}{@{}lrrrrrr@{}}
%        \toprule
%        Region & $n_\mathrm{H}$             & $kT$  & $\tau$                        & Si  & S   & $\chi^2_{\mathrm{red}} (\mathrm{dof}$) \\
%               & ($10^{22} \unit{cm^{-2}}$) & (keV) & ($10^{10} \unit{s\;cm^{-3}}$) & (-) & (-) &  \\
%        \midrule
%        north clump & 2.50 & 1.07 & 4.79e+10 & 5.73 & 5.22 & 1.339 (693) \\  % File src_north_clump_nH-2.5_row.tex
%        E lobe & 2.50 & 1.74 & 2.71e+10 & 5.47 & 3.69 & 1.016 (208) \\  % File src_E_lobe_nH-2.5_row.tex
%        SW lobe & 2.50 & 1.27 & 1.37e+10 & 3.46 & 5.78 & 1.429 (439) \\  % File src_SW_lobe_nH-2.5_row.tex
%        SE dark & 2.50 & 0.68 & 8.80e+10 & 2.79 & 2.30 & 0.983 (223) \\  % File src_SE_dark_nH-2.5_row.tex
%        \midrule
%        ridge & 2.50 & 0.81 & 8.69e+10 & 2.63 & 2.10 & 1.467 (226) \\  % File src_ridge_nH-2.5_row.tex
%        SE dark ridge & 2.50 & 0.78 & 2.51e+11 & 1(fr) & 1(fr) & 1.232 (105) \\  % File src_SE_ridge_dark_nH-2.5_row.tex
%        \bottomrule
%    \end{tabular}
%\end{table*}
%
%\begin{table*}
%    \centering
%    \caption{G309.2-0.6 -- annulus fits, $n_H=2.5$ fixed}
%    \begin{tabular}{@{}lrrrrrr@{}}
%        \toprule
%        Region & $n_\mathrm{H}$             & $kT$  & $\tau$                        & Si  & S   & $\chi^2_{\mathrm{red}} (\mathrm{dof}$) \\
%               & ($10^{22} \unit{cm^{-2}}$) & (keV) & ($10^{10} \unit{s\;cm^{-3}}$) & (-) & (-) &  \\
%        \midrule
%        $0$--$100\arcsec$ & 2.50 & 2.33 & 1.58e+10 & 1.74 & 1.27 & 2.756 (357) \\  % File ann_000_100_nH-2.5_row.tex
%        $100$--$200\arcsec$ & 2.50 & 1.22 & 2.72e+10 & 5.29 & 5.04 & 1.210 (560) \\  % File ann_100_200_nH-2.5_row.tex
%        $200$--$300\arcsec$ & 2.50 & 1.07 & 4.51e+10 & 3.83 & 3.56 & 1.291 (793) \\  % File ann_200_300_nH-2.5_row.tex
%        $300$--$400\arcsec$ & 2.50 & 0.78 & 9.05e+10 & 4.11 & 3.74 & 1.275 (800) \\  % File ann_300_400_nH-2.5_row.tex
%        $400$--$500\arcsec$ & 2.50 & 0.99 & 3.99e+10 & 8.57 & 10.19 & 1.188 (823) \\  % File ann_400_500_nH-2.5_row.tex
%        \bottomrule
%    \end{tabular}
%\end{table*}

%\begin{table*}
%    \centering
%    \caption{sub-region fits, varying $n_H$}
%    \begin{tabular}{@{}lrrrrrr@{}}
%        \toprule
%        Region & $n_\mathrm{H}$             & $kT$  & $\tau$                        & Si  & S   & $\chi^2_{\mathrm{red}} (\mathrm{dof}$) \\
%               & ($10^{22} \unit{cm^{-2}}$) & (keV) & ($10^{10} \unit{s\;cm^{-3}}$) & (-) & (-) &  \\
%        \midrule
%        north clump & 1.50 & 8.50 & 1.81e+10 & 8.76 & 8.92 & 1.933 (692) \\  % File 20160420_src_north_clump_nH_1.5_row.tex
%        north clump & 2.00 & 2.42 & 2.08e+10 & 6.52 & 5.95 & 1.623 (692) \\  % File 20160420_src_north_clump_nH_2.0_row.tex
%        north clump & 2.50 & 1.33 & 3.23e+10 & 5.15 & 4.59 & 1.361 (692) \\  % File 20160420_src_north_clump_nH_2.5_row.tex
%        north clump & 3.00 & 0.91 & 5.46e+10 & 4.27 & 3.98 & 1.338 (692) \\  % File 20160420_src_north_clump_nH_3.0_row.tex
%        \midrule
%        E lobe & 1.50 & 8.62 & 2.02e+10 & 9.71 & 7.85 & 1.331 (207) \\  % File 20160420_src_E_lobe_nH_1.5_row.tex
%        E lobe & 2.00 & 5.68 & 1.89e+10 & 5.80 & 4.25 & 1.086 (207) \\  % File 20160420_src_E_lobe_nH_2.0_row.tex
%        E lobe & 2.50 & 2.75 & 2.03e+10 & 4.53 & 2.97 & 0.979 (207) \\  % File 20160420_src_E_lobe_nH_2.5_row.tex
%        E lobe & 3.00 & 1.73 & 2.46e+10 & 3.65 & 2.30 & 0.941 (207) \\  % File 20160420_src_E_lobe_nH_3.0_row.tex
%        \midrule
%        SW lobe & 1.50 & 9.21 & 1.44e+10 & 3.98 & 4.79 & 1.390 (438) \\  % File 20160420_src_SW_lobe_nH_1.5_row.tex
%        SW lobe & 2.00 & 3.09 & 1.27e+10 & 3.40 & 4.11 & 1.326 (438) \\  % File 20160420_src_SW_lobe_nH_2.0_row.tex
%        SW lobe & 2.50 & 1.64 & 1.28e+10 & 3.06 & 4.50 & 1.403 (438) \\  % File 20160420_src_SW_lobe_nH_2.5_row.tex
%        SW lobe & 3.00 & 1.13 & 1.30e+10 & 2.93 & 5.32 & 1.522 (438) \\  % File 20160420_src_SW_lobe_nH_3.0_row.tex
%        \midrule
%        SE dark & 1.50 & 5.47 & 1.69e+10 & 3.26 & 2.39 & 1.079 (222) \\  % File 20160420_src_SE_dark_nH_1.5_row.tex
%        SE dark & 2.00 & 1.79 & 2.43e+10 & 2.54 & 1.56 & 0.996 (222) \\  % File 20160420_src_SE_dark_nH_2.0_row.tex
%        SE dark & 2.50 & 1.00 & 4.60e+10 & 2.10 & 1.32 & 0.956 (222) \\  % File 20160420_src_SE_dark_nH_2.5_row.tex
%        SE dark & 3.00 & 0.70 & 7.14e+10 & 1.83 & 1.35 & 0.985 (222) \\  % File 20160420_src_SE_dark_nH_3.0_row.tex
%        \midrule
%        ridge & 1.50 & 3.46 & 2.13e+10 & 4.58 & 3.22 & 2.476 (225) \\  % File 20160420_src_ridge_nH_1.5_row.tex
%        ridge & 2.00 & 1.64 & 2.88e+10 & 3.03 & 2.13 & 1.764 (225) \\  % File 20160420_src_ridge_nH_2.0_row.tex
%        ridge & 2.50 & 0.96 & 5.67e+10 & 2.33 & 1.74 & 1.504 (225) \\  % File 20160420_src_ridge_nH_2.5_row.tex
%        ridge & 3.00 & 0.66 & 1.03e+11 & 1.96 & 1.70 & 1.370 (225) \\  % File 20160420_src_ridge_nH_3.0_row.tex
%        \midrule
%        % NOTE: may need to re-do these fits
%        % (1) without SNR component, (2) with SNR, Si,S fixed
%        SE dark ridge & 1.50 & 2.26 & 8.48e+11 & 1.36 & 0.45 & 1.252 (102) \\  % File 20160420_src_SE_ridge_dark_nH_1.5_row.tex
%        SE dark ridge & 2.00 & 10.00 & 7.39e+09 & 3.95 & 10.00 & 1.081 (102) \\  % File 20160420_src_SE_ridge_dark_nH_2.0_row.tex
%        SE dark ridge & 2.50 & 3.48 & 6.78e+09 & 3.68 & 6.48 & 1.080 (102) \\  % File 20160420_src_SE_ridge_dark_nH_2.5_row.tex
%        SE dark ridge & 3.00 & 1.39 & 7.11e+09 & 3.85 & 7.64 & 1.073 (102) \\  % File 20160420_src_SE_ridge_dark_nH_3.0_row.tex
%        %SE dark ridge & 1.50 & 1.73 & 1.21e+11 & 1.00 & 1.00 & 1.344 (105) \\  % File src_SE_ridge_dark_nH-1.5_row.tex
%        %SE dark ridge & 2.00 & 1.36 & 6.70e+10 & 1.00 & 1.00 & 1.284 (105) \\  % File src_SE_ridge_dark_nH-2.0_row.tex
%        %SE dark ridge & 2.50 & 0.78 & 2.51e+11 & 1.00 & 1.00 & 1.232 (105) \\  % File src_SE_ridge_dark_nH-2.5_row.tex
%        %SE dark ridge & 3.00 & 0.79 & 5.14e+10 & 1.00 & 1.00 & 1.196 (105) \\  % File src_SE_ridge_dark_nH-3.0_row.tex
%        \bottomrule
%    \end{tabular}
%\end{table*}

%\begin{table*}
%    \centering
%    \caption{Annulus fits with varying $n_H$}
%    \begin{tabular}{@{}lrrrrrr@{}}
%        \toprule
%        Region & $n_\mathrm{H}$             & $kT$  & $\tau$                        & Si  & S   & $\chi^2_{\mathrm{red}} (\mathrm{dof}$) \\
%               & ($10^{22} \unit{cm^{-2}}$) & (keV) & ($10^{10} \unit{s\;cm^{-3}}$) & (-) & (-) &  \\
%        \midrule
%        000--100$\arcsec$ & 1.50 &10.00 & 1.75e+10 & 2.51 & 2.15 & 2.184 (352) \\  % File 20160420_ann_000_100_nH_1.5_row.tex
%        000--100$\arcsec$ & 2.00 & 4.55 & 1.59e+10 & 2.04 & 1.53 & 2.338 (352) \\  % File 20160420_ann_000_100_nH_2.0_row.tex
%        000--100$\arcsec$ & 2.50 & 2.43 & 1.58e+10 & 1.77 & 1.25 & 2.884 (352) \\  % File 20160420_ann_000_100_nH_2.5_row.tex
%        000--100$\arcsec$ & 3.00 & 1.58 & 1.78e+10 & 1.57 & 1.11 & 3.479 (352) \\  % File 20160420_ann_000_100_nH_3.0_row.tex
%        \midrule
%        100--200$\arcsec$ & 1.50 & 8.47 & 1.67e+10 & 7.56 & 7.97 & 1.375 (559) \\  % File 20160420_ann_100_200_nH_1.5_row.tex
%        100--200$\arcsec$ & 2.00 & 2.83 & 1.72e+10 & 5.96 & 5.64 & 1.196 (559) \\  % File 20160420_ann_100_200_nH_2.0_row.tex
%        100--200$\arcsec$ & 2.50 & 1.58 & 2.14e+10 & 4.93 & 4.63 & 1.210 (559) \\  % File 20160420_ann_100_200_nH_2.5_row.tex
%        100--200$\arcsec$ & 3.00 & 1.04 & 2.85e+10 & 4.29 & 4.21 & 1.303 (559) \\  % File 20160420_ann_100_200_nH_3.0_row.tex
%        \midrule
%        200--300$\arcsec$ & 1.50 & 6.67 & 1.84e+10 & 5.98 & 5.85 & 1.742 (792) \\  % File 20160420_ann_200_300_nH_1.5_row.tex
%        200--300$\arcsec$ & 2.00 & 2.39 & 2.12e+10 & 4.35 & 3.92 & 1.412 (792) \\  % File 20160420_ann_200_300_nH_2.0_row.tex
%        200--300$\arcsec$ & 2.50 & 1.37 & 3.09e+10 & 3.41 & 3.03 & 1.284 (792) \\  % File 20160420_ann_200_300_nH_2.5_row.tex
%        200--300$\arcsec$ & 3.00 & 0.91 & 5.39e+10 & 2.84 & 2.65 & 1.280 (792) \\  % File 20160420_ann_200_300_nH_3.0_row.tex
%        \midrule
%        300--400$\arcsec$ & 1.50 & 8.59 & 1.82e+10 & 4.57 & 4.19 & 1.255 (799) \\  % File 20160420_ann_300_400_nH_1.5_row.tex
%        300--400$\arcsec$ & 2.00 & 2.95 & 1.87e+10 & 3.56 & 2.98 & 1.202 (799) \\  % File 20160420_ann_300_400_nH_2.0_row.tex
%        300--400$\arcsec$ & 2.50 & 1.61 & 2.31e+10 & 2.92 & 2.49 & 1.201 (799) \\  % File 20160420_ann_300_400_nH_2.5_row.tex
%        300--400$\arcsec$ & 3.00 & 1.00 & 3.86e+10 & 2.56 & 2.19 & 1.220 (799) \\  % File 20160420_ann_300_400_nH_3.0_row.tex
%        \midrule
%        400--500$\arcsec$ & 1.50 & 10.00 & 2.20e+10 & 2.07 & 2.50 & 1.254 (822) \\  % File 20160420_ann_400_500_nH_1.5_row.tex
%        400--500$\arcsec$ & 2.00 &  4.34 & 4.34e+13 & 0.00 & 0.00 & 1.267 (822) \\  % File 20160420_ann_400_500_nH_2.0_row.tex
%            % This fit obviously failed
%        400--500$\arcsec$ & 2.50 & 10.00 & 1.53e+10 & 1.34 & 2.16 & 1.184 (822) \\  % File 20160420_ann_400_500_nH_2.5_row.tex
%        400--500$\arcsec$ & 3.00 & 10.00 & 1.32e+10 & 1.50 & 2.48 & 1.171 (822) \\  % File 20160420_ann_400_500_nH_3.0_row.tex
%        \bottomrule
%    \end{tabular}
%\end{table*}


\end{document}
