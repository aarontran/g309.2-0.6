\documentclass[preprint2,tighten,trackchanges]{aastex6}
%\documentclass[iop, tighten, apj, numberedappendix]{emulateapj} % Still more compact than aastex

% Customize figures/sizing for 2 column vs. manuscript printout
%%\usepackage{etoolbox}
%%\newtoggle{manuscript}
%%\toggletrue{manuscript}  % Set TRUE if using manuscript / 1-col layout
%%%\togglefalse{manuscript}  % Set FALSE if using 2-col layout

\shorttitle{Stuff (\today)}  % <~ 44 char
\shortauthors{Draft (\today)}  % Max three
\slugcomment{Draft, \today}

% Packages and commands
%\usepackage{amsmath}  % Included in aastex
\usepackage{booktabs}
\usepackage{hyperref}

% My "standard" TeX aliases
\newcommand*{\mt}{\mathrm}
\newcommand*{\unit}[1]{\;\mt{#1}}  % vemod.net/typesetting-units-in-latex
\newcommand*{\abt}{\mathord{\sim}} % tex.stackexchange.com/q/55701
\newcommand*{\ptl}{\partial}
\newcommand*{\del}{\nabla}
\newcommand*\mean[1]{\bar{#1}}
\renewcommand{\vec}[1]{\mathbf{#1}}  % Bold vectors
\newcommand*{\tsup}{\textsuperscript}

% Paper-specific commands
\newcommand*{\nH}{n_{\mathrm{H}}}
\newcommand*{\nHunits}{\times 10^{22} \unit{cm^{-2}}}
\newcommand*{\AV}{A_{\mathrm{V}}}
\defcitealias{rakowski2001}{RHS01}
\defcitealias{gaensler1998}{GGM98}

\begin{document}

\title{Fuzzy blobs in G309.2-0.6}

% Not quite in line with recommended aastex style
\author{
Joe Bob\altaffilmark{1}, Joe Smith\altaffilmark{2}
}

\affil{
\tsup{1}Smithsonian Astrophysical Observatory, 60 Garden Street MS-70, Cambridge, MA 02138, USA \\
\tsup{2}Department of Basket Weaving, Belafonte Hall, Univ. Calumny, Belabored, CA 00000, USA
}

%\received{receipt date}
%\revised{\today}
%\accepted{acceptance date}


\begin{abstract}
We image and fit spectra of G309.2-0.6 from archival XMM-Newton data.
The remnant is ejecta-dominated and hotter than expected for its HI-inferred
distance and hence physical size.

Sketch of a writeup, and a lot of scribbled notes.
\end{abstract}

% (!) no longer used by AAS as of late February, 2016
\keywords{acceleration of particles ---
    ISM: supernova remnants ---
    ISM: individual objects (SNR G309.2-0.6) ---
    X-rays: ISM
%    ISM: magnetic fields ---
%    shock waves
}

% =============================================================================
% Introduction
% =============================================================================
\section{Introduction} \label{sec:intro}

Motivations:
(1) typing remnants by ejecta,
(2) resolve ejecta spatial structure (layering, mixing),
(3) resolve interaction with surrounding medium

Individual object study to build information for more ambitious global studies
of remnant properties and interaction with the ISM -- not undertaken here.
Just one piece of a puzzle.

We present the first ``modern'' high-resolution X-ray image of G309.2-0.6 based
on two archival XMM-Newton observations.  We fit the CCD spectra of
morphologically distinct regions of the remnant to map the distribution of
ejecta abundances, plasma temperature, and ionization states.

% TODO Interpretation TBD -- so throw out text for now

% \subsection{SNR stuff}
%
% MMSNRs (Rho and Petre, 1998).  Tilley+ (2006) - model anisotropic thermal
% conduction in remnants in regular ISM vs. dense medium.
% Is this a mixed morphology SNR?
% Answer: not quite.  What other remnants look similar to this one?
% Rakowski did not label as such but noted ejecta dominated nature.
%
% Possible for MM to have some mixture of ejecta -- how have definitions /
% classifications evolved?
%
% MC interaction: what are the most common proxies and
% claims?
% Kilpatrick, Bieging, Rieke (2016) give a very lucid ALMA study of CO(2-1)
% broadening: http://iopscience.iop.org/article/10.3847/0004-637X/816/1/1/pdf
%
% On barrel morphology and magnetic field orientation: Kesteven and Caswell (1987),
% Gaensler (1998), West et al. (2016),
%
% General question: how can we safely infer that the ISM is solar in abundance,
% outside of our area?
%
% A comparative study may be interesting.
% G352.7-0.1 (Pannuti+ 2014, XMM/Chandra) exhibits similar features...
% G298.6-0.0 (Bamba+ 2015, Suzaku)


\subsection{Previous observations of G309.2-0.6}

% Radio remnant observations and morphology
G309.2-0.6 (J2000 RA 13h46m30s, dec. $-62\arcdeg 54\arcmin 00\arcsec$) was
discovered in Molonglo $408 \unit{MHz}$ and Parkes $5000 \unit{MHz}$ surveys
\citep{clark1973, green1974, clark1975}, and has since been observed multiple
times by southern radio telescopes \citep{caswell1981, kesteven1987,
whiteoak1996, gaensler1998}.
\citet{gaensler1998} (hereafter, \citetalias{gaensler1998}) provide the most
comprehensive radio study to date based upon Australia Telescope Compact Array
(ATCA) $1.344 \unit{GHz}$ continuum and $1.420 \unit{GHz}$ HI observations at
angular resolution $24\arcsec$.
The radio remnant is a circular shell with two bright lobes to the NE and SW,
hereafter referred to as ``ears'' (Fig \ref{fig:snr}).
Its flux density at $843 \unit{MHz}$ is $\abt 6 \unit{Jy}$, corresponding to a
surface brightness $5.4\times10^{-21} \unit{W \; m^{-2} \; Hz^{-1} \; sr^{-1}}$
\citep{whiteoak1996}.
% Comment: Bryan's ATCA data are publically available, but no real motivation
% to download and reduce at this time
% TODO -- we could provide surface brightness for the bright ears alone

% X-ray remnant observation(s) and morphology
X-ray emission was discovered in an Advanced Satellite for Cosmology and
Astrophysics (ASCA) survey of small remnants by \citet{rakowski2001}
(hereafter, \citetalias{rakowski2001}).
The X-ray remnant sits within the radio shell and is brightest towards the
north (Figure \ref{fig:snr}).
A faint X-ray arc on the remnant's northeast limb coincides with the radio
shell's limb and is most apparent in our $0.8$--$1.4 \unit{keV}$ band image.
Both radio ears are X-ray dark.

% Other wavelengths
The field of G309.2-0.6 has been well-observed in galactic plane surveys in
many wavelengths.
No H$\alpha$ filaments are observed (\textbf{SOURCE}) % TODO
% OH maser
No $1720.5 \unit{MHz}$ OH maser emission is observed above $40 \unit{mJy}$
\citep{green1997}.
Infrared images (Spitzer IRAC 3.6,4.5,5.8,$8 \unit{{\mu}m}$, MIPSGAL, WISE)
don't show any morphological features that, by eye, could be associated with
the remnant.
% Mopra CO survey - no public data yet, in works

\paragraph{Properties}

\textbf{These paragraphs could be each condensed into 1-2 sentences.}

% Galactic environment
The sight line towards G309.2-0.6 crosses the tangent of the Centaurus Arm
(variously Scu-Cen, Crux-Cen, Scu-Crux) and the near and far Carina arm at
$\abt 1.5 \unit{kpc}$ and $\abt 14 \unit{kpc}$.  Molecular and HI emission is
dominated by the Centaurus Arm at line-of-sight velocities $-70$ to
$-20 \unit{km/s}$ \citep[e.g.,][Figure 4]{dame2011}, and a good number of
molecular clouds can be identified \citep{rice2016}.  The field is populated
with unrelated but independently interesting features.

% Remnant distance in HI
The remnant distance is estimated to be within $5$--$14 \unit{kpc}$ based on HI
absorption out to the galactic rotation curve's tangent point, $v_{\mt{LSR}}
\sim -50 \unit{km/s}$, and a lack of absorption above $v_{\mt{LSR}} \sim +40
\unit{km/s}$ \citepalias{gaensler1998}.  The galactic rotation curve is from
\citet{fich1989} with galactic center distance $R_0 = 8.5 \unit{kpc}$ and local
circular velocity $\Theta = 220 \unit{km/s}$.

% sigma-D estimate of distance
From the somewhat questionable $\Sigma$-$D$ relation, \citet{pavlovic2014}
provide a distance estimate of $5.9 \unit{kpc}$.
This is fraught with peril.
At least the relation of \citet{pavlovic2014} is calibrated for shell-type
radio remnants.
%See also \citet{huang1985} which presents a $\Sigma$-$D$ relation for
%shell-like remnants also interacting with clouds -- which helps constrain the
%expected brightness.
I wonder if there's some way to adjust the results based on local density
estimates, or just blindly regress against multiplicative combinations of other
salient physical variables.
Nevertheless, present the result, since it is at least partially physically
motivated.
% Shklovskii 1960
% Clark & Caswell 1976
% Milne 1979

\begin{figure*}[]
    \plottwo{fig/fig_snr_xmm-broadband_most-sparse.pdf}{fig/fig_snr_most.pdf}
    %\caption{Broadband 0.8--3.3 keV image of SNR G309.2-0.6.}
    % figcaption is an aastex alias -- cannot use with emulateapj
    \figcaption{Left: XMM broadband 0.8--3.3 keV image of SNR G309.2-0.6,
        smoothed and log scaled, with $1.4 \unit{GHz}$ radio contours from
        right-side image.
        Right: MOST $1.4 \unit{GHz}$ image, arcsinh scaled.}
    \label{fig:snr}  % TODO this must follow figcaption, don't remember why
    % TODO properly explain MOST image or reference this image in text
%Plotted with 0.843 GHz radio contours from the  supernova remnant catalogue,
%with resolution $\abt 43\arcsec$ and sensitivity $2$ mJy/beam
%\citep{whiteoak1996}.
\end{figure*}

\paragraph{Environment}

% HD 119682 - previous interest and subsequent disassociation
A bright ROSAT source (1WGA J1346.5--6255) within G309.2-0.6 is the foreground
Be star HD 119682.
The source's position within a bilobed radio remnant led
\citetalias{gaensler1998} and \citetalias{rakowski2001} to suggest that
G309.2-0.6 might be a jet- or outflow-driven remnant, similar to the SS 443 /
W50 system.
% bilobed beats bilobate on Google by 278 to 66 (cf. Google ngrams)
But, HD 119682 is likely a member of the open cluster NGC 5281 at a distance
$\abt 1.4 \unit{kpc}$, based on X-ray and optical astrometry, cluster proper
motions, and X-ray spectrum fits \citep{rakowski2006, safi-harb2007,
torrejon2013}.
% The X-ray fitted absorption of $\nH \sim 0.2 \times 10^{22} \unit{cm^{-2}}$
% is consistent with the inferred distance to HD 119682 \citep{rakowski2006,
% safi-harb2007, torrejon2013}.

% Gum 48d / RCW 80
The H II region Gum 48d (RCW 80) lies to the north of G309.2-0.6.
Gum 48d is well traced out in HI, polycyclic aromatic hydrocarbon (PAH)
emission, and warm dust; assuming these emission features all correspond to the
same H II region, the distance to Gum 48d is $\abt 3.5\unit{kpc}$, based on
ionized gas emission velocities (H$\alpha$, HI) distance estimates to the
central star system HR 5171 \citep{karr2009}.
The distance estimate to Gum 48d, in conjunction with a lack of evidence for
remnant-cloud interaction, suggest that Gum 48d is also a foreground feature
unrelated to the remnant G309.2-0.6.  The practical consequence is that we will
subsequently ignore a north-south wisp of radio emission that appears to
connect the radio remnant and H II region.

% Centaurus CO cloud
Given the remnant's rough radio and X-ray symmetry, we may be able to disfavor
a location within the Centaurus arm.  \citet{saito2001} imaged
an extensive molecular CO complex in ${}^{12}$CO, ${}^{13}$CO, and C${}^{18}$O
J=(1-0) transitions at line-of-sight velocities $-64$ to $-36 \unit{km/s}$.
\textbf{TODO} this would need some CO to ISM density conversion estimate so
that we could estimate a density gradient across the remnant, if it were
associated with this cloud (which we think it is not).
For this see Bolatto, Wolfire and Leroy (2013).
It is clear that remnants expanding into a density gradient need not look that
asymmetric \citep{williams2013}, so \textbf{this is speculative and may be
outright wrong}.
But, I think we can start thinking about what kind of environments may be
relevant for this remnant's evolution.
% H II region = blob of natal stardust, shoved out and compressed by
% young OB star's radiation pressure, interacting with ambient ISM, nearby
% stars and clouds, etc...
%
% Here's a lovely optical photograph, clearly showing the bright H II region and
% the bluer star cluster NGC 5281 to the south:
% https://it.wikipedia.org/wiki/File:RCW_80.jpg

% Mopra CO J(1-0) survey, resolution 30 arcsec and 0.1 km/s, would be huge!
% Looking at our crude rotation curve -- beyond tangent point, 100 km/s per 10kpc
% is the approximate slope; 0.1 km/s means we have a distance resolution of order
% 10 parsecs, comparable to or smaller than this remnant size.
% Compare, Dame survey has beamwidth ~ 1/8 deg. = 7.5 arcmin = 450 arcsec

% ATCA data -- cannot access at http://www.atnf.csiro.au/research/HI/sgps/queryForm.html
% SGPS survey -- taken 1998-2000.
% Emailed a feedback query on 2016 April 13... probably won't have much luck.

% TODO: check out ATCA J134649--625235, mentioned in Gaensler

% Summary / wrap-up
\citetalias{gaensler1998} and \citetalias{rakowski2001} have given the most comprehensive radio and
X-ray imaging studies of G309.2-0.6 to date.
The remnant has received little attention since -- not unsurprising for its
location in the southern sky, overlapping foreground objects, and non-detection
of clearly associated emission outside of radio and X-ray wavelengths.
Here, we provide the first spatially resolved X-ray study of this remnant based
entirely on archived XMM-Newton observations.

%
%Safi-Harb report:
%* vnei fit with supersolar Si/S, subsolar O/Ne/Mg/Ca/Fe.
%  kT = 2 +/- 0.6 keV,
%  nH = 0.65 (+0.45/-0.25) x 10^22 cm^{-2}
%
%  This is extremely odd to me -- does not match my results at ALL.
%
%  Suggest mismatch with Rakowski et al. due to confusion w/ bright star
%  main effect is to add a fair amount of broad soft emission
%
%* multiple arguments for distance to HD119682
%  spectrum fit -> nH ~ 0.2 x 10^22 cm^{-2}
%  based on both Chandra and XMM-Newton fits
%
%Rakowski 2001
%    kT ~ 2 keV,
%    nH = 2 x 10^22
%    Tau ~ 4.6e3 cm^{-3} year = 1.45e11 cm^{-3} s
%
%Safi-Harb don't discuss how background modeling is done, nor instrumental lines
%etc -- which I imagine would affect results...
%Rakowski+ does thoroughly explain their model for modeling out point source
%contamination.
%
%Safi-Harb region is basically the "north clump"


% =============================================================================
% Observations + Reduction
% =============================================================================
\section{Observations and Data Reduction} \label{sec:obs}

% Summary of observations -- same as Table 1 and could be cut to 1-2 sentences
XMM-Newton's European Photon Imaging Camera (EPIC) observed G309.2-0.6 for a
total of $97.73 \unit{ks}$ in two pointings on 2001 August 28 and 2009 March
6--7.  The first pointing (obsid 0087940201; PI John P. Hughes) was $40.46
\unit{ks}$ with MOS1/2 in Full Frame mode, PN in Extended Full Frame mode, and
XMM's ``thick'' optical filter.
The second pointing (obsid 0551000201; PI Christian Motch) was $57.27
\unit{ks}$  with MOS1/2 in Full Frame mode, PN in Large Window mode, and XMM's
``medium'' optical filter.
The Motch pointing targeted the foreground Be star HD 119682, but as a
byproduct captured most of G309.2-0.6 on all three XMM detectors.

% TODO I have not thought at all about the effects of optical loading on these
% observations...

\begin{table*}
    \centering
    \caption{XMM Observations of G309.2-0.6\label{tab:obs}}
    %% TODO use AASTEX table environment -- allow column hiding etc...
    \begin{tabular}{@{}lrrrlrlr@{}}
        \toprule
        Obs. ID & Dur. & $t_{\mt{\,live},\,\mt{MOS}}$ & $t_{\mt{\,live},\,\mt{PN}}$
            & Date & Rev. & Filter & PI \\
        \midrule
        0087940201 & $40.5$ & $25.3$ & $18.0$ & 2001 August 28 & 315 & Thick & Hughes \\
        0551000201 & $57.3$ & $21.9$ &  $9.0$ & 2009 March 6--7 & 1692 & Medium & Motch \\
        \bottomrule
    \end{tabular}
    \tablecomments{Durations are in kiloseconds (ks).
    Good duration $t_{\mt{live}}$ is central CCD live time after flare
    filtering; MOS value averages MOS1 and MOS2 times ($\sim 0.6 \unit{ks}$
    difference in both obsids).
    Rev. is XMM-Newton orbit (revolution) number.}
\end{table*}

% TODO how to refer to each observation in a simple way?
% for now just using ``Hughes'' and ``Motch''

% Position angles listed in FITS header differs by ~0.2 degrees from column
% values given on XMM-Newton science archive query pages.  Why?
%   0087940201 PA_PNT = 311.11 deg.
%   0551000201 PA_PNT = 138.77 deg.
% XMM-Newton page: 310.9, 138.5 deg.
% Anyways I decided to omit it because we don't need to report it.

% Current good time values (2016 April 21. Somewhat old extraction, pipeline
% last re-run in Feb or Mar after SAS v15 release)
%   0087940201 mos1S001 livetime 25.016 ks, ontime 25.290 ks
%   0087940201 mos2S002 livetime 25.621 ks, ontime 25.891 ks
%   0087940201 pnS003   livetime 18.011 ks, ontime 20.979 ks
%   0551000201 mos1S001 livetime 21.621 ks, ontime 21.989 ks
%   0551000201 mos2S002 livetime 22.167 ks, ontime 22.545 ks
%   0551000201 pnS003   livetime  8.981 ks, ontime  9.720 ks

% Event lists, soft proton light curve filtering
We reduce the data with XMM's Science Analysis System (SAS) v15.0.0 and
Extended Science Analysis System (ESAS) v5.9 \citep{snowden2008, kuntz2008}.
\footnote{\href{http://heasarc.gsfc.nasa.gov/docs/xmm/esas/cookbook/xmm-esas.html}{http://heasarc.gsfc.nasa.gov/docs/xmm/esas/cookbook/xmm-esas.html}}
Event lists are created with SAS tasks \texttt{emchain} and \texttt{epchain}
(with \texttt{epchain} run twice to generate out-of-time events).
% TODO shorten emchain/epchain sentence
Strong soft proton flares are filtered by ESAS tasks \texttt{mos-filter} and
\texttt{pn-filter}, which fit a Gaussian to a histogram of time-binned count
rates for each exposure and discard time intervals where the count rate is
$1.5\sigma$ above the fitted Gaussian mean.  % TODO - sentence needs retooling
We further inspect the $2.5$--$12 \unit{keV}$ field-of-view light curves and
manually cut some brief good-time intervals surrounded by higher count rate
flares.
About half to two-thirds of each observation is lost to flares, with more
severe loss for PN exposures.
The resulting live times (good time intervals less CCD readout time) are
given in Table~\ref{tab:obs}.

% Point sources
We remove point sources with the ESAS task \texttt{cheese} and manually inspect
the masks to confirm removal of obvious point sources.
\textbf{TODO: merge pt source removal lists from both obsids}  %TODO
In particular, HD 119682 requires an exclusion mask $\sim 1\arcmin$ in radius.

% CCD exclusions
In obs. ID 0551000201, we excluded two MOS CCDs and the entire PN exposure from
our spectral analysis.
MOS2 CCD5 showed anomalously high soft X-ray background noise in its corner
counts \citep[cf.][]{kuntz2008}; MOS1 CCD6 was disabled by a presumed
micrometeorite in 2005.
The PN in Large Window mode does not collect corner counts, so we are unable to
estimate the detector background from ESAS tasks; moreover, the good time is
$\lesssim 10\%$ of our available exposures due to severe flare contamination.
% We do include PN counts in our non-background-subtracted broad- and
% narrow-band images.


\subsection{Spectrum Extraction and Modeling}

% Spectrum and QPB creation
Spectra are created from ``good'' single and double events (i.e.,
\texttt{PATTERN} $\leq 12,4$ for MOS, PN respectively; \texttt{FLAG} $= 0$ for
both MOS and PN), using ESAS tasks \texttt{\{mos,pn\}-spectra}.
Our models and spectral fits are performed in XSPEC 12.9.0d \citep{arnaud1996}
using the T{\"u}bingen-Boulder abundances and absorption model
\citep{wilms2000}
% TODO TODO download tb_new and use that!!!!!
% If downloading new version, update this text accordingly

% QPB
The detector background is modeled by ESAS tasks \texttt{\{mos,pn\}\_back}
by (1) computing a quiescent spectrum from unexposed CCD corner counts,
(2) augmenting the quiescent spectrum with corner counts from public
observations with similar count rates and spectral hardness, and (3) scaling
the augmented quiescent spectrum shape to that expected for the source region
of interest, using the ratio of quiescent spectra across corner / source
regions from XMM's filter wheel closed observations and assuming that the ratio
of spectra across a given MOS or PN chip is time invariant.
See Sec. 3.4 of \citet{kuntz2008} for more detail and motivation.
\textbf{TODO: shorten this text}
% TODO this desperately needs to be shortened.
% but I do want make clear that this is a somewhat ad hoc model

% Instrumental lines
We model instrumental lines as zero-width, fixed-energy Gaussians and assume
that relative line strengths (i.e., line ratios) in a given CCD region are time
invariant, or, the same between FWC and observation data.
In spectrum fits, we use line normalizations from FWC data and vary a constant
prefactor for all instrumental lines in a given exposure.
For MOS exposures, we fit FWC spectra to two Gaussian lines at $1.49$ (Al) and
$1.75 \unit{keV}$ (Si) with a broken power-law continuum.
For PN, we fit seven Gaussian lines at $1.49$ (Al), $4.54$ (Ti), $5.44$ (Cr),
$7.49$ (Ni), $8.05$ (Cu), $8.62$ (Zn), and $8.90 \unit{keV}$ (Cu K-beta) with a
broken power-law continuum.
The PN Ti and Cr lines do not contribute significantly to observation spectra,
but are clearly visible in FWC spectra.

% SWCX
We neglect solar wind charge exchange (SWCX) emission, which affects a subset
of XMM-Newton observations \citep{snowden2004}.
The Hughes observation (0087940201) was not flagged by \citet{carter2011} as
containing time-variable soft X-ray emission indicative of exospheric SWCX,
% TODO what is the difference between these types of swcx?
Magnetospheric SWCX emissivity can be estimated by combining a magnetosphere
model \citep{spreiter1966} with Advanced Composition Explorer (ACE) solar wind
velocity and density measurements, implemented via an online tool by the United
States XMM-Newton Guest Observer Facility (XMM GOF).
\footnote{\href{https://heasarc.gsfc.nasa.gov/docs/xmm/scripts/xmm_trend.html}{https://heasarc.gsfc.nasa.gov/docs/xmm/scripts/xmm\_trend.html}}
\textbf{TODO find better reference?}  % TODO revise text, too
During the Hughes observation, modeled SWCX emission decreases by a
factor of $10$.
We extracted 0087940201 MOS1 remnant spectra early and late in the observation
and, by eye, saw no obvious differences in soft X-ray emission.
% TODO I will be re-doing this.
The XMM GOF magnetosphere model additionally predicts weaker SWCX emission for
0551000201 compared to 0087940201.
We thus argue that SWCX can safely be neglected in both observations.
% TODO reword, too verbose

% NOTE: see my notes from 2016 Jan 13 on spectrum cuts to evaluate SWCX
% 0551000201 was observed March 2009
%   Carter/Sembay analysis performed August 2009, so 0551000201 was likely
%   still in its proprietary period
% 0087940201 is not listed in Carter/Sembay, but meets basic selection criteria
% (namely, MOS1/MOS2 operating in full frame mode

% Residual soft proton contamination
Residual soft protons dominate the X-ray background above $\sim 3 \unit{keV}$
and are not removed by flare filtering due to their slow time variation.
We model residual soft protons by a power law not passed through CCD response
and effective area functions, following the ESAS prescription.

% =============================================================================
% Spectrum fits
% =============================================================================
\section{Integrated and spatially resolved spectral fits} \label{sec:spec}

\begin{figure*}[]
    \plotone{fig/fig_snr_regs-src-bkg.pdf}
    %\caption{Broadband 0.8--3.3 keV image of SNR G309.2-0.6.}
    % figcaption is an aastex alias -- cannot use with emulateapj
    \figcaption{Broadband 0.8--3.3 keV image of SNR G309.2-0.6 with region overlay.}
    \label{fig:regs}  % TODO this must follow figcaption, don't remember why
%Plotted with 0.843 GHz radio contours from the  supernova remnant catalogue,
%with resolution $\abt 43\arcsec$ and sensitivity $2$ mJy/beam
%\citep{whiteoak1996}.
\end{figure*}

We first fit the integrated remnant spectrum and sky background simultaneously
to a four component model non-equilibrium ionized plasma spectrum, implemented
as \texttt{vnei} \citep{hamilton1983} in XSPEC
\textbf{(is there a better reference?)}.
We also fit to \texttt{vpshock} and \texttt{vsedov} models, which integrate
emission over an assumed plane shock or Sedov geometry \citep{borkowski2001},
but find no significant difference in fitted plasma parameters, expected given
fitted electron temperatures of order $1 \unit{keV}$ \citep{borkowski2001}.
% The \texttt{vpshock} fit converges to $\tau_{\mt{l}} \sim 0$
% Discuss assumptions and arguments laid out by Borkowski, Lyerly, Reynolds
% (2001, ApJ 548:820).
% For a young-ish shock, ejecta-dominated, hot, we do expect vnei to be a good
% approximation (immediate post-shock electron temperature ~ 1/2000th proton
% temperature, Tau initial -> 0).
% Therefore forget overly detailed modeling.
A \texttt{vnei} model fits the emission-weighted plasma to a single temperature
and ionization age, which may obscure spatial structure in temperature,
ionization, and ejecta abundances.
Nevertheless, we find that it gives a good initial fit to the remnant and
allows us to cleanly characterize the X-ray background.
From the integrated
The fit is most representative if the reverse shock has completely propagated
through the ejecta, ejecta abundances are well-mixed, and the thermal state of
the plasma does not show strong gradients.
Obviously this will never be the case.

% Takeaway point from Borkowski+ 2001:
%   vpshock is a better approx to full Sedov solution than vnei model
%   at high electron temperatures (order ~ a few keV)
% BUT, at x-ray photon energies > ~ 0.5 keV, difference is very small.
% so vnei is not that bad.
% ...
% why do all these plots have emission measure on the x-axis?
    % See Sec. 3.3 of borkowski 2001 for vpshock model

% X-ray background model
The X-ray sky background model comprises:
(1) an unabsorbed collisional ionization equilibrium (CIE) plasma for local
bubble emission,
(2) an absorbed CIE plasma from the hotter galactic halo,
and
(3) an absorbed power law for the unresolved extragalactic X-ray background.
In XSPEC, this is \texttt{apec + TBabs*(apec + powerlaw)}.
We allow the plasma temperatures and absorption column to vary,
but fix the extragalactic X-ray power law to index $1.4$ and normalization
$10.9 \unit{cm^{-2}\, s^{-1}\, sr^{-1}\, keV^{-1}}$ from \textit{Chandra} Deep Field
South analysis \citep{hickox2006}.
These values are generally within 10\% of values derived from a range of other
observatories (\textit{Swift}, \textit{ASCA}, \textit{ROSAT}) and pointings
\citep[e.g.,][]{chen1997, kushino2002, de-luca2004, moretti2009}.

% TODO - ...
% XSPEC manual states apec norm = 10^-14 integral(...)dV / (4 pi D_A^2 (1+z)^2)
%
%   Unabsorbed apec: 2.9e-4 x 10^14 = 2.9e10 cm^-6 cm
%   Absorbed apec: 2.6e-3 x 10^14 = 2.6e11 cm^-6 cm
%
% I'm very confused.  The brightnesses from Henley and Shelton (2013)
% have range 0.4e-3 -- 7e-3 cm^-6 pc, or 1.23e15 -- 2.1e16 cm^-6 cm
% which is 5-6 orders of magnitude off!..
%
% The norm straight out of XSPEC, when compared with values stated as "cm^-6
% pc", looks about right.  But it seems unlikely to be an error...
%
% Temperature ~ 2.2e6 K (interquartile range 0.63e6 K).
% Assuming E = 3/2 k T, this translates to 0.29 keV
% Our fit value of 0.64 keV is ~2x larger.
%
% But this seems reasonable for an observation in the galactic plane.
% TODO QUESTION does the galactic halo have a temperature gradient?

We obtain unabsorbed plasma temperature $0.26 \unit{keV}$, absorption column
$1.34 \times 10^{22} \unit{cm^{-2}}$, and absorbed halo temperature
$0.64 \unit{keV}$.  The \texttt{apec} normalizations are $2.9\times10^{-4}$
and $2.6\times10^{-3}$ for each plasma.
% TODO: validate per notes...
The absorbed plasma emission measure is within the range expected for galactic
halo emission, but its temperature is $\abt 2\times$ that deduced from high
galactic latitude XMM pointings \citep{henley2013}.
The absolute temperature discrepancy $\abt 2 \times 10^6 \unit{K}$
is larger that what might be expected for cross-satellite and statistical
uncertainties \citep{henley2015} -- although, the cloud shadowing measurements
of \citet{henley2015} were also well out of the galactic plane.

Fixing the X-ray background to that derived from our integrated source vnei fit
could introduce a systematic bias, if the vnei assumption for the integrated
source is NOT GOOD and therefore the x-ray background fit is systematically
skewed.
Because of this I would argue that fitting the x-ray background from the full
annulus is preferable.
This is probably a small / subtle effect.
But I'll revisit this analysis to be sure.

% Variability of the sky X-ray background
%%It is unclear whether the sky X-ray background can be treated as constant over
%%scales of several arcminutes.
%%\citet{henley2013} modeled galactic halo emission using 110 XMM-Newton
%%observations well outside the galactic plane.
%%Most observations were widely separated, but a group of pointings within $\abt
%%30\arcmin$ showed halo temperature variations $\abt 10$--$20\%$ and emission
%%measure variation within a factor of $\abt 2$
% TODO I am totally just eyeballing the numbers right now -- not that useful.
% Possibly remove this.
% Henley 2013: looking at 28 clustered observations well out of the galactic
% plane, sightlines 103.1 to 103.27; 103.8 is two observations.
% SZE SurF project to complement SPT,APEX,ACT study.

\begin{table*}
    \centering
    \caption{Integrated remnant and X-ray background fit
    \label{tab:fits}}
    \begin{tabular}{@{}lrrrrrr@{}}
        \toprule
        Region & $n_\mathrm{H}$             & $kT$  & $\tau$   & Si  & S   & $\chi^2_{\mathrm{red}} (\mathrm{dof}$) \\
               & ($10^{22} \unit{cm^{-2}}$) & (keV) & ($10^{10} \unit{s\;cm^{-3}}$) & (-) & (-) &  \\
        \midrule
        Integrated SNR & 2.12 & 2.39 & 1.89 & 3.81 & 3.36 & 1.213 (3768) \\  % 20160421_src_and_bkg_row.tex
        \midrule
        N clump & 2.86 & 0.98 & 4.99 & 4.51 & 4.11 & 1.336 (691) \\  % 20160420_src_north_clump_nH_free_row.tex
        E lobe  & 3.34 & 1.34 & 2.77 & 3.25 & 2.07 & 0.939 (206) \\  % 20160420_src_E_lobe_nH_free_row.tex
        SW lobe & 1.87 & 4.04 & 1.29 & 3.47 & 4.06 & 1.324 (437) \\  % 20160420_src_SW_lobe_nH_free_row.tex
        SE dark & 2.45 & 1.01 & 4.71 & 2.17 & 1.37 & 0.958 (221) \\  % 20160420_src_SE_dark_nH_free_row.tex
        \midrule
        % Bad fits (!)
        Pre ridge & 1.34 & 8.53 & 2.86 & 0.50 & 0.00 & 1.238 (25) \\  % 20160420_src_pre_ridge_nH_free_row.tex
        Ridge         & 2.81 & 0.76 & 8.52 & 2.07 &  1.67 & 1.390 (224) \\  % 20160420_src_ridge_nH_free_row.tex
        SE dark ridge & 2.86 & 1.11 & 0.53 & 6.00 & 10.00 & 1.103 (101) \\  % 20160420_src_SE_ridge_dark_nH_free_row.tex
        \bottomrule
    \end{tabular}
\end{table*}

\begin{figure*}[]
    \plotone{fig/fig_src_and_bkg_fit_temp.pdf}
    \figcaption{Messy plot of individual spectra + x-ray background with residuals below}
    \label{fig:spec-src}  % TODO this must follow figcaption, don't remember why
\end{figure*}

% Fit discrepancy wrt. Rakowski / Safi-Harb
Our fitted absorption column density $2.1 \times 10^{22} \unit{cm^{-2}}$ is
higher than $0.65 \times 10^{22} \unit{cm^{-2}}$ from a similar \texttt{vnei}
fit of the Hughes observation, briefly reported by \citet{safi-harb2007}.
\citet{safi-harb2007} freed O, Ne, Mg, Ar, Ca, Fe, and Ni abundances in
addition to Si and S and obtained qualitatively similar results (i.e.,
emission-weighted averaged plasma temperature $\abt 2 \unit{keV}$ and enhanced
Si and S abundances), so only column density is discrepant.
The reason for this is unclear, though we speculate that different background
treatment (modeling, subtraction) may play a role.
% See my notes: 2016 March 15-22
% with a somewhat older version of fitting code, but results should be about
% the same... I did not explore further by e.g. bounding elemental abundances

% Discussion: nH value discrepancies
We comment -- though it may be already obvious to the X-ray community -- that
our fitted absorption columns are significantly larger than those derived from
HI and dust maps \citep[e.g.,][]{willingale2013}.
For G309.2-0.6, the \textit{Swift} galactic $\nH$ calculator implementation of
\citet{willingale2013} provides an expected column density $1.81 \nHunits$.
Dust scattering can bias X-ray spectrum fits of absorption alone to
overestimate the absorbing column by $\abt25\%$ \citep{corrales2016}.
% TODO -- possibly pull out their model and use it... cite randall2016 as well if needed
Our absorption column estimate varies strongly with our model assumptions and
spectrum extraction regions (Table~\ref{tab:fits}), but a 25\% overestimate of
column density is consistent with our integrated remnant fit column of $2.1
\nHunits$.
The \textit{spectrum fit value} of $\nH$ may also be estimated from empirical
linear regression against optical extinction $\AV$ \citep[e.g.,][]{foight2015}
-- which benefits from a strong sampling of sources within the galactic plane,
and empirical validation -- but does not offer physical insight into the
discrepancy. We find approximate agreement ($2$--$3 \nHunits$) with the various
relations put
\textbf{TODO: for now detailed accounting of absorption is beyond the scope of
this work, but may be revisited...}


% =============================================================================
% Discussion, interpretation, ... more descriptive name tbd
% =============================================================================
\section{Morphology}

We resolve the bulk of SNR emission to a bright blob towards the north of the
remnant (assuming that the remnant may be delineated by the MOST contours),
part of a slightly dimmer ridge arcing towards the northwest.
The south is relatively dark.

\citetalias{rakowski2001} previously found an unresolved blob blended with the bright O
star.

So what?

This doesn't seem interesting unless we can start drawing conclusions from
systematic study of a population of remnants.



Main observation: nH values are generally quite high (2-3e22).

If nH is low, we can still get reasonable kT and Si/S abundances
by driving down O,Ne,Fe emission. (based on north clump fits)
IF we force solar O/Ne/Fe, lower nH (below 2) quickly starts to look
unphysical.

north clump fit is ok not amazing.
SE dark gives OK fit with both Si/S solar...

Fits with: PN power law frozen, generally Si or both Si/S thawed



Balmer shocks taken as evidence for progenitor NOT photo-ionizing too much
stuff (requires neutral H ahead of shock).  So Balmer shocks at small radii can
be taken as evidence that the progenitor cannot have ionized "too much" stuff.
But, the absence of Balmer shocks doesn't tell us as much.
The shock could be weak, the density could be low, the material could be fully
ionized.

\begin{table*}
    \centering
    \caption{G309.2-0.6 -- annulus fits}
    \begin{tabular}{@{}lrrrrrr@{}}
        \toprule
        Region & $n_\mathrm{H}$             & $kT$  & $\tau$                        & Si  & S   & $\chi^2_{\mathrm{red}} (\mathrm{dof}$) \\
               & ($10^{22} \unit{cm^{-2}}$) & (keV) & ($10^{10} \unit{s\;cm^{-3}}$) & (-) & (-) &  \\
        \midrule
        $0$--$100\arcsec$ & 1.59 & 9.52 & 1.68e+10 & 2.34 & 2.01 & 1.893 (356) \\  % File ann_000_100_row.tex
        $100$--$200\arcsec$ & 2.17 & 1.68 & 2.24e+10 & 5.96 & 5.56 & 1.189 (559) \\  % File ann_100_200_row.tex
        $200$--$300\arcsec$ & 2.86 & 0.82 & 7.33e+10 & 3.31 & 3.21 & 1.268 (792) \\  % File ann_200_300_row.tex
        $300$--$400\arcsec$ & 2.59 & 0.73 & 1.03e+11 & 4.01 & 3.76 & 1.276 (799) \\  % File ann_300_400_row.tex
        $400$--$500\arcsec$ & 4.77 & 0.40 & 8.90e+10 & 4.34 & 10.99 & 1.186 (822) \\  % File ann_400_500_row.tex
        \midrule
        $0$--$100\arcsec$ (ONeMg) & 1.59 & 6.22 & 1.72e+10 & 1.97 & 1.42 & 1.543 (353) \\  % File ann_000_100_ONeMg_row.tex
        \bottomrule
    \end{tabular}
\end{table*}


NOTE: SW lobe and ridge both show weird shifts in PN spectral lines (Si, S)
Details tbd (pending updates to plotting tools...)


\acknowledgments

X acknowledges support by contract ...

This research is based on observations obtained with XMM-Newton, an ESA science
mission with instruments and contributions directly funded by ESA Member States
and NASA.
The MOST is operated by The University of Sydney with support from the
Australian Research Council and the Science Foundation for Physics within The
University of Sydney.

This research has made extensive use of NASA's Astrophysics Data System.
% TODO is software acknowledgement sufficient?
This research made use of Astropy, a community-developed core Python package
for Astronomy \citep{astropy2013}.
This research also made use of APLpy, an open-source plotting package for
Python hosted at \href{http://aplpy.github.com}{http://aplpy.github.com}.
This research was expedited in part by Jonathan Sick's \texttt{ads2bibdesk}.

{\it Facilities:} \facility{XMM (EPIC)}, \facility{Molonglo Observatory}

\software{APLpy, Astropy}

%\listofchanges

% ==========
% References
% ==========
\bibliographystyle{aasjournal}
\bibliography{refs-snr}

% ========
% Appendix
% ========
\newpage
\clearpage  % Use \clearpage over \newpage
\appendix

\setcounter{table}{0}
\renewcommand{\thetable}{A\arabic{table}}
\setcounter{figure}{0}
\renewcommand{\thefigure}{A\arabic{figure}}


%\begin{table*}
%    \centering
%    \caption{G309.2-0.6 -- sub-source region fits, $n_H=2.5$ fixed}
%    \begin{tabular}{@{}lrrrrrr@{}}
%        \toprule
%        Region & $n_\mathrm{H}$             & $kT$  & $\tau$                        & Si  & S   & $\chi^2_{\mathrm{red}} (\mathrm{dof}$) \\
%               & ($10^{22} \unit{cm^{-2}}$) & (keV) & ($10^{10} \unit{s\;cm^{-3}}$) & (-) & (-) &  \\
%        \midrule
%        north clump & 2.50 & 1.07 & 4.79e+10 & 5.73 & 5.22 & 1.339 (693) \\  % File src_north_clump_nH-2.5_row.tex
%        E lobe & 2.50 & 1.74 & 2.71e+10 & 5.47 & 3.69 & 1.016 (208) \\  % File src_E_lobe_nH-2.5_row.tex
%        SW lobe & 2.50 & 1.27 & 1.37e+10 & 3.46 & 5.78 & 1.429 (439) \\  % File src_SW_lobe_nH-2.5_row.tex
%        SE dark & 2.50 & 0.68 & 8.80e+10 & 2.79 & 2.30 & 0.983 (223) \\  % File src_SE_dark_nH-2.5_row.tex
%        \midrule
%        ridge & 2.50 & 0.81 & 8.69e+10 & 2.63 & 2.10 & 1.467 (226) \\  % File src_ridge_nH-2.5_row.tex
%        SE dark ridge & 2.50 & 0.78 & 2.51e+11 & 1(fr) & 1(fr) & 1.232 (105) \\  % File src_SE_ridge_dark_nH-2.5_row.tex
%        \bottomrule
%    \end{tabular}
%\end{table*}
%
%\begin{table*}
%    \centering
%    \caption{G309.2-0.6 -- annulus fits, $n_H=2.5$ fixed}
%    \begin{tabular}{@{}lrrrrrr@{}}
%        \toprule
%        Region & $n_\mathrm{H}$             & $kT$  & $\tau$                        & Si  & S   & $\chi^2_{\mathrm{red}} (\mathrm{dof}$) \\
%               & ($10^{22} \unit{cm^{-2}}$) & (keV) & ($10^{10} \unit{s\;cm^{-3}}$) & (-) & (-) &  \\
%        \midrule
%        $0$--$100\arcsec$ & 2.50 & 2.33 & 1.58e+10 & 1.74 & 1.27 & 2.756 (357) \\  % File ann_000_100_nH-2.5_row.tex
%        $100$--$200\arcsec$ & 2.50 & 1.22 & 2.72e+10 & 5.29 & 5.04 & 1.210 (560) \\  % File ann_100_200_nH-2.5_row.tex
%        $200$--$300\arcsec$ & 2.50 & 1.07 & 4.51e+10 & 3.83 & 3.56 & 1.291 (793) \\  % File ann_200_300_nH-2.5_row.tex
%        $300$--$400\arcsec$ & 2.50 & 0.78 & 9.05e+10 & 4.11 & 3.74 & 1.275 (800) \\  % File ann_300_400_nH-2.5_row.tex
%        $400$--$500\arcsec$ & 2.50 & 0.99 & 3.99e+10 & 8.57 & 10.19 & 1.188 (823) \\  % File ann_400_500_nH-2.5_row.tex
%        \bottomrule
%    \end{tabular}
%\end{table*}

\begin{table*}
    \centering
    \caption{sub-region fits, varying $n_H$}
    \begin{tabular}{@{}lrrrrrr@{}}
        \toprule
        Region & $n_\mathrm{H}$             & $kT$  & $\tau$                        & Si  & S   & $\chi^2_{\mathrm{red}} (\mathrm{dof}$) \\
               & ($10^{22} \unit{cm^{-2}}$) & (keV) & ($10^{10} \unit{s\;cm^{-3}}$) & (-) & (-) &  \\
        \midrule
        north clump & 1.50 & 8.50 & 1.81e+10 & 8.76 & 8.92 & 1.933 (692) \\  % File 20160420_src_north_clump_nH_1.5_row.tex
        north clump & 2.00 & 2.42 & 2.08e+10 & 6.52 & 5.95 & 1.623 (692) \\  % File 20160420_src_north_clump_nH_2.0_row.tex
        north clump & 2.50 & 1.33 & 3.23e+10 & 5.15 & 4.59 & 1.361 (692) \\  % File 20160420_src_north_clump_nH_2.5_row.tex
        north clump & 3.00 & 0.91 & 5.46e+10 & 4.27 & 3.98 & 1.338 (692) \\  % File 20160420_src_north_clump_nH_3.0_row.tex
        \midrule
        E lobe & 1.50 & 8.62 & 2.02e+10 & 9.71 & 7.85 & 1.331 (207) \\  % File 20160420_src_E_lobe_nH_1.5_row.tex
        E lobe & 2.00 & 5.68 & 1.89e+10 & 5.80 & 4.25 & 1.086 (207) \\  % File 20160420_src_E_lobe_nH_2.0_row.tex
        E lobe & 2.50 & 2.75 & 2.03e+10 & 4.53 & 2.97 & 0.979 (207) \\  % File 20160420_src_E_lobe_nH_2.5_row.tex
        E lobe & 3.00 & 1.73 & 2.46e+10 & 3.65 & 2.30 & 0.941 (207) \\  % File 20160420_src_E_lobe_nH_3.0_row.tex
        \midrule
        SW lobe & 1.50 & 9.21 & 1.44e+10 & 3.98 & 4.79 & 1.390 (438) \\  % File 20160420_src_SW_lobe_nH_1.5_row.tex
        SW lobe & 2.00 & 3.09 & 1.27e+10 & 3.40 & 4.11 & 1.326 (438) \\  % File 20160420_src_SW_lobe_nH_2.0_row.tex
        SW lobe & 2.50 & 1.64 & 1.28e+10 & 3.06 & 4.50 & 1.403 (438) \\  % File 20160420_src_SW_lobe_nH_2.5_row.tex
        SW lobe & 3.00 & 1.13 & 1.30e+10 & 2.93 & 5.32 & 1.522 (438) \\  % File 20160420_src_SW_lobe_nH_3.0_row.tex
        \midrule
        SE dark & 1.50 & 5.47 & 1.69e+10 & 3.26 & 2.39 & 1.079 (222) \\  % File 20160420_src_SE_dark_nH_1.5_row.tex
        SE dark & 2.00 & 1.79 & 2.43e+10 & 2.54 & 1.56 & 0.996 (222) \\  % File 20160420_src_SE_dark_nH_2.0_row.tex
        SE dark & 2.50 & 1.00 & 4.60e+10 & 2.10 & 1.32 & 0.956 (222) \\  % File 20160420_src_SE_dark_nH_2.5_row.tex
        SE dark & 3.00 & 0.70 & 7.14e+10 & 1.83 & 1.35 & 0.985 (222) \\  % File 20160420_src_SE_dark_nH_3.0_row.tex
        \midrule
        ridge & 1.50 & 3.46 & 2.13e+10 & 4.58 & 3.22 & 2.476 (225) \\  % File 20160420_src_ridge_nH_1.5_row.tex
        ridge & 2.00 & 1.64 & 2.88e+10 & 3.03 & 2.13 & 1.764 (225) \\  % File 20160420_src_ridge_nH_2.0_row.tex
        ridge & 2.50 & 0.96 & 5.67e+10 & 2.33 & 1.74 & 1.504 (225) \\  % File 20160420_src_ridge_nH_2.5_row.tex
        ridge & 3.00 & 0.66 & 1.03e+11 & 1.96 & 1.70 & 1.370 (225) \\  % File 20160420_src_ridge_nH_3.0_row.tex
        \midrule
        % NOTE: may need to re-do these fits
        % (1) without SNR component, (2) with SNR, Si,S fixed
        SE dark ridge & 1.50 & 2.26 & 8.48e+11 & 1.36 & 0.45 & 1.252 (102) \\  % File 20160420_src_SE_ridge_dark_nH_1.5_row.tex
        SE dark ridge & 2.00 & 10.00 & 7.39e+09 & 3.95 & 10.00 & 1.081 (102) \\  % File 20160420_src_SE_ridge_dark_nH_2.0_row.tex
        SE dark ridge & 2.50 & 3.48 & 6.78e+09 & 3.68 & 6.48 & 1.080 (102) \\  % File 20160420_src_SE_ridge_dark_nH_2.5_row.tex
        SE dark ridge & 3.00 & 1.39 & 7.11e+09 & 3.85 & 7.64 & 1.073 (102) \\  % File 20160420_src_SE_ridge_dark_nH_3.0_row.tex
        %SE dark ridge & 1.50 & 1.73 & 1.21e+11 & 1.00 & 1.00 & 1.344 (105) \\  % File src_SE_ridge_dark_nH-1.5_row.tex
        %SE dark ridge & 2.00 & 1.36 & 6.70e+10 & 1.00 & 1.00 & 1.284 (105) \\  % File src_SE_ridge_dark_nH-2.0_row.tex
        %SE dark ridge & 2.50 & 0.78 & 2.51e+11 & 1.00 & 1.00 & 1.232 (105) \\  % File src_SE_ridge_dark_nH-2.5_row.tex
        %SE dark ridge & 3.00 & 0.79 & 5.14e+10 & 1.00 & 1.00 & 1.196 (105) \\  % File src_SE_ridge_dark_nH-3.0_row.tex
        \bottomrule
    \end{tabular}
\end{table*}

\begin{table*}
    \centering
    \caption{Annulus fits with varying $n_H$}
    \begin{tabular}{@{}lrrrrrr@{}}
        \toprule
        Region & $n_\mathrm{H}$             & $kT$  & $\tau$                        & Si  & S   & $\chi^2_{\mathrm{red}} (\mathrm{dof}$) \\
               & ($10^{22} \unit{cm^{-2}}$) & (keV) & ($10^{10} \unit{s\;cm^{-3}}$) & (-) & (-) &  \\
        \midrule
        000--100$\arcsec$ & 1.50 &10.00 & 1.75e+10 & 2.51 & 2.15 & 2.184 (352) \\  % File 20160420_ann_000_100_nH_1.5_row.tex
        000--100$\arcsec$ & 2.00 & 4.55 & 1.59e+10 & 2.04 & 1.53 & 2.338 (352) \\  % File 20160420_ann_000_100_nH_2.0_row.tex
        000--100$\arcsec$ & 2.50 & 2.43 & 1.58e+10 & 1.77 & 1.25 & 2.884 (352) \\  % File 20160420_ann_000_100_nH_2.5_row.tex
        000--100$\arcsec$ & 3.00 & 1.58 & 1.78e+10 & 1.57 & 1.11 & 3.479 (352) \\  % File 20160420_ann_000_100_nH_3.0_row.tex
        \midrule
        100--200$\arcsec$ & 1.50 & 8.47 & 1.67e+10 & 7.56 & 7.97 & 1.375 (559) \\  % File 20160420_ann_100_200_nH_1.5_row.tex
        100--200$\arcsec$ & 2.00 & 2.83 & 1.72e+10 & 5.96 & 5.64 & 1.196 (559) \\  % File 20160420_ann_100_200_nH_2.0_row.tex
        100--200$\arcsec$ & 2.50 & 1.58 & 2.14e+10 & 4.93 & 4.63 & 1.210 (559) \\  % File 20160420_ann_100_200_nH_2.5_row.tex
        100--200$\arcsec$ & 3.00 & 1.04 & 2.85e+10 & 4.29 & 4.21 & 1.303 (559) \\  % File 20160420_ann_100_200_nH_3.0_row.tex
        \midrule
        200--300$\arcsec$ & 1.50 & 6.67 & 1.84e+10 & 5.98 & 5.85 & 1.742 (792) \\  % File 20160420_ann_200_300_nH_1.5_row.tex
        200--300$\arcsec$ & 2.00 & 2.39 & 2.12e+10 & 4.35 & 3.92 & 1.412 (792) \\  % File 20160420_ann_200_300_nH_2.0_row.tex
        200--300$\arcsec$ & 2.50 & 1.37 & 3.09e+10 & 3.41 & 3.03 & 1.284 (792) \\  % File 20160420_ann_200_300_nH_2.5_row.tex
        200--300$\arcsec$ & 3.00 & 0.91 & 5.39e+10 & 2.84 & 2.65 & 1.280 (792) \\  % File 20160420_ann_200_300_nH_3.0_row.tex
        \midrule
        300--400$\arcsec$ & 1.50 & 8.59 & 1.82e+10 & 4.57 & 4.19 & 1.255 (799) \\  % File 20160420_ann_300_400_nH_1.5_row.tex
        300--400$\arcsec$ & 2.00 & 2.95 & 1.87e+10 & 3.56 & 2.98 & 1.202 (799) \\  % File 20160420_ann_300_400_nH_2.0_row.tex
        300--400$\arcsec$ & 2.50 & 1.61 & 2.31e+10 & 2.92 & 2.49 & 1.201 (799) \\  % File 20160420_ann_300_400_nH_2.5_row.tex
        300--400$\arcsec$ & 3.00 & 1.00 & 3.86e+10 & 2.56 & 2.19 & 1.220 (799) \\  % File 20160420_ann_300_400_nH_3.0_row.tex
        \midrule
        400--500$\arcsec$ & 1.50 & 10.00 & 2.20e+10 & 2.07 & 2.50 & 1.254 (822) \\  % File 20160420_ann_400_500_nH_1.5_row.tex
        400--500$\arcsec$ & 2.00 &  4.34 & 4.34e+13 & 0.00 & 0.00 & 1.267 (822) \\  % File 20160420_ann_400_500_nH_2.0_row.tex
            % This fit obviously failed
        400--500$\arcsec$ & 2.50 & 10.00 & 1.53e+10 & 1.34 & 2.16 & 1.184 (822) \\  % File 20160420_ann_400_500_nH_2.5_row.tex
        400--500$\arcsec$ & 3.00 & 10.00 & 1.32e+10 & 1.50 & 2.48 & 1.171 (822) \\  % File 20160420_ann_400_500_nH_3.0_row.tex
        \bottomrule
    \end{tabular}
\end{table*}


\end{document}
